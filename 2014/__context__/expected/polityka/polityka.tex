\usemodule[pycon-2014]
\starttext


\section[co-tam-panie-w-polityce-czyli-czym-zaskoczył-nas-python-3.4---marcin-bardź]{Co
tam, panie, w polityce? Czyli czym zaskoczył nas Python 3.4 - Marcin
Bardź}

Python w wersji 3.4 światło dzienne ujrzał 16 marca 2014. To wydanie nie
wprowadza żadnych zmian do samego języka, zamiast tego mamy kilka nowych
bibliotek, usprawnienia w już istniejących modułach oraz wiele ulepszeń
\quotation{pod maską}.

Ta pozorna stagnacja w rozwoju języka została zaplanowana oraz
wprowadzona w życie z pełną premedytacją (PEP 3003 o złowróżebnym tytule
{\em Python Language Moratorium}) i ma ona na celu umożliwienie
\quotation{dogonienia} bazowej implementacji (CPython) przez inne
implementacje języka, takie jak Jython czy PyPy.

Brak zmian w składni nie oznacza jednak, że przeciętny pythonista nie
dostanie do rąk nowych zabawek, ciekawych narzędzi, a jego życie nie
stanie się jeszcze prostsze.

\subsection[nowe-biblioteki]{Nowe biblioteki}

Najnowsza odsłona naszego ulubionego gada raczy nas pokaźną baterią
całkiem nowych bibliotek, wśród których każdy użytkownik powinien
znaleźć coś dla siebie.

\section[asyncio]{\type{asyncio}}

Potężna biblioteka umożliwiająca tworzenie kodu współbieżnego,
przełączany dostęp do zasobów we/wy, uruchamianie klientów/serwerów
sieciowych, a to wszystko w jednym jedynym wątku! Dla osób znających
Twisted nie będzie to nic nowego, mimo to wprowadzenie tak potężnego
narzędzia do biblioteki standardowej otwiera wiele nowych możliwości.

Szczegółowy opis biblioteki wykracza daleko poza ramy niniejszego
artykułu, dlatego wymienię tylko główne różnice pomiędzy Twisted i
\type{asyncio}:

\startitemize
\item
  \type{asyncio} jest prostsze i składa się z mniejszej liczby modułów.
  Z jednej strony \type{asyncio} nie ma wszystkich możliwości Twisted, z
  drugiej jednak powinno być łatwiejsze w użyciu i nowy użytkownik
  powinien móc się szybciej wdrożyć (korzystniejsza krzywa uczenia się).
\item
  Dokumentacja \type{asyncio} jest przejrzysta i ułożona w sposób
  intuicyjny dla przeciętnego programisty Pythona.
\item
  \type{asyncio} wspiera najnowsze wersje Pythona i potrafi wykorzystać
  jego dobrodziejstwa (np. składnię \type{yield from}).
\stopitemize

\section[ensurepip]{\type{ensurepip}}

Zaczęło się od PEP 453, który namaścił \type{pip} jako rekomendowane
narzędzie zarządzania bibliotekami. Gdy już PEP został~zaakceptowany,
należało się więc upewnić, że użytkownik będzie miał dostęp do tego
błogosławionego narzędzia.

Zasadniczo, standardowa instalacja Pythona powinna zawierać \type{pip} w
wersji, która była najnowszą w momencie pojawenia się RC danego wydania.
Jeśli jednak \type{pip} by się nam gdzieś zapodział (np. w wirtualnym
środowisku), wystarczy wywołać z linii polecń:

\starttyping
$ python -m ensurepip
\stoptyping

I już mamy pewność, że \type{pip} jest dostępny. Co ciekawe, żeby
wykonać powyższą komendę, nie jest potrzebny dostęp do internetu, gdyż
\type{ensurepip} przechowuje na swoje potrzeby lokalną kopię biblioteki
\type{pip}.

Przeciętny użytkownik może nie musieć w ogóle igrać z modułem
\type{ensurepip}, a cały ten cyrk wynika z faktu, że \type{pip} jest
niezależnym projektem, posiadającym własny cykl wydawniczy.

\section[enum]{\type{enum}}

Po wielu latach i po wielu niezależnych implementacjach, Python doczekał
się w końcu swoich własnych typów wyliczeniowych. Dzięki nim można teraz
pisać elegancki i mniej podatny na błędy kod.

W myśl zasady, że jedna linijka kodu znaczy więcej, niż tysiąc słów,
przedstawiam poniżej próbkę możliwości modułu:
\mono{python     \lettermore{}\lettermore{}\lettermore{} from enum import Enum     \lettermore{}\lettermore{}\lettermore{} class Osoba(Enum):     ...     ja = 1     ...     ty = 2     ...     on\letterunderscore{}ona\letterunderscore{}ono = 3     ...     ona = 3     ...     ono = 3     ...      \lettermore{}\lettermore{}\lettermore{} Osoby = Enum('Osoby', 'my wy oni\letterunderscore{}one', module=\letterunderscore{}\letterunderscore{}name\letterunderscore{}\letterunderscore{})     \lettermore{}\lettermore{}\lettermore{} być = \letteropenbrace{}     ...     Osoba.ja: "jestem",     ...     Osoba.ty: "jesteś",     ...     Osoba.on\letterunderscore{}ona\letterunderscore{}ono: "jest",     ...     Osoby.my: "jesteśmy",     ...     Osoby.wy: "jesteście",     ...     Osoby.oni\letterunderscore{}one: "są"     ... \letterclosebrace{}     \lettermore{}\lettermore{}\lettermore{} for lp, lmn in zip(Osoba, Osoby):     ...     poj = "\letteropenbrace{}\letterclosebrace{}. \letteropenbrace{}\letterclosebrace{} \letteropenbrace{}\letterclosebrace{}".format(lp.value, lp.name, być{[}lp{]})     ...     mn = "\letteropenbrace{}\letterclosebrace{}. \letteropenbrace{}\letterclosebrace{} \letteropenbrace{}\letterclosebrace{}".format(lmn.value, lmn.name, być{[}lmn{]})     ...     print("\letteropenbrace{}:20\letterclosebrace{}\letteropenbrace{}:20\letterclosebrace{}".format(poj, mn))     ...      1. ja jestem        1. my jesteśmy     2. ty jesteś        2. wy jesteście     3. on\letterunderscore{}ona\letterunderscore{}ono jest  3. oni\letterunderscore{}one są     \lettermore{}\lettermore{}\lettermore{} print("Myślę, więc \%s" \% być{[}Osoba.ja{]})     Myślę, więc jestem     \lettermore{}\lettermore{}\lettermore{} print(Osoba(3))     Osoba.on\letterunderscore{}ona\letterunderscore{}ono     \lettermore{}\lettermore{}\lettermore{} print(Osoba.ono)     Osoba.on\letterunderscore{}ona\letterunderscore{}ono     \lettermore{}\lettermore{}\lettermore{} Osoby{[}'oni\letterunderscore{}one'{]}     \letterless{}Osoby.oni\letterunderscore{}one: 3\lettermore{}}

\section[pathlib]{\type{pathlib}}

Kolejna obszerna biblioteka, przenosząca operacje na ścieżkach i plikach
z prehistorii do świata programowania obiektowego. Moduł łączy w sobie
funkcjonalności \type{os.path}, \type{glob} oraz wielu funkcji z innych
bibliotek (głównie \type{os}), opakowując wszystko w przepyszną
obiektową otoczkę.

Klasy biblioteki podzielone są na dwie odrębne części:

\startitemize
\item
  Czyste ścieżki ({\em pure paths}) - umożliwiające operacje na samych
  ścieżkach.
\item
  Konkretne ścieżki ({\em concrete paths}) - dodające jeszcze operacje
  we/wy.
\stopitemize

Ponadto, moduł udostępnia dwa warianty klas, jeden dla systemów
POSIXowych i jeden dla ścieżek Windowsowych, ale zwykły zjadacz chleba
do szczęścia potrzebuje tylko jednej klasy \type{Path}, która
reprezentuje konkretną ścieżkę zgodną z aktualnym systemem operacyjnym.

Czasy poszukiwań potrzebnych funkcji plikowych odchodzą w niepamięć,
gdyż użycie \type{pathlib} jest tak wygodne i intuicyjne, że aż trudno
uwierzyć, że to cudeńko pojawiło się w bibliotece standardowej dopiero
teraz.

Oto kilka przykładów użycia \type{pathlib}:

\starttyping
    >>> from pathlib import Path
    >>> p = Path('/')
    >>> p.is_dir()
    True
    >>> q = p / 'etc' / 'resolv.conf'
    >>> q
    PosixPath('/etc/resolv.conf')
    >>> q.exists()
    True
    >>> q.owner()
    'root'
    >>> with q.open() as f:
    ...     print(f.readline())
    ... 
    #

    >>> q.parts
    ('/', 'etc', 'resolv.conf')
    >>> q.parents[1]
    PosixPath('/')
    >>> q.parents[2]
    Traceback (most recent call last):
    ...
    raise IndexError(idx)
    IndexError: 2
    >>> q.name
    'resolv.conf'
    >>> q.suffix
    '.conf'
    >>> q.relative_to('/home')
    Traceback (most recent call last):
    ...
    ValueError: '/etc/resolv.conf' does not start with '/home'
    >>> list(q.parent.glob('a*.conf'))
    [PosixPath('/etc/asl.conf'), PosixPath('/etc/autofs.conf')]
\stoptyping

\section[selectors]{\type{selectors}}

Moduł ten udostępnia wysokopoziomowe mechanizmy przełączania we/wy. Jest
to abstrakcyjny i w pełni obiektowy, a co za tym idzie łatwiejszy w
użyciu odpowiednik niskopoziomowej biblioteki \type{select}.

Trik z parą modułów służących do realizacji tego samego zadania nie jest
niczym nowym w Pythonie, zastosowano go już wcześniej do obsługi
wielowątkowości (niskopoziomowy \type{_thread} i wysokopoziomowy
\type{threading}).

Programista otrzymuje do ręki (między innymi) klasę
\type{DefaultSelector}, która jest abstrakcją najbardziej efektywnej
implementacji selektora dla danej platformy, a dzięki wspólnej klasie
bazowej \type{BaseSelector}, dostępny jest jednolity interfejs obsługi,
co przekłada się na czytelny, niezależny od platformy kod, bez
niepotrzebnych klauzul \type{if}.

\section[statistics]{\type{statistics}}

Tym modułem twórcy Pythona starają się uszczęśliwić statystyków,
księgowych, maklerów oraz wszystkich pozostałych im podobnych.
Znajdziemy tu funkcje liczące średnią, medianę, odchylenie standardowe i
kilka innych tajemniczych rzeczy pokroju wariancji.

Wśród wszystkich tych funkcji, na szczególną uwagę zasługuje
\type{mode()}, wyszukująca najczęściej występujący element w dyskretnym
zbiorze danych (lub rzucająca wyjątkiem \type{StatisticsError}, jeśli
takiego elementu nie ma). Funkcja \type{mode()} bywa przydatna nie tylko
księgowym i może czasem zaoszczędzić kilka linijek kodu.

Moduł zapewne będzie się jeszcze rozwijał, obrastając w nowe funkcje,
gdyż na chwilę obecną zawiera tylko niewielki wycinek tego, co dla dobra
ludzkości wymyślili statystycy.

\subsection[tracemalloc]{\type{tracemalloc}}

Ostatnia na liście, ale zdecydowanie warta uwagi biblioteka, która
pozwala na tworzenie migawek ({\em snapshot}) alokowanych bloków pamięci
oraz przetwarzanie ich na różne sposoby.

Moduł udostępnia trzy rodzaje informacji:

\startitemize
\item
  Traceback miejsca alokacji obiektu.
\item
  Statystyki przydzielonej pamięci (dla pliku, dla linii kodu).
\item
  Porównywanie migawek w celu wykrycia wycieków.
\stopitemize

Poniższy plik (nazwałem go \type{t.py}) pokazuje przykładowe użycie
niektórych funkcji biblioteki:

\starttyping
    01  import tracemalloc
    02
    03  tracemalloc.start()
    04
    05  a = 'A'*1000000
    06  b = 'ź'*1000000
    07  c = list(range(1000000))
    08  d = set(range(1000000))
    09  e = {_: None for _ in range(1000000)}
    10
    11  snap = tracemalloc.take_snapshot()
    12  top = snap.statistics('lineno')
    13  for stat in top[:5]:
    14      print(stat)
    15
    16  f = list(range(1000000))
    17  g = c + f
    18
    19  snap2 = tracemalloc.take_snapshot()
    20  top2 = snap2.compare_to(snap, 'lineno')
    21  print()
    22  for stat in top2[:2]:
    23      print(stat)
\stoptyping

A tak wygląda efekt uruchomienia powyższego pliku:

\starttyping
    $ python3.4 t.py
    t.py:9: size=74.7 MiB, count=999744, average=78 B
    t.py:8: size=58.7 MiB, count=999745, average=62 B
    t.py:7: size=35.3 MiB, count=999746, average=37 B
    t.py:6: size=1953 KiB, count=1, average=1953 KiB
    t.py:5: size=977 KiB, count=1, average=977 KiB

    t.py:16: size=35.3 MiB (+35.3 MiB), count=999745 (+999745), average=37 B
    t.py:17: size=15.3 MiB (+15.3 MiB), count=1 (+1), average=15.3 MiB
\stoptyping

Wszystko jest widoczne jak na dłoni, litera \type{ź} zajmuje więcej
miejsca niż \type{A}, jednak to wszystko nic w porównaniu z rozmiarem
słownika. Tego rodzaju dane mogą być nieocenione przy debugowaniu oraz
podczas optymalizacji kodu.

Na koniec muszę ostrzec, że \type{tracemalloc} dość mocno spowalnia
wykonywanie programu, więc po pierwsze -- żeby otrzymać wyniki czasem
trzeba uzbroić się w cierpliwość, a po drugie -- nie należy stosować
\type{tracemalloc} w środowisku produkcyjnym.

\subsection[inne-co-ciekawsze-zmiany]{Inne, co ciekawsze zmiany}

\section[tryb-izolowany]{Tryb izolowany}

Pythona można teraz uruchomić z parametrem \type{-I}, który odcina
interpreter od \type{site-packages}, jak i od wszelkiego dostępu do
zewnętrznych bibliotek. Użycie tego trybu jest zalecane przy
zastosowaniu Pythona w skryptach systemowych.

\section[nowy-format-pickle-i-marshal]{Nowy format \type{pickle} i
\type{marshal}}

Python 3.4 wprowadza nowe formaty dla \type{pickle} i \type{marshal}
(odpowiednio 4 i 3).

W przypadku \type{pickle} rozwiązanych zostało wiele problemów
występujących w poprzednich wersjach protokołu, a także poprawiona
została wydajność. Tradycyjnie już, w celu zapewnienia wstecznej
kompatybilności, nowy format \type{pickle} nie jest domyślnym, tak więc
żeby go użyć, należy jawnie określić wersję protokołu, najlepiej (też
tradycyjnie już) używając stałej \type{pickle.HIGHEST_PROTOCOL}.

Jeśli chodzi o \type{marshal}, to dzięki uniknięciu powielania
niektórych obiektów, zmniejszył się rozmiar plików \type{.pyc} (i
\type{.pyo}), a co za tym idzie, spadła ilość pamięci zajmowanej przez
moduły wczytane z tychże plików.

\section[single-dispatch-w-functools]{{\em Single-dispatch} w
\type{functools}}

W module \type{functools} pojawił się~niepozorny dekorator
\type{singledispatch()}, który pozwala na zdefiniowanie funkcji
generycznej, posiadającej różną implementację w zależności od typu
argumentu.

Bez wdawania się w dywagacje, poniższy przykład powinien rzucić nieco
światła na to, ile dobrego kryje się za tą mętną definicją:

\starttyping
    >>> from collections.abc import Sequence
    >>> from functools import singledispatch
    >>> @singledispatch
    ... def fun(arg):
    ...     print("Łapię całą resztę, tym razem był to", type(arg))
    ... 
    >>> @fun.register(int)
    ... @fun.register(float)
    ... def _(arg):
    ...     print("Mamy numerek!")
    ... 
    >>> @fun.register(Sequence)
    ... def _(arg):
    ...     print("Sekwencja o długości ", len(arg))
    ... 
    >>> @fun.register(tuple)
    ... def _(arg):
    ...     print("Nie lubię tupli!")
    ... 
    >>> fun(11)
    Mamy numerek!
    >>> fun("tekścik")
    Sekwencja o długości  7
    >>> fun([1, 3, 5])
    Sekwencja o długości  3
    >>> fun((3, 4))
    Nie lubię tupli!
    >>> fun(set())
    Łapię całą resztę, tym razem był to <class 'set'>
\stoptyping

\section[poprawa-bezpieczeństwa]{Poprawa bezpieczeństwa}

Nowy, bezpieczny, algorytm hashowania, obsługa TLS v1.1 i v1.2,
możliwość pobierania certyfikatów z Windows system cert store, obsługa
serwerowego SNI (Server Name Indication), bezpieczniejsze deskryptory
plików. To tylko niektóre z licznych zmian, wpływających korzystnie na
bezpieczeństwo nowego Pythona.

\subsection[podsumowanie]{Podsumowanie}

Python 3.4 zdaje się być bardzo dobrym wydaniem, pomimo braku (a może
dzięki brakowi?) zmian w składni. Wprowadza on szereg usprawnień oraz
oddaje w ręce użytkownika kilka bardzo dobrych bibliotek wbudowanych,
które z pewnością znajdą sobie miejsce w wielu plikach z rozszerzeniem
\type{.py}.

Zarówno nowi użytkownicy, jak i starzy wyjadacze znajdą coś dla siebie w
tym wydaniu, a dzięki praktycznie pełnej kompatybilności wstecz, nowy
Python nie powinien też nastręczać problemów przy aktualizacji*.

* Oczywiście chodzi o aktualizację z Pythona 3.3 ;)

\subsection[źródła]{Źródła}

\startitemize
\item
  https://docs.python.org/3/whatsnew/3.4.html - oficjalny dokument
  \quotation{what's new} dla Pythona 3.4.
\item
  https://mail.python.org/pipermail/python-ideas/ - lista mailingowa
  wyznaczająca nowe kierunki rozwoju Pythona.
\item
  http://hg.python.org/cpython/ - repozytorium zawierające źródła (oraz
  nader ciekawe commit-logi) Pythona.
\item
  http://legacy.python.org/dev/peps/ - propozycje usprawnień Pythona.
\stopitemize


\stoptext
