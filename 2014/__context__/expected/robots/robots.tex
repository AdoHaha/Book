\usemodule[pycon-2014]
\starttext


\section[how-to-make-killer-robots-with-python-and-raspberry-pi---radomir-dopieralski]{How
to make killer robots with Python and Raspberry Pi - Radomir
Dopieralski}

Robots in popular culture differ substantially from what is available
today. You may see yourself in a role of a mad scientist constructing an
army of robots to take over the world, but when you go to any website
about building robots, you are presented with those\ldots{} toys. They
can't fly, don't have lasers and death rays, they ride around on wheels,
and their main function is to avoid obstacles. Imagine that! You unleash
your robot on the city to wreck havoc and destruction, and what it does
when it faces a building or a police car? Meekly turns around and rides
on its silly plastic wheels? No! The robots I build will trample the
policemen and walk straight into the building, with no regard to their
own safety. That's how it's done!

And no wheels. Wheels are easy -- you basically only need an H-bridge
and two electric motors with gear boxes. And a small battery. Connect
the bridge to two GPIO pins, and you have full control. Pathetic.

Now, legs. Legs are interesting. First of all, your robot will need to
be strong enough to carry its own weight, plus some extra strength for
inertia. That means servomotors with enough torque, and a battery with
low enough internal resistance to power them. If you want anything
better than the silly chopstick walkers, or the ridiculous bipedals with
huge feet, you will also need a lot of those servomotors. At least three
per leg, to have full inverse kinematics. Multiply that with the number
of legs, from four to eight, and you you get from 12 to 24 servomotors.
Expensive and power-hungry, but that's the price of awesomeness! You can
improve the situation a little by initializing the servomotors in a
sequence, not all at once, and by adding a large capacitor next to the
battery -- that will smoothen out any spikes in power consumption. Of
course you won't connect all those servomotors to your central unit
directly, it doesn't even have so many pins. You will need a servo
controller, and you will need to send the commands to it through the
serial interface or I²C, depending on what you get. You can even make
your own out of an Arduino, especially if you need some custom
functionality.

Now, suppose you have all the parts, and you start assembling it all
together. It's easy to just glue everything together and solder all the
wires. But then you realize, that you attached one of the legs up-side
down, one of the servomotors is fried and you need a larger capacitor.
So you have to break it all apart again. What a pain! Better to use gold
pins, plugs, screws, and even velcro tape. This way you can easily
modify your prototype and replace parts. Oh, and by the way, have some
spares on hand. They {\em will} break. That's how the physical world
works.

As for software, it's in equal parts programming and system
administration. Remember that this is practically a mobile data center
-- multiple electronic devices communicating with each other. For
instance, to use the serial interface you will need to edit
\type{/boot/cmdline.txt} and \type{/etc/inittab} to disable the system
console that normally runs there. And to use I²C, you will need to
comment out the blacklisted modules from
\type{/etc/modprobe.d/raspi-blacklist.conf}, and so on. Even if your
program doesn't need any fancy settings or permissions, you will still
need to make it run at system startup, or have some other way to start
it.

I won't write much about the actual code here. After all, you are a
programmer and it's what you want to solve by yourself. Just a few
hints. Look for \quotation{inverse kinematics} -- that's what you will
need to translate the coordinates of the leg tips into actual servomotor
angles. And the position of the robot's body into the positions of
individual legs. Then, remember that animating a physical thing is very
similar to animating computer graphics -- just instead of writing pixels
to the screen, you are sending commands to the servos. You are on your
own with the rest.

Controlling your robot can also be a challenge. You might try to go
fully autonomous, give it a lot of sensors, connected over ISP, I²C,
1Wire, TTL, USB or even directly to the pins as analog input. There is a
wide range of sensors available, from proximity (both sonar and
light-based) and motion sensors, through light, color, temperature,
ambient noise, touch, acceleration, direction, GPS, to moisture. You can
even have a camera and use OpenCV for image analysis! However, it's nice
to have some kind of manual override. You can use a WiFi dongle and
simply SSH to it. Or get one of those wireless game pads. Just remember
that the Sony Sixaxis controller is not supported very well, especially
if it's not the original one. You could even have your computer talk to
your robot through a Bluetooth or IRDA or one of those
transmitter-receiver sets for Arduino. Whatever you choose, make sure
it's always possible to just connect a good old monitor screen and
keyboard to it, in case something goes very wrong and you lose access.

Finally, there is the equipment that your robot will carry. Of course,
crawling around and trampling enemies is cool in itself, but you can
make it even better. A speaker for making beeps or playing the
\type{DESTROY.wav} file over and over (remember to get a small audio
amplifier too, so that it can be heard). Blinking LEDs. A nerf rocket
turret. Maybe even an Airsoft gun, so that you can take part in a Mech
Warfare contest?

Last, but not least, remember about your own safety. Conquering the
world is so much harder when you have no hands and you are bleeding from
your eyes. Don't look into lasers, don't use power tools without
glasses, don't play with high voltages, always be careful which end of
the soldering iron you are grabbing. And remember that the LiPo
batteries will erupt in flames if you so much as look funny at them.
Seriously, search YouTube for videos of burning LiPos. You don't want
that to happen in your secret lair. And never shoot anything at humans
-- you don't want them to become suspicious before your plans are fully
prepared!

Now go forth and create robots! It's great fun. It's also addictive,
expensive and takes all your free time -- a perfect hobby.


\stoptext
