\usemodule[pycon-2014]
\starttext

\Title{Journey to the center of the asynchronous world}
\Author{Maciej Szulik}
\MakeTitlePage

\subsection[introduction]{Introduction}

Since the release of Python 3.4, one of the hottest and most frequently
mentioned topics is the
\useURL[url1][https://docs.python.org/3/library/asyncio.html][][asyncio]\from[url1]
module, introduced in
\useURL[url2][http://legacy.python.org/dev/peps/pep-3156/][][PEP
3156]\from[url2]. In the following short article I'll try to show you
some cool stuff that lies at the core of this module. But before we dig
in, there is one warning: most of the samples presented in this article
are written in Python 3.4, so make sure to use at least that version
when running the examples. These can be found at
\useURL[url3][https://github.com/soltysh/talks/tree/master/coroutines_generators/examples][][my
github account]\from[url3].

\subsection[generators]{Generators}

Let's start with this sample piece of code,
\useURL[url4][https://github.com/soltysh/talks/tree/master/coroutines_generators/examples/generator1.py][][generator1.py]\from[url4]:

\starttyping
def countdown(n):
    while n > 0:
        yield n
        n =- 1
\stoptyping

As you've probably noticed, this is one of the simplest generator
functions imaginable. Its typical usage is as following:

\starttyping
for x in countdown(10):
    print("Got ", x)
\stoptyping

This prints every number starting from 10 down to 1. So we can conclude
that every function written using the \type{yield} statement is a
generator, which can then be used to feed all kinds of loops and
iterations. If we look under the hood, this iteration calls
\type{next()} to get the next value from the generator, until it reaches
the \type{StopIteration} exception. We can illustrate that with the
following piece of code:

\starttyping
c = countdown(3)
print(c)
next(c)
next(c)
next(c)
next(c)
\stoptyping

The result of running this code is:

\starttyping
<generator object countdown at 0x7f2b0fa7a0d0>
3
2
1
Traceback (most recent call last):
  File "generator1.py", line 21, in <module>
    print(next(c))
StopIteration
Traceback (most recent call last):
  File "stdin", line 1, in ?
    print(next(c))
StopIteration
\stoptyping

This is just the beginning, to make sure that everyone is on the same
page. Expect more of the promised awesomeness to come! What is probably
a lesser known fact, is that the \type{yield} statement can be used to
receive values,
\useURL[url5][https://github.com/soltysh/talks/tree/master/coroutines_generators/examples/generator2.py][][generator2.py]\from[url5]:

\starttyping
def receiver():
    while True:
        item = yield
        print("Got: ", item)

c = receiver()
print(c)
next(c)
c.send(43)
c.send([1, 2, 3])
c.send("Hello")
\stoptyping

The output of the above code is as following:

\starttyping
<generator object receiver at 0x7f1690d88f58>
Got:  43
Got:  [1, 2, 3]
Got:  Hello
\stoptyping

This was introduced in
\useURL[url6][http://legacy.python.org/dev/peps/pep-0342/][][PEP
342]\from[url6], where the idea of coroutines appeared for the first
time. This PEP extends the functionality of the generators presented in
the first example with the possibility to send values to the generators.
Essentially any function having the \type{yield} statement in its body
is actually a generator. This means that it is not going to execute, but
rather it will return a generator object, which provides the following
operations: * \type{next()} - advance code to the \type{yield} statement
and emit a value, if such was passed as a parameter. That's the only
operation you can call after creating the generator. * \type{send()} -
sends a value to the \type{yield} statement, making it produce a value
instead of emitting one. Remember to call \type{next()} beforehand. *
\type{close()} - a way to inform the generator that it should finish its
work. It generates the \type{GeneratorExit} exception upon calling the
\type{yield} statement. * \type{throw()} - gives you the opportunity to
send an error to generator upon a call to the \type{yield} statement.

In Python 3.4 specifically you can have both the \type{yield} and the
\type{return} statement. In previous Python versions this would be a
syntax error. Currently if you write,
\useURL[url7][https://github.com/soltysh/talks/tree/master/coroutines_generators/examples/generator3.py][][generator3.py]\from[url7]:

\starttyping
def returnyield(x):
    yield x
    return "Hi there"

ry = returnyield(5)
print(ry)
print(next(ry))
print(next(ry))
\stoptyping

The output of the above code is as following:

\starttyping
<generator object returnyield at 0x7f27bf38bf58>
5
Traceback (most recent call last):
  File "generator3.py", line 15, in <module>
    print(next(ry))
StopIteration: Hi there
\stoptyping

If you carefully study the output, you'll notice that the value of the
\type{return} statement is actually passed as a value of the
\type{StopIteration} exception. Interesting, isn't it?

\subsection[delegating-to-a-subgenerator]{Delegating to a subgenerator}

\useURL[url8][http://legacy.python.org/dev/peps/pep-0380/][][PEP
380]\from[url8] proposed the syntax for a generator to delegate part, or
all, of its work to another generator. This basically means that instead
of manually iterating, we're passing the generation to somebody else,
who will do it for us. This is presented in
\useURL[url9][https://github.com/soltysh/talks/tree/master/coroutines_generators/examples/yieldfrom.py][][yieldfrom.py]\from[url9]:

\starttyping
def yieldfrom(x, y):
    yield from x
    yield from y

x = [1, 2, 3]
y = [4, 5, 6]
for i in yieldfrom(x, y):
     print(i, end=' ')
\stoptyping

The expected output is a series of numbers from 1 to 6. What is
happening here is that both of these \type{yield from} statements take
values from both lists, consume them and join them as if they were one
list. In its simplest form, these can be seen as hidden \type{for}
loops, but soon you'll see there's more to it. Beyond this, there is the
concept of generator chaining, meaning iteration can be delegated even
further. Let's create something more complicated:

\starttyping
for i in yieldfrom(yieldfrom(a, b), yieldfrom(b, a)):
    print(i, ' ')
\stoptyping

In the above example, the outermost call will delegate iteration to the
inner generators until we reach a single value that will be yielded.

\subsection[context-managers]{Context managers}

Let's leave the generator and coroutines topic for a bit and look at
something different. I'm hoping the reader is familiar with these
constructs:

\starttyping
file = open()
# do some stuff with file
file.close()

lock.acquire()
# do some stuff with lock
lock.release()
\stoptyping

These constructs are currently nicely handled by context managers,
introduced in
\useURL[url10][http://legacy.python.org/dev/peps/pep-0343/][][PEP
343]\from[url10] - the \type{with} statement. Context managers are
basically normal objects implementing two methods: *
\type{__enter__(self)} - start work with your object, returning it *
\type{__exit__(self, exc, val, tb)} - release the object, or handle the
exception

Let's create a simple context manager for working with a temporary
directory,
\useURL[url11][https://github.com/soltysh/talks/tree/master/coroutines_generators/examples/contextmanager1.py][][contextmanager1.py]\from[url11]:

\starttyping
class tempdir(object):
    def __enter__(self):
        self.dirname = tempfile.mkdtemp()
        return self.dirname
    def __exit__(self, exc, val, tb):
        shutil.rmtree(self.dirname)

with tempdir() as dirname:
    print(dirname, os.path.isdir(dirname))
\stoptyping

This sample context manager will create a temporary directory, whose
name will be printed and then checked for it's existence. Thanks to the
awesome Python core developers, the \type{yield} statement and the
\type{@contextmanager} decorator, the above code can be rewritten as
follows,
\useURL[url12][https://github.com/soltysh/talks/tree/master/coroutines_generators/examples/contextmanager2.py][][contextmanager2.py]\from[url12]:

\starttyping
@contextmanager
def tempdir():
    dirname = tempfile.mkdtemp()
    try:
        yield dirname # the magic happens here
    finally:
        shutil.rmtree(dirname)
\stoptyping

You will use this piece of code in exactly the same way as in the
previous context manager. The only difference is how you define your
context manager. In the latter example, the decorator is creating the
context manager for you, and \type{yield} returns the temporary
directory. If you look under the covers you'll see that calling
\type{tempdir()} in the first example will return
\type{<__main__.tempdir object at 0x7f3e4778f5a0>} whereas the later -
\type{<contextlib._GeneratorContextManager object at 0x7fd94c7ce538>}.
Do you see the difference? If you look closely at the
\type{@contextmanager} decorator you'll find out that it sets up the
\type{__enter__()} and \type{__exit__()} methods, with some additional
error checking, see:
\useURL[url13][http://hg.python.org/cpython/file/3.4/Lib/contextlib.py\#l96][][contextlib.py\#96]\from[url13].
For those of you concerned about performance, my test shows the
decorator solution runs \lettertilde{}9\% slower than its class
counterpart, but think of how much easier to read the decorator solution
is.

\subsection[asynchronous-processing]{Asynchronous processing}

Finally we've reached the last part - asynchronous processing. The usual
way of processing in such cases is: we have some main thread; in it we
run some asynchronous function, and after some time we reach for the
results. This is a very common programming pattern, which can be
presented with the following code,
\useURL[url14][https://github.com/soltysh/talks/tree/master/coroutines_generators/examples/future1.py][][future1.py]\from[url14]:

\starttyping
def executor(x, y):
    time.sleep(10)
    return x + y

pool = ThreadPoolExecutor(8)
fut = pool.submit(executor, 2, 3)
fut.result()
\stoptyping

The above code runs in a different thread, but here we're blocked; we
wait until we get the result. The next example shows how to use callback
functions that will return when the result is ready, whereas in the
meantime we still have control over the main thread,
\useURL[url15][https://github.com/soltysh/talks/tree/master/coroutines_generators/examples/future2.py][][future2.py]\from[url15]:

\starttyping
def handle_result(result):
    """Handling result from previous function"""
    print("Got: ", result.result())

pool = ThreadPoolExecutor(8)
fut = pool.submit(executor, 2, 3)
fut.add_done_callback(handle_result)
\stoptyping

A quick note: if an exception happens inside the executor method, it
will be returned when getting the result. Testing this will be left as
an exercise to the reader.

\subsection[asyncio-basics]{\type{asyncio} basics}

OK, we've reached a point where I've shown you a couple of cool tricks
with generators, but you may be asking \quotation{How is this useful?
What can we do with it?} Let's then move to the final part where I'll
show you how, using previous parts we can bypass certain Python
limitations and create \type{asyncio} core functionality.

Let's start with creating a task object, which is similar to what I've
shown you before, but this time, we'll put the idea into a reusable
object,
\useURL[url16][https://github.com/soltysh/talks/tree/master/coroutines_generators/examples/task1.py][][task1.py]\from[url16]:

\starttyping
class Task:
    def __init__(self, gen):
        self._gen = gen

    def step(self, value=None):
        try:
            fut = self._gen.send(value)
            fut.add_done_callback(self._wakeup)
        except StopIteration as exc:
            pass

    def _wakeup(self, fut):
        result = fut.result()
        self.step(result)
\stoptyping

If you look at the code you'll notice it is almost identical to the
previous one, but placed inside of some sort of a context manager class.
The only difference being method names, \type{step()} in place of
\type{__enter__()} and \type{_wakeup()} for \type{__exit__()}. What we
have here, actually, is a task object accepting a generator as the only
initialization parameter, with the main function \type{step()}
responsible for advancing the generator to the next \type{yield}
statement, sending in a value and a callback to do something with the
result. There's also one little trick at the end of the attached
callback method called \type{_wakeup()} where we feed ourselves with the
result to proceed execution to the next \type{yield} statement.

So let's create a recursive function, to show that using the above
tricks we can bypass Python's recursion limit which by default is
\useURL[url17][http://hg.python.org/cpython/file/3.4/Python/ceval.c\#l687][][1000]\from[url17].

\starttyping
def recursive(pool, n):
    yield pool.submit(time.sleep, 0.001)
    print(n)
    Task(recursive(pool, n+1)).step()
\stoptyping

If you run it long enough, you'll notice that in using this little trick
Python doesn't have a stack limit any more. What's more, the execution
does not provide any overhead when run. You should definitely check it
if you don't believe me.

There's still one more modification to our \type{Task} object which I'd
like to show you. So far, this class can only process background tasks,
but how do you return something from that background task? Let's use the
\type{concurrent.futures.Future} object as a base class for our
\type{Task}. To perform this we need to do a little Python patching,
meaning we need to make the \type{Task} class iterable to be used inside
the \type{yield from} statement:

\starttyping
def patch_future(cls):
    def __iter__(self):
        if not self.done():
            yield self
        return self.result()
    cls.__iter__ = __iter__
\stoptyping

Then we can properly modify our \type{Task} object to return the result:

\starttyping
class Task(Future):                     # <--

    def __init__(self, gen):
        super().__init__()              # <--
        self._gen = gen

    def step(self, value=None):
        try:
            ...
        except StopIteration as exc:
            self.set_result(exc.value)  # <--
\stoptyping

Now we can use the \type{Task} object to do some intensive calculation
and retrieve the result,
\useURL[url18][https://github.com/soltysh/talks/tree/master/coroutines_generators/examples/task2.py][][task2.py]\from[url18].

\starttyping
def calc(x, y):
    print("I'm going to sleep for a while...")
    time.sleep(10)
    return x + y

def do_calc(pool, x, y):
    result = yield from pool.submit(calc, x, y)
    return result

if __name__ == '__main__':
    pool = ProcessPoolExecutor(8)
    t = Task(do_calc(pool, 2, 3))
    t.step()
    print("Got ", t.result())
\stoptyping

\subsection[summary]{Summary}

The attentive reader might say at this point, \quotation{We're at the
summary already, but you've promised to show what \type{asyncio}
internals look like!} But I just did that: the last example of the
\type{Task} object is almost exactly the same as the one in the
\type{asyncio} module. For reference see
\useURL[url19][http://hg.python.org/cpython/file/3.4/Lib/asyncio/tasks.py\#l25][][tasks.py\#25]\from[url19],
with more error handling, and most importantly, some additional useful
functions which hide implementation details from users to make it easier
to use. Of course, there's also the most important thing, which is the
event loop being the core runner instead of thread pools.

Hopefully, this article made you eager to get more from generators. I
highly recommend reading David Beazley's trilogy on this topic:

\startitemize[n][stopper=.]
\item
  \useURL[url20][http://www.dabeaz.com/generators/][][Generator Tricks
  for Systems Programmers]\from[url20]
\item
  \useURL[url21][http://www.dabeaz.com/coroutines/][][A Curious Course
  on Coroutines and Concurrency]\from[url21]
\item
  \useURL[url22][http://www.dabeaz.com/finalgenerator/][][Generators:
  The Final Frontier]\from[url22]
\stopitemize

I also would like to thank him for giving me the chance to use his
materials as an input for my article and presentation. Additionally,
check out his awesome, mind-blowing book
\useURL[url20][http://shop.oreilly.com/product/0636920027072.do][][Python
Cookbook]\from[url20].


\stoptext
