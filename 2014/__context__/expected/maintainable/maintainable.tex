\usemodule[pycon-2014]
\starttext


\section[writing-maintainable-code---peter-inglesby]{Writing
Maintainable Code - Peter Inglesby}

A few thoughts on maintainability and complexity

\quotation{Always write code as if the person who ends up maintaining it
is a violent psychopath who knows where you live\ldots{}} {[}0{]}

Maintainable code is easy to read, easy to understand, and easy to
extend. Very little of the code we write is never maintained, so making
it maintainable must be one of our goals. I don't think anybody would
disagree with this.

However, we've all experienced code that is a nightmare to maintain.
Week-long sprints to extend a little bit of functionality have become
months-long marathons once the true extent of maintainability problems
becomes clear.

Moreover, we've all been responsible for producing unmaintainable code.
(If you haven't, you can stop reading now!)

So, despite our best efforts, we fail at writing maintainable code. Why
is it so hard?

Code becomes unmaintainable when it is too complex to understand. You
might argue that having complexity in our code is natural because, as
programmers, we're solving complex problems. Ironically, I think this
view is too simple.

In his 1987 essay \quotation{No Silver Bullet} {[}1{]}, Fred Brooks
demonstrates that there are two kinds of complexity. He contrasts
\quotation{essential} {[}2{]} complexity with \quotation{accidental}
{[}3{]} complexity.

Essential complexity is inherent to a problem. It's unavoidable, and it
must be understood before we can attempt to properly solve it.

On the other hand, accidental complexity is what we bring to the problem
when we try to solve it. Accidental complexity comes in two kinds.

The first relates to our tools. Do our tools help us solve problems, or
do they get in the way? I think this is something we're getting better
at. For instance, with a high level language such as Python, we don't
have to think about things like manual memory management.

The second kind of accidental complexity is what we, as programmers,
create through our code. This is something that we still struggle with.

For instance, we often create accidental complexity when we spend too
much time writing code to solve problems that we might face in the
future, leading to unnecessary abstractions that are hard to understand.

Conversely, we create accidental complexity when we write new code
quickly and don't give enough thought to how this new code relates to
the rest of a codebase. This can lead to repetitious and poorly
structured code that is, again, hard to understand.

In order for our code to be maintainable, it must expose the essential
complexity of the problem that we're trying to solve, while avoiding
adding too much accidental complexity. For this to happen, the structure
of the code must correspond with the essential structure of the
underlying problem.

My workshop, \quotation{Writing Maintainable Code}, will be about
identifying the essential structures of the problems we're trying to
solve, and about writing code so that these essential structures are
clear and visible.

During the workshop, we'll practice refactoring code to become better
structured. Although the examples will all be on a small scale, the
ideas and the techniques we'll explore will be applicable at a larger
scale.

My hope is that workshop will help you write code that is easier to
read, to understand, and to extend, and that this will help protect you
from that violent psychopath who knows where you live!

{[}0{]} http://stackoverflow.com/questions/876089\crlf
{[}1{]}
http://www.cs.nott.ac.uk/\lettertilde{}cah/G51ISS/Documents/NoSilverBullet.html\crlf
{[}2{]} \quotation{Essence}, from a Latin word meaning \quotation{to
be}\crlf
{[}3{]} \quotation{Accidence}, from a Latin word meaning \quotation{to
happen alongside}


\stoptext
