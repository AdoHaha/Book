\usemodule[pycon-2014]
\starttext


\section[proper-code-reviews---tomasz-maćkowiak]{Proper code reviews -
Tomasz Maćkowiak}

Code quality is the ultimate goal that every programmer (and his
manager) strives for. High code quality means fewer bugs, better
performance and easier maintenance. One of the most popular ways of
ensuring quality of the code (apart from peer programming) is to
implement code reviews.

Not all code reviews are equal. Some developers are better at it, while
others add no value in their comments.This article presents my approach
to code reviews that has worked well for me and my team for about a year
already.

\subsection[what-are-code-reviews]{What are code reviews?}

This article is about informal, individual code reviews. We develop our
code using Git version control system. The model we use is
feature-branching, which means that there is a new Git branch created
off the master branch for each feature or bug fix. Developers work on
this branch and, when done, they release a pull request. The pull
request is then reviewed by at least 2 other developers. Either the
original developer needs to address the review comments and release
fixes for issues found during code review, or the pull request can be
merged straight away.

\subsection[why-do-you-do-code-reviews]{Why do you do code reviews?}

The most obvious answer to the above question is to keep high code
quality. Two additional developers scrutinising all the code which gets
into the repository are supposed to find mistakes, ineffective bits,
convention violations and other issues that might be improved. The value
of code review seems to be indisputable, but there are different ways of
approaching reviews, some more beneficial than others.

Code quality is not the only benefit of code reviews. What are the
others?

\section[standards]{Standards}

Thanks to code reviews, the team is bound to develop coding standards
and adhere to them. Be it standards like PEP8, the usage of \type{%}
operator against \type{format} function, to certain ways of using
external libraries; over time the team reaches a consensus and everybody
needs to act on its terms, unless they want to have their pull request
rejected. At first there might be a few ways of doing something, but in
the process of performing regular code reviews alternatives are being
flagged and one winning solution emerges. In our team history we
organised meetings following some particularly heated code review
discussions, where we had to pick one standard over the other.

\section[knowledge-sharing]{Knowledge sharing}

If code reviews are performed properly (the whole team is obliged to
carry them out and is notified about all pull requests and comments),
then code reviews also serve the purpose of sharing knowledge among team
members. That is how the above mentioned standards are being announced -
team members learn from pull review's comments that a certain solution
is preferable to the other. Also, the team learns other good practices.
We are supposed to learn from our own mistakes, but learning from the
mistakes of our colleagues is even better. The team starts to single out
undesirable patterns if they are being flagged in code reviews. Provided
that code review is done properly and the comments hold value (for
example: {\em for performance use \type{set} here}), it can constitute
an invaluable lesson. Another important function of code reviews is that
they enable team members to track what is going on in the project. The
team sees the code being committed to the repository and they know what
functionalities are being introduced and how they are implemented.

\section[ownership]{Ownership}

The code is not supposed to be the private property of the developer who
first wrote it. When practising code reviews, the reviewers share the
responsibility for the code. If the code works poorly or does not work
at all, the reviewers are guilty of letting its defects slip through
their fingers. Another benefit is that there is always an opportunity
for somebody else to fix the mistake in the code, because they are at
least vaguely familiar with the code, since they browsed through it
during code review.

\section[personal-development]{Personal development}

There is no faster way of improving your skill than to have somebody
systematically point out your mistakes and (maybe) show you how to fix
them. You improve your code every time you receive a comment in pull
request. You gain knowledge and learn good practices from other people's
pull requests. You post your own comments to other people's code and you
are able to see if your opinions are valid.

\subsection[how-to-do-code-reviews]{How to do code reviews?}

Code reviews only hold value if they are done properly.

\section[tools]{Tools}

One of the key factors in effective code reviews are tools. If
submitting your code for pull request or reviewing the code takes too
much effort, people will simply refuse to do it. The process needs to be
automatised as much as possible. Email notifications about pull requests
and comments help a lot, because even if you do not actively follow open
pull requests, you are still notified about the progress. My
recommendation is having a private Github. It supports email
notifications, integrates into your source code repository, has a great
interface and makes it really effortless to create a pull request as
well as to review one.

\section[the-actual-review]{The actual review}

Depending on the size of the pull request, code review can take anywhere
from 5 minutes to half a day in extreme cases. Since concentrating on
the code under review requires a clear head, you need to allocate an
appropriate amount of time for the review. It is preferable to do review
in-between tasks,at the beginning of the day (as a warm-up), or at the
end of the day when you are too tired to keep on developing your own
task anymore.

The code you are reviewing is most probably connected with a story or a
bug in your issue tracking system. It is a good idea to first read about
the issue so that you know what is to be implemented with the code.

In my opinion, you should never run the code you are reviewing. Running
the code might give you a false impression that the code is acceptable
and that it works as intended, while under the hood it is a mess.

\subsection[good-review-and-bad-review]{Good review and bad review}

What should you point out during code review to make your colleagues
regard it as valuable input and not nit-picking?

Code review is done by qualified humans for a reason. You should not
treat code review as another round of PEP8 checks - these should be done
by the code's author before he creates a pull request. What if the
author did not run his changes through PEP8 check? You still should not
bother since doing something that a machine can do in under 1 second is
never worth your time and effort. It is helpful to have PEP8 as part of
your Continuous Integration suite as it prevents any potential PEP8
violation from slipping through to the master branch. Nobody appreciates
comments about different possibilities of breaking newline in their
code.

There are a few other types of comments that are annoying to developers,
whereas, contrary to popular belief, they do bring in value. People very
much dislike comments (for Python2) about changing strings to unicodes,
for example in labels, especially if the string does not contain any
non-ascii characters. Their argument is that {\em it doesn't matter
anyway}, but in reality this error shows that the programmer does not
understand the difference between byte-strings and text-strings. Even
though such comments might seem annoying and repetitive, they do improve
the code and foster the understanding of the language's fundamental
structures.

A different kind of apparent nuisance is when the reviewer points out
the inconsistencies of your code with the project's standards. For
example, the usage of \type{format} might be preferred over the \type{%}
operator. The usage of the latter is not an error per se but it goes
against project policy. This kind of comment might be annoying, but in
the long run abiding to the policy improves the coherence of the code
and makes it easy to get into it by junior developers.

Another annoyance might be calling for docstring of functions or
classes. Docstrings are almost always a good idea, unless you have a
simple getter function with a sensible name; then it might not be
needed. Asking for docstrings to all functions in review might be
alleviated by using automatic tools like pylint, which can do that for
you.

During code review one might encounter pieces of code that are complex
and non-trivial to understand at first glance, or require significant
effort to analyze and understand. It is worth to ask the original
developer to insert a comment describing the particularly hard bit so
that programmers down the line will not have to spend half an hour
analyzing the code just to learn what it was supposed to do.

When developers write comments and docstring, they usually need to use
English which is a foreign language to most. The level of familiarity
with the language is often visible in the comments. Incorrect grammar,
misspellings etc. can make the comment hard to understand. It is a good
idea to point out such issues during code review so that all your
documentation is grammatically correct and understandable.

The most valuable comments during code review, though, are about the
more high-level issues. Comments about architecture should be greatly
appreciated. Pointing out that some code can be extracted to a helper or
that some piece of code does not belong here but somewhere else, can
greatly improve the maintainability of the code.

Quite often, experienced developers can point out that code under review
does not need to be written at all, because there is already a built-in
or a helper somewhere in the project that performs the same function.
Such comments are usually only to be expected from highly experienced
developers, but they carry a great learning potential for the more
junior ones.

A similar case can be made with applying correct data structures and
using efficient algorithms. An experienced developer can spot fragments
of the reviewed code where the usage of a dictionary data structure can
greatly improve its performance. Such a remark requires a higher order
understanding of the purpose of the code - it goes beyond the meaning of
a single line of the code, but pertains to the the overall result that
the developer is trying to achieve.

It is fairly easy to review a code change where somebody has just added
a few lines to a function. It is, however, much harder to analyze code
that introduces conditional logic. This is usually the point at which
the reviewer needs to stop and look deeply into what is happening.
Analyzing the conditions and their effects can provide one of the most
valuable feedbacks possible. The reviewer needs to ask himself the
following questions: {\em what does this condition really mean?},
{\em when is this condition not met?}, {\em what happens when the
condition is not met?}, {\em is this condition really needed?}. Some
programmers try to solve problems by introducing a number of \type{if}
statements with complicated conditions. In many instances such pieces of
code can be refactored to a much simpler form, decreasing the code
complexity significantly and making it much more readable.

An experienced reviewer is often able to provide (just by looking at the
code) instances of corner cases in which the code would fail or predict
a situation that the original developer did not anticipate. Spotting
such cases directly prevents live system bugs and is therefore of
immense value. It often requires applying a very deep and meticulous
review process, but the time spent on it is not wasted. You should not,
therefore, cut down the amount of time spent on a pull request, but
rather review it until you understand it fully.

\subsection[summary]{Summary}

Code review is one of the most valuable tools for assuring software
quality. When done by experienced developers who devote enough time to
it, it provides an opportunity to spot potential bugs, performance
bottlenecks, or hard-to-maintain bits. Apart from the reviewers doing
their job well, the developers under review also need to learn to
embrace the critical comments and understand that code review is not a
nuisance, but something that is done for their benefit.


\stoptext
