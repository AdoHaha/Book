\usemodule[pycon-2014]
\starttext


\Title{Projekty sprzętowe -- bo nie tylko webem człowiek żyje}
\Author{Piotr Maliński}
\MakeTitlePage

Rozwój urządzeń mobilnych i postępująca cyfryzacja naszego życia, czy
popularyzacja bezprzewodowego dostępu dostępu do Internetu i
bezprzewodowej komunikacji otaczających nas urządzeń, tworzą nowe rynki,
na których twórcy i ich startupy odnoszą sukcesy oferując nowe
urządzenia i usługi. Projekty sprzętowe otwierają przed nami nowy
segment rynku oraz stanowią przyjemną odskocznię od np. webowej rutyny.

Żeby skryptować elektronikę i tworzyć nowe urządzenia, czy sprzętowe
usługi, nie trzeba już znać się na programowaniu mikrokontrolerów. Nie
trzeba wkuwać wielkich ksiąg poświęconych poszczególnym układom. Rozwój
narzędzi deweloperskich pozwala nam już skryptować elektronikę z poziomu
Pythona, czy nawet JavaScriptu. Czego więc potrzebujemy, by zacząć
przygodę ze sprzętowymi projektami?

\subsection[pierwsze-kroki-i-prototypowanie]{Pierwsze kroki i
prototypowanie}

Osoby stawiające pierwsze kroki ze skryptowaniem elektroniki zapewne
natrafią na Raspberry Pi, czy Arduino. Są to obecnie chyba
najpopularniejsze marki w branży {\em sprzętowej}.

{\bf Raspberry Pi} to mini komputer, na którym po uruchomieniu Linuksa
możemy tworzyć np. aplikacje desktopowe zintegrowane z peryferiami USB,
czy też tymi podłączonymi za pomocą pinów GPIO. Dostępne na rynku mini
komputery (a jest ich wiele) pozwalają tworzyć nowe urządzenia i usługi
integrujące biblioteki i możliwości komputera z elektroniką taką jak
czujniki, wyświetlacze, czy różnego rodzaju silniki.

{\bf Arduino} to przedstawiciel mikrokontrolerów. Płytka Arduino możemy
przypominać Raspberry Pi, ale nie dostajemy na niej procesora, tylko
znacznie prostszy mikrokontroler. Za pomocą dostarczanego SDK i edytora
jesteśmy w stanie zaprogramować mikrokontroler na płytce. Zazwyczaj
oznacza to, że np. odpada nam obsługa peryferiów USB, czy też możliwość
wykorzystania bibliotek znanych z Linuksa. Z drugiej strony
mikrokontrolery są znacznie bardziej energooszczędne i tańsze (nawet od
Raspberry Pi). Świetnie sprawdzają się w projektach, w których musimy
zasilać układ z baterii przez długi okres czasu, lub kiedy ważną rolę
odgrywają koszty, a zastosowanie komputera nie jest potrzebne.

Na rynku dostępnych jest wiele mikrokontrolerów i mini komputerów. Nie
wszystkie oferują biblioteki i API dla Pythona. Spośród mini komputerów
przyjazne dla Pythona jest np. Raspberry Pi, Beaglebone Black, czy
PcDuino. W przypadku mikrokontrolerów nadal króluje C (Arduino), czy C++
(mbed), ale Python też zaczyna coraz częściej zaznaczać swoją obecność.
MicroPython (pyboard), czy PyMCU to płytki skryptowane z poziomu
Pythona. PyBoard posiada własną (ograniczoną) implementację Pythona na
mikrokontrolery, natomiast PyMCU jest skryptowane z poziomu Pythona na
komputerze, do którego jest podłączona. Także Arduino w pewnym zakresie
można skryptować Pythonem (za pomocą PyFirmata i nakładek). Dodatkowo
wymienić można projekty takie jak TinkerForge, Phidgets, czy Synapse
Snap.

Jeżeli {\bf chcesz zacząć} przygodę z elektroniką, to polecam wybranie
jakiegoś zestawu startowego. Nauka to nic innego jak budowanie kolejnych
prostych układów i skryptowanie ich. Czy to będzie miganie diodą LED,
poruszanie serwomechanizmami, silnikami krokowymi, albo DC, czy też
wykorzystanie wyświetlaczy LCD, komunikacja szeregowa z innymi złożonymi
układami. Jak opanujesz API danej platformy, będziesz w stanie łączyć
działanie poszczególnych elementów ze sobą, np. będziesz mógł zbudować
samojezdnego robota, który będzie unikał przeszkód dzięki czujnikom
zbliżeniowym.

Co do {\bf startowej platformy} to możliwości jest co najmniej kilka.
Dla PyMCU opublikowałem już szereg artykułów. PyBoard zapewne też
niebawem stanie się popularny, bo m.in. może działać bez podłączenia do
PC. Arduino z poziomu Pythona nie oferuje wszystkich możliwości, ale
proste układy da się w nim obsłużyć (biblioteka BreakfastSerial). Pośród
mini PC mamy Raspberry Pi, ale także Beaglebone Black, czy PcDuino -
choć do tych dwóch ostatnich liczba tutoriali i przykładów będzie
mniejsza niż w przypadku Raspberry Pi. Ich zalety to np. większa liczba
pinów GPIO, czy też silniejszy procesor.

\subsection[rozwiązania-praktyczne]{Rozwiązania praktyczne}

W pewnym momencie będziesz chciał przejść z poziomu prototypu na płytce
stykowej do bardziej praktycznej wersji swojego układu. W zależności od
tego, czy to ma być jeden układ, czy też produkcja masowa dla szerszej
grupy odbiorców, czeka Cię szereg kroków, jakie należy wykonać, by
osiągnąć wyznaczony cel. Nikt nie kupi Twojego urządzenia jeżeli będzie
drogie, czy nieporęczne, bo np. siedzi w nim Raspberry Pi, płytka
stykowa i szereg przewodów.

\section[produkty-masowe---podbijamy-światowe-rynki]{Produkty
masowe - podbijamy światowe rynki}

Zacznijmy od produktów masowych, które mają dość wyraźne cechy i
wymagania. Załóżmy, że masz gotowy układ kamery z monochromatyczną
matrycą i filtrami do fotografii w ultrafiolecie i podczerwieni, czyli
coś, czego zwykły aparat fotograficzny nie potrafi. Zrobiłeś wstępną
analizę i wynika z niej, że taki produkt znajdzie wystarczającą liczbę
nabywców. Załóżmy, że działa to tak: mini PC, jak np. Raspberry Pi,
steruje małą kamerą przemysłową oraz poprzez GPIO prostym kołem
filtrowym zmieniającym filtry. Do tego zasilanie z akumulatora,
wyświetlacz i inne komponenty. {\bf Jak z płytki stykowej przejść do
produktu konsumenckiego?}

Proces przejścia od prototypu do gotowego produktu nie będzie ani
krótki, ani tani. Aparat musi być relatywnie mały i trwały. Oznacza to,
że elektronika powinna być zintegrowana na płytce PCB, tak by nie było
konieczności stosowania różnego rodzaju przewodów i innych nietrwałych
lub nieporęcznych konstrukcji. Tworzeniem projektów i wykonaniem płytek
PCB zajmują się wyspecjalizowane fabryki, w których automaty potrafią
budować tysiące, jak nie setki tysięcy, gotowych układów w krótkim
okresie czasu. Twoim pierwszym krokiem powinno być nawiązanie współpracy
z taką fabryką - w celu stworzenia i zoptymalizowania PCB dla układu, a
następnie by rozpocząć produkcję. Jako etap pośredni można skorzystać z
sieciowych usług, takich jak
\useURL[url1][http://123d.circuits.io/circuits/][][123d.circuits.io]\from[url1],
gdzie możesz narysować swój układ, a następnie wygenerować projekt PCB i
zamówić kilka sztuk, czy też poprosić społeczność o porady przy
podstawowej optymalizacji płytki.

Prędzej czy później dojdziesz do etapu składania zamówienia w fabryce na
PCB. Problem w tym, że takie fabryki przyjmują zamówienia na duże
ilości, by było to opłacalne. W zależności od skomplikowania układu mogą
to być setki, a nawet tysiące sztuk. 5000 sztuk po 50 zł każda to już
250 tysięcy złotych. Bez zewnętrznego wsparcia finansowego mało który
startup sprzętowy byłby w stanie wyłożyć takie pieniądze. Na szczęście
na ratunek przychodzi {\bf crowdfunding}, czyli zbiórki na
Kickstarterze, Indiegogo i podobnych serwisach. Dzięki takim zbiórkom
udało zrealizować się wiele projektów, które inaczej nie byłyby w stanie
sfinansować budowy. Dodatkowa zaleta to weryfikacja produktu w oczach
przyszłych konsumentów - jeżeli dobrze przeprowadzona zbiórka nie
przekłada się na prognozowany poziom zainteresowania to może jednak Twój
pomysł nie jest aż tak dobry?

Projekt Kano zebrał półtora miliona dolarów (zestaw edukacyjny złożony z
Raspberry Pi, pomarańczowej klawiatury i materiałów edukacyjnych). Udoo,
czyli połączenie Arduino z dość silnym mini PC zebrał 650 tysięcy, choć
jako minimum wyznaczył sobie jedynie 27 tysięcy. Z drugiej strony są też
projekty, które nie osiągnęły swoich kwot minimalnych. W chwili pisania
tego artykułu na Kickstarterze trwa zbiórka na projekt SnaPiCam - czyli
aparat cyfrowy na bazie Raspberry Pi i dedykowanego modułu kamery. Za 30
funtów (jakieś 150 zł) możemy dostać obudowę. Do tego musimy doliczyć
koszt Raspberry Pi, modułu kamery, wyświetlacza, akumulatora i innych
niezbędnych części. Czy nie byłoby prościej kupić gotową i lepszą
cyfrówkę w sklepie? Dla wielu konsumentów tak zapewne jest i nic nie
wskazuje, by ten projekt osiągnął dobry wynik (o ile dojdzie do
minimum). Jeżeli uważasz, jako programista i twórca, że twój produkt
jest dobry, to {\bf zainwestuj też w ludzi od marketingu i finansów}.
Bez tego nie sprzedasz efektywnie produktu w crowdfundingu, jak i możesz
mieć problemy z wyceną całego przedsięwzięcia. Dobre {\em opakowanie}
jest kluczowe.

Jeżeli tworzysz nowe urządzenie i chcesz {\bf wprowadzić je do obrotu},
to czeka Cię jeszcze spełnienie wymogów prawnych. Znaczek zgodności CE
na urządzeniach jakie znajdziesz w sklepach, nie bierze się sam z
siebie. Musisz znaleźć regulacje prawne dopuszczenia do obrotu urządzeń
z danej kategorii i upewnić się, że Twoje urządzenie spełnia wszystkie
wymagania. W przypadku kontroli urządzenia nie spełniające tych wymagań
mogą marnie skończyć. Zainteresowanym polecam np.
\useURL[url2][http://www.uke.gov.pl/uke-skontrolowal-rynek-tabletow-13185][][wyniki
kontroli]\from[url2] tabletów przeprowadzonej przez UKE. Wyobraź sobie
sytuację, w której musisz wymienić zasilacze dołączone do każdego
tabletu, a tych na magazynie masz np. tysiące. Musisz więc uwzględnić
{\bf obsługę prawną} twojego przedsięwzięcia.

Niektóre projekty sprzętowe stosują swego rodzaju optymalizację
{\bf wykorzystując istniejące urządzenia}. Najłatwiej o to, gdy tworzysz
nową usługę w oparciu o istniejące urządzenia, np. do sieci klubów
wprowadzasz karty lojalnościowe oparte o układ RFID. Potrzebujesz więc
kart RFID, jak i czytników i terminali. Można połączyć np. gotowy
czytnik RFID i np. tablet, by stworzyć działający terminal. Dzięki temu
nie musisz wprowadzać produktu do obrotu, a jedynie tworzysz
oprogramowanie, czy też optymalizujesz dobór sprzętu. W przypadku
działalności międzynarodowej możesz korzystać ze sprzętu dostępnego na
danym lokalnym rynku. Odpada logistyka oraz cło i podatki przy
przekraczaniu granicy UE. Tworząc dedykowane urządzenie można by zadbać
o dodatkowe funkcjonalności, czy lepszy interfejs, ale nie zawsze jest
to opłacalne, bo zyskać można niewiele, a wydać trzeba wtedy bardzo
dużo. Przykładem takiego produktu może być
\useURL[url3][http://socialwifi.com/pl/][][Social WiFi]\from[url3],
który wykorzystuje istniejące na rynku routery, by dostarczyć nowe
usługi, wcześniej nieobecne dla tego typu urządzeń.

Ręczna drukarka 3D {\bf 3D Simo} przeszła odwrotną drogę. Jest zupełnie
nowym produktem i musiała przejść całą drogę, jaką opisałem w
poprzednich paragrafach. Dopiero po kilku zbiórkach crowdfundingowych,
po wykonaniu szeregu prototypów, twórcy mogli zacząć produkcję
urządzenia i wysyłkę do osób, którzy zakupili je w zbiórkach - z
kilkumiesięcznym opóźnieniem. Tutaj głównym problemem nie była
elektronika, ale np. wykonanie form do bicia obudów, czy zamawianie
części od różnych dostawców (np. dysz, czy układów grzewczych). Znaczek
CE też pojawił się na obudowie drukarki. Odbiorca produktu tego nie
widzi, ale dla twórców były to długie miesiące ciężkiej i czasami
zapewne stresującej pracy. Z drugiej strony weszli na rynek z nowym
produktem, a to już jest coś.

\section[produkty-unikatowe---czyli-radosna-twórczość-na-małą-skalę]{Produkty
{\em unikatowe} - czyli radosna twórczość na małą skalę}

Ręcznie tworzone urządzenia, czy usługi na nich oparte zdarzają się
znacznie częściej niż projekty na masową skalę. Chcesz rozwiązać jakiś
problem, dostarczyć jakieś proste urządzenie dla jednego klienta? Wtedy
sprawa jest prostsza, bo odpadają wysokie koszty, problemy prawne czy
technologiczne. Wystarczy, że stworzysz urządzenie i/lub oprogramowanie,
upewnisz się, że wytrzyma trudy pracy i gotowe. Twórcza i niecodzienna
praca może być bardzo wciągająca.

Możesz zapytać, {\bf czy jest zapotrzebowanie na takie projekty?}. Otóż
jest, choć może niezbyt duże. Wspomniałem już o terminalu z RFID. Jedna
firma, z którą kiedyś współpracowałem, używała technologii RFID na
przyjęciach organizowanych z jednym z ich klientów. Niestety dość
rzadko, bo firma zapewniająca obsługę RFID wysoko się ceniła. Z drugiej
strony wymagania stawiane RFID na tych imprezach były bardzo proste - po
odczytaniu tagu sparować go z kontem Facebooka użytkownika, czyli dać
stronę logowania Facebooka. Tablet, czy mini PC z czytnikiem podłączonym
na USB (np. udającym klawiaturę), dałoby się łatwo oprogramować, by
zapewnić takie funkcjonalności niskim kosztem.

Kolejny przykład to panel reklamowy, który miał za zadanie wyświetlać
zdjęcia produktu po zeskanowaniu jego kodu kreskowego. Można to zrobić
stosując prosty czytnik kodów kreskowych (który udając klawiaturę jest
łatwy do oprogramowania), albo też tablet z kamerą i zewnętrznym ekranem
na HDMI (gdzie wbudowana kamera wykrywa i skanuje kod kreskowy).

\subsection[dobór-sprzętu]{Dobór sprzętu}

W {\em amatorskich} projektach na niską skalę można poświęcić nieco
czasu na dobór komponentów. Może to znacznie ułatwić stworzenie gotowego
urządzenia oraz zredukować koszty. Jeżeli jesteś w stanie zrobić coś bez
użycia mini PC, tylko za pomocą samego mikrokontrolera, to warto do tego
się przyłożyć. Z drugiej strony mini PC mogą mieć przewagę, jeżeli
możesz skorzystać z gotowych urządzeń USB i systemowych bibliotek do ich
obsługi. Krótszy czas poświęcony na napisanie oprogramowania też
zmniejsza koszty.

Praktyczność łączy się z pomysłowością, innowacyjnością. Nie jestem w
stanie podać na to wzoru, reguły. To już zależy od Ciebie i Twojego
projektu. Mogę jedynie podać Ci kilka przykładów, czy pomysłów, jakie
możesz wykorzystać, czy też użyć jako punkt wyjścia.

Raspberry Pi może być mini komputerem, o jakim pomyślisz w pierwszej
kolejności, gdy zajdzie potrzeba wykorzystania takowego. A co, jeśli
będziesz potrzebował czegoś mocniejszego, albo mniejszego? Są inne mini
komputery z silniejszymi procesorami, ale takie rozwiązania są droższe.
A gdyby zamiast mini komputera dla miłośników elektroniki użyć mini
komputera dostępnego powszechnie na rynku konsumenckim? Androidowe
dongle HDMI są takim urządzeniem. Te wyposażone w znacznie wydajniejszy
czterordzeniowy układ RK3188 z 2GB RAM nie dość, że są już tańsze od
Raspberry Pi (ceny w chińskim sklepie dx.com), to oferują nam Androida,
czy też zwykłego Linuksa (Picuntu, Linuxium), choć z nieco ograniczoną
obsługą (brak obsługi VPU, czyli Video Processing Unit). W małej
obudowie kryje się moduł WiFi, a na USB możemy podłączyć np. potrzebny
nam czytnik. Wyjście HDMI możemy podłączyć od razu do monitora. Nieco
droższe i większe dongle zaoferują nam też złącze Ethernetowe.

Androidowe dongle, czy też tablety, albo taniejące nettopy z procesorami
Intela, czy AMD, mogą dostarczyć nam potrzebnej wydajności, ale nie
zaoferują nam pinów GPIO. Wybranie odpowiednio silnego mini PC z pinami
GPIO może okazać się problematycznym wyborem - np. mini PC Radxa Rock da
nam ten sam układ RK3188 co dongle, ale przynajmniej na chwilę obecną
obsługa GPIO tego mini komputera zostawia wiele do życzenia pod względem
dokumentacji, bibliotek, czy przykładowego kodu. Co wtedy? Na ratunek
mogą przyjść nam płytki z mikrokontrolerami. Możemy podłączyć je poprzez
USB, czy w przypadku tabletów (lub smartphonów) bez odpowiedniej obsługi
USB host i z brakiem sterowników do układów UART możemy skorzystać z
komunikacji poprzez Bluetooth (gdzie mikrokontroler wykorzystać będzie
układ do komunikacji szeregowej poprzez Bluetooth).

Jeżeli jesteś w stanie przeboleć użycie C, to otwiera się przed tobą
świat Arduino - od dużych płytek do prototypowania aż po małe płytki do
urządzeń miniaturowych. Jeżeli potrzebujesz obsługi Ethernetu, WiFi lub
Bluetooth, to też coś się znajdzie. Arduino oferuje SDK dla wielu
różnych płytek, podczas gdy Raspberry Pi trzyma się jednego urządzenia.

Rodzina mikrokontrolerów na platformie mbed wykorzystuje C++, choć nie
jest to kluczową cechą. Największa zaleta to programowanie przez
przeglądarkę za pomocą webowego IDE i zdalnego kompilatora, który
przygotuje nam plik, jaki należy wrzucić na nasz mikrokontroler.
Przydatne jest też duże zróżnicowanie płytek dostępnych na tej
platformie. Za najbardziej rozbudowane (np. te z obsługą USB Host i
Ethernetem) zapłacimy kilkaset złotych. Natomiast za najprostsze bez
takich funkcjonalności około 50 - 60 zł.

MicroPython jest opcją dla chcących wykorzystać Pythona. Obecnie
dostępna jest tylko płytka PyBoard za około 30 funtów. Jeżeli Chińczycy
zaimplementują MicroPythona na jednej ze swoich płytek z kompatybilnym
mikrokontrolerem, to wtedy powinny pojawić się tańsze płytki. PyMCU jest
tańsze, ale wymaga połączenia z komputerem, by działać. Obie Pythonowe
platformy niestety nie oferują tylu bibliotek i przykładów, co Arduino.
Może to mieć kluczowe znaczenie, gdy będziesz musiał wykorzystać inne
układy komunikujące się za pomocą SPI, czy I2C.

Czasami można zupełnie odejść od mikrokontrolerów i takiej elektroniki
\quotation{dla twórców}. Jeżeli potrzebujesz bardziej złożonego
urządzenia, np. kamery przemysłowej, czy laboratoryjnej, to nie zrobisz
jej z prostych modułów dla Arduino. Gotowe kamery tego typu nie
dostarczają na ogół bibliotek do Pythona, ale też muszą wystawiać jakieś
API, żeby aplikacje mogły z nich korzystać. Pod MS Windows dość często
jest to API dla .NET. Jeżeli tak się dzieje, to możemy wykorzystać
IronPythona do oskryptowania urządzenia. Sprzęt astronomiczny operuje na
standardzie ASCOM, który dostarcza API nie tylko dla .NET, ale też
wystawia obiekty COM, które można wykorzystać w zwykłym Pythonie. Dzięki
temu możesz zyskać dostęp do zaawansowanych kamer, czy kół filtrowych.
Jeżeli pod Linuksem twórca sprzętu daje binarny plik *.so z interfejsem
do urządzenia, to też da się go wykorzystać z poziomu Pythona za pomocą
modułu ctypes.cdll.

\subsection[na-zakończenie]{Na zakończenie}

Mam nadzieję, że zainteresowałem Ciebie elektroniką i możliwościami,
jakie dają projekty sprzętowe - czy to dla własnej zabawy, czy dla
nowych klientów Twojej firmy. Zaprezentowałem różne przykłady, różne
platformy sprzętowe. Możesz użyć ich jako punkt wyjścia we własnych
przygodach z elektroniką. Rynek rozwija się dość dynamicznie i za parę
miesięcy nie wszystkie zawarte tu informacje mogą być już aktualne.

\subsection[w-sieci]{W sieci}

\startitemize
\item
  http://www.dx.com/c/electrical-tools-499/arduino-scm-supplies-436 -
  chiński sklep z ogromną ilością elektroniki do Arduino i przyjaciół.
\item
  http://kamami.pl, http://nettigo.pl, http://botland.com.pl,
  http://electropark.pl, http://arduinosolutions.com - polskie sklepy z
  elektroniką i komponentami
\item
  http://www.circuitsforfun.com - strona PyMCU
\item
  http://micropython.org - strona MicroPythona i PyBoard
\item
  http://www.python.rk.edu.pl/w/p/elektronika-i-python/ - moja strona
  poświęcona Pythonowi {\em w elektronice}
\stopitemize


\stoptext
