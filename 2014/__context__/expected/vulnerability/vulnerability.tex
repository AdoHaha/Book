\usemodule[pycon-2014]
\starttext

\Title{Python w pracy vulnerability researchera/reversera}
\Author{Patryk Branicki}
\Author{Radosław Matusiak}
\MakeTitlePage

\subsection[wprowadzenie]{Wprowadzenie}

Prezentacja ma na celu przedstawienie wykorzystania języka programowania
Python, jako jednego z narzędzi wspomagających pracę osoby zajmującej
się analizą oprogramowania pod kątem odnajdywania błędów bezpieczeństwa
w dostarczonym oprogramowaniu. Przez analizę oprogramowania mamy na
myśli w większości przypadków podejście blackbox, czyli źródła
aplikacji/systemu nie są dostępne i to warsztat pracy, doświadczenie
oraz stworzone narzędzia umożliwiają odnajdywanie nieznanych dotychczas
błędów.

\subsection[słownik]{Słownik}

\startitemize
\item
  Vulnerability Researcher -- Osoba zajmująca się analizą oprogramowania
  pod kątem odnajdywania błędów bezpieczeństwa.
\item
  Reverser -- Osoba zajmująca się analizą rewersyjną/wsteczną (Reverse
  Engineering).
\item
  Vulnerability Research -- Analiza oprogramowania pod kątem wykrywania
  błędów bezpieczeństwa.
\item
  Vulnerability (vuln) -- Podatność, błąd bezpieczeństwa. Przykładowo:
  DoS, LCE, RCE.
\item
  DoS -- Denial of Service
\item
  LCE -- Local Code Execution
\item
  RCE -- Remote Code Execution
\item
  Reverse Engineering -- Proces analizy kodu bez posiadania źródeł.
  Przykładowo mamy plik wynikowy EXE i dokonujemy analizy (statycznej
  bądź dynamicznej), w celu odtworzenia logiki aplikacji/algorytmów w
  niej użytych. Obszary zastosowania: Malware, Vulnerability Research,
  Cracking, Software Protection, DRM etc.
\item
  Exploit - Konkretne wykorzystanie danej podatności. Click \& Run.
\item
  0day - Nieznany szerszej grupie osób exploit. Pierwszy exploit
  wykorzystujący daną podatność.
\item
  Fuzzing -- Automatyczny proces testowania oprogramowania z
  wykorzystaniem poprawnych oraz niepoprawnych danych wejściowych
  (input).
\item
  Input -- Dane, na których pracuje aplikacja. Może to być plik TXT,
  AVI, HTML, JS, dane przesyłane po socketach, endpointach, argumenty
  funkcji etc. Generalnie dane aplikacji, na które ma wpływ użytkownik
  pracujący z daną aplikacją.
\item
  Shellcode -- Wstrzyknięty kod atakującego, który ma zostać wykonany
  podczas wykorzystania danej podatności.
\stopitemize

\subsection[debugging]{Debugging}

Debugging w pracy reversera/researchera, niezależnie od wykonywanego
zajęcia (security research, malware/anti-malware, cracking/software
protections, vulnerability research) jest podstawowym
narzędziem/umiejętnością wykorzystywaną w czasie pracy. Pozwala
odtworzyć logikę/zastosowane algorytmy programu bez potrzeby posiadania
plików źródłowych aplikacji. W przypadku odnalezienia nowej podatności,
pozwala na dokładną analizę problemu oraz jego potencjalne
wykorzystanie. Implementując konkretny exploit musimy znać dokładny stan
aplikacji w czasie wywołania podatności. Przez „stan” rozumiem stan
całej przestrzeni adresowej procesu: stack(s), heap(s), thread(s),
moduły wykonywalne załadowane w procesie, a w szczególności ich adresy
oraz właściwości, ich tablice IAT (import)/EAT(export), systemowe
struktury danych pozwalające systemowi operacyjnemu na zarządzanie
procesem i wykonywaniem wątków etc.

\section[narzędzia]{Narzędzia}

\subsection[win32-debug-api-oraz-ctypes]{Win32 Debug API oraz
CTYPES}

System operacyjny Windows posiada bogate API umożliwiające pisanie
własnych debuggerów dla R3 (Ring 3 -- User Mode) {[}1{]}. Istnieje wiele
powodów, dla których napisanie własnego debuggera jest często
najrozsądniejszym rozwiązaniem. Debugger to nic innego jak swoisty
nadzorca procesu, który może reagować, w zaprogramowany przez nas
sposób, na zachodzące w procesie zdarzenia:

\starttyping
* CREATE_PROCESS_DEBUG_EVENT
* CREATE_THREAD_DEBUG_EVENT
* EXCEPTION_DEBUG_EVENT
* EXIT_PROCESS_DEBUG_EVENT
* EXIT_THREAD_DEBUG_EVENT
* LOAD_DLL_DEBUG_EVENT
* OUTPUT_DEBUG_STRING_EVENT
* UNLOAD_DLL_DEBUG_EVENT
* RIP_EVENT
\stoptyping

Jednym z najciekawszych zdarzeń z naszej perspektywy jest
EXCEPTION\letterunderscore{}DEBUG\letterunderscore{}EVENT. Zostaje on
wywołany przez system w przypadku np.: dzielenia przez zero, dostępu do
nieprzydzielonej pamięci, bądź zatrzymaniu się na ustawionym przez nas
breakpointcie {[}2{]}.

Exception codes {[}3{]}:

\starttyping
* EXCEPTION_ACCESS_VIOLATION
* EXCEPTION_ARRAY_BOUNDS_EXCEEDED
* EXCEPTION_BREAKPOINT
* EXCEPTION_DATATYPE_MISALIGNMENT
* EXCEPTION_FLT_DENORMAL_OPERAND
* EXCEPTION_FLT_DIVIDE_BY_ZERO
* EXCEPTION_FLT_INEXACT_RESULT
* EXCEPTION_FLT_INVALID_OPERATION
* EXCEPTION_FLT_OVERFLOW
* EXCEPTION_FLT_STACK_CHECK
* EXCEPTION_FLT_UNDERFLOW
* EXCEPTION_ILLEGAL_INSTRUCTION
* EXCEPTION_IN_PAGE_ERROR
* EXCEPTION_INT_DIVIDE_BY_ZERO
* EXCEPTION_INT_OVERFLOW
* EXCEPTION_INVALID_DISPOSITION
* EXCEPTION_NONCONTINUABLE_EXCEPTION
* EXCEPTION_PRIV_INSTRUCTION
* EXCEPTION_SINGLE_STEP
* EXCEPTION_STACK_OVERFLOW
\stoptyping

Dzięki własnemu debuggerowi, możemy w dowolny sposób nadzorować oraz
modyfikować wykonywane w procesie wątki.

Z naszej perspektywy najbardziej interesujące są: *
Odczytanie/modyfikacja danych w pamięci. * Nadzorowanie wyjątków i ich
automatyczna analiza w przypadku sesji fuzzowania. * Modyfikacje
wykonywanego kodu. * Monitorowanie wykonywania kodu. * Wstrzykiwanie
kodu.

Dzięki wykorzystaniu CTYPES, w skryptach możemy wykorzystywać wywołania
natywnych Win32 API w skryptach Python.

\subsection[pydbg]{PyDBG}

PyDBG {[}4{]}, stworzony przez Pedram Amini, jest niczym innym jak
przyjemnym dla programisty wrapperem na Debug API napisanym w Pythonie.
Umożliwia stworzenie małego, dedykowanego debuggera dosłownie w kilku
linijkach kodu.

Funkcjonalność PyDBG: * Zarządzanie software breakpoints * Zarządzanie
hardware breakpoints * Zarządzanie memory breakpoints * Zarządzanie
uchwytami procesu * Zarządzanie i manipulacja rejestrami procesora *
Zarządzanie modułami wykonywalnymi (ładowanie, odładowywanie) *
Mapowanie adresów VA na załadowane moduły wykonywalne * Przetwarzanie
debug events * Podłączanie się do procesu i odłączanie (ang. process
attach/detach) * Disassemble * Dump context * Enumeracja modułów
wykonywalnych w procesie * Enumeracja procesów i wątków * Tworzenie
procesów * Tworzenie wątków w procesie * Kończenie procesów i wątków *
Otwieranie procesów i wątków * Event handlery dla obsługi debug events *
Obsługa łańcuchów znakowych znajdujących się w pamięci procesu *
Zrzucanie snapshotu procesu * Odczyt/zapis pamięci procesu *
Zatrzymywanie/wznawianie wątków w procesie * Zarządzanie stosami *
Zarządzaniem stronami pamięci (zmiany flag: RWX)

\subsection[pydbg-hooking]{PyDBG: Hooking}

PyDBG umożliwia wygodne ustawianie breakpointów dla wykonywanych w
procesie wątków. Dzięki temu mamy łatwą możliwość odczytu argumentów
funkcji, zmiennych globalnych, zmiennych lokalnych, buforów danych
alokowanych na stosie bądź stercie (ang. heap) oraz oczywiście ich
modyfikacji. Hooking umożliwia wygodne nadzorowanie pracy procesu oraz
analizę działania wykonywanego kodu. Dzięki możliwości zaprogramowania
obsługi każdego interesującego nas zdarzenia, możemy wprowadzić dowolną
logikę, która pozwala nam analizować skomplikowane przetwarzanie danych
przez dany wątek.

\subsection[pydbg-dynamiczne-modyfikowanie-kodu]{PyDBG: Dynamiczne
modyfikowanie kodu}

Kolejnym krokiem, po odczytywaniu i modyfikowaniu danych używanych przez
dany proces, jest nadzorowanie i modyfikacja kodu wykonywanego w danym
wątku. Posiadamy możliwość reagowania na ustawione breakpointy, możemy
dokonywać zmiany wartości rejestrów, zmiany wartości pamięci, zmieniać
miejsce wykonania kodu, dokonywać zmiany decyzji o skokach warunkowych.
Dodatkowo, jeśli sytuacja tego wymaga, możemy wstrzyknąć nasz kod w
postaci shellcodu, bądź dodatkowej biblioteki, w której znajduje się
nasz kod.

\subsection[immunity-debugger]{Immunity Debugger}

Immunity Debugger {[}5{]} bazuje na debuggerze OllyDbg {[}6{]}
stworzonym przez Oleh Yuschuk. Immunity Debugger rozszerza OllyDbg o
interpreter Pythona wraz z pokaźnym API, umożliwiający tworzenie nowych
narzędzi na powyższej platformie.

\subsection[ida-idapython]{IDA \& IDAPython}

Podobnie jak w przypadku Immunity Debuggera, stworzenie pluginu
IDAPython {[}7{]} otworzyło środowisko IDA {[}8{]} na łatwe tworzenie
nowych narzędzi w Pythonie, wspomagających analizę modułów
wykonywalnych.

\subsection[budowa-fuzzerów-w-pythonie]{Budowa fuzzerów w Pythonie}

Proces fuzzowania aplikacji można podzielić na kilka kroków: 1.
Stworzenie danych wejściowych dla aplikacji na podstawie wzorca. 2.
Uruchomienie aplikacji w nadzorowanym środowisku (np. z podłączonym
debuggerem). 3. Załadowanie stworzonych w punkcie 1. danych do
aplikacji. 4. Analiza zachowania aplikacji. a. Gdy aplikacja zachowuje
się stabilnie (powrót do punktu 2.). b. Gdy aplikacja zachowuje się
niestabilnie -- zalogowanie błędu. 5. Końcowa analiza znalezionych
błędów.

Każdy z tych punktów wymaga niezależnych narzędzi. Każde z narzędzi
często wymaga też modyfikacji, w zależności od aplikacji, która ma być
celem fuzzowania. Tworzenie danych wejściowych może być tak proste, jak
tylko zmiany pojedynczych bitów/bajtów (dumb fuzzing) lub tak złożonym
jak analiza danego formatu i zmiana tylko interesujących nas elementów
(smart fuzzing). Niezależnie od wybranej strategii, generowanie danych
za pomocą Pythona jest zadaniem stosunkowo łatwym. Stworzenie
nadzorowanego środowiska wymaga dokładnej znajomości zagadnień z wielu
dziedzin (protokołów sieciowych, architektury systemów operacyjnych,
błędów bezpieczeństwa, reverse engineeringu, etc.). Najczęściej
stosowanymi mechanizmami do nadzorowania aplikacji, jako procesu
systemowego, są debuggery oraz monitory sieciowe (np. Wireshark).
Tworzenie własnych debuggerów jest czasochłonne, ale pozwala na pełną
kontrolę uruchomionej aplikacji. Dodatkowym atutem stosowania własnych
debuggerów jest możliwość automatycznej analizy odnalezionego błędu, co
skraca znacząco czas pracy researchera. Analiza wyników jest zadaniem
niepowtarzalnym. Każda aplikacja/biblioteka może generować wyniki w
innym formacie (core dump, plik tekstowy JVM, wyniki z własnego
debuggera). Python jest doskonałym narzędziem do pisania własnych
analizatorów plików wynikowych i wydobywania z nich istotnych
szczegółów.

\section[sulley]{Sulley}

Sulley {[}9{]} jest frameworkiem przeznaczonym do fuzzowania aplikacji.
Jest on w całości napisany w Pythonie. Jego niewątpliwymi zaletami są
prostota oraz kompleksowość. Pozwala on na tworzenie w intuicyjny sposób
reprezentacji danych (request), które później są wykorzystywane w
sesjach fuzzowania. Sulley zawiera wbudowane agenty, które pozwalają na
monitorowanie środowiska: * network\letterunderscore{}monitor.py -
logowanie komunikacji sieciowej do plików PCAP. *
process\letterunderscore{}monitor.py - pozwala na monitorowanie
fuzzowanego procesu. * etc. Dodatkowym atutem jest wbudowany serwer
HTTP, pozwalający śledzić postępy fuzzowania. Jest to proste, ale
potężne narzędzie, z którym definitywnie warto się zapoznać. Często może
stanowić pierwszy krok w rozpoczęciu procesu testowania danej aplikacji.
Sulley jest jednym z ważniejszych frameworków wykorzystywanych przez
security researcherów w ich pracy.

\section[fuzzmyapp-fuzzing-framework-fff]{FuzzMyApp Fuzzing
Framework „FFF”}

W FuzzMyApp rozwijamy własny framework, używany przez nas w czasie
fuzzowania aplikacji. Powodem podjęcia takiej decyzji było ciągłe
napotykanie powtarzalnych problemów, które wybitnie nadawały się do
automatyzacji. Framework jest stworzony całkowicie w Pythonie 2.x. Jego
głównymi zadaniami są: * Tworzenie zbioru danych wejściowych, dla
zadanych strategii mutacji. * Uruchamianie aplikacji w nadzorowanym
środowisku. * Agregacja i sortowanie wyników zebranych w czasie sesji
fuzzowania. * Wstępna analiza odnalezionych podatności. * Generacja
statystyk dla sesji fuzzowania. * Wielozadaniowość.

Wielozadaniowość w naszym kontekście oznacza dany scenariusz testowy.
Czyli przykładowo fuzzowanie danego formatu pliku dla konkretnej
aplikacji. System stworzony jest w taki sposób, by wykonać jak najwięcej
powtarzalnych zadań, pozostawiając researcherowi/reverserowi analizę
odkrytych podatności. Większość tych zadań nie jest krytyczna czasowo,
wiec zdecydowaliśmy się właśnie na Pythona, w celu szybkiej i przyjemnej
implementacji naszego frameworku. Decyzja o wykorzystaniu Pythona
została również podjęta dzięki tak dużej liczbie zewnętrznych
narzędzi/bibliotek, które znacząco ułatwiają pracę. Framework został
napisany w sposób modułowy, dzięki czemu jego dostosowanie do nowej
aplikacji wymaga minimalnego nakładu pracy.

\section[fuzzowanie-aplikacji-napisanych-w-javie]{Fuzzowanie
aplikacji napisanych w Javie}

Jednym z napotkanych problemów podczas pracy w FuzzMyApp było stworzenie
narzędzi do fuzzowania aplikacji napisanych w Javie. Proces ten
podzieliliśmy na dwie części: 1. Znalezienie potencjalnych
niebezpiecznych metod. 2. Stworzenie kodu fuzzującego dla metod
znalezionych w punkcie 1.

Punkt 2 został rozwiązany poprzez stworzenie dedykowanego generatora
kodu Javy, który importował interesujące nas klasy i fuzzował wskazane
metody. Generacja kodu Javy oraz jego kompilacja została
zaimplementowana jako skrypt Pythona, bazujący na pliku wzorcowym. Sam
proces fuzzowania odbywał się poprzez uruchomienie wszystkich
wygenerowanych plików klas. Dodatkowym ułatwieniem było samo zachowanie
JVM, która zrzuca bardzo opisowe tekstowe pliki dump w wypadku
nieobsłużonego błędu w samej wirtualnej maszynie Javy. Aby znaleźć
metody, które moglibyśmy poddać fuzzowaniu, postanowiliśmy statycznie
analizować bytecode Java. W tym celu stworzyliśmy dedykowane narzędzie w
Pythonie, w oparciu o dokumentację specyfikacji JVM. Do tego zadania
doskonale nadaje się moduł struct. Naszym celem nie było odtworzenie
pełnej specyfikacji, lecz tylko stworzenie narzędzia, pozwalającego
wydobyć sygnatury metod dostępnych w danej klasie. Dzięki tym sygnaturom
mogliśmy przefiltrować listę dostępnych metod i wybrać tylko te, które:
a. przyjmowały parametry (metody bezparametrowe nie mogą być fuzzowane).
b. były metodami publicznymi (brak potrzeby wykorzystania refleksji). c.
były metodami zaimplementowanym natywnie (słowo kluczowe native) przy
pomocy JNI.

Takie podejście okazało się dużym sukcesem. Udało nam się znaleźć błędy
bezpieczeństwa m.in. w JRE7 oraz JOAL: * FMA-2013-010 {[}10{]} -- wiele
błędów w JRE7. Oracle nie uznał ich jako błędy bezpieczeństwa, w
przeciwieństwie do firmy specjalizującej się w Vulnerability Research -
Beyond Security {[}11{]}. * FMA-2012-038 {[}12{]} (CVE-2013-4099) --
prawie 70 błędów Remote Code Execution w projekcie JOAL (OpenAL.dll).
Nasza praca doprowadziła do największego przeglądu bezpieczeństwa
projektu {[}13{]}.

Wykorzystanie Pythona pozwoliło na pełną automatyzację procesu, którego
czas oscylował wokół kilkunastu godzin ciągłej pracy.

\subsection[emulacja-cpu-ia-32---pyemu]{Emulacja CPU IA-32 - PyEmu}

PyEmu {[}14{]} jest emulatorem dla procesora o architekturze IA-32
napisanym w Pythonie. Wprowadza abstrakcję na procesor oraz pamięć
procesu. Podobnie jak Immunity Debugger oraz IDAPython dla produktu IDA,
stanowi kolejny framework umożliwiający tworzenie nowych narzędzi do
dynamicznej analizy kodu w oparciu o warstwę abstrakcji, jaką jest
emulator procesora.

\subsection[podsumowanie]{Podsumowanie}

Niniejszy artykuł jest tylko wstępem do problemu wyszukiwania błędów
bezpieczeństwa w aplikacjach, frameworkach, bibliotekach i samych
systemach operacyjnych. Jest to zagadnienie pełne problemów i wyzwań, w
których znacząco pomaga zastosowanie Pythona jako języka programowania.
Dodatkowym atutem jest liczna baza narzędzi zaimplementowanych i
dostępnych publicznie w tym języku. Narzędzia napisane w Pythonie w
łatwy sposób można szybko modyfikować i przystosowywać do nowych
zastosowań. Python jest fantastycznie prostym, czytelnym, jednakże
potężnym językiem programowania. Pisząc obiektowo/proceduralnie w innych
językach może nam brakować wielu elementów w Pythonie, jednak przeważnie
znajdzie się sposób na ich analogiczną reprezentację. Python umożliwia
budowanie i utrzymywanie nie tylko prostych skryptów ale i dużych
obiektowych rozwiązań. Jednak przy dużych rozwiązaniach warto trzymać
się sprawdzonych procesów produkcji i utrzymania oprogramowania i
\ldots{} pisać dużo testów jednostkowych! Czasami pełen dynamizm oraz
prostota w Pythonie może być negatywnym czynnikiem, jeśli mamy duże
rozwiązanie z n warstwami abstrakcji. Brak prostych mechanizmów
definiowania zakresu widoczności bądź wymuszania implementacji, innych
niż rzucanie wyjątkami w czasie runtime. Jednak powyższe aspekty
(zalety/wady) Pythona pozostawmy na kuluarowe dyskusje, do których
serdecznie zapraszamy.

\subsection[referencje]{Referencje}

\startitemize
\item
  {[}1{]} Linki MSDN Debugging Reference
  http://msdn.microsoft.com/en-us/library/windows/desktop/ms679304(v=vs.85).aspx
\item
  {[}2{]} Structured Exception Handling
  http://msdn.microsoft.com/en-us/library/windows/desktop/ms680660(v=vs.85).aspx
\item
  {[}3{]} SEH Exception Codes
  http://msdn.microsoft.com/en-us/library/windows/desktop/aa363082(v=vs.85).aspx
\item
  {[}4{]} PyDBG https://github.com/OpenRCE/pydbg
\item
  {[}5{]} Immunity Debugger http://www.immunityinc.com/products/debugger
\item
  {[}6{]} OllyDbg http://www.ollydbg.de
\item
  {[}7{]} IDAPython http://code.google.com/p/idapython
\item
  {[}8{]} IDA https://www.hex-rays.com
\item
  {[}9{]} Sulley https://github.com/OpenRCE/sulley
\item
  {[}10{]} http://www.fuzzmyapp.com/pl/advisories.html
\item
  {[}11{]} http://www.securiteam.com/securitynews/5DP3K0AAVA.html
\item
  {[}12{]}
  http://www.fuzzmyapp.com/advisories/FMA-2012-038/FMA-2012-038-PL.xml
\item
  {[}13{]} https://www.jogamp.org/blog/index1.html
\item
  {[}14{]} PyEmu http://code.google.com/p/pyemu
\stopitemize


\stoptext
