\usemodule[pycon-2014]
\starttext


\section[common-programming-mistakes-in-python---dariusz-śmigiel]{Common
programming mistakes in Python - Dariusz Śmigiel}

\subsection[introduction]{Introduction}

Python is known as one of the simplest programming languages. You need
to know syntax, some basic things like indentation and voilà! You're a
developer! But\ldots{} not so fast. Often it turns out that that's not
everything that you need to know about Python to be an efficient
developer. You need to know some subtle things. Without this, you can
have big problems, because behaviour that you have, is different than
expected.

\section[full-disclosure]{Full disclosure}

Idea for this talk was taken from work done by Martin Chikilian,
originally published at
\useURL[url1][http://www.toptal.com/python/top-10-mistakes-that-python-programmers-make][][www.toptal.com/python/top-10-mistakes-that-python-programmers-make]\from[url1]

\subsection[mistake-1-expressions-as-defaults-for-function-arguments]{Mistake
\#1: Expressions as defaults for function arguments}

Python allows to specify default values for function arguments. When
function is called without the argument, argument will have assigned
value provided as default. It's a big advantage, because user doesn't
have to remember, or even know, about providing values to function. It
works pretty good, as long, as you won't give any mutable values there.

\starttyping
    >>> def foo(bar=[]):
    ...     bar.append("baz")
    ...     return bar
\stoptyping

\section[current-behaviour]{Current behaviour}

It's simple function, where we want to append simple string to list.
Expected behaviour of this function, would be:

\startitemize
\item
  create list called \type{bar}
\item
  append to it a string \type{baz}
\item
  return list
\stopitemize

Unfortunately, it doesn't work like expected. When function is called,
we're receiving the same list, with growing number of strings.

\starttyping
    >>> foo()
    ['baz']
    >>> foo()
    ['baz', 'baz']
    >>> foo()
    ['baz', 'baz', 'baz']
\stoptyping

\section[whats-happening]{What's happening?}

List \type{bar} is initialized only once; when function definition is
evaluated. That's why, when we're calling \type{foo}, every time we're
appending new string to the same list. This behaviour is implemented on
purpose.
\useURL[url2][http://docs.python-guide.org/en/latest/writing/gotchas/\#mutable-default-arguments][][Early
binding]\from[url2] means that the compiler is able to directly
associate the identifier name (such as a function or variable name) with
a machine address.

\section[solution]{Solution}

\starttyping
    >>> def foo(bar=None):
    ...     if bar is None:
    ...         bar = []
    ...     bar.append("baz")
    ...     return bar

    >>> foo()
    ['baz']
    >>> foo()
    ['baz']
\stoptyping

\subsection[mistake-2-binding-variables-in-closures]{Mistake \#2:
Binding variables in closures}

This time, we'll look at
\useURL[url3][http://docs.python-guide.org/en/latest/writing/gotchas/\#late-binding-closures][][late
bindings]\from[url3]. Assume, we have function to build 5 next
multipliers of given value.

\starttyping
    >>> def create_multipliers():
    ...     return [lambda x : i * x for i in range(5)]
    >>> for multiplier in create_multipliers():
    ...     print(multiplier(2))
    ...
\stoptyping

\section[current-behaviour-1]{Current behaviour}

We're expecting to retrieve:

\starttyping
0
2
4
6
8
\stoptyping

But, instead of this we've got:

\starttyping
8
8
8
8
8
\stoptyping

\section[whats-happening-1]{What's happening?}

Closures are {\em late binding}. The values of variables used in
closures are looked up at the time the inner function is called.
Whenever any of the returned functions are called, the value of \type{i}
is looked up in the surrounding scope at call time. By then, the loop
has completed and \type{i} is left with its final value of 4. And,
contrary to popular belief, it has nothing to do with {\em lambdas}.

\starttyping
    >>> def create_multipliers():
    ...     multipliers = []
    ...
    ...     for i in range(5):
    ...         def multiplier(x):
    ...             return i * x
    ...         multipliers.append(multiplier)
    ...     return multipliers
    ...
    >>> for multiplier in create_multipliers():
    ...     print(multiplier(2))
    ...
    ...
    8
    8
    8
    8
    8
    >>>
\stoptyping

\section[solution-1]{Solution}

As mentioned in \#1, Python evaluates early the default arguments of a
function, so we can create a closure that binds immediately to its
arguments by using default arg:

\starttyping
    >>> def create_multipliers():
    ...     return [lambda x, i=i : i * x for i in range(5)]
    ...
    >>> for multiplier in create_multipliers():
    ...     print (multiplier(2))
    ...
    ...
    0
    2
    4
    6
    8
    >>>
\stoptyping

But, even better idea is change it to explicit:

\starttyping
    >>> def get_func(i):
    ...     return lambda x: i * x
    ...
    >>> def create_multipliers():
    ...     return [get_func(i) for i in range(5)]
    ...
    >>> for multiplier in create_multipliers():
    ...     print (multiplier(2))
    ...
    ...
    0
    2
    4
    6
    8
    >>>
\stoptyping

\subsection[mistake-3-local-names-are-detected-statically]{Mistake \#3:
Local names are detected statically}

Again, our main concern is about
\useURL[url4][http://www.onlamp.com/pub/a/python/2004/02/05/learn_python.html?page=2][][closures]\from[url4].
We want to print global \type{x}, and later change it to new value.

\starttyping
    >>> x = 99
    >>> def func():
    ...     print(x)
    ...     x = 88
    ...
\stoptyping

\section[current-behaviour-2]{Current behaviour}

And, again. Something goes wrong.

\starttyping
    >>> func()
    Traceback (most recent call last):
      File "<input>", line 1, in <module>
      File "<input>", line 2, in func
    UnboundLocalError: local variable 'x' referenced before assignment
\stoptyping

\section[whats-happening-2]{What's happening}

While parsing this code, Python sees the assignment to \type{x} and
decides that \type{x} will be a local variable in the function. But
later, when the function is actually run, the assignment hasn't yet
happened when the print executes, so Python raises an undefined name
error.

\section[solution-2]{Solution}

In this case, we have two solutions: nasty (\type{global}) and better
(reference it through enclosing module name)

\starttyping
    >>> x = 99
    >>> def func():
    ...     global x
    ...     print(x)
    ...     x = 88
    ...
    >>> func()
    99
\stoptyping

\subsection[mistake-4-class-variables]{Mistake \#4: Class variables}

We have three classes:

\starttyping
    >>> class A(object):
    ...     x = 1
    ...
    >>> class B(A):
    ...     pass
    ...
    >>> class C(A):
    ...     pass
    ...
\stoptyping

\section[current-behaviour-3]{Current behaviour}

After initialization, we're able to get \type{x} from all classes

\starttyping
    >>> print(A.x, B.x, C.x)
    1 1 1
\stoptyping

We can also assign variables:

\starttyping
    >>> B.x = 2
    >>> print(A.x, B.x, C.x)
    1 2 1
\stoptyping

And another time:

\starttyping
    >>> A.x = 3
    >>> print(A.x, B.x, C.x)
    3 2 3
\stoptyping

Oops.

\section[whats-happening-3]{What's happening?}

\useURL[url5][http://legacy.python.org/dev/peps/pep-0253/][][MRO]\from[url5]
is happening. Because \type{C} class has no attribute \type{x}, it goes
to class \type{A} and returns value for it.

\subsection[mistake-5-exception-handling]{Mistake \#5: Exception
handling}

Python has nice feature to catch any raised exception. It allows you to
prevent the software from crashing, reacting for expected or
non-expected situations. In this case, we're using \type{try/except}
clause:

\starttyping
    >>> try:
    ...     list_ = ['a', 'b']
    ...     int(list_[2])
    ... except ValueError, IndexError:
    ...     pass
\stoptyping

\section[current-behaviour-4]{Current behaviour}

Our code seems to be very well protected. In one place we're catching
value conversion problem and also index bigger than list. Unfortunately,
it doesn't work like we would like to:

\starttyping
    ...
    Traceback (most recent call last):
      File "<input>", line 3, in <module>
    IndexError: list index out of range
\stoptyping

\section[whats-happening-4]{What's happening?}

In Python 2
\useURL[url6][https://docs.python.org/2/tutorial/errors.html\#handling-exceptions][][old
syntax is still supported for backwards compatibility]\from[url6]. In
modern Python, we would write: \type{except ValueError as e} what is
equal to \type{except ValueError, e} In our case,
\type{except ValueError, IndexError} is equivalent to
\type{except ValueError as IndexError} which is not what we want.

\section[solution-3]{Solution}

Python 3 has no problems with above code. It throws an error, and shows
exact place, where something doesn't work:

\starttyping
    File "<stdin>", line 4
      except ValueError, IndexError:
\stoptyping

But if you're still using Python 2 and don't see any warning signs,
you'll need to follow below solution:

\starttyping
    >>> try:
    ...     list_ = ['a', 'b']
    ...     int(list_[2])
    ... except (ValueError, IndexError) as e:
    ...     pass
\stoptyping

Works OK, for both, Python 2 and Python 3.

\subsection[mistake-6-scope-rules]{Mistake \#6: Scope rules}

Sometimes, we want to know, how many times some code was run. It would
be good, to count it. So, we'll write a simple snippet:

\starttyping
    >>> bar = 0
    >>> def foo():
    ...    bar += 1
    ...    print(bar)
    ...
\stoptyping

\section[current-behaviour-5]{Current behaviour}

And run it:

\starttyping
    >>> foo()
    Traceback (most recent call last):
      File "<input>", line 1, in <module>
      File "<input>", line 2, in foo
    UnboundLocalError: local variable 'bar' referenced before assignment
\stoptyping

BAM! The same happens, when we change line \type{i += 1} to
\type{i = i + 1} It doesn't matter which one is used, both throw an
error.

\section[whats-happening-5]{What's happening?}

Python scope resolution is based on LEGB:

\startitemize
\item
  Local - names assigned in any way within a function ({\em def} or
  {\em lambda}) and not declared global in that function;
\item
  Enclosing function locals - name in the local scope of any and all
  enclosing ({\em def} or {\em lambda}), from inner to outer;
\item
  Global (module) - names assigned at the top-level of a module file, or
  declared global in a {\em def} within the file;
\item
  Built-in (Python) - names preassigned in a built-in names module:
  {\em open}, {\em range}, {\em SyntaxError}, \ldots{}
\stopitemize

So, when you make an assignment to variable, Python considers it as in
local scope. It shadows everything, that is outside this scope. In this
case, we don't {\em see} variable \type{i} declared before function
definition.

\section[solution-4]{Solution}

Variable needs to be modified in the scope that it was declared in. We
can achieve this by passing it as an argument to a function that will
return it's new value.

\starttyping
    >>> bar = 0
    >>> def foo(bar=bar):
    ...     bar += 1
    ...     return bar
    ...
    >>> foo()
    1
    >>> foo()
    1
\stoptyping

\subsection[mistake-7-modifying-list-while-iterating-over-it]{Mistake
\#7: Modifying list while iterating over it}

Often we need to do something with lists. For instance, remove all odd
values from list.

\starttyping
    >>> odd = lambda x: bool(x % 2)
    >>> numbers = list(range(10))
    >>> for i in range(len(numbers)):
    ...     if odd(numbers[i]):
    ...         del numbers[i]
    ...
    Traceback (most recent call last):
          File "<stdin>", line 2, in <module>
    IndexError: list index out of range
\stoptyping

\section[whats-happening-6]{What's happening}

We're iterating over this list, and when value is odd, we're removing it
from list. In the same time list is shrinking, so after few iterations,
list is shorter than expected.

\section[solution-5]{Solution}

Suggested solution, probably the simplest one.

\starttyping
    >>> odd = lambda x: bool(x % 2)
    >>> numbers = list(range(10))
    >>> for n in numbers:
    ...     if odd(n):
    ...         numbers.remove(n)
    ...
    >>> numbers
    [0, 2, 4, 6, 8]
\stoptyping

But, we cannot fall into false impression, that this works.

\starttyping
    >>> odd = lambda x: bool(x % 2)
    >>> numbers
    [0, 0, 1, 1, 2, 2, 3, 3, 4, 4, 5, 5, 6, 6, 7, 7, 8, 8, 9, 9]
    >>> for n in numbers:
    ...     if odd(n):
    ...         numbers.remove(n)
    ...
    >>> numbers
    [0, 0, 1, 2, 2, 3, 4, 4, 5, 6, 6, 7, 8, 8, 9]
\stoptyping

As you may see, it iterates, removes, but also leaves odd numbers. Does
it mean, there is no proper solution for it? No! To solve it, we will
use
\useURL[url7][https://docs.python.org/2/tutorial/datastructures.html\#list-comprehensions][][list
comprehensions]\from[url7]

\starttyping
    >>> odd = lambda x: bool(x % 2)
    >>> numbers = list(range(10))
    >>> numbers = [n for n in numbers if not odd(n)]
    >>> numbers
    [0, 2, 4, 6, 8]
\stoptyping

\subsection[mistake-8-name-clashing-with-python-standard-library-modules]{Mistake
\#8: Name clashing with Python Standard Library modules}

Sometimes, we forget about simple things. Not every possible module name
should be used. Assume, you're creating an application for sending
emails.

\starttyping
app
|-sender.py
|-receiver.py
|-email.py
\stoptyping

\type{sender.py}

\starttyping
    from email.mime.multipart import MIMEMultipart

    msg = MIMEMultipart('alternative')
    msg['Subject'] = "Link"
    msg['From'] = 'from@email.com'
    msg['To'] = 'to@email.com'
\stoptyping

\section[current-behaviour-6]{Current behaviour}

When we're trying to run this application, we've receiving
\type{ImportError}

\starttyping
    Traceback (most recent call last):
      File "<input>", line 1, in <module>
    ImportError: No module named mime.multipart
\stoptyping

\section[whats-happening-7]{What's happening}

We've mixed Standard Library module called \type{email} with local
module \type{email.py}. In this case, application sees local module, and
imports it, instead of expected one.

\section[solution-6]{Solution}

In this case, we have to be very careful. Python uses pre-defined order
of
\useURL[url8][https://docs.python.org/2/tutorial/modules.html\#the-module-search-path][][importing
modules]\from[url8]. When we're trying to import \type{spam} it looks
in:

\startitemize
\item
  built-in module (string, re, datetime, etc.)
\item
  searches for a file named \type{spam.py} in directories given by the
  variable \type{sys.path} in
\item
  the directory containing the input script (or the current directory).
\item
  PYTHONPATH (a list of directory names, with the same syntax as the
  shell variable PATH).
\item
  the installation-dependent default.
\stopitemize

\subsection[bonus-mistake-differences-between-python-2-and-python-3]{Bonus
Mistake: Differences between Python 2 and Python 3:}

Simple thing can make a big difference. Python 3 handles exception in
\useURL[url9][https://docs.python.org/3/reference/compound_stmts.html\#except][][local
scope]\from[url9].

\starttyping
    import sys

    def bar(i):
        if i == 1:
            raise KeyError(1)
        if i == 2:
            raise ValueError(2)

    def bad():
        e = None
        try:
            bar(int(sys.argv[1]))
        except KeyError as e:
            print('key error')
        except ValueError as e:
            print('value error')
        print(e)

    bad()
\stoptyping

\section[current-behaviour-7]{Current behaviour}

\type{Python 2}

\starttyping
    $ python foo.py 1
    key error
    1
    $ python foo.py 2
    value error
    2
\stoptyping

\type{Python 3}

\starttyping
    $ python3 foo.py 1
    key error
    Traceback (most recent call last):
      File "foo.py", line 19, in <module>
    bad()
      File "foo.py", line 17, in bad
    print(e)
    UnboundLocalError: local variable 'e' referenced before assignment
\stoptyping

\section[whats-happening-8]{What's happening}

When an exception has been assigned to a variable name using
\type{as target}, it is cleared at the end of the except clause:

\starttyping
    except E as N:
        try:
            foo
        finally:
            del N
\stoptyping

This means the exception must be assigned to a different name to be able
to refer to it after the except clause. Exceptions are cleared because
with the traceback attached to them, they form a reference cycle with
the stack frame, keeping all locals in that frame alive until the next
garbage collection occurs.


\stoptext
