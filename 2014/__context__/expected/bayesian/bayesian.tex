\usemodule[pycon-2014]
\starttext


\section[bayesian-ab-testing-with-python---bogdan-kulynych]{Bayesian A/B
testing with Python - Bogdan Kulynych}

\subsection[introduction]{Introduction}

A/B testing is a simple yet powerful instrument to evaluate design
decisions for web applications. It's a kind of behavioral research study
akin to clinical trials, conducted to assess the effectiveness of
alternative variations of graphical designs, wording solutions, logical
decisions in terms of different performance metrics. It is to be
contrasted to {\em personalization}: the goal of A/B testing is to find
a solution that fits best to all users, whereas personalization aims to
find a best solution for every particular user.

Traditionally, a classical framework of statistical hypothesis testing
is used to evaluate the performance of variations. This text discusses
the Bayesian approach to A/B testing, its pros and cons, and a way of
implementing it using Python's \type{scikit} and an author's library
called
\useURL[url1][https://github.com/bogdan-kulynych/trials][][\hyphenatedurl{trials}]\from[url1].

\subsection[classical-approach]{Classical approach}

Suppose there are two different variations of the landing web page
design: call them A and B. A common problem is to find the one that will
probably produce more sign-ups based on collected data.

Within the classical frequentist approach, the following model is
commonly used. Let $A$, $B$ be two independent finite binary populations
corresponding to all possible page views by target audience. Each item
in population is either a $1$ (success, user viewed the page and signed
up) or $0$, (failure, user viewed the page and didn't sign up). We
assume that both populations are Bernoulli-distributed with {\em fixed
parameters} $p_A, p_B$. We don't know the parameters, but want to
estimate them. This is an important point that will be returned to
later: the parameters are {\em fixed yet unknown}.

We conduct an experiment by randomly showing the users different designs
and logging the results. Sample data $X_A=\{x_A^{(i)}\}$, and
$X_B=\{x_B^{(i)}\}$ are obtained, where $x_*^{(i)} \in \{0, 1\}$. We
assume that every observation $x_A^{(i)}$, $x_B^{(i)}$ is randomly
picked from the respective population in such a way that every item in
population has an equal chance to be picked, i.e. $X_A$ and $X_B$ are
randomly sampled from respective populations. Therefore,
$x_A^{(i)} \sim \text{Bern}(p_A)$, $x_B^{(i)} \sim \text{Bern}(p_B)$.
Moreover, this means that all observations $x_A^{(i)}$, $x_B^{(i)}$ are
independent. This is a strong assumption that almost never holds in real
life. It is natural that there exist some common factors that influence
multiple observations (for example, geographic location, time zone,
screen size, etc.). If the influence of these factors is quite
substantial, then probably this model isn't best fit for the task.

One way to reason about the true population parameters $p_A$ and $p_B$
is to use two-sample
\useURL[url2][https://en.wikipedia.org/wiki/Student%27s_t-test][][t-test]\from[url2].
We will test the null hypothesis about the equality of expected values
of A and B:
\startformula  H_0: p_A = p_B\ \Rightarrow\ p_A - p_B = 0  \stopformula
Since $p_A$, $p_B$ can be estimated by sample means
$\hat p_A = \bar X_A$, $\hat p_B = \bar X_B$, the distribution of the
test statistic
$T = \frac{\bar X_A - \bar X_B}{\sigma_{\hat p_A - \hat p_B}}$
(standardized $\hat p_A - \hat p_B$) approaches standard normal as the
number of observations gets large, and approximately equals the
Student's t-distribution if the number is not large enough. It is
straightforward to find the value of the test statistic, knowing a
formula for estimated variance:
\startformula \sigma_{\hat p_A - \hat p_B}^2 = \frac{\sigma_{X_A}^2}{n_{X_A}} + \frac{\sigma_{X_B}^2}{n_{X_B}} \stopformula

And for the number of degrees of freedom $\nu$ of the t-distribution:
\startformula  \nu = \frac{(\sigma_{X_A}^2/n_{X_A} + \sigma_{X_B}^2/n_{X_B})^2}{(\sigma_{X_A}^2/n_{X_A})^2/(n_{X_A}-1) + (\sigma_{X_B}^2/n_{X_B})^2/(n_{X_B}-1)}.  \stopformula

(\useURL[url3][https://en.wikipedia.org/wiki/Student%27s_t-test\#Equal_or_Unequal_sample_sizes.2C_unequal_variances][][Welch--Satterthwaite]\from[url3]
formula, for the the most general case).

Then we find corresponding data likelihoods under the null hypothesis
$P(X~|~H_0)$. An $\alpha$-confidence interval for the test statistic can
be obtained as well:
$\{ T \pm \sigma_{\hat p_A - \hat p_B} \cdot t_{\nu, \alpha}\}$.

In practice, if the number of observations is big enough, z-test is
often conducted instead, i.e.~the $T$-statistic is assumed to be
normally distributed.

To control the false positive rate we use $\alpha$ (usually $5\%$),
which actually equals to the probability of a false positive
$\text{P}(|T| < t_{\nu,\alpha}~|~\neg H_0 ) = \alpha$. The probability
of a false negative $\text{P}(|T| \geq t_{\nu,\alpha}~|~H_0)$ is a
called a power function of the test. For t-test, it depends on
difference between $p_A$ and $p_B$ and the number of observations. The
larger is the number of observations and the larger is the difference
between parameters, the smaller false negative rate is. For example, if
the relative difference between true population parameters is small,
e.g., 1\%, around 80,000 observations is needed to provide at least 20\%
false negative rate. If the difference is 50\%, 1000 is enough.

\section[problems]{Problems}

Frequentist treats a probability as a frequency of an event's occurrence
in a number of repeated experiments. Frequentist techniques rely greatly
on the Law of Large Numbers and Central Limit Theorem. For A/B testing
we would want to get the results as quickly as possible, possibly before
the CLT can be applied.

There's also a philosophical issue with hypothesis testing: it answers
the wrong question. For the example above, what we would like to do is
estimate the true population parameters based on the observed data. In
other words, we want to know something like $\text{P}(p_A > p_B~|~X)$.
That formula doesn't make much sense within the frequentist paradigm,
since probability is a frequency of an event in series of repeated
experiments, yet $p_A$ and $p_B$ are {\em fixed} values, they're not
outcomes of any experiments, not random variables, therefore, you can't
make any probabilistic statements about them. The only thing hypothesis
testing can provide is the likelihood of the data given the hypothesis,
$\text{P}(X~|~p_A > p_B)$. We need to pick a hypothesis, then check how
well the obtained data supports it. That makes sense, but the inverse
$\text{P}(H~|~X)$ would fit our question so much better. In fact, the
two probabilities $\text{P}(X~|~H)$ and $\text{P}(H~|~X)$ are connected
by the Bayes rule, but one first has to change the definition of
probability to apply it.

From a Bayesian perspective, probability is nothing but a degree of
belief on a scale from 0 to 1. This interpretation not only drops the
requirement for large repeated experiments, but allows to answer the
question directly: what are our best estimates on population parameters
given what we can observe.

\subsection[bayesian-approach]{Bayesian approach}

Whereas $p_A, p_B$ where fixed population parameters in the previous
model, for Bayesian approach let $p_A, p_B$ be independent random
variables. The data now should be considered fixed. Previously, it was
vice versa: parameters were fixed, and data was random; now parameters
are random variables, and data is fixed.

Let {\em prior} distributions of $p_A$ and $p_B$ be Beta-distributed:
\startformula p_A \sim \text{Beta}(\alpha_A, \beta_A),\ p_B \sim \text{Beta}(\alpha_B, \beta_B) \stopformula
The choice of Beta distribution will be explained later. Let number of
sign-ups $k_A = |\{x_A^{(i)}=1\}|$, number of page views $n_A = |X_A|$.
Assume that the likelihood of data obtained by logging views and
sign-ups is binomial:
$\text{P}(X_A ~|~ p_A) = \text{Binomial}(k_A; n_A, p_A)$. Analogically,
for B.

Applying Bayes theorem, we can find the posterior:
The nice result in the final step occurs because Beta is a conjugate
prior to Bernoulli and Binomial distributions, and this is the reason
why it was chosen as a prior for $p_A$ and $p_B$.

From the posterior distribution we can find interesting values like
$P(p_A > p_B ~|~X)$, $P(p_A < p_B ~|~ X)$, and e.g., {\em lifts}
$\frac{p_B - p_A}{p_A}$ and $\frac{p_A - p_B}{p_B}$, or pretty much any
function we'd like using Monte Carlo techniques.

This model relies on similar assumptions (independent Bernoulli trials,
implied by Binomial likelihood), but doesn't rely on large numbers like
the previous one. An important observation is that instead of finding
$\text{P}(\text{data} ~|~ \text{hypothesis})$ for a set of predefined
hypotheses about the parameters, we integrate over all possible values
of parameters (hypotheses) and get
$\text{P}(\text{parameter} ~| ~\text{data})$ using Bayes theorem. In
such sense, this approach is a generalization of hypothesis testing.

In the end, this approach doesn't suffer from the problems outlined
previously. This comes at the cost of expensive MCMC computations for
more complicated posteriors.

\subsection[example]{Example}

Suppose we've gathered some data. Say, variation A was viewed 44 times,
and produced 2 sign-ups, and variation B was viewed 96 times and
produced 11 sign-ups. Let's use \type{scipy} to calculate some
statistics.

```python from scipy import stats

data = \letteropenbrace{} \quote{A}: \letteropenbrace{} \quote{views}:
42, \quote{signups}: 2 \letterclosebrace{}, \quote{B}:
\letteropenbrace{} \quote{views}: 85, \quote{signups}: 11
\letterclosebrace{} \letterclosebrace{}

posteriors = \letteropenbrace{} variation:
stats.beta(logs{[}\quote{signups}{]}, logs{[}\quote{views}{]} -
logs{[}\quote{signups}{]}) for variation, logs in data.items()
\letterclosebrace{} ``` Calculate expected sign-up rate
$\text{E}[p_A~|~X]$:

\starttyping
posteriors['A'].mean()
\stoptyping

Get \$ \text{E}{[}p\letterunderscore{}A\low{\letterbar{}}X{]} =
5.81\%\$, \$\text{E}{[}p\letterunderscore{}B\low{\letterbar{}}X{]} =
13.37\% \$.

Calculate 95\%-credible intervals:

\type{python lower = posteriors['A'].ppf(0.025) upper = posteriors['A'].ppf(0.975)}
\$ \text{P}(1.00\% \letterless{} p\letterunderscore{}A \letterless{}
14.41\%) = 0.95 \$, \$\text{P}(7.07\% \letterless{}
p\letterunderscore{}B \letterless{} 21.28\%) = 0.95 \$

 Monte Carlo approach to compute
$P(p_B > p_A~|~X) \approx \frac{1}{n}\sum_i \blackboard{I}[y_A^{i} > y_B^{i}]$
and expected lift: ```python import numpy as np

sample\letterunderscore{}size = 10000 samples = \letteropenbrace{}
variation: posterior.sample(sample\letterunderscore{}size) \crlf
 for variation, posterior in posteriors.items() \letterclosebrace{}

dominance = np.mean(samples{[}\quote{B}{]} \lettermore{}
samples{[}\quote{A}{]}) lift = np.mean((samples{[}\quote{B}{]} -
samples{[}\quote{A}{]}) \crlf
 / samples{[}\quote{A}{]}) ```

Variation B performs better, so $P(p_B > p_A) = 92.90\%$. Expected lift
of sign-up rate under variation B is $+271.68\%$.

\type{trials} is a tiny library that does all of the above and a little
bit more.

\starttyping
from trials import Trials

test = Trials(['A', 'B'], vtype='bernoulli')

test.update({
    'A': (2, 40),
    'B': (11, 79),
})
\stoptyping

Statistics supported by \type{trials} for Bernoulli experiments:
expected posterior, posterior CI, expected lift, lift CI, empirical
lift, dominance.

$\text{P}(p_A > p_B ~|~ X)$:
\type{python dominances = test.evaluate('dominance')}

Expected lift $\text{E}(\frac{p_B - p_A}{p_A} ~|~ X)$:
\type{python lifts = test.evaluate('expected lift')}

Lift 95\%-credible interval:
\type{python intervals = test.evaluate('lift CI', level=95)}

\subsection[conclusion]{Conclusion}

We showed how to do A/B testing the Bayesian way using sign-up rate as
an example metric. The technique is flexible enough to use any kind of
interesting metrics by changing the prior and posterior distributions
appropriately. Log-normal posterior could be used to evaluate variations
based on time users spent on a page, Poisson posterior for number of
clicks users made, etc. One could choose \quotation{nice} conjugate
priors and posteriors, but, essentially, any distributions can be used
at the cost of MCMC computation. The Bayesian approach is more general
than the usual hypothesis testing approach, doesn't rely on large
numbers in theory, and produces results that are easier to interpret.


\stoptext
