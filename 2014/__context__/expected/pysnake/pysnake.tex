\usemodule[pycon-2014]
\starttext


\section[pysnake-dojo---kuba-wasielak-grzegorz-nosek]{PySnake Dojo -
Kuba Wasielak, Grzegorz Nosek}

Zadaniem uczestników jest napisanie programu, który pokieruje wężem w
zaproponowanej przez nas grze przypominającej tradycyjnego snake'a, lecz
z dwoma wężami na planszy. Zadaniem węża, oprócz tradycyjnego zjadania
kropek, będzie doprowadzenie do takiej sytuacji, w której wąż
przeciwnika uderzy w ścianę albo w ogon któregokolwiek z węży.
Początkowym zadaniem uczestników będzie stworzenie takich algorytmów
poruszania wężami, w których uda im się pokonać przygotowanych przez
organizatorów przeciwników. Później natomiast, wszyscy uczestnicy wezmą
udział w wielkim turnieju, gdzie ich węże zmierzą się ze sobą parami,
aby wyłonić najsilniejszego spośród nich.

Jako wstęp do warsztatów zapoznamy uczestników z podstawami algorytmów
wyszukiwania ścieżek z wyróżnieniem algorytmu A*.

Algorytm A* operuje na grafie węzłów, w którym każdy posiada określoną
wartość heurystyki, a dodatkowo posiadamy dwa wyróżnione węzły - węzeł
początkowy i końcowy. Heurystyka oznacza szacunkową wartość pomiędzy
danym węzłem, a naszym wyznaczonym celem. Dla przykładu, stosując
algorytm A* w wyszukiwaniu optymalnej trasy pomiędzy dwoma rzeczywistymi
mapami heurystyką dla każdego miasta jest geograficzna odległość
pomiędzy nim samym, a celem. Tak więc odległość heurystyki dla Gdyni,
podczas gdy chcemy się dostać do Sztokholmu wynosi 536 km. Heurystyka
jednak nie uwzględnia faktu, że auto nie pojedzie przez Bałtyk i
prawdziwa odległość może być znacznie większa.

W przypadku snake, zarówno graf jak i heurystyka zostają zredukowane do
płaszczyzny punktów, z których każdy posiada odpowiednik liczbowy na osi
x i osi y. Gdyby zlikwidować własny ogon i całego węża przeciwnika,
algorytm znajdywania jedzenia można by sprowadzić do:

\starttyping
    if your.x < food.x:
        go_up()
    elif your.x > food.x:
        go_down()
    elif your.y < food.y:
        go_right()
    elif your.y > food.y:
        go_left()
    else:
        munch_munch()
\stoptyping

Jednak nasz graf posiada przeszkody, miejsca zabronione i niedostępne.
Wracając więc do algorytmu A* : dla każdego punktu nasz wąż ma co
najwyżej trzy możliwe punkty wyboru drogi. Jeżeli liczba możliwości
wynosi 0, zwiastuje to szybki koniec węża. Początkiem startowym jest
głowa, natomiast końcowym jedzenie. Dla każdego punktu planszy można w
bardzo prosty sposób obliczyć heurystykę, a najlepiej w przestrzeni
dwuwymiarowej sprawdzi się do tego twierdzenie Pitagorasa. Znając
odległości punktu i jedzenia na obu osiach, x i y nie jest problemem
obliczenie odległości pomiędzy tymi dwoma węzłami. Jeżeli więc jedzenie
znajduje się w punkcie (22, 36), wartość heurystyki dla punktu (10, 12)
będzie wynosić pierwiastek z (22-10 + 36-12), czyli 6. Sposób liczenia
heurystyki jest umowny - ważne, aby był stały dla każdego punktu. Czyli
zamiast faktycznej odległości liczonej za pomocą twierdzenia Pitagorasa,
równie dobrze możemy liczyć sumę odległości liczonej po obu osiach,
która dla powyższego przykładu będzie wynosić 36.

Tak więc zaczynając od głowy węża, dla każdego z osiągalnych punktów
liczymy odległość potrzebną do pokonania tego dystansu powiększoną o
wartość heurystyki danego punktu. Spośród wszystkich punktów znajdujemy
ten, dla którego obliczony koszt jest najniższy. Następnie do listy
osiągalnych punktów dodajemy wszystkie punkty, które są osiągalne z tego
właśnie dodanego. Wracając do powyższego przykładu, w którym głowa
znajduje się w punkcie (10, 12), a jedzenie w punkcie (22, 36):

\startitemize[n][stopper=.]
\item
  Osiągalne punkty wraz z odległościami od głowy węża g():
\stopitemize

\startitemize
\item
  A (11, 12) i g(A) = 1
\item
  B (10, 11) i g(B) = 1
\item
  C (9, 12) i g(C) = 1 są to tylko 3 punkty, ponieważ wąż oprócz głowy
  ma ogon i nie może zawrócić w miejscu.
\stopitemize

\startitemize[n][start=2,stopper=.]
\item
  Dla każdego z punktów wartość heurystyki wynosi:
\stopitemize

\startitemize
\item
  h(A) = 5.92
\item
  h(B) = 6.08
\item
  h(C) = 6.08
\stopitemize

\startitemize[n][start=3,stopper=.]
\item
  Wybrany zostaje punkt A, dla którego suma odległości i heurystyki
  wynosi 6.92, jednocześnie otwierając drogę na nowe punkty:
\stopitemize

\startitemize
\item
  D (11, 13) i g(D) = 2
\item
  E (12, 12) i g(E) = 2
\item
  F (11, 11) i g(F) = 2
\stopitemize

\startitemize[n][start=4,stopper=.]
\item
  Spośród wszystkich dostępnych punktów suma g() i h() wynosi:
\stopitemize

\startitemize
\item
  g(A) + h(A) = 7.08
\item
  g(B) + h(B) = 7.08
\item
  g(D) + h(D) = 7.83
\item
  g(E) + h(E) = 7.83
\item
  g(F) + h(F) = 8
\stopitemize

Wybieramy więc dowolny z punktów A lub B oraz szukamy ich sąsiedztwa
wraz z nową odległością potrzebną do dotarcia do tego punktu wynoszącą
3. Jeżeli jeden z nowych punktów znajduje się już w liście osiągalnych
punktów, możemy go zignorować, ponieważ oznaczałoby to, że nasz wąż
zacząłby robić pętlę. Kontynuujemy algorytm tak długo, aż jednym z
punktów osiągalnych będzie cel. Wtedy znamy już trasę dla naszego węża.

W naszej grze jednak planszą są same węże, więc ulega ona zmianie
podczas każdej tury. Pozostawia to pole do popisu dla uczestników przy
programowaniu. Można postawić kilka pytań, np. jak powinien się zachować
nasz wąż w sytuacji, gdy jest całkowicie odgrodzony od pożywienia? Albo
czy można napisać algorytm, który zamiast poszukiwać jedzenia będzie
starał się przeciąć drogę przeciwnikowi albo uwięzić go na zamkniętym
obszarze planszy.

Same warsztaty będą podzielone na trzy części:

\startitemize[n][stopper=.]
\item
  Powitanie, wstęp teoretyczny, wybranie zespołów (ok. 30 minut)
\item
  Pracę nad własnym algorytmem (ok. 2 godziny).
\item
  Rozgrywki oraz wyłonienie zwycięzcy (ok. 30 minut)
\stopitemize


\stoptext
