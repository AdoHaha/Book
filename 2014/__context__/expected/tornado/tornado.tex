\usemodule[pycon-2014]
\starttext


\section[tornado-and-friends-workshop---wesley-mason]{Tornado and
Friends workshop - Wesley Mason}

\subsection[learn-how-to-write-fast-asynchronous-web-services-with-tornado-and-python]{Learn
how to write fast asynchronous web services with Tornado and Python}

The world isn't blocking; why should your app be?

Asynchronous - {\em non-blocking} - I/O is a form of input/output
processing that permits other processing to continue before the
transmission has finished.

It's a technique that can help bypass bottlenecks in program operation,
and vastly improve performance of software.

Tornado is an asynchronous networking library that can handle tens of
thousands of open connections, and can readily be deployed alongside
Django and other Python applications.

\subsection[workshop-goals]{Workshop goals}

\startitemize
\item
  Create your first Tornado app
\item
  Integrate it with thirdparty libraries
\item
  Decide where and when to use Tornado and its components
\item
  Make a kitten smile ({\em subject to availability of kitten})
\stopitemize

\subsection[prerequisites]{Prerequisites}

\startitemize
\item
  A laptop (or desktop, but I'm not helping carry it.)
\item
  Working knowledge of Python (somewhere just above {\bf Hello World!})
\item
  Conversational English, or someone to translate, as I suck at Polskie.
\stopitemize

\subsection[programme]{Programme}

All the workshop material is available via
\useURL[url1][http://tornadoandfriends.org/][][www.tornadoandfriends.org]\from[url1].

\startitemize[n][stopper=.]
\item
  Introduction to Tornado
\item
  Getting started (first app)
\item
  Making RESTful services
\item
  Making your app asynchronous
\item
  WebSockets!
\item
  Templates
\item
  Coroutines and friends
\item
  Tips and Tricks
\stopitemize


\stoptext
