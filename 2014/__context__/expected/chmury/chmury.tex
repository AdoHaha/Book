\usemodule[pycon-2014]
\starttext


\section[python-w-chmurach---dariusz-aniszewski]{Python w Chmurach -
Dariusz Aniszewski}

Wyobraźmy sobię osobę związaną z branżą IT, która nie spotkała się
jeszcze z terminem {\em chmura obliczeniowa}. Nie da się. Słowa {\em as
a Service} robią ogromną karierę i cieszą się rosnącą popularnością. Z
chmur korzysta coraz więcej użytkowników, od małych developerów po
wielkie korporacje. Jeśli ktoś myśli, że chmury obliczeniowe nie są dla
niego, najprawdopodobniej jest w błędzie. Jeśli ktoś uważa, że nie używa
ich na co dzień, również najprawdopodobniej nie ma racji. Serwisy takie
jak Facebook albo Gmail to typowe usługi chmurowe. Dropbox i Dysk Google
to niemal książkowe przykłady chmury oferującej usługę przechowywania
danych. Przykłady można mnożyć. Skupmy się jednak na tym, jak
wykorzystać {\em cloud computing} do rozwijania własnych Pythonowych
aplikacji. Na początek jednak krótka nota historyczna.

\subsection[krótka-historia]{Krótka historia}

Mogłoby się wydawać,~że chmury obliczeniowe to wynalazek ostatnich kilku
lat. A jednak\ldots{} Ich idea sięga końca lat 60-tych ubiegłego wieku i
zakłada istnienie rozproszonych, skalowalnych systemów, dostępnych z
dowolnego miejsca na świecie. Przez lata jednak niewiele się działo w
tym kierunku, przynajmniej w sferze rozwiązań dostępnych publicznie. W
1999 roku firma Salesforce zaczęła dostarczać swoje produkty przez
przeglądarkę - był to kamień milowy i prekursor do modelu {\em Software
as a Service}. Następny kamień położył Amazon, startując w 2002 roku z
Amazon Web Services, a 4 lata później udostępniając usługę {\em Elastic
Cloud Computing} (EC2) będącą udzieleniem dostępu do maszyny wirtualnej
pracującej w chmurze. Od tego dnia Internet nie był już taki sam. W 2009
roku do wyścigu dołączył Google, a chwilę później Microsoft. Następnie
nastąpił boom na różne usługi dostępne w modelu chmury obliczeniowej i
trwa on po dzień dzisiejszy.

\subsection[chmura-obliczeniowa]{Chmura obliczeniowa}

Czym więc jest w obecnym czasie chmura? Najprościej - usługą. Usługą
przechowywania plików, albo usługą serwisu bazodanowego. Może to też być
usługa udostępnienia maszyny wirtualnej, albo przeprowadzania obliczeń.
Może to być jakakolwiek usługa, możliwa do dostarczenia przez Internet,
za jaką użytkownicy będą skłonni zapłacić. Obecnie chmury możemy
podzielić na trzy typy:

\startitemize
\item
  Infrastructure as a Service (IaaS) - w tym modelu usługą jest
  najczęściej maszyna wirtualna
\item
  Platform as a Service (PaaS) - dostęp do platformy, na której możemy
  uruchamiać aplikacje
\item
  Software as a Service (SaaS) - oprogramowanie dostępne przez
  przeglądarki
\stopitemize

Idąc po kolei, w modelu {\em IaaS} klient dostaje dostęp do pewnej
części infrastruktury teleinformatycznej. Najczęściej jest to po prostu
maszyna wirtualna, ale mogą to być też np. serwery SMTP, czy baza
danych. Konfiguracja i instalacja oprogramowania spoczywa na kliencie.
Model {\em PaaS} jest rozwinięciem infrastruktury, w której operator
dostarcza już pewne mechanizmy odciążające użytkowników z instalacji
oprogramowania, czy konfiguracji serwerów. Klient musi jedynie wgrać
swoją aplikację oraz wskazać plaftormie sposób jej uruchomienia.
{\em SaaS} jest natomiast niczym innym jak oprogramowaniem dostępnym
przez przeglądarkę. Oprogramowanie tego typu jest bardzo zróżnicowane -
mogą to być pakiety biurowe, programy graficzne, księgowe i wiele
innych.

Koszty usług chmurowych są bardzo zróżnicowane i każdy operator ma swój
cennik za poszczególne produkty. Panuje jednak jedna zasada - klient
jest rozliczany z używanych zasobów. Dla przykładu jeśli trzeba
przeprowadzić skomplikowane obliczenia, można utworzyć kilka maszyn
wirtualnych, połączyć je w macierz i po zakończeniu obliczeń je
wyłączyć. Koszty będą generowane tylko podczas pracy tych maszyn. Jest
to ogromna zaleta chmur, bez których konieczny byłby zakup fizycznych
komputerów, które byłyby wykorzystywane sporadycznie, a koszt ich zakupu
i konserwacji znacząco przewyższałby koszty takiego samego zestawu w
chmurze.

\subsection[przygotowania]{Przygotowania}

Skoro wiemy już co nieco o chmurach, czas je wykorzystać w praktyce. W
chmurze możemy serwować wszelkie aplikacje Pythonowe posiadające
wywołania po protokole HTTP. Możemy zatem użyć dowolnego frameworka do
aplikacji internetowych, np. Django, czy też Flask. Możemy również sami
zaimplementować nasz serwer używając np. modułu {\em SimpleHTTPServer} z
biblioteki standardowej. Dla dalszych rozważań przyjmijmy, że nasza
aplikacja będzie korzystała z Django. Ta aplikacja będzie korzystała z
relacyjnej bazy danych PostgreSQL (choć może być też np. MySQL) i będzie
umożliwiała użytkownikom wgrywanie plików do 30MB. Komunikację będziemy
prowadzili po protokole HTTPS. Mamy wykupioną własną domenę oraz
certyfikat SSL. Spodziewamy się umiarkowanego ruchu, który z czasem
będzie wzrastał. Dobrze i co dalej?

Musimy teraz wybrać typ chmury, z jakiej chcemy korzystać, oraz jej
dostawcę. Dla naszej aplikacji rozważmy wady i zalety modeli IaaS oraz
PaaS. Jako reprezentantów obu modeli przyjmijmy: Amazon EC2 dla
infrastruktury oraz Heroku dla platformy. Sprawdźmy najpierw, jak
dokładnie działają obydwa te serwisy.

\section[amazon-ec2]{Amazon EC2}

Usługa Elastic Cloud Computing jest niczym innym, jak udostępnianiem
wirtualnych maszyn. Możemy zatem utworzyć maszynę posiadającą określoną
moc obliczeniową, czy dysk twardy. Dodatkowo wybieramy system
operacyjny, jaki ma na niej zostać zainstalowany oraz generujemy parę
kluczy RSA, potrzebnych do połączenia się z maszyną przez SSH. Jeśli
chcemy wykorzystać swój zestaw kluczy, możemy podać swój klucz publiczny
zamiast generować nową parę. W obecnej chwili możemy wybierać z systemów
UNIXowych, np. Ubuntu, Debian, CentOS oraz Microsoft Server. Każda
maszyna przy uruchomieniu ma przyznawany publiczny adres IP, można też
zarezerwować stały adres IP i przypiąć go do danej maszyny, wtedy pomimo
restartów publiczny adres IP nie ulegnie zmianie. Po utworzeniu maszyny
i przyznaniu jej stałego adresu musimy przystąpić do konfiguracji, czyli
m.in.:

\startitemize
\item
  otworzyć porty 22 (SSH), 80 (HTTP), 443 (HTTPS), ewentualnie inne
  wedle potrzeby
\item
  zainstalować wymagane komponenty systemowe
\item
  zainstalować i skonfigurować niezbędne serwisy
\item
  zainstalować wymagane paczki Pythonowe
\item
  wgrać, skonfigurować i uruchomić naszą aplikację i wszystkie jej
  komponenty
\stopitemize

Jak widać, na uruchomienie aplikacji na EC2 należy poświęcić nieco czasu
i pracy, jednak w zamian mamy pełną kontrolę nad tym, co się właściwie
dzieje. Jedyne ograniczenia wynikają głównie z możliwości utworzonej
maszyny.

\section[heroku]{Heroku}

Heroku jest platformą uruchomieniową dla aplikacji napisanych w różnych
językach - Ruby, Python, Java, NodeJS i innych. Heroku do swojego
działania wykorzystuje wirtualizowane kontenery UNIXowe, marketingowo
zwane {\em Dyno}. Dla uproszczenia można przyjąć, że jedno {\em Dyno} to
odpowiednik jednej maszyny, choć w praktyce na jednej wirtualnej
maszynie może istnieć wiele wyizolowanych kontenerów. Aby uruchomić
aplikację na Heroku, należy poinformować platformę, jak należy to
zrobić. W przypadku Pythona należy wykonać dwa kroki:

\startitemize[n][stopper=.]
\item
  dostarczyć plik {\bf requirements.txt} z wylistowanymi paczkami
  Pythonowymi, jakie muszą zostać zainstalowane
\item
  dostarczyć plik {\bf Procfile}, w którym podamy komendę uruchamiającą
  aplikację, np. \type{web: python manage.py runserver 0.0.0.0:$PORT}
\stopitemize

Aby plafroma Heroku mogła sprawnie i bezobsługowo działać, narzucone są
pewne ograniczenia:

\startitemize
\item
  ulotny system plików - jeśli w trakcie działania naszej aplikacji coś
  zostanie zapisane na dysk, w momencie jej restartu będzie bezpowrotnie
  utracone. Jest to bardzo ważne ograniczenie, które należy mieć w
  świadomości już na etapie projektowania aplikacji.
\item
  30 sekundowy timeout - jeśli nasza aplikacja nie wyrobi się w ciągu 30
  sekund z rozpoczęciem wysyłania odpowiedzi, Dyno Manager zwróci status
  503. Należy mieć na uwadze, że pomimo wysłania statusu błędu, Dyno
  będzie ciągle przetwarzać zapytanie, w tym przeprowadzać rozpoczęte
  operacje na bazie danych, co może prowadzić do niespójności.
\item
  512MB pamięci operacyjnej - jeśli aplikacja potrzebuje dużej ilości
  pamięci operacyjnej, Heroku nie jest najlepszym wyborem - po
  przekroczeniu 512MB Heroku zacznie używać pliku wymiany (SWAP), co
  drastycznie wpłynie na wydajność.
\item
  usypianie Dyno - jeśli nasza aplikacja wykorzystuje tylko jedno Dyno i
  w ciągu godziny nie było wykonane żadne zapytanie, Dyno zostaje
  uśpione. Uśpione Dyno nie pracuje, więc jeśli zostanie wysłane do
  niego zapytanie to Dyno Manager musi je najpierw wybudzić, co
  powoduje, że pierwsze zapytania mogą mieć dłuższy czas oczekiwania na
  odpowiedź.
\stopitemize

Niewątpliwym plusem Heroku jest natomiast minimalna ilość pracy, jaką
należy wykonać, aby uruchomić aplikację. Dodatkowo Heroku oferuje bardzo
dużo dodatków, np.:

\startitemize
\item
  bazy danych: relacyjne i NoSQL
\item
  monitoring
\item
  analiza wydajności
\item
  agregowanie i przeszukiwanie logów
\item
  i wiele innych
\stopitemize

Te dodatki są de facto powiązanymi usługami chmurowymi, którymi możemy
rozwijać naszą aplikację.

Aplikacje na Heroku mają swoje własne domeny w postaci
\type{<nazwa-aplikacji>.herokuapp.com} działające zarówno po HTTP jak i
HTTPS. Dostajemy również repozytorium Git, służące do wgrywania
kolejnych wersji.

\subsection[wybór-chmury]{Wybór chmury}

Skoro wiemy już czym charakteryzują się EC2 i Heroku, nadszedł czas na
wybranie najbardziej pasującej usługi. W tym celu dokonajmy konfrontacji
wymagań naszej aplikacji z możliwościami serwisów.

\section[wsparcie-dla-django]{Wsparcie dla Django}

Zarówno Heroku jak i EC2 mają wsparcie dla aplikacji korzystających z
frameworka Django. Przy EC2 należy poświęcić więcej czasu na instalację
i konfigurację systemu niż ma to miejsce na Heroku, gdzie musimy tylko
dostarczyć plik requirements.txt (który prawdopodobnie i tak mamy) i
Procfile. Na EC2 sami musimy zainstalować paczki Pythonowe wraz z
wymaganymi pakietami systemowymi.

\section[baza-danych-postgresql]{Baza danych PostgreSQL}

W kwestii bazy danych sytuacja jest bardziej zawiła, ale oba serwisy
umożliwiają nam jej obsługę. Wybierając ofertę Amazona mamy do wyboru
dwie opcje. Pierwszą jest instalacja serwera bazodanowego na tej samej
maszynie, na której działa nasza aplikacja. Drugą opcją jest
skorzystanie z dedykowanej bazom danych usługi {\em Amazon Relational
Database Service} - {\bf RDS}. Wybór wariantu nie jest prosty - w
przypadku instalacji na tej samej maszynie będziemy mieli problemy ze
skalowalnością, ale korzystając z RDS generujemy dodatkowe koszta w
przypadku używania tylko jednej instancji EC2 dla obsługi naszej
aplikacji.

Na Heroku sprawa jest rozwiązana za nas. Baza danych dostępna jest jako
dodatek, a my musimy jedynie wybrać plan cenowy. Na początek wystarczy
plan darmowy, ale posiada on limit 10 000 wierszy. Po przekroczeniu tej
liczby baza przestanie dodawać nowe rekordy i będziemy zmuszeni do
zmiany planu na płatny.

\section[wgrywanie-plików-do-30-mb]{Wgrywanie plików do 30 MB}

To jest kluczowy punkt w specyfikacji wymagań. Umożliwienie wgrania
plików przez użytkowników oznacza, że musimy je gdzieś przechowywać, a
ich rozmiar będzie rósł wraz z liczbą użytkowników. Jeśli pamiętamy
ograniczenia zarówno EC2 jak i Heroku, dochodzimy do wniosku, że żadne z
nich nam nie wystarczy. Jeśli wybierzemy EC2 to prędzej czy później
wyczerpie nam się miejsce na dysku. W przypadku Heroku system plików
jest ulotny, więc wgrane pliki są nietrwałe. Rozwiązaniem obydwu
problemów będzie skorzystanie z kolejnej usługi - {\em Amazon Secure
Storage Service} (Amazon {\bf S3}). Ta usługa jest również polecana
przez Heroku do przechowywania plików. S3 jest, jak nazwa mówi, usługą
przechowywania plików, w której miejsce jest nieograniczone. Rozliczani
jesteśmy z ilości danych, jakie przechowujemy oraz transferu. Do
wgrywania plików na S3 możemy użyć modułu \type{boto}.

\section[obsługa-https-na-własnej-domenie]{Obsługa HTTPS na
własnej domenie}

Obydwa serwisy umożliwiają bezpieczną komunikację. Na EC2 należy wgrać
certyfikaty na serwer WWW i odpowiednio go skonfigurować. Niestety na
Heroku, chcąc korzystać z HTTPS z własną domeną, poza wgraniem
certyfikatów jesteśmy zmuszeni do dokupienia dodatku {\em SSL Endpoint}
w cenie 20\$/miesiąc.

\section[skalowalność]{Skalowalność}

Skalowalność jest bardzo ważnym aspektem aplikacji, obydwa serwisy
oferują wsparcie. Tak samo, jak w przypadku innych wymagań, zapewnienie
skalowalności na EC2 jest bardziej złożone i wymaga dodania maszyny typu
{\em Load Balancer} oraz utworzenia żądanej liczby maszyn z pracującą
aplikacją. Na Heroku można zmienić liczbę używanych Dynos korzystając z
przeglądarki i ustawiając żądaną liczbę w konfiguracji aplikacji.

\section[konkluzja]{Konkluzja}

Obydwa serwisy jak najbardziej nadają się do serwowania aplikacji
Pythonowych, a ich różnice wynikają w dużej mierze z wyboru modelu, w
jakich pracują. W ogólnym rozliczeniu EC2 w modelu IaaS będzie wypadało
korzystniej pod kątem miesięcznych opłat, lecz będzie wymagało większych
nakładów na konfigurację i utrzymanie systemu. Heroku jest wygodniejsze
w obsłudze, jednak w środowisku produkcyjnym zapewne będzie droższe od
EC2.

\subsection[podsumowanie]{Podsumowanie}

Jak widać wybór serwisu chmurowego dla naszej aplikacji nie jest sprawą
prostą i wymaga bardzo indywidualnego podejścia. Pod uwagę należy wziąć
zarówno obecne wymagania funkcjonalne i pozafunkcjonalne oprogramowania,
perspektywy rozwoju oraz budżet, jakim dysponujemy.

\section[źródła]{Źródła}

\startitemize
\item
  http://www.citeworld.com/article/2114518/cloud-computing/saas-top-50-list.html
  - artykuł wymieniający serwisy chmurowe używane na co dzień.
\item
  http://www.computerweekly.com/feature/A-history-of-cloud-computing -
  historia chmur obliczeniowych
\item
  https://aws.amazon.com/marketplace/search?page=1\&category=2649367011
  - dostępne systemy na EC2
\item
  https://devcenter.heroku.com/categories/language-support - języki
  programowania wspierane na Heroku
\item
  https://devcenter.heroku.com/articles/dynos - opis kontenerów Dyno
\item
  https://addons.heroku.com/ - dodatki do Heroku
\item
  Wykorzystanie technologii chmury obliczeniowej do zwiększenia zasięgu
  oprogramowania desktopowego - praca dyplomowa magisterska, autor mgr
  inż. Dariusz Aniszewski, promotor dr inż. Piotr Helt
\stopitemize


\stoptext
