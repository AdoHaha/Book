\usemodule[pycon-2014]
\starttext
\Title{{\tt \char`\_\char`\_init\char`\_\char`\_}}
\HeaderTitle{}
\gobbleoneargument \MakeTitlePage
\CheckPingPages


% Strona pierwsza, czyli okładka znaczy

\null
\noheaderandfooterlines
\vskip 0.2\vsize
\centerline{%
  \tfd PyCon PL 2014
  %\externalfigure[ping-ascii.100][width=0.7\hsize]
}


\vskip 0pt plus 1filll

\centerline{Polska Grupa Użytkowników Linuxa}
\break
\noheaderandfooterlines

\def \ListaPlac#1:{\blank[2*big]
     \begingroup
     \let \PAR=\par
     \def \par{\PAR\endgroup}
     \catcode`\^^M=13
     \noindent \begingroup \sc #1:%
     %\lccode `\~=`\^^M%
     \lowercase{\endgroup\def~{\hfil\break\hbox
     to2em{}\ignorespaces
}}
}%


\ListaPlac Komitet Organizacyjny:
 Filip Kłębczyk (przewodniczący)
 Sebastian Pawluś
 Piotr Tynecki
 Justyna Żarna
 Kamil Gałuszka
 Łukasz Porębski
 Piotr Skamruk
 Michał Węgrzynek

\ListaPlac Komitet Programowy:
 Sebastian Pawluś (przewodniczący)
 Marcin Baran
 Michał Chałupczak
 Konrad Hałas
 Marcin Jabrzyk
 Jakub Janoszek
 Piotr Kasprzyk
 Przemysław Kukulski
 Grzegorz Nosek
 Dawid Romaldowski
 Patrycja Waleśko
 Jakub Wiśniowski
 Maciej Wiśniowski

\ListaPlac Redakcja:
 Sebastian Pawluś
 Jakub Janoszek
 Piotr Kasprzyk
 % Ja nie robię, ja tylko makra --K.L.
 % Krzysztof Leszczyński
 Filip Kłębczyk
 Maciej Wiśniowski
 Piotr Wójcik

%\blank[2*big]
\vfill \break
\ListaPlac Platynowy Sponsor:~\par%\blank[2*big]
\centerline{\externalfigure[src/webinterpret-logo.pdf][width=0.6\hsize]}

\ListaPlac Złoci Sponsorzy:~\par%\blank[2*big]
\centerline{\externalfigure[src/Vector_B+W_MirantisLogo.pdf][width=0.45\hsize]\hskip10mm\externalfigure[src/intel_blk_100.pdf][width=0.45\hsize]}
\centerline{\externalfigure[src/logo-applause-gray.png][width=0.45\hsize]\hskip10mm\externalfigure[src/daftcode_logo.pdf][width=0.45\hsize]}
\centerline{\externalfigure[src/logo_wp_bw.png][width=0.45\hsize]\hskip10mm\externalfigure[src/mslogo_b.png][width=0.45\hsize]}

\ListaPlac Srebrni Sponsorzy:\par
\vskip12mm
\centerline{\externalfigure[src/compendium.png][width=0.4\hsize]\hskip10mm\externalfigure[src/Samsung_bw.pdf][width=0.4\hsize]}

\centerline{\externalfigure[src/transporeon_bw.pdf][width=0.4\hsize]\hskip10mm\externalfigure[src/stx-logo_gray.png][width=0.4\hsize]}
\centerline{\externalfigure[src/Codility_logo_bw.pdf][width=0.4\hsize]\hskip10mm\externalfigure[src/Red_Hat_BW_rgb.pdf][width=0.4\hsize]}
\centerline{\externalfigure[src/smt_CMYK.pdf][width=0.2\hsize]\hskip10mm\externalfigure[src/BaseLab-Black-RGB.pdf][width=0.4\hsize]}

\ListaPlac Brązowi Sponsorzy:\par
\vskip12mm
\centerline{\externalfigure[src/hurra_Logo_SW.pdf][width=0.25\hsize]\hskip20mm\externalfigure[src/ORM_logo_cmyk.pdf][width=0.4\hsize]}
\centerline{\externalfigure[src/hb-full.pdf][width=0.4\hsize]\hskip10mm\externalfigure[src/egnyte-lightbg-vector.pdf][width=0.4\hsize]}

\ListaPlac Grafitowi Sponsorzy:\par
\centerline{\externalfigure[src/helion.png][width=0.6\hsize]\hskip10mm\externalfigure[src/logo-raspi.pdf][width=0.2\hsize]}
\centerline{\externalfigure[src/packt-logo-med-res-transparent.png][width=0.4\hsize]\hskip10mm\externalfigure[src/two-scoops-press.png][width=0.4\hsize]}
\centerline{\externalfigure[src/FINAL_LOGO_TECHLAND-black-ver8.pdf][width=0.4\hsize]\hskip10mm\externalfigure[src/logo_kinetise.png][width=0.4\hsize]}
\break

\iffalse
  % wersja prezesa

Witajcie. Znowu.


\else

\begingroup
\setupwhitespace[big]
\setupindenting[none,never]

Witamy na siódmej edycji konferencji PyCon PL!

Podobnie jak w zeszłym roku, spotykamy się w Orlim Gnieździe w Szczyrku. Trochę niespodziewanie rozrosła się nam konferencja - zarówno o uczestników, prelegentów jak i sponsorów. Mamy jednak gorącą nadzieję, że wszystko pójdzie w miarę sprawnie, a integracja uda się jak zwykle. Jeśli chcesz w przyszłych latach pomóc kształtować tę konferencję, koniecznie skontaktuj się z organizatorami. Dziękujemy za czas i~zaangażowanie osób pomagających przy tegorocznej edycji.

Chcielibyśmy też przy okazji zwrócić szczególną uwagę na powstające stowarzyszenie Polska Grupa Użytkowników Pythona (PLPUG). Warto zaangażować się w działania grupy, która ma na celu rozwój społeczności programistów Pythona w Polsce i już niedługo zakończy się formalizować.  Zapraszamy więc gorąco na piątek wieczór do szerokiej dyskusji w tym temacie.

Przyzwyczailiśmy się już do integracyjnego grilla w sobotę, a tym razem dodatkowo proponujemy skorzystanie z okazji do rozmów podczas dłuższej, czwartkowej kolacji z ciekawą muzyką w tle. Ponadto niezmiernie cieszy nas fakt, że w tym roku będziemy mieli więcej uczestników zza granicy. Liczymy, że wyniosą oni pozytywne wrażenia z polskiego PyCona i zachęcą swoich znajomych z lokalnych społeczności do odwiedzenia naszej konferencji w przyszłych latach.

Ta książka, którą trzymasz w rękach, powstała wspólnym wysiłkiem za pośrednictwem publicznego repozytorium  na Githubie, gdzie każdy może wnieść swój wkład w treść i formę zamieszczonych tam materiałów. Jeśli podoba Ci się ta idea, to zapraszamy do współpracy w przyszłym roku:
https://github.com/PyConPL/Book

Na koniec mamy nadzieję, że również Ty zaliczysz swój pobyt na PyConie PL 2014 do tych udanych, a piękna okolica Beskidu Śląskiego będzie dla Ciebie inspirująca i~pozwoli podładować baterie.


\endgroup

\fi

\vfil\break
\begingroup

\setupwhitespace[big]

\null\vfill
\noindent{\sc Kolofon:}

\blank[big]\parindent=0pt \rightskip=0pt
plus 1fill

Wszystkie prace związane z~przygotowaniem publikacji do~druku
zostały wykonane wyłącznie w~systemie Linux (m.in. Ubuntu 12.04.1 LTS).

Skład został wykonany w systemie \TeX\ z~wykorzystaniem,
opracowanego na~potrzeby konferencji, środowiska redakcyjnego
opartego na~formacie \CONTEXT.

Teksty referatów złożono krojem Bonum z kolekcji \TeX-Gyre.
Tytuły referatów złożono krojem Antykwa Toruńska, opracowanym
przez Janusza Nowackiego na~podstawie rysunków Zygfryda
Gardzielewskiego. 


\endgroup
\break


\iffalse
\def \defEvent{\dodoubleargumentwithset \dodefEvent}
\def \dodefEvent[#1][#2]{\getparameters[Event::#1::][#2]}

\def \vskipreserve #1#2{\vskip #1\nobreak 
                        \vskip 0pt plus \dimexpr #2\relax\penalty-1000
                        \vskip 0pt plus -\dimexpr #2\relax}

\def \startDay#1{\vskipreserve{3\baselineskip}{0.3\vsize}
     \centerline{\tfa #1}
     \blank[big]
}

\def \AgendaEvent[#1]{\par
     \def \AgendaEventfrom{\errmessage{Undefined AgendaEvent from}}
     \def \AgendaEventto{\errmessage{Undefined AgendaEvent to}}
     \def \AgendaEventid{\errmessage{Undefined AgendaEvent id}}
     \getparameters[AgendaEvent][#1]%
     \begingroup\edef \a{\endgroup
       \noexpand\doAgendaEvent
       \AgendaEventid \space \AgendaEventfrom \space \AgendaEventto\relax}\a
     }

\def \numifty#1{\ifnum 99=0#1\relax\infty \else #1\fi}

\setvalue{start::ref}{}
\setvalue{stop::ref}{}
\setvalue{start::event}{\qquad --- \vrule width 0pt height 5ex depth 0ex\relax}
\setvalue{stop::event}{ ---\vrule width 0pt height 0ex depth 3ex\relax}


\def \doAgendaEvent #1 #2:#3 #4:#5\relax{
     \hbox to \hsize \bgroup
     \hbox to 6em{\hfil
       $\numifty{#2}^{\numifty{#3}}$~--~%
       $\numifty{#4}^{\numifty{#5}}$\quad}%
     \vtop \bgroup
       \advance \hsize -9em
       \noindent \rightskip=0pt plus 1fil
       \vrule width 0pt height 3ex
       \if $\getvalue{Event::#1::author}$\else
          {\sc \getvalue{Event::#1::author}}\hfil\break
       \fi
       \getvalue{start::\getvalue{Event::#1::type}}
      {\it \getvalue{Event::#1::title}}%
       \getvalue{stop::\getvalue{Event::#1::type}}
      \ifnum 0=0\ArtPage{#1::firstpage}\relax
      \else 
      \parfillskip=0pt \dotfill\null
      \rlap{\tt \inhex\ArtPage{#1::firstpage}}
      \setgvalue{Art::#1::found}{1}
      \fi\par
     \egroup\hfil\egroup}
\def \stopDay{}
             

\catcode `\_=13
\def_{{\tt \char `\_}}
% \input agenda

\iffalse
\startDay{pozostałe artykuły}
      \setgvalue{Art::ping2008::found}{1}

\def \Article#1#2{\ifnum1=0\getvalue{Art::#1::found}\relax
     \else #1\par\doAgendaEvent #1 ~:~ ~:~\relax\fi}

\PingArticles

\fi
\fi
%\xxxxxxxxxxxxxxxxx


\centerline{\bf Spis treści}
\blank[2*big]
\toks0={}

\newread \testarticle
\def \Article#1#2{\message{**ARTICLE: #1 ***}%
     \toks0=\expandafter{\the\toks0\relax\Article{#1}}
     \toks1={}% Authors
     \toks2={}% Title
     \openin \testarticle=#1/#1.tex\relax
     \ifeof \testarticle
        \closein \testarticle
     \else
        \closein \testarticle
        \expandafter \ScanArt \normalinput #1/#1.tex\relax\ScanArt
     \fi}

\def \ScanArt {\afterassignment \ScanArtA \let \next}
\def \ScanArtA{%
     \ifcase 0%
            \ifx\next \ScanArt 1\fi
            \ifx\next \Title   2\fi
            \ifx\next \Author  2\fi
            \ifx\next \MakeTitlePage 2\fi
            \relax
            % case 0, unknown token
            \expandafter \ScanArt
            \or
            % case 1, found \ScanArt, i.e. endofinput
            \expandafter \FinishScan
            \or
            % case 2
            \expandafter \next
            \else
              \errmessage{internal if/fi error}
            \fi}

\def \Title#1{\toks2=\expandafter{\the\toks2\relax #1\relax}%
     \ScanArt}
\def  \Author#1{\toks1=\expandafter{\the\toks1, #1}\ScanArt}
\def \MakeTitlePage{\endinput\ScanArt}
\def \FinishScan{\edef\next{{\the\toks1}{\the\toks2}}%
     \toks0=\expandafter{\the\toks0\next}}
\PingArticles
\def \Article#1#2#3{\bTR
     \bTD \tt 0x00\inhex \ArtPage{#1::firstpage}\relax\eTD
     \bTD #3\relax~--~\it \gobbleoneargument#2\relax\eTD
     \eTR}
\bTABLE
        \the\toks0
\eTABLE

\stoptext
