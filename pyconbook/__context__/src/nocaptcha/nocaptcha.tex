\usemodule[pycon-2013]
\starttext


\Title{NoCaptcha~-- jak zabezpieczyć formularz przed zautomatyzowanym uzupełnianiem bez
udziału użytkownika}
\HeaderTitle{NoCaptcha~-- jak zabezpieczyć formularz}
\Author{Jakub Wasielak}
\MakeTitlePage

\section[wstęp]{Wstęp}

W poniższym artykule przedstawiam alternatywne podejście do bardzo
popularnej metody ochrony przed botami -- CAPTCHA, kompletny,
zautomatyzowany i publiczny test Turinga do odróżniania ludzi od
robotów, który już od 2000 roku „broni” formularzy internetowych przed
zautomatyzowanym dostępem. Obecnie, w dobie powszechnego upraszczania
interfejsu użytkownika, CAPTCHA staje się coraz większym obciążeniem dla
projektantów stron internetowych. Jednak okazuje się, że CAPTCHA może
zostać bardzo łatwo zastąpiona poprzez kombinację takich mechanizmów,
jak: Honeypot, sprawdzanie czasu wypełniania formularza oraz metod
opartych o szyfrowanie md5. Celem tego artykułu jest zaprezentowanie i
rozpowszechnienie innych metod zabezpieczania formularzy niż CAPCHA.

\section[o-captcha]{O CAPTCHA}

\subsection[powstanie]{Powstanie}

Pierwsze wzmianki na temat zastosowania CAPTCHA pochodzą już z 1997
roku, kiedy została ona wykorzystana w wyszukiwarce AltaVista w celu
ochrony przed automatycznym dodawaniem adresów do bazy. W 1998 roku
mechanizm przypominający dzisiejsze działanie CAPTCHA zostało
opatentowane przez zespół programistów. Przez lata generatory CAPTCHA
były udoskonalane, aby możliwie jak najbardziej ułatwić użytkownikom
odczytywanie graficznie przedstawionych tekstów, a jednocześnie
maksymalnie je utrudnić mechanizmom OCR (\quotation{Optical Character
Recognition}).

\subsection[recaptcha]{ReCAPTCHA}


W roku 2007 Luis Von Ahn po raz pierwszy zaprezentował ideę reCAPTCHA.
Każda akcja użytkownika polegająca na przepisaniu tekstu z obrazka do
odpowiedniego pola była w rzeczywistości pracą nie przynoszącą realnych
korzyści. Biorąc pod uwagę jednego użytkownika, rozpoznanie tekstu nie
ma żadnej korzyści, lecz przesuwając skalę do milionów użytkowników
wypełniających zabezpieczone formularze na całym świecie, wykorzystanie
tej aktywności okazało się istotne. Pomysł Luisa Von Ahn polegał na
zastosowaniu w obrazkach CAPTCHA fragmentów rzeczywistych skanów
książek, których nie potrafiły rozpoznać skanery OCR. Standardowe pole
reCAPTCHA składa się z dwóch słów -- jednego sprawdzającego użytkownika,
oraz drugiego mającego na celu rozpoznanie niemożliwego do
automatycznego odczytania tekstu. W 2009 roku mechanizm reCAPTCHA został
wykupiony przez Google i od tamtej pory jest najbardziej popularnym
mechanizmem stosowanym do zabezpieczania formularzy.

\subsection[nowoczesne-alternatywy]{Nowoczesne alternatywy}


Wraz z rozwojem języka JavaScript oraz powstawaniem coraz liczniejszych
bibliotek ułatwiających dynamiczną obsługę treści stron internetowych
powstawało dużo rozwiązań opartych wyłącznie o grafiki oraz akcje inne,
niż przepisywanie tekstu -- np. przeciągnięcie elementu za pomocą myszy.
Największym serwisem prezentującym gotowe wtyczki jest strona
www.areyouahuman.com. Te metody, choć bezpieczne i proste, bardzo często
nie pasują do treści prezentowanej na stronie, uniemożliwiają obsługę
strony wyłącznie za pomocą klawiatury oraz najprawdopodobniej nawet nie
zostaną wyświetlone bez włączonej obsługi języka JavaScript.

\subsection[wady]{Wady}

Największą wadą CAPTCHA jest konieczność poświęcenia czasu przez
użytkowników na czynność, która nie przynosi im samym żadnej korzyści. W
czasach, gdy ogromny nacisk jest kładziony na zatrzymanie użytkownika na
swojej stronie, zastosowanie CAPTCHA może mieć duży wpływ na realne
straty finansowe. W artykule \quotation{F**k CAPTCHA} opublikowanym na
stronie www.90percentofeverything.com opisano wpływ likwidacji
standardowego mechanizmu CAPTCHA na konwersję użytkowników. Okazało się,
że współczynnik konwersji wzrósł o 33,3\%. Realnie, przez rejestrację
przechodziło o ponad 15\% więcej użytkowników, niż poprzednio. Same
obrazki bardzo często są zupełnie nieczytelne. W tym miejscu można by
zamieścić dziesiątki zrzutów zupełnie nieczytelnych obrazków CAPTCHA,
które są dostępne w internecie. Ze statystyk prowadzonych w systemie
Sympatia.pl można odczytać, że kiedy jeszcze CAPTCHA była stosowana, co
czwarta osoba była zmuszona wybrać opcję „wygeneruj ponownie”. Nie można
również zapomnieć o utrudnieniu dla osób niepełnosprawnych --
niedowidzących i niewidomych. Choć niektóre systemy posiadają opcję
automatycznego czytania tekstu, wciąż jest to niewielki odsetek, a także
wciąż duże obciążenie dla osób, które i bez tego posiadają utrudniony
dostęp do treści.

\section[nocaptcha]{NoCAPTCHA}

Poniższe rozwiązanie jest wzorowane na propozycji przedstawionej przez
Neda Batcheldera na stronie nedbatchelder.com w artykule pt.
\quotation{stopbots}. Aby poradzić sobie z likwidacją mechanizmu CAPTCHA
postanowił on wprowadzić szereg metod, które automatycznie miały
rozróżniać ludzi od robotów. Ogólna zasada dotycząca obrony przed
robotami polega nie na zastosowaniu pojedynczego, uniwersalnego
rozwiązania, lecz na kombinacji wielu metod, z których każda zablokuje
jeden konkretny rodzaj robotów. Wszystkie poniższe zabezpieczenia wiążą
się z ukrytymi polami, niewidocznymi dla użytkownika. Obecnie boty są w
stanie łatwo rozpoznać pola ukryte poprzez znacznik:

\starttyping
<input type="hidden">
\stoptyping

Są one nauczone, aby nie uzupełniać takich pól -- skoro są one
niewidoczne dla użytkownika, zapewne ich uzupełnienie nie jest
konieczne, aby przejść do kolejnego kroku formularza. Dlatego o wiele
lepiej sprawdza się ukrywanie elementu przez klasy CSS. W dużym
uproszczeniu takie pole będzie wyglądać następująco:

\starttyping
<input style="display: none;">
\stoptyping

Oczywiście powyższy napis można modyfikować jeszcze bardziej. Zamiast
stylu w kodzie HTML, można zastosować klasę zadeklarowaną w plikach CSS.
Ponadto znacznik input można zawrzeć w innych znacznikach, które to
dopiero będą ukryte.

\subsection[timestamp]{Timestamp}

Najprostsza metoda obrony przed robotami polega na dodaniu jednego
ukrytego pola do formularza. Pole to, zwane tutaj
\quotation{timestampem} zawiera aktualną datę wygenerowania formularza
zapisaną w UNIX-owym formacie timestamp (liczonym w sekundach od 1
stycznia 1970). Podczas wysyłki formularza należy porównywać obecny czas
z czasem podanym w polu timestamp. Jeżeli formularz składający się z
kilkunastu pól został uzupełniony i wysłany w np. mniej niż 5 sekund,
wprost oznacza to, że nie mógł go uzupełnić człowiek. Oczywiście ta
zapora może zostać bardzo łatwo ominięta -- wystarczy tak zaprogramować
robota, aby po wygenerowaniu formularza odczekał oczekiwaną wartość
czasu i dopiero po tym wysłał dane. Niemniej jest to najprostsza, a
zarazem najbardziej skuteczna z metod ochronnych. Większość botów nie
jest wymierzona w jeden konkretny cel -- przemierzają one Internet od
linku do linku starając się wypełnić każdy napotkany formularz. Jeżeli
posiadamy mały lub średni serwis, zapewne zastosowanie wyłącznie tej
metody ochroni nas przed robotami, lecz jeżeli prowadzimy duży serwis,
musimy być gotowi aby odeprzeć te skierowane wyłącznie w nas ataki.

\subsection[honeypot]{Honeypot}

Honeypot, czyli \quotation{lep na muchy} jest tak samo prostą metody
obrony jak timestamp. Robot uzupełniający formularz szuka pól input o
specyficznych, znanych mu nazwach, takich jak \quotation{login},
\quotation{email}, czy \quotation{password}. Aby oszukać automat, należy
pomiędzy normalnymi polami formularza zamieścić pułapki, których
uzupełnienie powodowałoby błąd podczas walidacji formularza. Formularz,
w którym trzeba podać imię i hasło mógłby wyglądać następująco:

\starttyping
<input type="text" name="name">
<input type="text" name="email" class="hidden">
<input type="password" name="password">
\stoptyping

Jeżeli w powyższym formularzu zwrócone zostałyby dane dla pola
\quotation{email}, które jest schowane przed użytkownikiem w pliku CSS,
oznaczałoby to wprost, że został on wygenerowany automatycznie, a nie
przez prawdziwego użytkownika. Oczywiście przy złożonym formularzu
zawierającym wiele różnych pół ciężko będzie znaleźć honeypoty o takich
nazwach, które jeszcze nie występują, lecz rozwiązanie tego problemu
pojawi się w dalszej części. Aby jeszcze bardziej usprawnić działanie
\quotation{lepów}, można wprowadzić losowość w generowanym formularzu --
pola mogą mieć różne nazwy, dodatkowo mogą pojawiać się losowo między
różnymi realnymi polami w formularzu. Wprowadzenie dynamiki do
formularza pozwoli ochronić się przed atakami skierowanymi bezpośrednio
w naszą stronę. Jednak samo to rozwiązanie jest również proste do
obejścia. Jeżeli osoba programująca robota zapozna się ze stroną, może
zadeklarować, jakie pola powinien uzupełniać robot oraz nakazać mu, żeby
resztę pól omijał.

\subsection[md5]{MD5}

Ostatnia linia zabezpieczeń polega na zakodowaniu nazw rzeczywistych pól
w formularzu. Kodowanie może odbywać się na różne sposoby, ale
proponowana jest następująca metoda:

\starttyping
nazwa = md5 ( hasło + nazwa pola + timestamp )
\stoptyping

Kodowanie nazw odbywa się po stronie aplikacji w trakcie generowania
formularza, dlatego użytkownik widzi tylko rezultat końcowy, czyli
gotową nazwę. Hasło jest zabezpieczeniem wewnętrznym przed złamaniem
metody -- tak długo, jak nie zostanie one poznane, nie będzie istniała
metoda na odwzorowanie algorytmu przetwarzania nazwy. Kodowanie MD5 jest
nieodwracalne, natomiast zmiana już jednego znaku zmienia zupełnie całą
sumę. Nazwa pola służy do rozpoznawania, które konkretnie pole chcemy
przekodować. Ostatnie pole, timestamp wprowadza dynamikę w nazwach pól.
Każde odświeżenie formularza będzie powodować wyglądające na losowe
zmiany nazw pól. Poniżej przykład dwóch nazw jednego pola:

Pole \quotation{email} przy haśle \quotation{bardzotrudnehasło}
wygenerowane w odstępie 1 sekundy:

\starttyping
md5( „bardzotrudnehasłoemail1378206571” ) = „3d4d9a18d1584fb54a049cc79416d7b7”
md5( „bardzotrudnehasłoemail1378206572” ) = „99adcde65db4b8a6d4b98916a708867b”
\stoptyping

Tak więc pole użyte w formularzu zamiast wyglądać w następujący sposób:

\starttyping
<input type="text" name="name">
\stoptyping

będzie wyglądać na przykład tak:

\starttyping
<input type="text" name="3d4d9a18d1584fb54a049cc79416d7b7">
\stoptyping

Po wysłaniu formularza z powrotem do aplikacji na podstawie powyższej
formuły oraz biorąc jako timestamp ten przekazany w polu timestamp
będzie możliwe ponowne wygenerowanie tych samych sum md5, które zostały
przekazane jako nowe nazwy pól. Dzięki temu można będzie uzyskać
oryginalne nazwy pól bez prezentowania ich nawet przez chwilę
użytkownikowi.

Dzięki temu rozwiązaniu każdy formularz prezentowany użytkownikowi
będzie inny. Niemożliwe więc będzie zastosowanie żadnych sztywnych metod
wpisywania danych.

\subsection[formularz-przed-i-po]{Formularz przed i po}

Dla użytego wcześniej formularza zawierającego imię i hasło:

\starttyping
<input type="text" name="name">
<input type="password" name="password">
\stoptyping

wygenerowany bezpieczny formularz może wyglądać następująco:

\starttyping
<input type="hidden" name="timestamp" value="1378201731">
<input type="text" name="1d8d80c45b7b66368d4dfe28df74d1d2">
<input type="text" name="email" class="hidden">
<input type="password" name="3b1c1e6e4c0275fc468aa94fc4e973f5">
<input type="text" name="login" class="hidden">
\stoptyping

Natomiast po kolejnym odświeżeniu strony ten sam formularz będzie bardzo
odmienny od poprzedniego:

\starttyping
<input type="hidden" name="timestamp" value="1378201771">
<input type="text" name="password" class="hidden">
<input type="text" name="city" class="hidden">
<input type="text" name="bf4123f857426f9f60c395520f0c5ee1">
<input type="password" name="5a4f20be80fc229c16154269260e9bee">
\stoptyping

Nie jest więc możliwe przygotowanie się do tego, jak będzie wyglądać
wygenerowany przez stronę formularz. Pomimo braku standardowego machizmu
CAPTCHA bez udziału ludzkiego czynnika prawidłowe wypełnienie formularzu
przez robota nie będzie możliwe. A wszystko to dzieje się za plecami
użytkownika, który nie zauważy nawet jednej linijki odróżniającej taki
formularz od standardowego.

\section[nocaptcha-w-django]{NoCAPTCHA w Django}

Instalacja i korzystanie z biblioteki NoCAPTCHA w Django jest wyjątkowo
proste i wymaga jedynie kilku akcji. Po zainstalowaniu pakietu przez
pip: \type{pip install nocaptcha} wystarczy go dodać do listy
\type{INSTALLED_APPS} w ustawieniach:

\starttyping
INSTALLED_APPS = (
    ... 
    'nocaptcha',
    ... )
\stoptyping

Po tym kroku NoCAPTCHA jest już zainstalowana i można z niej korzystać.
Aby stworzyć przykładowy formularz stosujący wszystkie opisane powyżej
zabezpieczania dodajemy wyłącznie kilka linijek do standardowego
formularza Django:

\starttyping
from nocaptcha.forms import NoCaptchaForm

class ContactForm(NoCaptchaForm, forms.Form):
    secret_password = "NoCAPTCHA rządzi!"
    name = forms.CharField(label="Imię”)
    password = forms.PasswordField(label="Hasło")
\stoptyping

Tylko tyle aby nasze formularze pozostawały bezpieczne przed robotami.

\section[podsumowanie]{Podsumowanie}

Powyższy artykuł ma na celu wskazanie jednej z wielu możliwych ścieżek
realizacji zagadnienia ochrony formularzy przed automatycznym
uzupełnianiem. Nie jest to jedyne możliwe wyjście, z pewnością nie jest
też stuprocentowo skuteczne. Jednak to, na czym należy się skupić
projektując stronę, to znalezienie środka pomiędzy bezpieczeństwem, a
wygodą użytkownika. Rozwiązania NoCAPTCHA nadaje się do łatwego
dodawania kolejnych poziomów usprawnień bez wymaganej interakcji ze
strony użytkownika. Koniec końców istotne jest to, aby twórcy botów
zamiast zająć się pracochłonnym zadaniem pokonania naszych poziomów
zabezpieczeń spróbowali szczęścia na innych serwisach -- być może nawet
konkurencyjnych.


\stoptext
