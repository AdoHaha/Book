\usemodule[pycon-2015]
\starttext

\Title{Condemned to re-invent SQL, poorly}
\Author{Michał Lowas-Rzechonek}
\MakeTitlePage

\subsection[introduction]{Introduction}

The days when developers were expected to write SQL by hand are long
gone. ORMs are getting more and more popular, especially in web
development.

Django, as one of the most popular (if not the most popular?) Python
frameworks for web development comes with the ORM built-in. Because of
its prevalence, I'm going to use it in my examples. If you don't know
how it works, their tutorial is really nice and short {[}1{]}.

While it's hard to beat the convenience of using such an ORM, it lures
us into thinking about a database as a simple \quotation{object
storage}. One of the problems is the fact that in most ORMs, querying
the database happens pretty much exclusively via a model class, which
constrains the set of results to a known attributes of our objects.

Focus on collections of objects might be one of the reasons why NoSQL
approaches are more alluring: just throw your objects into a bin and
save them on the disk. Indeed, there is no need to worry about tables,
constraints and indexes if all you do is load data from the storage and
process it inside the application!

I think the \quotation{object storage} way of thinking severely limits
our ability to process the data. In a relational model, columns are
computed ad-hoc, when returning query result, not when defining the
schema. This is one of the aspects of so-called
\quotation{object-relational impedance mismatch} that results in
headaches for people smarter than us.

In this article, I'd like to show a few features of a relational model
which, while mapping poorly to object-oriented world, give the
programmer very powerful data manipulation tools.

As an example, I'm going to use PostgreSQL, which I find the most
powerful open-source database management system. Note that most of the
\quotation{fancy} constructs supported by PostgreSQL are actually
standard SQL clauses. Still, the big warning is in order: if you use
MySQL, you might as well stop reading right now.

I'm also going to suggest ways to map these features into application
code. While I believe the complete, robust translation is not achievable
for reasons mentioned above, there are techniques that allow us to tie
the data and application layers together, especially in a flexible,
multi-paradigm language like Python.

\subsection[views]{Views}

Consider an university schedule. There are buildings, rooms and lectures
happening in these rooms. The schema is a fairly straightforward one:

\starttyping
class Room(models.Model):
    building = models.CharField(max_length=64, null=False, blank=False)
    number = models.IntegerField(null=False, blank=False)

class Lecture(models.Model):
    title = models.CharField(max_length=120, null=False, blank=False)
    room = models.ForeignKey(Room, null=False)
    date = models.DateTimeField(null=False, blank=False)
\stoptyping

The most obvious case where the ORM falls short, is any non-trivial
aggregation. Let's say we would like to know which months are the most
(or least) busy, per building, so we can plan construction work on the
campus.

With SQL, that's easy. First, fetch number of lectures happening in each
building in each month:

\starttyping
select
    "lectures_room"."building" as "building",
    date_trunc('month', "lectures_lecture"."date") as "month",
    count(*) as "count"
from "lectures_lecture"
left join "lectures_room" on "lectures_room"."id" =
"lectures_lecture"."room_id" group by "building", "month"
\stoptyping

result is going to be something like this:

\starttyping
 building |         month          | count
----------+------------------------+-------
 A2       | 2015-06-01 00:00:00+02 |   275
 B3       | 2015-08-01 00:00:00+02 |   291
 C3       | 2015-12-01 00:00:00+01 |   288
 B4       | 2015-03-01 00:00:00+01 |   299
 B2       | 2015-08-01 00:00:00+02 |   315
 B1       | 2015-01-01 00:00:00+01 |   294
 ...
\stoptyping

Then, partition the result by building name, within each partition order
rows by number of lectures, then fetch first and last value of
\quotation{month} column.

The way to do that is called a window function. This feature allows the
query to aggregate results into subsets (windows) and constructs rows by
selecting values from a subset. More on that in {[}2{]}.

Notice that in this case we're going to run a query on top of another
query. This is called \quotation{composition} and is a cornerstone of
most programming languages. Think about that as an equivalent of nested
function calls - call this, pass the result as an argument to call that,
and so on. Just that in SQL, you don't define functions, but rather
assign names to such subqueries. The mechanism is called
\quotation{Common Table Expressions} {[}3{]} and uses
\type{with foo as (select ...)} syntax.

\starttyping
with "busy_months" as (
    select
        "lectures_room"."building" as "building",
        ... -- the rest of the first query
)
select distinct on ("building")
    "building",
    first_value("month") over "building_window" as "least_busy_month",
    last_value("month")  over "building_window" as "most_busy_month"
from
    "busy_months"
window "building_window" as (
    partition by "building"
    order by "count"
    rows between unbounded preceding and unbounded following
);
\stoptyping

Which gives us the final report:

\starttyping
condemned=> select * from lectures_busy_months;
 building |    least_busy_month    |    most_busy_month
----------+------------------------+------------------------
 A1       | 2015-02-01 00:00:00+01 | 2015-03-01 00:00:00+01
 A2       | 2015-04-01 00:00:00+02 | 2015-01-01 00:00:00+01
 A3       | 2015-02-01 00:00:00+01 | 2015-03-01 00:00:00+01
 B1       | 2015-02-01 00:00:00+01 | 2015-03-01 00:00:00+01
 B2       | 2015-02-01 00:00:00+01 | 2015-12-01 00:00:00+01
 B3       | 2015-04-01 00:00:00+02 | 2015-07-01 00:00:00+02
 B4       | 2015-02-01 00:00:00+01 | 2015-10-01 00:00:00+02
 C1       | 2015-02-01 00:00:00+01 | 2015-12-01 00:00:00+01
 C2       | 2015-01-01 00:00:00+01 | 2015-07-01 00:00:00+02
 C3       | 2015-02-01 00:00:00+01 | 2015-07-01 00:00:00+02
(10 rows)
\stoptyping

Mapping this to the ORM is a bit of a problem, though: * There is no
\type{Building} model, so we can't say
\type{Building.objects.annotate(count=Count(...))} * There is no
straightforward way to call \type{date_trunc} function and aggregate by
that * The ORM doesn't have any idea about \type{partition by} and
\type{over} clauses.

This doesn't mean that all hope is lost, though. Sane databases allow
you to create a sort-of dynamic table out of that query. This is called
a \quotation{view} - just a name for a query result. Don't confuse that
with Django views.

The view might seem like a trivial feature, but the point here is that
it allows us to define database-side abstractions, so do not
underestimate them.

\starttyping
create view "lectures_busy_months" as
    with ...
    select ...;
\stoptyping

This, for all intents and purposes, behaves just like a (read-only)
table. This means we can explicitly map this view to a model:

\starttyping
class BusyMonths(models.Model):
    building = models.CharField(max_length=64, primary_key=True)
    least_busy_month = models.DateField()
    most_busy_month = models.DateField()

    class Meta:
        managed = False
        db_table = 'lectures_busy_months'
        ordering = ('building',)
\stoptyping

There's one shortcoming though: Django requires all models to have a
single-column primary key. In our case, building names are unique in the
result, so we can tell Django to just use that. This is not always the
case though, and you might end up with adding superficial
auto-incrementing column just to make the ORM happy:

\starttyping
create view "needs_id" as
    select
        row_number() over () as "id",
        ...
    from ...
\stoptyping

The only remaining thing is creating a schema migration that would
install our SQL view in the database. This can be done by executing an
SQL script via \type{migrations.RunSQL} operation. You can find the
details in the sample project {[}4{]}.

\subsection[functions]{Functions}

Pure ORM simply does not allow any kind of imperative logic to be
defined in the database. If you need to do non-trivial processing, you
need to fetch everything and do the computation in Python. In some
cases, this is a huge I/O overhead.

What about using this feature to crunch the numbers right where they
are, and passing only the result to the application? The best part is,
we can do it in Python!

Continuing the \quotation{university} theme, consider a table of grades:

\starttyping
class Grade(models.Model):
    student = models.ForeignKey(settings.AUTH_USER_MODEL)
    date = models.DateTimeField(auto_now=True, null=False)
    grade = models.IntegerField(null=False)
\stoptyping

Now, we would like to know who of the students is improving over time
and who's getting worse. One way of doing that is finding a linear
interpolation of grades over time with a \quotation{least squares
method} {[}5{]}

\starttyping
given a set `S` of points `(x, y)` on a plane
and a line L: `y = A * x + B`
find the values of A and B
such that sum of squared distances between line L and each point in S
is minimal
\stoptyping

As said before, to calculate that in your application, you would need to
fetch {\em all} grades from the database, group them by student
{\em yourself} and run the fitting algorithm (i.e. \type{numpy}) on each
set. But what if you could write an ORM query that looks like this?

\starttyping
User.objects.annotate(grade_trend=LinearFit(
    'grade__grade', 'grade__date'))
\stoptyping

Fortunately, in PostgreSQL, we can implement database-side functions in
Python. Using these, we can create our own aggregates, besides standard
ones like \type{count}, \type{min} and \type{max}!

The way it works, you need a data type and two functions: the type
stores the \quotation{state} of your aggregate. In our case, it's just a
list of grades. Then, first function is supposed to \quotation{add} new
value to the \quotation{state}: in our case, just append the grade to
the list. The final function takes the \quotation{state}, and calculates
the final result.

Sounds scary, but really isn't. The code below is all you need:

\starttyping
create language plpythonu;

create or replace function linear_fit_finalfunc(p_state float[])
returns float[] as
$$
    import numpy
    return numpy.polyfit(xrange(len(p_state)), p_state, 1)
$$
language plpythonu immutable;

create aggregate linear_fit(float)
(
    stype = float[],      -- our "state" is a list of grades
    initcond = '{}',      -- starting from an empty list
    sfunc = array_append, -- adding grade to the "state" is just appending to
                          -- the list
    -- final function does the line fitting on accumulated points
    finalfunc = linear_fit_finalfunc
);
\stoptyping

Mind the \type{create aggregate} clause which is very non-standard and
specific to PostgreSQL. The parameters are:

\startitemize
\item
  \type{stype}: the data type holding the \quotation{state}
\item
  \type{initcond}: initial value of the \quotation{state}
\item
  \type{sfunc}\quotation{: function that inserts (aggregates) a}value"
  into the \quotation{state}
\item
  \type{finalfunc}: function that takes the \quotation{state} and
  returns final value of the aggregation
\stopitemize

This allows us to write the following query:

\starttyping
select
    "auth_user"."username",
    linear_fit("grades_grade"."grade" order by "grades_grade"."date")
from "auth_user"
left join "grades_grade" on "grades_grade"."student_id" = "auth_user"."id"
\stoptyping

Looks good, but as with the view, we still need to map this to ORM
somehow. This is slightly more complex, we need to define an ORM wrapper
for our custom aggregation function. This is similar to what Django
already provides: \type{Count} wraps the \type{count()}, \type{Avg}
wraps \type{avg} and so on. We need to make our own \type{LinearFit}
that wraps \type{linear_fit}.

This is little complicated, as we need to dig deeper into the
ORM\ldots{}

Long story short: when we write a query, Django constructs a tree of
\quotation{expressions} to reflect our wishes {[}6{]}. Then, this tree
is \quotation{compiled} to SQL code and executed as a raw SQL query. The
result is translated back from a table to a set of \type{Model}
instances.

To add a custom aggregate, we need to define a custom
\quotation{expression}. Luckily, we can base it on built-in base
expression classes. There is one for database-side functions,
appropriately named \type{Func}. Unfortunately, it doesn't support

\starttyping
select aggregate_function(column order by another_column)
\stoptyping

syntax, which is vital for our calculations - grades need to be fitted
in order in which they were given (otherwise the whole \quotation{trend}
doesn't make any sense). This means we need to extend it a little by
modifying \type{Func.template} attribute, like this:

\starttyping
class LinearFit(Func):
    contains_aggregate = True
    function = 'linear_fit'
    template = '%(function)s(%(expressions)s order by %(ordering)s)'

    def __init__(self, expression, ordering, **extra):
        super(LinearFit, self).__init__(
            expression,
            output_field=ArrayField(models.FloatField()),
            **extra)
        self.ordering = self._parse_expressions(ordering)[0]

    def resolve_expression(self, *args, **kwargs):
        c = super(Func, self).resolve_expression(*args, **kwargs)
        c.ordering = c.ordering.resolve_expression(*args, **kwargs)
        return c

    def as_sql(self, compiler, connection, function=None, template=None):
        ordering_sql, ordering_params = compiler.compile(self.ordering)
        self.extra['ordering'] = ordering_sql

        return super(LinearFit, self).as_sql(compiler, connection,
            function, template)

    def get_group_by_cols(self):
        return []
\stoptyping

This seems to do the job (note that grades were given randomly, so
trends are mosly flat)!

\starttyping
>>> for i in User.objects.annotate(grade_trend=LinearFit(
            'grade__grade', 'grade__date')):
...     print i.username, i.grade_trend
...
john [-0.0017908297019, 3.77233748271]
mary [-0.000803833399885, 3.63772475795]
barbara [-0.0018932620358, 3.65124481328]
peter [-0.00281602111148, 3.87818118949]
thomas [0.000325526484835, 3.48609958506]
kevin [0.00103517422177, 3.3887966805]
betty [-0.000701401065991, 3.54215076072]
april [5.72926613309e-05, 3.53482019364]
bob [-0.000128474452681, 3.49868603043]
sean [0.000461379537839, 3.54069847856]
denise [0.000466587961597, 3.36507607192]
\stoptyping

\subsection[summary]{Summary}

SQL is often considered an \quotation{ugly duckling} in the stack, and
most of developers avoid it like a plague. True, the syntax isn't the
prettiest and sometimes it works in a counter-intuitive way. On the
other hand, you can achieve amazing results with very little amount of
rather readable code.

The best part is that you don't need to know upfront what kind of
reporting you're going to do. Most of the time, you can just design your
models as usual and write your views/aggregates/whatnot right when you
need them.

I would like to encourage everyone dealing with databases to dig a bit
deeper into the relational world and stop relying exclusively on the
ORM. The two examples are just a tip of an iceberg, there is {\em much}
more in SQL than you think.

\subsection[further-reading]{Further reading}

\startitemize[n][stopper=.]
\item
  \useURL[url1][https://docs.djangoproject.com/en/1.8/intro/tutorial01/\#creating-models][][Django:
  Model intro]\from[url1]
\item
  \useURL[url2][http://tapoueh.org/blog/2013/08/20-Window-Functions][][Dimitri
  Fontaine: Understanding Window Functions]\from[url2]
\item
  \useURL[url3][http://www.postgresql.org/docs/8.4/static/queries-with.html][][PostgreSQL:
  Common Table Expressions]\from[url3]
\item
  \useURL[url4][https://github.com/mrzechonek/sql_condemned][][Github:
  SQL, Condemned]\from[url4]
\item
  \useURL[url5][https://en.wikipedia.org/wiki/Simple_linear_regression][][Wikipedia:
  Simple linear reqression]\from[url5]
\item
  \useURL[url6][https://docs.djangoproject.com/en/1.8/ref/models/expressions/][][Django:
  Query expressions]\from[url6]
\stopitemize


\stoptext
