\usemodule[pycon-2015]
\starttext

\Title{Kosmiczny Python}
\Author{Sławomir Piasecki}
\MakeTitlePage

\subsection[krótko-o-astronomii]{Krótko o astronomii}

Od wieków człowiek spoglądał w niebo i próbował dowiedzieć się czegoś na
temat tajemniczych świateł na nocnym niebie. Prehistoryczne artefakty
służące do obserwacji nieba znajdujemy na całym świecie. Dzięki tym
odkryciom możemy śmiało powiedzieć, że astronomia jest jedną z
najstarszych dziedzin nauki. Przez wieki jedynym sposobem badania
wszechświata było spoglądanie w niebo każdej nocy gołym okiem. Milowym
krokiem w badaniach kosmosu okazał się wynalazek Galileusza.

Teleskop ze szklanymi soczewkami umożliwił głębsze sięgnięcie wzrokiem w
nieboskłon. Mogliśmy odkryć księżyce Jowisza oraz kolejne planety w
Układzie Słonecznym. Następnym krokiem w badaniu wszechświata okazało
się wynalezienie fotografii, dzięki której zaobserwowano wiele asteroid
poruszających się w Układzie Słonecznym. Najnowsza astronomia nie może
się obejść bez komputerów, a co za tym idzie, również bez programowania.
Najczęściej używanym językiem programowania w astronomii jest Fortran,
który został stworzony do obliczeń numerycznych. Jego największą zaletą
jest szybkość oraz szeroki zasób dodatkowych bibliotek.

Kolejnym równie często używanym językiem programowania jest C/C++. Jako
język strukturalny, podobnie jak Fortran, jest bardzo szybki. Większość
oprogramowania wykorzystywanego obecnie w astronomii jest napisane w
jednym lub obu tych językach. Astronomowie korzystają również z
programów do obliczeń numerycznych takich jak Mathematica czy MatLab.

\subsection[python-w-astronomii]{Python w Astronomii}

Mimo dominującej roli Fortrana i C w Astronomii coraz więcej osób
korzysta do obliczeń z języka Python. Mimo, iż jest to język
interpretowany, a więc dużo wolniejszy od wcześniej wspomnianych, ma
bardzo wielką zaletę, jest bardzo prosty do nauki. Astronomowie nie mają
w swoim programie nauczania wystarczającej ilości godzin na naukę
programowania, stąd Python, ze względu na swoją składnię oraz bardzo
bogatą dokumentację, jest bardzo przyjazny początkującym i zdobywa coraz
więcej zwolenników. Moduł matplotlib przypadł do gustu naukowcom i służy
jako główne narzędzie do wykreślania danych. Kolejnym atutem
przemawiającym za Pythonem jest możliwość zintegrowania z już
istniejącym kodem napisanym w C/C++ lub Fortranie.

\section[astropy]{AstroPy}

W astronomii najważniejsze są obserwacje, to dzięki nim weryfikowane są
teorie, które rodzą się w umysłach fizyków i astronomów. Obserwacje
polegają głównie na detekcji fal elektromagnetycznych w całym zakresie.
Badanie światła widzialnego, które do nas dociera, odbywa się z
wykorzystaniem kamer CCD (Charge-Coupled Device). Każda kamera (również
kamery CCD), ze względu na swoją budowę, emituje różnego rodzaju szumy.
Aby uzyskać wartościowe zdjęcie, które może posłużyć do badań, musimy
pozbyć się tych szumów. Takie zdjęcie/ramkę należy jeszcze skalibrować,
czyli usunąć szum termiczny (DARK), który pojawia się podczas
wykonywania zdjęcia, oraz podczas szczytywania (BIAS). Ostatnim
elementem potrzebnym do kalibracji jest FLATFIELD, czyli zarejestrowana
czułość pikseli, która jest normowana do 1 (najjaśniejszy piksel).
Podsumowując, każde zdjęcie ma \quote{zanieczyszczenia}, które muszą
zostać usunięte, aby uzyskać wartościowe zdjęcie do dalszej analizy.
Cały proces nie jest łatwy, gdyż ramek BIAS, czy FLATFIELD, robi się od
kilku do kilkunastu sztuk podczas obserwacji. Pierwsze ramki są
uśredniane, a drugie sumowane, a następnie ramki są odejmowane od
orginalnego zdjęcia. Każde zdjęcie ma od kilku do kilkunastu
megapikseli, więc wykonanie takich redukcji ręcznie jest niemożliwe.

Z pomocą przychodzą komputery oraz programy zewnętrzne, takie jak Gaia
czy Iraf. Wyczyszczone zdjęcia ze \quote{śmieci} zapisywane są w
specjalnym formacie FITS (Flexible Image Transport System). AstroPy
umożliwia odczytywanie tych plików oraz wykorzystanie do dalszej pracy.

Wykonane zdjęcie jest katalogowane, a współrzędne obiektów są zapisywane
do specjalnych katalogów. Do opisania pozycji gwiazd i planet na niebie
nie wystarczy znany wszystkim układ kartezjański, potrzeba układu
współrzędnych wzbogaconych o pomiar czasu. Układy opisane poniżej to
najczęściej używane w astronomii układy sferyczne, aby podać współrzędne
obiektu najpierw taki układ musi być dobrze zdefiniowany. Każdy z
układów posiada określone koło wielkie oraz punkt na tym kole, od
którego zaczynamy liczyć daną współrzędną. Drugą wartość otrzymamy przez
dodanie płaszczyzny prostopadłej do koła wielkiego. Druga płaszczyzna
wyznacza na sferze dwa bieguny, a południk początkowy to ten, który
przecina punkt początkowy.

Najczęściej stosowane układy odniesienia to horyzontalny, godzinowy,
równonocny, ekliptyczny oraz galaktyczny. Podstawowym układem jest układ
horyzontalny, tutaj kołem wielkim jest płaszczyzna horyzontu. Biegunami
w tym układzie będzie zenit i nadir. Współrzędnymi układu współrzędnych
horyzontalnych to Azymut astronomiczny {\bf A}, mierzony w stopniach
($0^{\circ}$-$360^{\circ}$) oraz wysokość {\bf h} podawana w zakresie
{[}$-90^{\circ}$-$90^{\circ}${]}.

Kolejny układ współrzędnych wykorzystywany w astronomii nazywa się
godzinowy. Współrzędne to kąt godzinny {\bf t} mierzony w godzinach
($0^{h}$-$24^{h}$). Deklinacja $\delta$ to druga współrzędna, która
podawana jest w stopniach w zakresie {[}$-90^{\circ}$-$90^{\circ}${]}.
Są one zdefiniowane dzięki kołu wielkiemu przechodzącemu przez równik
niebieski. Bieguny tego układu będą się pokrywać~z osią obrotu ziemi.

Układ współrzędnych równikowych równonocnych jest zdefiniowany jak
godzinowy z tą różnicą, że punktem początkowym jest punkt Barana, a
współrzędne to deklinacja $\delta$ {[}$-90^{\circ}$-$90^{\circ}${]} oraz
rektascensja $\alpha$ mierzona jako kąt godzinny w przedziale
0\letterhat{}\letteropenbrace{}h\letterclosebrace{},
24\letterhat{}\letteropenbrace{}h\letterclosebrace{}.

Koło wielkie, po którym pozornie porusza się Słońce, to ekliptyka, i
jest ono podstawowym kołem w układzie ekliptycznym. Długość ekliptyczna
$\lambda$ oraz szerokość ekliptyczna $\Beta$ są liczone w stopniach
(0\letterhat{}\letteropenbrace{}\circ\letterclosebrace{},
360\letterhat{}\letteropenbrace{}\circ\letterclosebrace{}).

Ostatnim układem będzie galaktyczny, jak łatwo się domyślić, podstawową
płaszczyzną będzie płaszczyzna galaktyki. I podobnie jak w ekliptycznym
układzie mamy tutaj długość i szerokość galaktyczną
{[}$-90^{\circ}$-$90^{\circ}${]}.

Wszystkie układy mają swoje mocne i słabe strony, jeden jest zależny od
ruchu obrotowego Ziemi, inny od Słońca. Ponadto instrumenty obserwacyjne
są również tak wykonywane, aby śledziły obiekty w jednym z powyższych
układów. Stąd współrzędne obiektów astronomicznych są zapisywane w
różnych systemach, dlatego też trzeba współrzędne przetransformować do
innego układu, który będzie bardziej przydatny na danym instrumencie.
Takie transformacje są również dostępne w module AstroPy.

\section[matplotlib]{MatPlotLib}

Jest to moduł do rysowania wykresów, które tak często pojawiają się w
publikacjach. Biblioteka umożliwia rysowanie bardzo prostych w $2D$
linii, diagramów, aż po bardziej skomplikowane histogramy i kontury.
Dołączona jest również biblioteka do wykresów $3D$, dzięki której
wykreślimy atraktory, powierzchnie i wiele innych. Jednocześnie mamy
wpływ na wszystkie elementy wykresów od opisów, przez osie, wielkość
podziałki na osiach. Możemy wprowadzać wzory matematyczne, manipulować
czcionką, wykreślać kilka, a nawet kilkanaście wykresów. Na jednym
obrazku możemy łączyć wykresy gęstości $2D$ z histogramami $1D$.
Matplotlib daje nam bardzo dużo swobody w tworzeniu różnorakich
wykresów.

\section[auto]{AUTO}

Bifurkacja (rozdwojenie) jest to zjawisko skokowej zmiany własności
modelu matematycznego, nawet przy drobnej zmianie parametrów. W
mechanice nieba, jednej z dziedzin astronomii, bifurkacja następuje
wtedy, gdy zmienia się liczba rozwiązań równania różniczkowego podczas
zmian parametru równania. Rozdwojenie przedstawione na wykresie pozwala
na odróżnienie zachowań okresowych od chaotycznych. Dla takich
przypadków powstało oprogramowanie AUTO, używające języków Fortran i
Python. AUTO jest w stanie określać punkty bifurkacji, a także wyznaczać
stabilne i niestabilne rodziny periodycznych rozwiązań zwyczajnych
równań różniczkowych.

Model matematyczny oraz równania różniczkowe zapisuje się w języku
Python, a do przechowania wszelkich ustawień dotyczących sposobu
obliczeń stosuje się strukturę listy. Obliczenia wykonywane są natomiast
we Fortranie na podstawie zapisanych wcześniej danych.

\subsection[zakończenie]{Zakończenie}

Niniejszy tekst niech posłuży jako słowniczek, odnośnik, który ma za
zadanie w najprostszy sposób wyjaśniać poszczególne zagadnienia, o
których będzie mowa podczas prezentacji. Mam nadzieję, że po
przeczytaniu słuchacz będzie miał większą świadomość, o czym jest
prezentacja.

\subsection[bibiografia]{Bibiografia}

\startitemize
\item
  {[}1{]} Astronomia - https://pl.wikipedia.org/wiki/Astronomia
\item
  {[}2{]} Układy współrzędnych w astronomii -
  http://www.astro.amu.edu.pl/\lettertilde{}jopek/\crlf
JopekTJ/Dydaktyka/A\_Sf/2014-15/Prezentacje/wyklad\_03.pdf
\item
  {[}3{]} Astropy - http://docs.astropy.org/
\item
  {[}4{]} Matplotlib - http://matplotlib.org/
\item
  {[}5{]} Bifurikacja -
  http://zasoby1.open.agh.edu.pl/dydaktyka/matematyka/\crlf
c\_fraktale\_i\_chaos/chaos.php?rozdzial=2
\item
  {[}6{]} AUTO - http://cmvl.cs.concordia.ca/auto/
\item
  {[}7{]} Notatki z wykładów astronomicznych.
\stopitemize


\stoptext
