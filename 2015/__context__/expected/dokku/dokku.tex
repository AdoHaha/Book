\usemodule[pycon-2015]
\starttext

\Title{Dokku. Build your own PaaS}
\Author{Sergey Parkhomenko}
\MakeTitlePage

\subsection[introduction]{Introduction}

Don't like configuring and satisfying system requirements? Your dream is
to deploy applications in one click? PaaS is the solution! During the
year 2014, we all did a great dive into the world of development
automation and optimization. Riding this curve, there are a lot of cloud
PaaS providers. The most famous are Heroku, Google Apps Engine, etc.
Each of them offers great opportunities for deploying, running and
scaling your application, but the first and the biggest issue is that
they don't offer enough flexibility in environment configuration. Every
developer had a situation when he needed something more from provider,
who gives you limited bunch of features. Another issue is cost. PaaS
providers, mentioned above, cost money. This is not bad, because they do
their job well, but it is possible to avoid those expenses. And now is
time to say one word: \quotation{Dokku}.

\subsection[definitions]{Definitions}

{\bf Platform as a service (PaaS)} is a category of cloud computing
services that provides a platform allowing customers to develop, run,
and manage Web applications without the complexity of building and
maintaining the infrastructure typically associated with developing and
launching an app. PaaS can be delivered in two ways: as a public cloud
service from a provider, where the consumer controls software deployment
and configuration settings, and the provider provides the networks,
servers, storage and other services to host the consumer's application;
or as software installed in private data centers or public
infrastructure as a service and managed by internal IT departments.

{\bf Heroku} is a cloud platform as a service (PaaS) supporting several
programming languages.

{\bf Dokku} is a Docker-powered PaaS implementation.

\subsection[paas-use-case]{PaaS use case}

Let's imagine that you have a small blog written in Django which you
need to deploy fast an easy. Also this blog is under development and you
are going to roll out updates every day during the upcoming month.
Simultaneously, lets imagine default actions which you do when you need
to match such requirements. You purchase a new virtual or dedicated
server, or buy a new EC2 instance, set up a new environment for the
project, or adopt your existing infrastructure to requirements of your
new blog, what can be especially painful if you have projects with other
system level requirements which can even use different languages. If you
manage with very diverse technologies stacks, sooner or later, your
server becomes a trash can. But this is still not all pain which you
experience in such kind routine. If your changes to the code base
require system configuration or packages modifications, you start
spending more time on deployment than on development. This is the right
time to say: \quotation{Stop! I'm not a system administrator! I don't
want to configure and maintain, I want to write code!}. And right after
this phrase you become a target audience for PaaS providers.

\subsection[installing-dokku]{Installing Dokku}

Let's assume that you already have configured and running server where
we will install Dokku. SSH into your host machine and run the following
command:

\starttyping
server $ wget -qO- https://raw.github.com/progrium/dokku/v0.3.26/
    bootstrap.sh | sudo DOKKU_TAG=v0.3.26 bash
\stoptyping

Please, {\bf notice} that I use the latest version of Dokku (0.3.26) for
this moment. Also, I want to emphasize that you have to run installation
using {\bf sudo} even if you are logged in via root user. This is
important as this installer creates a custom dokku user for executing
commands and it is important to make sure if this user has enough
permissions for building packages, creating configs and required
directories. Otherwise your future project builds will fail. After
script finishes you need to go back to your local machine console and
run the following:

\starttyping
$ cat ~/.ssh/id_rsa.pub | ssh
    root@<machine-address> "sudo sshcommand acl-add dokku <your-app-name>"
\stoptyping

Where \type{<machine-address>} is ip or hostname of your server and
\type{<your-app-name>} is the name of application we will deploy. This
command may be not required if you already have configured SSH keys and
you log in to your server without credentials.

\subsection[deploying-sample-django-application-to-dokku]{Deploying
sample Django application to Dokku}

This guideline shows how fast and easy you can set up your own PaaS. In
this example we will install Dokku on the server and deploy
\quotation{Hello, world} Django application.

Clone any sample django application from github, cd to the application's
directory and add dokku as remote server with:

\starttyping
local $ git remote add dokku dokku@<machine-address>:<your-app-name>
\stoptyping

Push application to Dokku:

\starttyping
local $ git push dokku <branch-name>:master
\stoptyping

Pay attention, that dokku will deploy your application {\bf only} if you
push to the master branch!

If your application requires additional dependencies like database or
cache server, application deployment is not finished yet. Take a look on
\quotation{Plugins} session.

\subsection[buildpacks-or-why-previous-section-is-so-short]{Buildpacks
or why previous section is so short}

If you are curious, why application deployment described in the previous
section is so short and where are dependencies installation and lots of
other stuff, this section is for you.

So what exactly is a {\bf buildpack} and what kind of black magic it
does?

{\bf Buildpack} is a bunch of scripts for building container image.

So what Dokku does is, determine exactly which buildpack is required for
your application and build a new container for your project. As the
result you have a very high level of isolation for your applications and
an awesome orchestration solution. Notice, that Dokku is compatible with
Heroku buildpacks.

But what if Dokku doesn't determine a buildpack for your application
correctly?

You can create an \type{.env} file in your application root directory
and specify a buildpack by yourself:

\starttyping
export BUILDPACK_URL=https://github.com/OShalakhin/
    heroku-buildpack-geodjango
\stoptyping

You can also just put a Dockerfile inside the project root and Dokku
will pick it up and build an environment from it by default!

\subsection[plugins]{Plugins}

As was mentioned before, Dokku provides a very high level of isolation
for your projects. Here you may ask, what if my application requires a
database or a cache server. You can either use a simple database or
cache server installed and configured on your server (recommended for
stable production purposes), or use Dokku plugins. Dokku provides quite
good set of plugins for setting up and running persistent containers for
databases, caches, message queues, etc.

Let's install postgresql Dokku plugin and wire it to our application as
an example.

First of all, you need to clone plugin to Dokku's plugins directory:

\starttyping
cd /var/lib/dokku/plugins
git clone https://github.com/jeffutter/dokku-postgresql-plugin postgresql
dokku plugins-install
\stoptyping

Then we create a database for application:

\starttyping
server $ dokku postgresql:create <your-app-name>
\stoptyping

This command automatically creates and wires a database to
\type{<your-app-name>} application. It also outputs an appropriate
\type{DATABASE_URL} to console which you need to put to your
application's configuration.

\subsection[debugging]{Debugging}

If you are experiencing troubles with building your application, run
\type{dokku trace} on your server and try to deploy your application one
more time. You will get a detailed verbose output of container building
process.

If you have issues with running application, run
\type{dokku logs <your-app-name>} to get logs of your app.

You can get more info about Dokku troubleshooting from documentation. It
has explanations with guidelines for lots of occasions.

\subsection[dokku-advantages-over-competitors]{Dokku advantages over
competitors}

\startitemize[n][stopper=.]
\item
  You pay only for the server where Dokku is installed
\item
  Fully customizable
\item
  Supports build from Dockerfile
\item
  Has a great plugins infrastructure
\item
  Easy change number of workers per application
\stopitemize

\subsection[dokku-disadvantages]{Dokku disadvantages}

\startitemize[n][stopper=.]
\item
  Doesn't scale on many machines. If you want this, explore Deis
\stopitemize

\subsection[conclusion]{Conclusion}

If you have a number of small applications you need to deploy, Dokku is
the right choice. It will provide you high level of isolation per
application and will improve your applications deployment experience.

\subsection[references]{References}

\startitemize[n][stopper=.]
\item
  PaaS Wiki
  https://en.wikipedia.org/wiki/Platform\_as\_a\_service
\item
  Dokku Documentation http://progrium.viewdocs.io/dokku/
\item
  Deis http://deis.io/
\stopitemize


\stoptext
