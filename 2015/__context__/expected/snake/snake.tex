\usemodule[pycon-2015]
\starttext

\Title{Latający wąż, czyli szybkie prototypowanie sprzętu przy pomocy Pythona}
\Author{Damian Gadomski}
\MakeTitlePage

Artykuł ten dotyczy prototypowania sprzętu na platformie MicroPython. W
pierwszej części omówiona została implementacja języka na mikrokontolery
z rodziny ARM, dostępne biblioteki oraz pokrewny projekt pyBoard.
Dokonano porównania wydajności MicroPythona względem innych popularnych
platform, a także opisano budowę i zasadę działania dronów latających w
zakresie umożliwiającym zrozumienie problemów, przed którymi staje
sterownik lotu takiego urządzenia.

\subsection[wstęp]{Wstęp}

Czasami zdarzają się projekty bardziej skomplikowane niż inne. Takie,
gdzie nie do końca wiadomo w jaki sposób osiągnąć oczekiwany rezultat.
Takie, gdzie potrzeba trochę pracy badawczej i testów, bo zagadnienie
nad którym pracujemy nie ma oczywistego rozwiązania. Często też nie
wiadomo jaki rezultat będzie najbardziej użyteczny i praktyczny.
Chcielibyśmy wtedy mieć możliwość zrobienia prototypu, który stale
będziemy ulepszać. Dobrze jeśli kolejne zmiany nie będą obarczone dużym
kosztem, ich wprowadzenie będzie szybkie. Może nawet chcielibyśmy
przetestować kilka algorytmów jednocześnie?

\section[inteligentna-łazienka]{Inteligentna łazienka}

Nie musi to być od razu ogromny system, lecz po prostu, niewielki
problem do rozwiązania. Przykładem może być kwestia włączania ogrzewania
i wyciągu (wentylatora) w inteligentnej łazience. Po wzięciu prysznica
chcielibyśmy, żeby ręcznik odwieszony na grzejnik jak najszybciej
wysechł, a wentylator jak najszybciej pozbył się wilgoci z łazienki.
Wbrew pozorom nie jest to takie banalne. W jaki sposób wykryć, że ktoś
wziął prysznic? Odpowiednio długo zapalone światło, podwyższona
wilgotność w łazience? Jaka zmiana wilgotności świadczy o wziętym
prysznicu? A może właśnie zostało rozwieszone pranie? Jak długo grzejnik
powinien być włączony aby ręcznik wysechł? Jak długo powinien działać
wentylator? Powinien uruchomić się od razu, czy po wyjściu z łazienki?

Chyba nalepiej zrobić prototyp, w którym później można „wyregulować”
kilka parametrów, aby dostosować go idealnie do potrzeb.

\subsection[micropython-i-pyboard]{Micropython i pyBoard}

Od dłuższego czasu powstają kolejne projekty, mające na celu uprościć
programowanie sprzętu: Arduino, SparkCore (obecnie Particle) czy
chociażby Raspberry Pi. Od ponad roku dostępny jest również MicroPython,
czyli implementacja języka Python na mikrokontrolery z rodziny ARM.
Projektowi temu towarzyszy pyBoard, czyli projekt prostej płytki
prototypowej na której działa MicroPython.

\section[język-i-dostępne-biblioteki]{Język i dostępne biblioteki}

MicroPython to implementacja języka Python w wersji 3.4. Co ważne,
zaimplementowany został cały język. Mamy więc do dyspozycji wszystkie
konstrukcje, począwszy od wyjątków, poprzez list i dict comprehensions,
a na wyrażeniach lambda skończywszy. W mikrokontroler wbudowane są
również poniższe standardowe biblioteki: - cmath - gc - math - os -
select - struct - sys - time

Należy wspomnieć, że bardzo dynamicznie tworzone są nowe
mikrobiblioteki. Mamy do dyspozycji chociażby moduły do wyrażeń
regularnych, dekodowania i kodowania JSON, kompresję zlib, i wiele
więcej. Temat ten jest bardzo żywy. W momencie pisania artykułu ostatnie
zmiany w repozytorium \type{micropython/micropython-lib} na Githubie
miały miejsce kilka godzin temu.

\section[część-sprzętowa---moduł-pyb]{Część sprzętowa - moduł pyb}

Poza wszystkimi rzeczami jakie możemy robić w \quotation{zwykłym}
Pythonie, musi być oczywiście coś jeszcze. Najbardziej
charakterystycznym modułem dla MicroPythona jest moduł \type{pyb}. Daje
on swobodny, wysokopoziomowy dostęp do peryferiów mikrokontrolera.
Dzięki niemu używanie przetworników analogowo - cyfrowych i cyfrowo
analogowych, akcelerometru, serwomechanizmów, przerwań, interfejsów
UART, I2C, SPI itd. jest banalnie proste i sprowadza się przeważnie do
jednej czy dwóch linijek kodu. To temu modułowi MicroPython zawdzięcza
tak duży potencjał do tworzenia szybkich prototypów. Wszelkich prostych
czujników użyjemy pisząc jedną linijkę kodu, sterowanie serwomechanizmem
ramienia robota to druga linijka. Zapalenie żarówki to podłączenie
układu wykonawczego i trzecia linijka. Do tego trochę logiki w Pythonie
i prototyp gotowy.

\section[wydajność]{Wydajność}

Kompilacja kodu dla MicroPythona jest wieloetapowa, w ostatnim etapie
domyślnie generowany jest bajtkod, który jest później uruchamiany na
wbudowanej w MicroPythona maszynie wirtualnej. Takie jest zachowanie
domyślne, zoptymalizowane względem zajętości pamięci RAM, której na
mikrokontrolerze jest niewiele. Możliwa jest jednak zmiana domyślnego
emitera kodu na natywny oraz natywny z optymalizacją (nazwany
\type{viper}). Aby użyć niedomyślych emiterów wystarczy zastosować
dekoratory, odpowiednio: \type{@micropython.native} i
\type{@micropython.viper}. Kod metody pozostaje bez zmian, jedynie
dekorator jest informacją dla kompilatora.

Aby zobrazować różnice w szybkości działania i zajętości pamięci przy
użyciu trzech powyższych emiterów, weźmy metodę która ma zapalić i
zgasić diodę LED milion razy. Wyniki to: - domyślny bajtkod: 44 bajty
pamięci, czas 10,4 sekund - natywny: 126 bajtów pamięci, czas 6,3
sekundy - „viper”: 114 bajtów pamięci, czas 5,0 sekundy

Dla porównania, odpowiedni kod napisany w C na Arduino (16MHz) wykonuje
się około 7 sekund, a na Raspberry Pi (700MHz) około 300 ms. Natomiat
odpowiedni kod napisany w Pythonie również na Raspberry Pi wykonuje się
niespełna 20 sekund!

Podandto istnieje czwarty emiter przydatny dla krytycznych fragmentów,
który pozwala pisać wstawki bezpośrednio w assemblerze. Niestety nie ma
możliwości pisania wstawek w języku C.

\subsection[drony-latające]{Drony latające}

Mikrośmigłowce wielowirnikowe zwane potocznie dronami czy
quadrokopterami stały się w ostatnim czasie bardzo popularne w wielu
zastosowaniach. Bezzałogowe statki powietrzne swoje początki
zawdzięczają celom militarnym, ale dzięki stabilności i łatwości
kontroli dronów wielowirnikowych, urządzenia te zyskują popularność w
zastosowaniach cywilnych, jak na przykład kinematografia.

\section[budowa-dronów-i-fizyka-silników]{Budowa dronów i fizyka
silników}

Klasycznym, najczęściej spotykanym wielowirnikowcem jest quadrokopter o
symetrycznych ramionach. W centrum konstrukcji znajdują się wszelkie
czujniki, moduły łączności bezprzewodowej i mikroprocesorowy sterownik
lotu. Do centralnej części urządzenia przymocowane są cztery
symetrycznie rozstawione ramiona o identycznej długości. Na końcu
każdego z ramion znajduje się silnik elektryczny poruszający śmigłem.

Parametrem każdego silnika jest jego ciąg oraz moment obrotowy. Ciąg
charakteryzuje siłę równoległą do osi obrotu śmigła, jaką silnik jest w
stanie wytworzyć. Moment obrotowy zaś, definiuję siłę jaka będzie
obracać silnik wokół osi obrotu śmigła. Obie te składowe należy wziąć
pod uwagę przy analizie sił działających na mikrośmigłowiec, aby
zagwarantować stabilny lot. W szczególności, nie zrównoważony moment
obrotowy będzie \quotation{obracał} dronem wokół pionowej osi.

W urządzeniach latających, zbudowanych w oparciu o wiele silników
wytwarzających ciąg przeciwstawny sile grawitacji, istnieje możliwość
zrównoważenia momentów obrotowych poprzez odpowiednie dobranie kierunków
obrotu poszczególnych silników. W typowym quadrokopterze silniki na
sąsiadującyh ramionach powinny obracać się w przeciwnych kierunkach.
Dzięki takiemu ustawieniu momenty obrotowe silników znoszą się, ponieważ
dla każdego kierunku obrotu mamy taką samą liczbę silników.

\section[fizyczne-podstawy-sterowania-lotem.]{Fizyczne podstawy
sterowania lotem.}

W klasycznym quadrokopterze sterowaniu podlegają cztery silniki.
Wysokość urządzenia nad powierzchnią ziemi można sterować równomiernie
zwiększając moc każdego z silników. Obracanie następuje przy zwiększeniu
mocy dwóch przeciwległych silników i zmniejszeniu dwóch pozostałych.
Dzięki temu sumaryczny ciąg pozostanie bez zmian i powstanie niezerowy
moment obrotowy, który obróci urządzenie. Nachyleniem urządzenia
względem płaszczyzny ziemi steruje się zwiększając moc jednego z
silników, nieznacznie obniżając przy tym pozostałych.

\section[czujniki]{Czujniki}

Aby sterowanie lotem urządzenia było możliwe niezbędne jest posiadanie
informacji o stanie w jakim obecnie pojazd się znajduje. Z takiego
punktu widzenia koniecznym jest wykorzystanie poniższych czujników przy
konstrukcji koptera: - Żyroskop -- pozwala mierzyć położenie kątowe co
pozwoli ocenić pod jakim kątem do płaszczyzny ziemi znajduje się
urządzenie. - Akcelerometr -- pozwala mierzyć przyspieszenie (we
wszystkich trzech wymiarach), wspomaga pracę żyroskopu - Barometr --
pozwala zmierzyć zmianę wysokości. Jako, że ciśnienie powietrza maleje
wraz z wysokością możliwa jest ocena zmiany wysokości na jakiej znajduje
się urządzenie. Dostępne są barometry pozwalające na pomiary z
dokładnością do dziesiątek centymetrów - Magnetometr -- pozwala zmierzyć
pole magnetyczne a tym samym ustalić położenie statku powietrznego
względem kierunków świata. Jest niezbędny do stabilizacji ruchu
obrotowego urządzenia.

\section[regulatory]{Regulatory}

Dane zebrane ze wszystkich czujników muszą zostać odpowiednio
przetworzone, aby mogły zostać wykorzystane do sterowania lotem
wielowirnikowca. Mamy tutaj do czynienia z pętlą sprzężenia zwrotnego.
Wskazania czujników odpowiadają pewnemu stanowi, chcąc np. przechylić
drona, zmieniamy moc silników, co skutkuje zmianą położenia urządzenia,
a więc też zmianą wskazań czujników. Problemem jest znalezienie
algorytmu, który tak dobierze moc silników, aby jak najszybciej osiągnąć
zamierzony stan.

Klasycznym rozwiązaniem tego problemu jest regulator PID. Na początku
obliczamy różnicę pomiędzy stanem obecnym, a oczekiwanym (np. wysokości
statku powietrznego nad poziomem ziemi) i nazywamy ją błędem. Wyjście
regulatora jest sumą trzech składowych: - P (proporcjonalnej) -- jest to
proste przemnożenie błędu przez stałą Kp. Różnica pomiędzy stanem
obecnym a oczekiwanym. - I (całkowej) -- przemnożenie całki błędu po
czasie przez stałą Ki. Zadaniem tej części jest akumulowanie błędów z
„przeszłości” tzn. kompensowanie stałych błędów. - D (różniczkowej) --
przemnożenie pochodnej błędu po czasie przez stałą Kd. Część
różniczkująca jest szczególnie wrażliwa na nagłe zmiany błędu.
Przyspiesza osiągnięcie zadanego położenia zwiększając szybkość reakcji.

Kluczowe znaczenie dla stabilności lotu i odpowiedniej reakcji
urządzenia jest poprawne dobranie parametrów Kp, Ki i Kd. Jest to
problem trudny, dla którego nie istnieją proste algorytmy. Współczynniki
te dobiera się doświadczalnie, pamiętając o ich znaczeniu.

\subsection[podsumowanie]{Podsumowanie}

Dzięki łatwości z jaką można zacząć pracę z MicroPythonem, szeregiem
wysokopoziomowych bibliotek do obsługi sprzętu oraz prostym (w
porównaniu do C) w użyciu językiem MicroPython wydaje się być idealną
platformą do tworzenia prototypów sprzętu. Nie bez znaczenia pozostaje
również maksymalnie uproszczony „deployment” - zaprogramowanie układu
oraz dostępna konsola REPL.

Na pewno układ ten nie jest rozwiązaniem wszystkich sprzętowych
problemów, ale warto mieć świadomość możliwości jakie ze sobą niesie.

\section[źródła]{Źródła}

\startitemize
\item
  Damien George \quotation{MicroPython -- Python for microconterollers}
  http://micropython.org/
\item
  Kick Starter - Micro Python: Python for microcontrollers
  https://www.kickstarter.com/
\item
  Jake Edge Micro \quotation{Python on the pyboard} http://lwn.net
\item
  Damien George \quotation{The 3 different code emitters} Update \#4 on
  https://www.kickstarter.com
\item
  Lauren Orsini „Why Copters Are The Next Big Thing In Robotics”
  http://readwrite.com
\item
  Oscar Liang „Quadcopter PID Explained and Tuning”
  http://blog.oscarliang.net
\item
  Naresh Kumar Thanigaivel „Building a Quadcopter”
  http://unmannedmulticopter.blogspot.com
\stopitemize


\stoptext
