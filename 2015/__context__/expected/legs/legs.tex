\usemodule[pycon-2015]
\starttext

\Title{Attaching legs to a snake - or - Python3 extensions}
\Author{Tomasz Maćkowiak}
\MakeTitlePage

\subsection[introduction]{Introduction}

Python is a powerful language by itself. The robust standard library and
the myriad of additional packages make it a Swiss Army knife for any
programming task. But sometimes it is not enough. When you need to
integrate with a low-level C API or execute heavy computations without
the interpreter overhead, writing a Python extension module might be a
good idea.

This article will show you how to write, build and run an extension;
what C API you can use in it and what to remember; how to link to shared
libraries and what does Boost.Python simplify.

Full source codes for all example modules below can be found on GitHub
{[}1{]}.

\subsection[overview-of-the-extensions-anatomy]{Overview of the
extension's anatomy}

\section[head]{Head}

Like any code written in C, Python extensions must load the appropriate
header file to let the compiler know what functions of the Python C API
are available. We usually only need to include one file:

\starttyping
#include <Python.h>
\stoptyping

\section[body]{Body}

Most Python functions take \type{PyObject} pointer argument (sometimes
more of them) and return a \type{PyObject} pointer. This is Python's
{\em everything is an object} put into practice.

\starttyping
static PyObject *
basic_hello(PyObject *self)
{
    const char *msg = "Hello world";
    return PyUnicode_FromString(msg);
}
\stoptyping

In the case of this module-level function, \type{self} parameter will be
the reference to the extension module the function is attached to.

\section[limbs]{Limbs}

We need to declare what functions our module exposes:

\starttyping
static PyMethodDef basic_methods[] = {
    {"hello", (PyCFunction)basic_hello,
        METH_NOARGS, "Return hello world."},
    {NULL, NULL, 0, NULL}        /* Sentinel */
};
\stoptyping

We declare a name under which the function will be visible, a pointer to
the function itself, type of arguments it expects (no arguments,
positional arguments or keyword arguments) and a docstring.

\section[nervous-system]{Nervous system}

We need to define the module itself:

\starttyping
static struct PyModuleDef basic_module = {
   PyModuleDef_HEAD_INIT,
   "basic",   /* name of module */
   "The simplest module", /* module documentation, may be NULL */
   -1,       /* size of per-interpreter state of the module,
                or -1 if the module keeps state in global variables. */
   basic_methods
};
\stoptyping

We pass its name, docstring, size of the optional module state memory
block and function definitions.

\section[heart]{Heart}

The most important, though, is the module initialization function that
creates the module on the basis of the module definition:

\starttyping
PyMODINIT_FUNC
PyInit_basic(void)
{
    return PyModule_Create(&basic_module);
}
\stoptyping

\subsection[animation---how-to-make-it-alive]{Animation - how to make it
alive}

The extensions are easy to build (on most platforms). We need a compiler
and Python development library:

\starttyping
sudo apt-get install python3-dev build-essential
\stoptyping

or an equivalent command for your system.

Extensions are declared in the \type{setup.py} file of our package:

\starttyping
from setuptools import find_packages, setup, Extension

basic = Extension('basic', sources=['src/basic_mod/basic.c'])

setup(
    name='pyext',
    version='0.1',
    ext_modules=[basic],
    packages=find_packages('src'),
    package_dir={'': 'src'},
)
\stoptyping

Once we have this, we can build our extension using

\starttyping
python3 setup.py build
\stoptyping

If we want to install it into the system or virtualenv, we need to
execute

\starttyping
python3 setup.py install
\stoptyping

As we can see no special commands are needed for installing packages
that have C extension modules. The extensions are automatically built
even if the package is just a dependency of another Python package.

\subsection[digestive-system---parsing-parameters]{Digestive system -
parsing parameters}

\section[positional-parameters]{Positional parameters}

To parse positional parameters we need to define our function with
\type{METH_VARARGS} flag in the module's functions definition:

\starttyping
{"hello", param_hello, METH_VARARGS, "Say hello."},
\stoptyping

With such a declaration, our function will get one more parameter,
similar to \type{*args} construct in pure Python:

\starttyping
static PyObject *
param_hello(PyObject *self, PyObject * args) {
    ...
\stoptyping

To parse the incoming parameters we can use the \type{PyArg_ParseTuple}
function. We need to pass into the function the incoming arguments
object and format string specifying what arguments you expect and a set
of pointers where the data parsed from the parameters should be
inserted. Before we call \type{PyArg_ParseTuple}, we need to allocate
space for the variables the parameters will be placed into:

\starttyping
const char * name;
unsigned age;
\stoptyping

Only then can we attempt to parse parameters:

\starttyping
if (!PyArg_ParseTuple(args, "sI", &name, &age)) {
    return NULL;
}
// Parameters parsed, carry on ...
\stoptyping

The description of the formatting parameters can be found in the docs
{[}2{]}. \type{s} means {\em convert to C string} (\type{char *}).
\type{I} means convert to \type{unsigned}. \type{O} means pass a Python
object and the parameter can be placed into a \type{PyObject *}
variable.

Notice that we are passing addresses of the variables (using the
\type{&} operator); this is what enables Python to write the values into
our variables in a return-parameter manner.

\section[keyword-parameters]{Keyword parameters}

If we want to give our users more freedom in passing parameters to our
extension function, we can use keyword parameters. We need to declare
the appropriate flag for our function:

\starttyping
{"belongs", (PyCFunction)key_belongs,
    METH_VARARGS | METH_KEYWORDS, "..."},
\stoptyping

thus making Python pass it one more \type{PyObject *}:

\starttyping
static PyObject *
key_belongs(PyObject *self, PyObject *args, PyObject *kwargs) {
    ...
\stoptyping

We need to define the names of the incoming parameters:

\starttyping
static char * keywords[] = {"mapping", "item", "category", NULL};
\stoptyping

and then we can call \type{PyArg_ParseTupleAndKeywords} function,
passing it both positional and keyword arguments:

\starttyping
if (!PyArg_ParseTupleAndKeywords(args, kwargs, "OOO", keywords,
        &mapping, &item, &category)) {
    return NULL;
}
// Carry on ...
\stoptyping

\subsection[hiccups---exceptions]{Hiccups - exceptions}

Most of Python API functions can indicate a failure. If the function is
supposed to return a \type{PyObject *}, it will return \type{NULL} when
it fails. The details of the exception are set in the per-thread
interpreter state.

If we detect a failed function call, we can just return the same
\type{NULL} from our function. The details of the original exception are
still stored within the interpreter, so if we don't modify the exception
state, the original exception will be used.

For example the \type{PyArg_ParseTuple} function can return \type{NULL}
if we pass an \type{int} where we were supposed to pass a \type{str}. It
will set the exception state to a \type{TypError} with message
\type{'must be str, not int'}. We can also set our own exception:

\starttyping
PyErr_SetString(PyExc_RuntimeError, "Cannot format output");
return NULL;
\stoptyping

Some functions (for example \type{__init__} C implementation) are
supposed to return an \type{int} status. To signal an encountered
exception we set the exception info using \type{PyErr_SetString} (or
leave the one already set if the exception is coming from a deeper
Python function call) and return \type{-1} from our function.

\subsection[bones---api]{Bones - API}

The API we can use in our Python C extensions is quite vast. We can read
all about it in the Python docs {[}2{]}. API is split into sections, so
all functions dealing with \type{str} are in one section (fun fact: in
the API \type{str} is still referred to as \type{Unicode}, for example
\type{PyUnicode_From...}), etc. We can find the equivalent calls for our
Python constructs. Here are some examples:

To get an item under a given key in a dictionary

\starttyping
category_sequence = mapping[category]
\stoptyping

, we use:

\starttyping
PyObject * category_sequence = PyObject_GetItem(mapping, category);
if (category_sequence == NULL) {
    return NULL;
}
// Deal with the object ...
\stoptyping

To check if a sequence contains given item

\starttyping
contains = item in category_sequence
\stoptyping

, we use:

\starttyping
int contains = PySequence_Contains(category_sequence, item);
if (contains == -1) {
    return NULL; // error, for example KeyError
} elif (contains == 0) {
    ... // doesn't contain
} else {
    ... // does contain
}
\stoptyping

\subsection[population-size---reference-counting]{Population size -
reference counting}

Python automatically manages memory using {\bf reference counting} and a
cyclic garbage collector. Reference counting means that for each Python
object (\type{PyObject}) the interpreter stores a count of how many
other objects are referencing it.

Supposing we have two dictionaries:

\starttyping
dict_a = {'a': 'VALUE'}
dict_b = {}
\stoptyping

The \type{str} object \type{'VALUE'} has a reference count of \type{1};
only the object \type{dict_a} is referencing it. Once we do:

\starttyping
dict_b['b'] = dict['a']
\stoptyping

our \type{str} object is referenced by both \type{dict_a} and
\type{dict_b} so its reference count is raising to \type{2}. If we
remove both references:

\starttyping
del dict_a['a']
del dict_b['b']
\stoptyping

then our \type{str} is no longer referenced by anything, its reference
count drops to \type{0} and the interpreter knows that this object is no
longer used, so the memory it occupied can be freed and reused later on.

This process of counting references is happening automatically in pure
Python, but requires manual support when dealing with Python objects in
C extensions.

We need to manually increase the reference count on an object using

\starttyping
Py_INCREF(result);
\stoptyping

and decrease it using

\starttyping
Py_DECREF(category_sequence);
\stoptyping

macros.

Knowing when to increase ref. count and when to decrease it is one of
the hardest things to get right. When using any Python API function we
need to check (in the docs {[}2{]}) if it returns {\em new reference} or
{\em borrowed reference}.

The former means that the object returned by the API function already
has the reference count increased, so we need to decrease it when we are
done with it.

In the latter case the reference count was not increased - our code
didn't become one of the owners of the object's reference, so there is
no need to decrease it when we are done with it. But if we would like to
return it from our function or store it, we need to increase the
reference count to make sure that Python will not deallocate that
object.

Check out the example of dealing with references:

\starttyping
PyObject *mapping = ...;
PyObject *item = ...;
PyObject *category = ...;

/* GetItem returns a new reference so the object's ref. count
   needs to be decremented */
PyObject *category_sequence = PyObject_GetItem(mapping, category);
if (category_sequence == NULL) {
    return NULL;
}

int contains = PySequence_Contains(category_sequence, item);
Py_DECREF(category_sequence); // We will no longer need this object.
if (contains == -1) {
    return NULL;
}

PyObject * result = contains ? Py_True : Py_False;
// True and False are singleton-like objects, if we want to return them
// from our code, we need to increase their ref. count
Py_INCREF(result);

return result;
\stoptyping

There are also \type{Py_XINCREF} and \type{Py_XDECREF} macros that,
first of all, check if the pointer is not \type{NULL}, and when it is,
they are a no-op.

\subsection[species---classes]{Species - classes}

It is possible to have classes implemented in C. It requires extending
the code of our extension module with additional elements.

An additional header needs to be included:

\starttyping
#include <structmember.h>
\stoptyping

Then the structure (fields) of the class objects needs to be defined:

\starttyping
typedef struct {
    PyObject_HEAD // Required header fields
    char * pointer;
    long number;
    PyObject * name;
} Native;
\stoptyping

The structure needs to start with the required fields from
\type{PyObject_HEAD} macro, but the rest of the members can be defined
freely by the developer. The fields can reference other Python objects
(\type{PyObject *}), they can be primitive types (\type{long}), pointers
(\type{char *}) or any other type, even if Python is not be able to
apply any default conversion to it.

Once we define the structure, we can also define a Python mapping of
fields, so that we will be able to access them straight from Python
(\type{obj = Native(...); obj.name}):

\starttyping
static PyMemberDef Native_members[] = {
    {"name", T_OBJECT_EX, offsetof(Native, name), 0, "Name"},
    {"number", T_LONG, offsetof(Native, number), 0, "Number"},
    {"pointer", T_STRING, offsetof(Native, pointer), READONLY, "Pointer"},
    {NULL}  /* Sentinel */
};
\stoptyping

For each member we define a name, type, offset in class structure, flags
and a docstring.

We can also define which methods will be available on our objects:

\starttyping
static PyMethodDef Native_methods[] = {
    {"summary", (PyCFunction)Native_summary, METH_NOARGS,
        "Return the name and the other attributes formatted"},
    {NULL}  /* Sentinel */
};
\stoptyping

The methods will receive a pointer to a \type{Native} instance as the
first parameter:

\starttyping
static PyObject *
Native_summary(Native *self)
{
    if (self->name == NULL) {
        PyErr_SetString(PyExc_AttributeError, "name");
        return NULL;
    }

    return PyUnicode_FromFormat(
        "Native %S number %li pointer %s",
        self->name, self->number, self->pointer
    );
}
\stoptyping

The most important bit is the class definition:

\starttyping
static PyTypeObject NativeType = {
    PyVarObject_HEAD_INIT(NULL, 0)
    "obj.Native",              /* tp_name */
    sizeof(Native),            /* tp_basicsize */
    0,                         /* tp_itemsize */
    (destructor)Native_dealloc,/* tp_dealloc */
    0,                         /* tp_print */
    0,                         /* tp_getattr */
    0,                         /* tp_setattr */
    0,                         /* tp_reserved */
    0,                         /* tp_repr */
    0,                         /* tp_as_number */
    0,                         /* tp_as_sequence */
    0,                         /* tp_as_mapping */
    0,                         /* tp_hash  */
    0,                         /* tp_call */
    0,                         /* tp_str */
    0,                         /* tp_getattro */
    0,                         /* tp_setattro */
    0,                         /* tp_as_buffer */
    Py_TPFLAGS_DEFAULT |
        Py_TPFLAGS_BASETYPE,   /* tp_flags */
    "Native objects",          /* tp_doc */
    0,                         /* tp_traverse */
    0,                         /* tp_clear */
    0,                         /* tp_richcompare */
    0,                         /* tp_weaklistoffset */
    0,                         /* tp_iter */
    0,                         /* tp_iternext */
    Native_methods,            /* tp_methods */
    Native_members,            /* tp_members */
    0,                         /* tp_getset */
    0,                         /* tp_base */
    0,                         /* tp_dict */
    0,                         /* tp_descr_get */
    0,                         /* tp_descr_set */
    0,                         /* tp_dictoffset */
    (initproc)Native_init,     /* tp_init */
    0,                         /* tp_alloc */
    Native_new,                /* tp_new */
};
\stoptyping

In this structure we can define some of the special functions that we
implement for our type. In the most common case we implement
\type{__new__}, \type{__init__} and object deallocation.

A sample \type{__new__} implementation:

\starttyping
static PyObject *
Native_new(PyTypeObject *type, PyObject *args, PyObject *kwargs)
{
    Native *self;
    /* Call the base allocator */
    self = (Native *)type->tp_alloc(type, 0);
    if (self == NULL) {
        return NULL; // Failed to allocate.
    }
    self->number = 0;
    self->name = PyUnicode_FromString("");
    if (self->name == NULL) {
        Py_DECREF(self);
        return NULL;
    }
    self->number = 0;
    self->pointer = (char *)malloc(sizeof(char) * 4);
    if (self->pointer == NULL) {
        Py_DECREF(self->name);
        Py_DECREF(self);
        return PyErr_NoMemory();
    }
    strcpy(self->pointer, "?");

    return (PyObject *)self;
}
\stoptyping

A sample \type{__init__} implementation:

\starttyping
static int
Native_init(Native *self, PyObject *args, PyObject *kwargs)
{
    PyObject * name = NULL;
    PyObject * tmp;
    int yes_no;

    static char *kwlist[] = {"name", "number", "yes", NULL};

    if (!PyArg_ParseTupleAndKeywords( // l = long, p = boolean evaluation
        args, kwargs, "Olp", kwlist, &name, &self->number, &yes_no
    )) {
        return -1;
    }

    if (name) {
        tmp = self->name;
        Py_INCREF(name);
        self->name = name;
        Py_XDECREF(tmp);
    }

    strcpy(self->pointer, yes_no ? "YES" : "NO");

    return 0;
}
\stoptyping

Notice how this function returns an \type{int}; \type{__init__} cannot
return any other object but is used to initialize the \type{self}
object.

Sample deallocation implementation:

\starttyping
static void
Native_dealloc(Native * self)
{
    Py_XDECREF(self->name);
    if (self->pointer != NULL) {
        free(self->pointer);
    }
    Py_TYPE(self)->tp_free((PyObject *)self);
}
\stoptyping

The last thing that remains, is to initialize and connect our type when
module loads:

\starttyping
PyMODINIT_FUNC
PyInit_obj(void)
{
    PyObject *module;

    if (PyType_Ready(&NativeType) < 0) {
        return NULL;
    }

    module = PyModule_Create(&obj_module);
    if (module == NULL) {
        return NULL;
    }

    Py_INCREF(&NativeType);
    PyModule_AddObject(module, "Native", (PyObject *)&NativeType);

    return module;
}
\stoptyping

\subsection[gil-and-threading]{GIL and threading}

Python has the Global Interpreter Lock; only one thread at a time can be
executing any Python code. This property of the language is making it
better suited for {\em multi-process} setups than for {\em multi-thread}
setups. But inside the code of our extension module we can declare a
block as {\em not-using-Python} - not executing {\bf any} Python API
functions and not operating on any Python structures passed to it as
arguments, etc. Such code can be executed {\em while} some other thread
{\em is} executing different Python code {\bf at the same time}. We
should be releasing the interpreter whenever a blocking operation is
being executed as long as it doesn't use any Python structures.

Here is an example of how to wrap computations so that they don't hold
the interpreter and so that they can be executed in threads in parallel
with other Python code:

\starttyping
static PyObject *
gil_calc_release(PyObject *self, PyObject *args)
{
    long n;
    if (!PyArg_ParseTuple(args, "l", &n)) {
        return NULL;
    }
    long result;

    Py_BEGIN_ALLOW_THREADS
    result = fibonacci(n);
    Py_END_ALLOW_THREADS

    return PyLong_FromLong(result);
}
\stoptyping

\subsection[boost.python]{Boost.Python}

A popular C++ library {\em Boost} provides the {\em Boost.Python} module
for easier writing of Python extensions in C++.

\section[c-code]{C++ code}

With {\em Boost} a different header is used:

\starttyping
#include <boost/python.hpp>

using namespace boost::python;
\stoptyping

Functions can be declared with C++ arguments that will be automatically
parsed by {\em Boost.Python}:

\starttyping
bool has_letter(const char * text, const char letter) {
    const char * ptr = text;
    while (char c = *(ptr++)) {
        if (c == letter) {
            return true;
        }
    }
    return false;
}
\stoptyping

To register such a function we don't need to create a module definition
or module members structure. All we need is the module initialization
function:

\starttyping
BOOST_PYTHON_MODULE(boost)
{
    def("has_letter", has_letter);
}
\stoptyping

Extensions written using {\em Boost.Python} can be much more concise.

\section[compiling-and-linking]{Compiling and linking}

{\em Boost.Python} comes as a shared library. That means that during
compilation and executing of our extension, Python needs to be able to
read the library's files. If the library is installed system-wide, we
don't have to worry about paths. If the library is installed locally, we
need to remember to pass the correct include path and library path
during compilation and to have the \type{LD_LIBRARY_PATH} system
variable correctly set during running.

The \type{setup.py} for a {\em Boost.Python} extension can look like
this:

\starttyping
boost = Extension(
    'boost',
    sources=['src/boost_mod/boost.cpp'],
    include_dirs=[os.path.join(BOOST_DIR, 'include')],
    libraries=["boost_python"],
    library_dirs=[os.path.join(BOOST_DIR, 'lib')],
)
\stoptyping

Please note, we specify here that our extension should be linked to the
\type{boost_python} shared library.

{\em Boost.Python} also allows us to define classes in a simplified
syntax.

\starttyping
struct Native
{
    std::string name;
    long number;
    std::string pointer;

    Native(std::string name, long number, bool yes
            ): name(name), number(number) {
        this->pointer = std::string(yes ? "YES" : "NO");
    }

    std::string summary() {
        std::stringstream ss;
        ss << "Native " << this->name << " number " << this->number <<
            " pointer " << this->pointer;
        return ss.str();
    }
};

BOOST_PYTHON_MODULE(boost)
{
    class_<Native>("Native", init<std::string, long, bool>())
        .def_readwrite("name", &Native::name)
        .def_readwrite("number", &Native::number)
        .def_readonly("pointer", &Native::pointer)
        .def("summary", &Native::summary);
}
\stoptyping

More information about {\em Boost.Python} can be found in the library's
docs {[}4{]}.

\subsection[summary]{Summary}

Python extensions are a great tool for pushing the boundaries of what
Python can do. Whether we want to code some calculations to work faster
without the interpreter's overhead or we want to integrate with a shared
library, extensions provide us with a way of doing it while still
keeping the usual package installation procedure.

Extensions are powerful and we can implement some really cool stuff with
them {[}3{]}, but we need to be very careful. Dealing with low level C
is dangerous on its own and dealing with Python's internals at the same
time adds to the complexity. One \type{PyDECREF} missing and we will
have memory leaks. One \type{PyDECREF} too many and our extension will
crash the whole interpreter with a core dump. Good luck!

\subsection[references]{References}

\startitemize
\item
  {[}1{]} GitHub repo with full source code of all examples\crlf
  https://github.com/kurazu/pyext
\item
  {[}2{]} Python/C API Reference Manual
  https://docs.python.org/3/c-api/index.html
\item
  {[}3{]} GitHub repo of a Python-SpiderMonkey integration library\crlf
  https://github.com/kurazu/bridge
\item
  {[}4{]} Boost.Python documentation\crlf
  http://www.boost.org/doc/libs/1\_59\_0/libs/python/doc/
\stopitemize


\stoptext
