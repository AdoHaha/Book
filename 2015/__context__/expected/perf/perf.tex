\usemodule[pycon-2015]
\starttext

\Title{What's Eating Python performance}
\Author{Piotr Przymus}
\MakeTitlePage

Have you ever wondered how to speed up your code in Python? This
presentation will show you how to start. I will begin with a guide how
to locate performance bottlenecks and then give you some tips how to
speed up your code. Also, I would like to discuss how to avoid premature
optimization as it may be \quote{the root of all evil} (at least
according to D. Knuth).

\subsection[introduction]{Introduction}

Code optimization is an important aspect of the development process,
but, when done improperly, it may do more harm than good. This talk is
intended as a gentle introduction into good practices in Python code
optimization. Certainly this does not exhaust the topic, and there are
various great resources where you can find more on the subject, see the
references for some starting points.

This talk mainly focuses on CPython (both \lettermore{}= 2.7 and
\lettermore{}= 3.4) Python implementation.

\subsection[improving-the-performance-overview]{Improving the
performance: overview}

\section[the-root-of-all-evil-premature-optimization]{The root of
all evil: premature optimization}

*Programmers waste enormous amounts of time thinking about, or worrying
about the speed of noncritical parts of their programs, and these
attempts at efficiency actually have a strong negative impact when
debugging and maintenance are considered. We should forget about small
efficiencies, say about 97\% of the time: premature optimization is the
root of all evil. \lettertilde{} * {\bf Donald Knuth},
\quotation{Structured Programming With Go To Statements}.

In short, premature optimization may be stated as optimizing code before
knowing whether we need to. This is a dangerous practice that impacts
your productivity, readability of the code and ease of maintenance and
debugging (it may also contradict many points of The Zen of Python).

Thus, it is important to learn how to do proper assessment of your code
in terms of optimization needs. (Remember that a strong felling that
your code falls into the remaining 3\% does not count!)

Yet do not be discouraged from learning the proper way of optimizing
your code, and remember about second part of the previous quote:

{\em Yet we should not pass up our opportunities in that critical 3\%.}
{\bf Donald Knuth}

Certain optimizations are a part of good programming style and good
practices, and therefore should not be considered as premature. For
example, moving computations that do not depend on the loop, outside the
loop -- as this also improves code readability.

\subsection[think-before-doing-think-before-coding]{Think before
doing (Think before coding)}

Going for higher performance without a deeper reason may be just a waste
of your time. So start with:

\startitemize
\item
  stating your reasons (Why do you need higher performance?),
\item
  defining your goals (What would be an acceptable speed of your code?),
\item
  estimating time and resources you are willing to spend to achieve
  these goals.
\stopitemize

Re-evaluate all the pros and cons.

\subsection[test-measure-track-down-bottlenecks-fix]{Test, Measure,
Track Down bottlenecks, Fix}

A starting point for optimization is a running code that gives correct
results. Then you should prepare a regression test suite, which will
stand on guard of the correctness of your code during the optimization.

Then rest of the optimization process may be summarized as:

\startitemize[n][stopper=.]
\item
  Test if the code works correctly.
\item
  Measure execution time and if code is not fast enough use a profiler
  to identify the bottlenecks.
\item
  Optimize.
\item
  Start from the beginning.
\stopitemize

\section[regression-test-suite]{Regression test suite}

Before you start, it is crucial to prepare regression test suite that is
comprehensive but yet quick-to-run. As test will be ran very often, it
is important that it takes a reasonable amount of time. Consider
breaking the tests into fast and slow categories if running full test
suite takes too much time.

\section[measuring-execution-time-and-tracking-down-the-bottlenecks-measuring-and-profiling-tools]{Measuring
execution time and tracking down the bottlenecks: Measuring and
profiling tools}

Next, you should measure execution time of your code. This is important
because:

\startitemize
\item
  it shows how current execution time relates to the desired execution
  time (i.e.~acceptable speed),
\item
  it allows you to compare various version of optimizations.
\stopitemize

There are various tools to do that, among them:

\startitemize
\item
  Python's timeit module,
\item
  custom made timer using Pythons time module,
\item
  unix time (use /usr/bin/time as time is also a common shell built in).
\stopitemize

Notes on measuring:

\startitemize
\item
  Try to measure multiple independent repetitions of your code to
  establish the lower bound of your execution time.
\item
  Prepare a testing environment that will allow you to get comparable
  results.
\item
  Consider writing a micro benchmark to check various alternative
  solutions of the same algorithm.
\item
  Be careful when measuring algorithm speed using artificial data,
  re-validate using real data.
\stopitemize

Profiling tools will give you a more in depth view of your code
performance. This will allow you to take a view of your program
internals in terms of execution time and used memory.

There are various possible tools, like:

\startitemize
\item
  cProfile -- a profiling module available in Python standard library,
\item
  line\letterunderscore{}profiler -- an external line-by line profiler,
\item
  tools for visualizing profiling results such as runsnakerun.
\stopitemize

\subsection[io-bound-vs-compute-bound]{IO bound vs compute bound}

It is important that you learn how to classify types of performance
bounds. {\bf The compute bound} case occurs when large number of
instructions is making your code slow, the {\bf I/O bound} case takes
place when your code is slow because of various I/O operations, like
network delays or disk access. Depending on the type of the bound,
different optimization strategies will apply.

\section[fixing-the-cause-performance-tips]{Fixing the cause:
Performance Tips}

\subsection[algorithms-and-data-structures]{Algorithms and data
structures}

Improving your algorithms time complexity is probably the best thing you
could do to optimize your code. Using micro optimization tricks will not
bring you anywhere near to the speed boost you could get from improving
time complexity of algorithm.

It is very common that innocently looking, searching or lookup code
placed in a large loop generates a performance issue. Often a trivial
change, like changing a list to a set, may be the key to solving the
problem.

That said, be sure to check
\useURL[url1][https://wiki.python.org/moin/TimeComplexity][][Time
complexity]\from[url1] from Python's wiki and confront it with the data
structures used in your algorithms. The big {\bf O} notation matters!

\subsection[improving-the-code-work-flow]{Improving the code work
flow}

Often you may encounter a loop with code that performs expensive
time/memory computations that do not change within loop. Fix it by
moving those operations outside a loop (just before the loop). Beware
that such operations may be hidden in a class method or in a free
function.

Try to avoid conditional branching in large loops. Check whatever
instead of having if/else statements in the loop body:

\startitemize
\item
  it is possible to do the conditional check outside the loop,
\item
  have separate loops for different branches.
\stopitemize

Python introduces relatively high overhead for function/method calls. In
some cases it may be worth to consider code inlining to avoid the
overhead -- but this comes at a cost of code maintenance and
readability.

There are also other techniques, like improving
function/method/variable/attribute lookup times or loop unrolling.

\subsection[memory-and-io-bounds]{Memory and I/O bounds}

Some performance issues may be memory related, so checking memory
utilization is also a good idea. Typical symptoms that indicate that
your code may have memory problems:

\startitemize
\item
  your program never releases memory,
\item
  or your program allocates way too much memory.
\stopitemize

It is also worth to check if your code uses memory efficiently. You may
find more information on this topic in my previous talk
\quotation{Everything You Always Wanted to Know About Memory in Python
But Were Afraid to Ask} and references included therein.

I/O bounds may require more effort to deal with. Depending on the
problem there may be various solutions, consider using:

\startitemize
\item
  asynchronous I/O with Python (see, for example, \quotation{Journey to
  the center of the asynchronous world}),
\item
  compressed data structures and lightweight compression algorithms
  (i.e.~algorithms that are primarily intended for real-time
  applications, which favours compression/decompression speed over
  compression ratio),
\item
  probabilistic data structures (like Bloom filters instead of real
  data).
\stopitemize

\subsection[notes-on-the-special-cases]{Notes on the special
cases}

When your code involves numerics, consider using numpy and scipy. These
libraries provide highly optimized routines (usually based on external
scientific libraries).

Some problems may just need more computing power, so it may be a good
idea to:

\startitemize
\item
  write code that utilizes multi core architecture (mutliprocessing),
\item
  or scale your code to multiple machines (task queues, spark, grid like
  environment),
\item
  or using hardware accelerators (pyOpenCL, pyCuda, pyMIC, etc.)
\stopitemize

You may also consider pushing performance-critical code into C.

Remember to check your code with PyPy, you may be pleasantly surprised.

\subsection[summary]{Summary}

Hopefully, this talk will help you start with Python optimization.

For more, check slides and source code examples on my website
przymus.org, under \quotation{What's Eating Python performance}.

\subsection[references]{References}

\startitemize[n][stopper=.]
\item
  PythonSpeed, https://wiki.python.org
\item
  PythonSpeed / Performance Tips, https://wiki.python.org
\item
  Time complexity, https://wiki.python.org
\item
  PythonSpeed / Profiling Python Programs,https://wiki.python.org
\item
  Performance, http://pypy.org
\item
  Everything You Always Wanted to Know About Memory in Python But Were
  Afraid to Ask, http://przymus.org
\item
  Journey to the center of the asynchronous world,
  https://github.com/PyConPL/Book
\stopitemize


\stoptext
