\usemodule[pycon-2015]
\starttext

\Title{Locust.io - nowoczesne narzędzie do load testów}
\Author{Jacek Nosal}
\MakeTitlePage

\subsection[wprowadzenie]{Wprowadzenie}

Testowanie obciążenia aplikacji webowych to temat bardzo złożony, który
nabiera na znaczeniu, jeżeli pod uwagę weźmiemy skalowalne architektury.
Coraz częściej spotyka się statystyki, które pokazują, jak wielu
użytkowników rezygnuje ze skorzystania z danej aplikacji, jeżeli czasy
odpowiedzi są wyższe niż pewien (niski) próg. Im szybsze odpowiedzi, tym
więcej użytkowników wejdzie na naszą stronę, ponieważ szybciej będą w
stanie ocenić czy znajduje się na niej to, czego szukają.

\startitemize
\item
  Google analizując ruch w internecie stwierdziło, że półsekundowe
  opóźnienie w ładowaniu strony skutkuje spadkiem ruchu na poziomie 20\%
\item
  Amazon podczas A/B testów zauważył, że wzrost czasów odpowiedzi o 100
  ms skutkował w ich przypadku spadkiem sprzedaży na poziomie 1\% (co
  dla tak dużej firmy jest znaczącą wartością)
\stopitemize

Typowy odwiedzający nie dba również o stronę techniczną, bo po prostu
się na tym nie zna. Aplikacja może cachować dużo danych, autoskalować
się, zepewniać failover - to wszystko nie ma wielkiego znaczenia, jeżeli
jest po prostu wolne. Naturalnym wydaje się więc posiadanie możliwości
symulacji dużego ruchu, który sprawdzi możliwości naszej aplikacji i
pokaże, jak cały stos sprawuje się, gdy zostanie \quotation{nawiedzony}
przez tysiące użytkowników jednocześnie.

Spośród wielu dostępnych narzędzi na rynku, jedno wyróżnia się
szczególnie - locust.io. Po pierwsze: napisane jest w Pythonie. Po
drugie: do jego obsługi wykorzystujemy Pythona. Po trzecie: jak
większość frameworków jest niesamowicie elastyczne i stworzone tak, żeby
podmiana podstawowych komponentów była transparentna i łatwa dla
programisty.

\subsection[locust.io]{Locust.io}

Locust to lekkie, open sourcowe narzędzie, które umożliwia stworzenie w
Pythonie scenariuszy do interakcji z aplikacją, które będą użyte do
symulacji użytkowników końcowych. Narzędzie potrafi symulować
praktycznie dowolną liczbę znajdujących się w osobnych sesjach
użytkowników.

Od strony technicznej narzędzie to spójne połączenie flaska, geventa
oraz pyzmq, które na pojedynczej maszynie doskonale nadają się do
symulowania tysięcy użytkowników. Jeżeli jednak chcielibyśmy zwiększyć
ruch do setek tysięcy bądź milionów jednoczesnych wizyt, to istnieje
możliwość uruchomienia testów rozłożonych na wiele maszyn.

Do zalet locusta należą:

\startitemize
\item
  możliwość pisania scenariuszy testowych w Pythonie
\item
  skalowalność
\item
  panel webowy do kontrolowania procesu testowania
\item
  możliwość przetestowania każdego systemu (nie tylko tych, opierających
  się na HTTP)
\stopitemize

Narzędzie to rozwiązuje jeszcze jeden problem: upraszcza proces
planowania i przeprowadzania testów obciążeniowych. Większość rozwiązań
jest konfigurowana przez skomplikowane xml-e. bądź udostępnia
\quote{tępe} panele webowe, które pozwalają nam wyklikać różne akcje,
jakie zostaną zasymulowane. W przypadku locusta wszystko sprowadza się
do Pythona i napisania skryptu, który wszystko kontroluje, czyli, innymi
słowy, typowego programistycznego zadania.

Instalacja locusta sprowadza się do zainstalowania jednej paczki:

\starttyping
pip install locustio
\stoptyping

Cała logika testowa znajduje się w pliku {\em locustfile.py}, który po
wpisaniu komendy {\em locust} jest użyty do skonfigurowania środowiska.

\starttyping
locust -f locustfile.py
\stoptyping

Po uruchomieniu środowiska możemy przystąpić do konfiguracji testu. Do
tego celu locust dostarcza panel webowy dostępny standardowo pod adresem
{\em localhost:8089}, w którym zostaniemy poproszeni o podanie liczby
użytkowników oraz hatch rate - parametru, który odpowiada za
częstotliwość przyrostu liczby użytkowników na sekundę, aż do
osiągnięcia zadanej liczby. Przykładowo podanie liczb 1000 i 50 skutkuje
dwudziestosekundową \quote{rozgrzewką} do właściwej symulacji dla
tysiąca użytkowników.

Jeżeli nie wystarcza nam pojedyncza maszyna, w prosty sposób możemy
zaprząc do współpracy inne środowiska: wystarczy uruchomić locusta w
trybie master na jednym z nich oraz w trybie slave na pozostałych:

Master: \type{python locust -f locustfile.py --master}

Slave:
\type{python locust -f locustfile.py --slave --master-host=<adres-ip-mastera>}

Maszyna w trybie master nie przeprowadza symulacji, a jedynie zbiera
statystyki z maszyn w trybie slave oraz udostępnia interfejs webowy z
podglądem całego procesu.

Komponenty:

\startitemize
\item
  Locust oraz HttpLocust
\item
  TaskSet
\item
  Event hooks
\stopitemize

{\em Locust} to klasa, której obecność w pliku locustfile.py jest
wymagana, żeby locust poprawnie się uruchomił. Instancje tej klasy
reprezentują użytkowników systemu (symulacja ruchu dla 5000 użytkowników
spowoduje 5000 instancji klasy Locust). Jeżeli nasz serwis działa na
podstawie protokołu HTTP, to możemy wykorzystać klasę {\em HttpLocust},
która dodatkowo zapewni obsługę sesji oraz ciasteczek.

{\em TaskSet} to nic innego jak kolekcja zadań (które mogą być funkcjami
bądź klasami), która będzie nieprzerwanie wykonywana przez instancję
klasy Locust.

{\em Event hooks} to zdarzenia, które służą do komunikacji z locustem w
celu zbierania statystyk dotyczących żądań i odpowiedzi. Do dyspozycji
mamy:

\startitemize
\item
  request\letterunderscore{}success
\item
  request\letterunderscore{}failure
\item
  locust\letterunderscore{}error
\item
  report\letterunderscore{}to\letterunderscore{}master
\item
  slave\letterunderscore{}report
\item
  hatch\letterunderscore{}complete
\item
  quitting
\stopitemize

\subsection[scenariusze-testowe]{Scenariusze testowe}

Wspomniany {\em TaskSet} służy jako scenariusz testowy - jest to zestaw
definicji operacji wykonywanych przez użytkownika, takich jak:

\startitemize
\item
  nawigacja do podstrony
\item
  interakcja z formularzem webowym
\item
  przeprowadzenie operacji wyszukiwania
\item
  zalogowanie / wylogowanie
\item
  w praktyce: jakiejkolwiek akcji, którą może wykonać użytkownik naszego
  serwisu
\stopitemize

Bardzo prosty {\em locustfile.py}, który posłuży do przeprowadzenia
symulacji wejścia na stronę główną, zalogowania i wyszukania produktu
może wyglądać na przykład tak:

\starttyping
from locust import HttpLocust, TaskSet, task


class MyCustomBehaviourTaskSet(TaskSet):
    # Klasa reprezentująca zestaw zadań jaki ma zostać
    # wykonany przez użytkownika

    @task(1)
    def get_index(l):
        l.client.get('/')

    @task(1)
    def search_something(l):
        l.client.get('/?q=%s' % 'query')

    @task(1)
    def submit_form(l):
        l.client.post('/submit/form/', data='form-data')


class MyCustomScenarioUser(HttpLocust):
    # Klasa reprezentująca użytkownika systemu, który ma wykonać
    # zdefiniowany zestaw zadań: MyCustomBehaviourTaskSet
    # względem serwisu bazującego o protokół HTTP
    task_set = MyCustomBehaviourTaskSet
    min_wait = 1000
    max_wait = 5000
\stoptyping

Nie pozostaje nam nic innego, jak uruchomić locust:

\starttyping
locust -f locustfile.py --host=http://host.pl
\stoptyping

\subsection[jak-to-wygląda-w-praktyce-i-co-dalej]{Jak to wygląda w
praktyce i co dalej}

Locust idealnie nadaje się do sprawdzania serwisów działających nie
tylko w oparciu o protokół HTTP. Poniżej znajduje się klient nanomsg
działający na socketach REQ \letterless{}-\lettermore{} REP. Kod byłby o
wiele krótszy, gdybyśmy korzystali z jednej z gotowych bibliotek
dostarczających klasy klienckie. Podpięcie całości pod locusta sprowadza
się do zmodyfikowania metody odpowiadającej za wysyłanie requestów do
serwisu i wysłaniu odpowiednich zdarzeń do locusta, które posłużą za
metryki. Następnie tworzymy własną klasę NanomsgLocust reprezentującą
użytkownika, który jest klientem serwisu działającego na nanomsg i
gotowe. Całość prezentuje się następująco:

\starttyping
# -*- coding: utf-8 -*-
import json
import time
import nanomsg

from locust import Locust, events, task, TaskSet


class MySocket(nanomsg.Socket):
    # Klasa dziedzicząca po nanomsg.Socket, dokładająca
    # dwie metody send_json oraz recv_json, które ułatwiają
    # pracę z tym formatem

    def send_json(self, msg, flags=0, **kwargs):
        msg = json.dumps(msg, **kwargs).encode('utf8')
        self.send(msg, flags)

    def recv_json(self, buf=None, flags=0):
        msg = self.recv(buf, flags)
        return json.loads(msg)


class NanomsgClient(object):
    # Właściwy klient nanomsg, pracujący w trybie komunikacji
    # REQ - REP (request - response)
    # więcej informacji o trybach komunikacji można znaleźć
    # w oficjalnej dokumentacji: http://nanomsg.org/

    socket_type = nanomsg.REQ

    # timeout dla operacji wysyłania, ustawiony w milisekundach
    default_send_timeout = 100

    def __init__(self, endpoint, **kwargs):
        self.endpoint = endpoint
        self.setup()

    def setup(self):
        # utworzenie i połączenie gniazda pod wskazany w __init__
        # endpoint oraz konfiguracja timeoutu
        self.socket = MySocket(self.socket_type)
        self.socket.connect(self.endpoint)
        self.socket._set_send_timeout(self.default_send_timeout)

    def get(self, msg):
        # metoda wysyła zserializowaną do formatu json
        # wiadomość i czeka na odpowiedź z serwisu, dodatkowo
        # wysyłając informację o czasach requestu do locusta
        # poprzez użycie komponentu Event
        start_time = time.time()

        try:
            self.socket.send_json(msg)
            result = self.socket.recv_json()
            print result
        except nanomsg.NanoMsgAPIError as e:
            total_time = int((time.time() - start_time) * 1000)
            events.request_failure.fire(
                request_type="nanomsg",
                name=msg.get('executable', ''),
                response_time=total_time,
                exception=e
            )
        else:
            total_time = int((time.time() - start_time) * 1000)
            events.request_success.fire(
                request_type="nanomsg",
                name=msg.get('executable', ''),
                response_time=total_time,
                response_length=0
            )

    def close(self):
        self.socket.close()


class NanomsgUser(Locust):
    endpoint = "tcp://127.0.0.1:5001"

    # Parametry reprezentujące (w milisekundach) minimalny
    # i maksymalny czas, jaki użytkownik powinien odczekać przed
    # wykonaniem kolejnego zadania.

    min_wait = 100
    max_wait = 1000

    def __init__(self, *args, **kwargs):
        # Przeładowujemy __init__ i podajemy naszą klasę
        # klienta, która będzie użyta do komunikacji z serwisem
        # działającym na nanomsg przez locusta.

        super(NanomsgUser, self).__init__(*args, **kwargs)
        self.client = NanomsgClient(self.endpoint)

    class task_set(TaskSet):
        # Zestaw zadań, jakie ma wykonać symulowany użytkownik
        # będzie to wysłanie dwóch wiadomości json do serwisu.

        # Zadania tworzymy poprzez dekorator @task
        # podając mu opcjonalnie wskaźnik wykonania
        # decydujący o tym jak często, w kontekście do innych
        # takie zadanie wykonywać

        @task(1)
        def ping(self):
            self.client.get({'executable': 'ping'})

        @task(1)
        def pong(self):
            self.client.get({'executable': 'pong'})
\stoptyping

Wnioski nasuwają się same - locust nie ogranicza się do pracy wyłącznie
z serwisami wspierającymi protokół HTTP. Testowanie architektur
mikroserwisów korzystających z Rabbitmq, zeromq czy nanomsg jest
banalnie prostym zadaniem. W kontekście samych mikroserwisów locust
idealnie nadaje się do znajdywania elementów będących słabym punktem
naszej aplikacji, znajdywania single point of failure czy też
sprawdzania, jak nasza architektura poradzi sobie, gdy jeden z serwisów
przestanie być responsywny lub całkowicie przestanie działać.

\subsection[bibliografia-użyteczne-linki]{Bibliografia / użyteczne
linki}

\startitemize[n][stopper=.]
\item
  \useURL[url1][http://docs.locust.io/en/latest/][][Locust.io:
  Dokumentacja]\from[url1]
\item
  \useURL[url2][http://shop.oreilly.com/product/9780596520670.do][][Ian
  Molyneaux: The Art of Application Performance Testing]\from[url2]
\item
  \useURL[url3][http://www.uvd.co.uk/blog/load-testing-vote-for-policies-with-locust-io/][][http://www.uvd.co.uk/blog/load-testing-vote-for-policies-with-locust-io/]\from[url3]
\item
  \useURL[url4][http://software.danielwatrous.com/load-testing-alternatives-for-large-scale-web-applications/][][http://software.danielwatrous.com/load-testing-alternatives-for-large-scale-web-applications/]\from[url4]
\item
  \useURL[url5][http://abhishek-tiwari.com/post/performance-testing-as-a-first-class-citizen][][http://abhishek-tiwari.com/post/performance-testing-as-a-first-class-citizen]\from[url5]
\item
  \useURL[url6][http://blog.codinghorror.com/performance-is-a-feature/][][http://blog.codinghorror.com/performance-is-a-feature/]\from[url6]
\stopitemize


\stoptext
