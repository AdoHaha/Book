\usemodule[pycon-2015]
\starttext


\section[import-deep-dive-detours---petr-viktorin]{Import Deep Dive
Detours - Petr Viktorin}

Python's \type{import} statement hides a lot of complexity. You could
talk about it for hours, and still not cover it all.

My talk should make you understand import machinery well enough that it
will make sense when you want to study it further. It shows the typical
case of loading a simple Python module (or package), omitting the
various customization hooks, and ignoring the ever-present hacks for
backward compatibility.

To summarize, the import machinery has two major parts: the {\em finder}
and the {\em loader}. The finder receives a name, and finds a file with
the corresponding code, typically by looking for \type{<modulename>.py}
in directories listed in \type{sys.path}. Its output is a
\type{ModuleSpec} object that contains, among other things,
\type{origin} (the name of the \type{.py} file to load) and a loader
object.

The loader then takes the ModuleSpec, creates a module object, and
executes code in the \type{.py} file to add the module's function class
objects.

There are two major complications: First, modules are cached in
\type{sys.modules}, so re-importing a module is little more than a dict
lookup. Second, for submodules, the parent package is always loaded
before the submodule.

When reasoning about module imports, it's useful to keep in mind the
exact order the loading and caching is done in. In simplified
pseudocode, the import machinery looks like this:

\starttyping
import_module(name):
    sys.modules[name]? return it

    if importing a submodule:
        load parent module
        sys.modules[name]? return it
        path = parent.__path__
    else:
        path = sys.path

    spec = find_spec(name, path)
    load(spec)  # sets sys.modules[name]

    module = sys.modules[name]
    if importing a submodule:
        set module as attribute of parent
    return module

find_spec(name, path):
    for finder in sys.meta_path:
        call its find_spec(name, path)
        return spec if successful

PathFinder.find_spec(name, path):
    for directory in path:
        get sys.path_hooks entry
        call its find_spec(name)
        return spec if successful

load(spec):
    module = spec.loader.create_module(spec)
    if module is None:
        module = types.ModuleType(spec.name)

    set initial module attributes
    sys.modules[spec.name] = module
    spec.loader.exec_module(module, spec)
\stoptyping

Most commonly, you will need to think about this when you're debugging
import cycles - the case where module A imports module B, which in turn
imports A again. Import cycles are hard to get right, and perhaps even
harder to keep working as your code changes. The best advice for import
cycles is to get rid of them, typically by putting common code in its
own module.

For packages, the rule of thumb to avoid import-related trouble are:

\startitemize
\item
  \type{__init__} should:
  \startitemize
  \item
    import from submodules
  \item
    set \type{__all__}
  \item
    nothing else
  \stopitemize
\item
  submodules should:
  \startitemize
  \item
    not import directly from \type{__init__}
  \item
    not have internal import cycles
  \stopitemize
\stopitemize

Now, there are still some topics that I haven't covered in the 30
minutes of my talk. This article dives into a few more tidbits and
special cases related to importing: namespace packages, real-world
examples of projects that use custom importers, Loader-assisted resource
loading, the rules for C extension modules, and more information on the
\type{__main__} module.

\subsection[namespace-packages-and-plugin-systems]{Namespace packages
and plugin systems}

Probably the most important omission in my talk are namespace packages.

When \type{PathFinder.find_spec} loops through the path entries, it
looks for either a file (which would become a module), or a directory
with an \type{__init__} file (which would become a package).

What I didn't tell you is that it actually looks for {\em all}
directories, even if they don't contain \type{__init__}. Any
\type{__init__}-less directories are remembered for later.

If the finder finishes going through \type{path} without finding a
suitable module, it makes a {\em namespace package}: an empty module
whose \type{__path__} contains all the directories remembered.

But why do that?

Namespace packages are usually used to implement plugin systems. Suppose
there is an extensible sound editor, where individual filters (like
\quotation{echo}, \quotation{reverb} and \quotation{speedup}) are
implemented as Python modules, and can be installed individually.

\quotation{Individually} means \quotation{in different locations}. Your
\type{sys.path} will usually include several kinds of locations. On a
Linux machine, the path will typically have:

\startitemize
\item
  Directories from \type{$PYTHONPATH}, and the current directory
\item
  The Python standard library (e.g. \type{/usr/lib/python3.4})
\item
  User-specific locations (e.g.
  \type{~/.local/lib/python3.4/site-packages})
\item
  A place for system-wide installed packages (e.g.
  \type{/usr/lib/python3.4/site-packages})
\stopitemize

Suppose you install the editor and a built-in \quotation{echo} filter
system-wide, a private \quotation{reverb} filter to your home directory,
and you have \quotation{speedup} (which you're currently developing) in
the current directory (on \type{$PYTHONPATH}). The paths to these
modules would be:

\startitemize
\item
  /usr/lib/python3.4/site-packages/sound\letterunderscore{}editor\letterunderscore{}filters/echo.py
\item
  \lettertilde{}/.local/lib/python3.4/site-packages/sound\letterunderscore{}editor\letterunderscore{}filters/reverb.py
\item
  ./sound\letterunderscore{}editor\letterunderscore{}filters/speedup.py
\stopitemize

And that repeated \type{sound_editor_filters} directory becomes a
namespace package:

\starttyping
>>> import sound_editor_filters
>>> sound_editor_filters.__path__
_NamespacePath([
    './sound_editor_filters',
    '~/.local/lib/python3.4/site-packages/sound_editor_filters',
    '/usr/lib/python3.4/site-packages/sound_editor_filters'])
\stoptyping

Importing \type{sound_editor_filters.echo} will get you the correct
module, because submodules are searched for in the \type{__path__} of
their parent.

Remember that adding an \type{__init__.py} to {\em any} of the
\type{sound_editor_filters} directories will cancel the magic. Any
directory with \type{__init__} is a regular package, its
\type{__init__}-less siblings are ignored.

Long before namespace packages were added to CPython, \type{setuptools}
contained
\useURL[url1][https://pythonhosted.org/setuptools/setuptools.html\#namespace-packages][][an
utility]\from[url1] to declare namespaces manually and explicitly. Of
course, this still works - and if you need to support Python 2, it's the
way to go.

\subsection[customization-examples]{Customization examples}

Another topic that didn't make it to my talk is customization: how to
extend importlib to do new, exciting things?

If you're interested in this, I'll just direct you to
\useURL[url2][https://docs.python.org/3/library/importlib.html][][importlib
documentation]\from[url2]. Now that you know the terminology and the
\quotation{normal} way of how things are done, those docs should make
much more sense.

Instead, I'll mention some real-world projects that use custom import
hooks, which you can use as an inspiration about what's possible.

Hy (hy.readthedocs.org) allows you to import modules written in a
dialect of Lisp. Or as its docs put it, it's \quotation{lisp-stick on a
Python}. It uses a custom finder that looks for \type{*.hy} files
instead of \type{*.py} ones, and a corresponding loader that parses the
Lisp code, runs it, and populates a Python module object with the
resulting functions and objects.

This is a good approach for any project that wants to embed a custom
language in the Python runtime.

The Macropy project uses a similar technique to add syntactic macros to
Python code. Conceptually, it's the same as Hy, except that the language
is a bit more similar to Python.

Perhaps most widely used of import customizations is Cython's
\type{pyximport}, which finds a \type{.pyx} file, compiles it into a C
extension, and then imports it.

If you just want to load Python modules that are stored in a
non-standard location, like a database, Git repository, or just a
string, look near the end of David Beazley's PyCon US 2015 tutorial for
his Redis importer.

\subsection[loading-module-resources]{Loading Module Resources}

Every module uses a \type{Loader} object when it is first initialized,
but the usefulness of loaders doesn't end there.

Loaders of most types of modules implement the \type{ResourceLoader}
ABC, a protocol that allows loading data files stored alongside a
module. For example, to open a data file stored \quotation{next to} the
current module on the filesystem, a common approach is to do:

\starttyping
>>> import os
>>> filename = os.path.join(os.path.dirname(__file__), "file.dat")
>>> with open(filename, 'rb') as f:
...     data = f.read()
\stoptyping

However, this will only work when the module and data are loaded from
regular files. If they were, for example, distributed in a zip file, the
\type{open} function would fail.

A more robust approach is to do:

\starttyping
>>> import os
>>> filename = os.path.join(os.path.dirname(__file__), "file.dat")
>>> data = __loader__.get_data(filename)
\stoptyping

While constructing the filename is still somewhat painful, this approach
will work with zipped packages, and any other loaders that implement
\type{get_data}.

(Sadly, David Beazley's Redis importer does not support this yet.)

\subsection[extension-and-built-in-modules]{Extension and built-in
modules}

A big part of CPython's appeal (especially back when it was just
starting) is that it is easily extensible with code written in C, so you
can interface with existing libraries or write code that runs fast.
Today, it's generally better to use CFFI or Cython for this, but
extensions remain important. If nothing else, a big part of the the
standard library is written in C. (And CPython extensions don't need to
be written in C - any language that can export C-like functions, such as
C++ or Rust, will do.) Anyway, how are extension modules loaded?

An extension is a shared library, such as a DLL on Windows or a
\type{.so} file on Linux. It exports one function called
\type{PyInit_<modulename>}.

When the shared library file is found (using \type{PathFinder} as
normal), an ExtensionFileLoader is created to handle the loading.

The difference from Python modules is in the loader's methods. Step one:
\type{create_module} loads the shared library and calls
\type{PyInit_<modulename>}. PyInit is -- for historical reasons --
responsible for both creating the module object and initializing it. So,
there's no step two: \type{ExtensionFileLoader.exec_module} does
nothing.

PyInit is called without any arguments, so it doesn't have access to the
\type{ModuleSpec} and the things it includes, like the filesystem path
of our module. This makes loading data from related files quite tricky.
Also, calling user-written code from the PyInit function is somewhat
dangerous, because \type{sys.modules} is not updated until after the
initialization is done. To overcome these problems, there's a new
mechanism in Python 3.5, which allows extensions to customize both the
\type{create_module} and \type{exec_module} steps. It's called
\quotation{multi-phase initialization}, and it's explained in PEP 489.

Built-in modules, like \type{io} or \type{zipimport}, are very similar
to extension modules. Pretty much the only difference is the loading
step: they're looked up in a compiled-in list rather than found in the
filesystem. If you embed Python, or build it from source, you can add
your own PyInit functions to this list.

\subsection[the-special-snowflake-__main__]{The special snowflake:
\type{_\_main\_\_}}

The \type{__main__} module is quite special.

The \type{loader.create_module} step for it is done very early during
interpreter startup - so early that the import machinery isn't loaded
yet. The module object is created directly, and remembered deep in C
code.

When the time comes to run some code - the file Python was given to
interpret, or a module given with the \type{-m} switch, or even commands
from interactive console - it is executed inside \type{__main__}'s
namespace.

Python has an \type{-i} command-line switch, which causes it to drop to
interactive prompt after running code. If you invoke:

\starttyping
$ python3 -i somemodule.py
\stoptyping

the contents of \type{somemodule.py} are run inside the \type{__main__}
module. When it's done, you're given a chance to give commands, which
are also executed in the \type{__main__} namespace. So, if the module
defined some functions, you're able to use them right away.

The mechanism for doing this assumes that the \type{__main__} module
object is the same as the one created during early initialization. This
has some implications for the types of modules Python can run directly.
As explained in the previous section, traditional C extensions implement
all of \type{loader.create_module} on their own, so they can't use the
pre-existing \type{__main__}. Sure enough, when you invoke
\type{python -m math}, you just get an error. (As for PEP 489 extension
modules: it would be theoretically possible to run some of these
directly, and hopefully the support will be added in Python 3.6)

However, Python can run {\em packages} directly. It doesn't run the
\type{__init__.py} when you ask to run a package. That would encourage
people to violate our rules of what goes in \type{__init__}. (Remember:
import from subpackages, set \type{__path__}, nothing else. Especially
not a big \type{if __name__ == "__main__"} block.) Rather, a submodule
named \type{__main__} is executed.

You can run a number of package-like things:

\startitemize
\item
  \type{$ python -m somepackage} (when the package has a \type{__main__}
  subpackage)
\item
  \type{$ python some-directory/} (when the directory has a
  \type{__main__.py} file)
\item
  \type{$ python archive.zip} (when the archive includes a
  \type{__main__.py} file)
\stopitemize

In fact, when you see a file with a \type{.pyz} extension, it's nothing
more than a Python module with a \type{__main__.py}, bundled in a zip
archive. You can run these directly -- on Unix, they need a correct
shebang; on Windows they need a correct extension. In Python 3.5, the
\type{zipapp} tool can produce such executable archives for you, but the
support for running them has been there since Python 2.6.

\subsection[references]{References}

The documentation of the import machinery's is abundant (if somewhat
perplexing for the uninitiated). You should definitely read the
importlib documentation:

\startitemize
\item
  https://docs.python.org/3/library/importlib.html
\stopitemize

or watch David Beazley's fun-filled tutorial:

\startitemize
\item
  \quotation{Modules and Packages: Live and Let Die!} (PyCon 2015):
  http://pyvideo.org/video/3387
\stopitemize

and talks by Brett Cannon, the author of importlib:

\startitemize
\item
  \quotation{How Import Works} (PyCon 2013):
  http://pyvideo.org/video/1707
\item
  \quotation{Import this, that, and the other thing} (PyCon 2010):
  http://pyvideo.org/video/341
\stopitemize

For interesting detours and change summaries, look no further than the
related PEPs:

\startitemize
\item
  PEP 328 -- Imports: Multi-Line and Absolute/Relative
\item
  PEP 3147 -- PYC Repository Directories
\item
  PEP 302 -- New Import Hooks
\item
  PEP 451 -- A ModuleSpec Type for the Import System
\item
  PEP 420 -- Implicit Namespace Packages
\item
  PEP 441 -- Improving Python ZIP Application Support
\item
  PEP 338 -- Executing modules as scripts
\item
  PEP 3149 -- ABI version tagged .so files
\item
  PEP 489 -- Multi-phase extension module initialization
\item
  PEP 235 -- Import on Case-Insensitive Platforms
\item
  PEP 263 -- Defining Python Source Code Encodings
\item
  PEP 3120 -- Using UTF-8 as the default source encoding
\stopitemize

And lastly, for the true deep dive, peek in the importlib sources:

\startitemize
\item
  https://hg.python.org/cpython/file/3.5/Lib/importlib
\stopitemize

If anything is unclear, ask! That's what I did :)

Many thanks to Nick Coghlan, Brett Cannon, Eric Snow, Stefan Behnel and
everyone else involved with PEP 489 discussions, which forced me to
learn all this.


\stoptext
