\usemodule[pycon-2015]
\starttext

\Title{Wykorzystanie klastra Apache Mesos w deploymencie aplikacji pythonowych}
\Author{Kamil Warguła}
\MakeTitlePage

Dzisiejsze aplikacje internetowe poddawane są zróżnicowanemu obciążeniu
w ciągu tygodni, dni, a nawet godzin. Wykorzystanie zasobów jest ważną
kwestią w utrzymywaniu aplikacji i panowaniu nad budżetem.

W tym artykule chciałbym przedstawić, jak w łatwy sposób można
wykorzystać potencjał klastra Apache Mesos do deployowania aplikacji
pythonowych.

Klaster Apache Mesos przykrywa warstwą abstrakcji zasoby takie jak
procesor, pamięć, przestrzeń dyskowa. Dzięki temu użytkownik, który
zleca poprzez framework {[}1{]} (w naszym przypadku będzie to Marathon
{[}2{]}) uruchomienie procesu na klastrze, nie potrzebuje się martwić,
na jakim serwerze zostanie on uruchomiony, oraz nie musi dbać o to, co
się stanie w przypadku awarii jednego serwera w klastrze, o wszystko
zadba Mesos.

\subsection[przygotowanie-aplikacji]{Przygotowanie aplikacji}

W celu zademonstrowania deploymentu na klaster Mesos zbudujmy prostą
aplikację, która będzie wystawiała REST API. Poniżej fragment pliku
\type{api.py} {[}3{]}:

\starttyping
import falcon
import json


class ExampleResource:
    def on_get(self, req, resp):
        data = {'foo': 'bar'}
        resp.body = json.dumps(data)

api = falcon.API()
api.add_route('/', ExampleResource())
\stoptyping

W powyższym fragmencie wykorzystany został framework Falcon {[}4{]} do
napisania aplikacji WSGI, która w połączeniu z serwerem Gunicorn pozwoli
serwować nasze API.

Do przygotowania paczki z naszą aplikacją wykorzystamy narzędzie PEX
{[}5{]}, które dostarczy nam wirtualne środowisko Pythona wraz
zależnościami w postaci jednego wykonywalnego pliku. Aby zbudować takie
środowisko, należy wykonać następujące polecenie:

\starttyping
pex -r <(printf "falcon==0.3.0\ngunicorn==19.3.0") -o app.pex
\stoptyping

Dopełnieniem procesu przygotowania naszej aplikacji jest spakowanie
pliku \type{app.pex} oraz \type{api.py} do archiwum \type{tar.gz}:

\starttyping
tar -zcvf app.tar.gz api.py rest_app.pex
\stoptyping

\subsection[deployment-aplikacji]{Deployment aplikacji}

Wcześniej przygotowane archiwum z naszą aplikacją musi być dostępne
poprzez protokół HTTP. Aby to zrobić, możemy wykorzystać wbudowany w
Pythona serwer HTTP.

Proces uruchomienia naszej aplikacji na klastrze jest bardzo prosty, a
mianowicie jedyne, co musimy zrobić, to wysłać do Marathona request do
REST API zawierający opis naszej aplikacji. Powinien on zawierać takie
dane jak: nazwa aplikacji, komenda, która uruchomi naszą aplikację,
liczbę instancji, wielkość zasobów takich jak CPU oraz RAM, oraz
lokalizacja paczki z aplikacją. Poniżej przykładowa zawartość opisująca
aplikację:

\starttyping
{
    "id":"rest_app",
    "cmd":"./app.pex api.py -p $PORT0",
    "cpus":0.5,
    "instances":1,
    "mem":128,
    "uris":[
        "http://10.141.141.10:8000/app.tar.gz"
    ]
}
\stoptyping

Komenda, która została wykorzystana do uruchomienia aplikacji, zawiera
zmienną środowiskową \type{$PORT0}, jest to zmienna określająca port
przydzielony automatycznie przez Marathona, pod którym zostanie
zbindowana nasza aplikacja.

Marathon po otrzymaniu informacji o definicji aplikacji przystąpi do
procesu deploymentu. Zdefiniowana przez nas paczka zostanie pobrana na
serwer slave w klastrze Mesos, rozpakowana, a następnie zostanie
uruchomiona komenda zawarta w definicji aplikacji.

Informację o wszystkich uruchomionych aplikacjach na klastrze Mesos
możemy zobaczyć w interfejsie Marathona dostępnym poprzez stronę WWW, a
także za pomocą REST API {[}6{]} wystawionego pod zasobem
\type{/v2/apps}.

W przypadku wystąpienia awarii serwera, na którym uruchomiony jest nasz
program, Marathon automatycznie przystąpi do procesu redeploymentu
naszej aplikacji na innym serwerze wchodzącym w skład klastra Mesos.
Dzięki temu mechanizmowi w automatyczny sposób możemy uzyskać odporną na
awarie infrastrukturę naszej aplikacji.

\subsection[skalowanie-manualne]{Skalowanie manualne}

Liczba użytkowników każdej aplikacji z czasem rośnie. Zarządzanie dużą
liczbą instancji aplikacji staje się kłopotliwe oraz czasochłonne. Z
pomocą przychodzi nam możliwość skalowania naszej aplikacji na klastrze
Mesos. Uruchomienie 10, czy 100 instancji aplikacji sprowadza się tylko
do dwóch kliknięć w interfejsie WWW Marathona, lub do wysłania requesta
pod zasób naszej aplikacji dostępny poprzez REST API. Jest to wygodny i
bardzo szybki sposób na obsłużenie dużej liczby użytkowników.

Request, dzięki któremu zeskalujemy naszą aplikację, powinien być typu
\type{PUT} oraz musi zostać wysłany pod zasób \type{/v2/apps/} wraz z
nazwą naszej aplikacji. W naszym przypadku będzie to
\type{/v2/apps/rest_app}. Zawartość przesłanego żądania wygląda
następująco:

\starttyping
{
    "instances":10
}
\stoptyping

Marathon po otrzymaniu takiego żądania przystąpi do procesu uruchomienia
zadeklarowanej przez nas liczby instancji na klastrze. Nie musimy się
martwić, na jakich serwerach zostanie uruchomiona nasza aplikacja,
wszystko stanie się automatycznie bez naszej ingerencji.

\subsection[skalowanie-automatyczne]{Skalowanie automatyczne}

Serwer pełniący role slave'a, będący częścią klastra Mesos, udostępnia
informację o aktualnym wykorzystaniu zasobów przez aplikacje, które
zostały na nim uruchomione. Dane te są dostępne poprzez REST API pod
zasobem \type{/monitor/statistics.json}.

Przykładowe dane dotyczące uruchomionej aplikacji:

\starttyping
{
    executor_id: "app-new.0d85b86a-4f60-11e5-9031-56847afe9799",
    framework_id: "20150827-194534-16842879-5050-1219-0001",
    statistics: {
        cpus_limit: 0.2,
        cpus_system_time_secs: 91.05,
        cpus_user_time_secs: 11.32,
        mem_limit_bytes: 50331648,
        mem_rss_bytes: 100966400,
        timestamp: 1441047277.2822
    }
},
\stoptyping

Zużycie zasobów przez nasze aplikacje zazwyczaj nie jest równomierne w
ciągu doby. Podczas dnia możemy potrzebować większej liczby instancji,
aby sprostać dużej liczbie użytkowników, zaś w nocy przy małym ruchu
większość instancji będzie się nudzić. Zbierając metryki dotyczące
zużycia zasobów przez nasze instancje aplikacji uruchomionych na
klastrze Mesos, jesteśmy w stanie automatycznie zarządzać liczbą
instancji.

Możemy stworzyć prosty mechanizm autoskalowania, który, w oparciu o
zebrane metryki zużyć naszej aplikacji, zdecyduje o dołożeniu
dodatkowych instancji, tak aby sprostać dużej liczbie użytkowników.

Pseudokod mechanizmu autoskalowania:

\starttyping
MIN_INSTANCES_COUNT = 4
MAX_INSTANCES_COUNT = 100

# wykrywamy sytuację, w której 90% naszych instancji jest obciążona,
# i zwiększamy liczbę wszystkich instancji o 2 pod warunkiem,
# że nie osiągnęliśmy maksymalnej liczby instancji.

if overloaded_instances >= 0.9 * total_instances:
    total_instances += 2
    if total_instances <= MAX_INSTANCES_COUNT:
        scale_app(instances_count=total_instances)

# wykrywamy sytuację, w której tylko 50% naszych instancji jest obciążona,
# i wtedy zmniejszamy całkowitą liczbę o 2 pod warunkiem,
# że nie osiągnęliśmy maksymalnej liczby instancji.

elif overloaded_instances <= 0.5 * total_instances:
    total_instances -= 2
    if total_instances >= MIN_INSTANCES_COUNT:
        scale_app(instances_count=total_instances)
\stoptyping

Powyższy pseudokod przedstawia sytuację, w której, w zależności od
liczby obciążonych instancji aplikacji, możemy automatycznie zwiększyć
lub zmniejszyć ich liczbę poprzez wysłanie odpowiedniego requesta do
Marathona.

Dobranie odpowiednich wartości parametrów sterujących procesem
autoskalowania nie jest rzeczą łatwą, lecz metodą prób i błędów jesteśmy
w stanie znaleźć optymalne dla nas wartości.

\subsection[podsumowanie]{Podsumowanie}

Klaster Apache Mesos w połączeniu z Marathonem umożliwa szybkie i
wygodne deployowanie aplikacji napisanych w Pythonie. Dzięki niemu
możemy w łatwy sposób zarządzać liczbą instancji, a także zasobami
używanymi przez aplikacje, a wszystko to dzięki prostocie działania i
możliwościom przeprowadzania wszystkich operacji poprzez REST API.

\subsection[referencje]{Referencje}

\startitemize
\item
  {[}1{]} Mesos Framework Development Guide
  \hyphenatedurl{http://mesos.apache.org/documentation/latest/app-framework-development-guide/}
\item
  {[}2{]} Marathon framework
  \hyphenatedurl{https://mesosphere.github.io/marathon/}
\item
  {[}3{]}
  \hyphenatedurl{https://gist.github.com/quamilek/4fd1f246feb6149dd1dd}
  - kod źródłowy pliku \type{api.py}
\item
  {[}4{]} Falcon framework http://falconframework.org
\item
  {[}5{]} PEX library https://github.com/pantsbuild/pex
\item
  {[}6{]} Marathon REST API
  \hyphenatedurl{https://mesosphere.github.io/marathon/docs/rest-api.html}
\stopitemize

\subsection[źródła]{Źródła}

\startitemize
\item
  Apache Mesos http://mesos.apache.org/
\item
  Marathon framework https://mesosphere.github.io/marathon/
\item
  PEX library documentation https://pex.readthedocs.org/en/stable/
\stopitemize


\stoptext
