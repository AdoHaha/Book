\usemodule[pycon-2016]
\starttext

\Title{Python for Networking Devices}
\Author{Elisa Jasinska}
\MakeTitlePage

\subsection[introduction]{Introduction}

The Internet is a series of tubes and at the end of those tubes are:
networking devices! To form the Internet as we know it, each provider
network has to be managed, maintained and interconnected. Traditional
Network Engineering is moving more and more towards automated device and
service management, a task often performed in Python due to the
availability of many useful libraries. We will walk you though common
tasks in Network Engineering and introduce a number of Python libraries
that are helpful in accessing and managing networking equipment.

\subsection[network-device-access]{Network Device Access}

Traditionally networking devices are managed via their command line
interface (CLI). The CLI is accessible though various forms of
transport. Originally Telnet has been used, later on devices started to
support SSH. SSH provides additional security mechanisms over Telnet,
but its usability doesn't differ much, it's still a CLI.

Configuration commands are sent via the CLI, line by line, and enable or
disable specific functionality on the device. For example, to add a
Border Gateway Protocol (BGP) neighbor on your router, your config might
need to contain a block similar to this one:

\starttyping
protocols {
    bgp {
        group external-peers {
            type external;
            neighbor 10.10.10.2 {
                peer-as 42;
            }
        }
    }
}
\stoptyping

The configurations differ per vendor, the above example would add BGP
neighbor on a Juniper device, but for a Cisco device, you might need to
send commands like this:

\starttyping
router bgp 23
neighbor 10.10.10.2 remote-as 42
address-family ipv4 unicast
next-hop-self
\stoptyping

The drawback of entering configurations line by line is the same as with
any other device (a server for instance): errors might occur somewhere
along the way, which will result in only a part of the configuration
being committed to the device - the change is not transactional.

\section[netconf]{Netconf}

Nowadays Netconf is the new hype in the networking world. With
underlying SSH transport it supports the same security feature set, but
in addition it offers support for transactions, structured data and
error reporting. Netconf allows to submit multiple changes at a time via
an RPC call, encoded as XML, and sends back an XML object with the
result in return.

Even though Netconf is an RFC standard and vendors should implement it
the same way - they don't. Not every vendor supports the full Netconf
feature set, if they offer support for it at all, in which case you are
stuck with managing your devices line by line via the CLI after all.

In addition to configuring specific functionality on the device,
\quote{show commands} are used to retrieve operational or state data of
the router. For example, output like this can be retrieved upon
requesting the current ARP table on an Arista device:

\starttyping
eos.edge1>show arp
Address         Age (min)  Hardware Addr   Interface
10.220.88.1             0  001f.9e92.16fb  Vlan1, Ethernet1
10.220.88.21            0  1c6a.7aaf.576c  Vlan1, not learned
10.220.88.28            0  5254.00ee.446c  Vlan1, not learned
10.220.88.29            0  5254.0098.69b6  Vlan1, not learned
10.220.88.30            0  5254.0092.13bb  Vlan1, not learned
10.220.88.38            0  0001.00ff.0001  Vlan1, not learned
\stoptyping

On a Juniper device, it will look more like this:

\starttyping
root@qfx.edge1> show arp
MAC Address       Address       Name          Interface Flags
00:1f:9e:92:16:fb 10.220.88.1   10.220.88.1   vlan.0    none
00:19:e8:45:ce:80 10.220.88.22  10.220.88.22  vlan.0    none
f0:ad:4e:01:d9:33 10.220.88.100 10.220.88.100 vlan.0    none
Total entries: 3
\stoptyping

Not only can the \quote{show commands} to execute on the device differ
per vendor, they also provide different text format output, which in
case of CLI access has to be parsed individually. Netconf-like access
methods support the retrieval of structured data, which is slightly
better, but this still doesn't cover fields not provided in one vendor
vs.~another (like for example the lack of an age timer in the Juniper
output above).

\subsection[generic-access-libraries]{Generic Access Libraries}

To access devices directly via their SSH or Netconf interface, generic
Python libraries such as
\useURL[url1][https://github.com/pexpect/pexpect][][pexpect]\from[url1],
\useURL[url2][https://github.com/paramiko/paramiko][][paramiko]\from[url2]
and
\useURL[url3][https://github.com/ncclient/ncclient][][ncclient]\from[url3]
can be used. They allow for programmatic access from scripts (for
network engineers who are trying to figure out how to code ;) ) but
don't provide any ease of use in terms of vendor specificity. You will
still have to deal with the little differences the vendors bring:
different login procedures or additional escape chars, different
configuration syntax, differences in \quote{show commands} or
differences with the varying support of Netconf on each device.

\subsection[specific-network-vendor-libraries]{Specific Network Vendor
Libraries}

Since Python became more and more popular amongst network engineers,
network vendors started to provide their own libraries to facilitate
device interaction - to mention a few:

\startitemize[packed]
\item
  Arista's
  \useURL[url4][https://github.com/arista-eosplus/pyeapi][][pyeapi]\from[url4]
\item
  Cisco IOS-XR
  \useURL[url5][https://github.com/fooelisa/pyiosxr][][pyiosxr]\from[url5]
\item
  Juniper's
  \useURL[url6][https://github.com/Juniper/py-junos-eznc][][py-junos-eznc]\from[url6]
\stopitemize

Each of them supports their specific device with its specific
capabilities. Juniper's py-junos-eznc provides access via Netconf,
whereas pyiosxr \quote{mimics} Netconf support (which is not directly
available) via SSH and pexpect. They ease config operations on the
devices by providing functions to send config, replace or merge it in
one go, or to retreive the configuration. Each of those libs is very
specific to the vendors device and takes all its nits into account, but
a lot of times networks are designed to use a mix of vendors, in which
case one library won't suffice.

\subsection[multivendor-libraries]{Multivendor Libraries}

Managing configurations in a multivendor environment takes on a whole
different level. In addition to managing differences per device-role,
differences in access have to be considered. Each device role needs a
similar base configuration for example (syslog servers, NTP servers, dns
servers) but different BGP neighbors on devices in different locations,
plus a different methodology to upload the configs to the device. Vendor
specific libs can be used for each of them individually, however
managing the use of different libraries per vendor is quite painful.

In an effort to improve upon this situation, wrappers around a set of
libraries have been built to unify multivendor access to networking
devices. For example
\useURL[url7][https://github.com/ktbyers/netmiko][][netmiko]\from[url7],
which is a wrapper around paramiko and provides easy SSH access for
plenty of networking equipment. On the Netconf front, or to be more
specific, access which supports transactions on the device, a project
called
\useURL[url8][https://github.com/napalm-automation/napalm][][napalm]\from[url8]
has been started, which unifies a set of vendor libraries into a single
set of methods.

Of course there are also standardization efforts, like open config or
Netconf, which aim to improve upon this situation. But as far as
standards go, interworkings between different vendors have been proven
to take a lot of time and effort, and typically aren't on the market as
quickly as the engineers would like.

\subsection[summary]{Summary}

How to start if you are looking to introduce automation into your
network? Start by reviewing your network design and determine what
vendors are in use. If it is one single vendor and they provide a
library, use that. If you operate a mix of vendors, check if the
multivendor module napalm supports all of them. If neither is an option,
you can decide to use a lower level access wrapper like netmiko, but you
will have to give up on using more advanced functionality that is
provided by the libraries.

And all of this, to get a few lines of config onto your router:

\starttyping
set system login message "> telnet xxx.xxx.xxx.xxx\nTrying xxx.xxx.xxx.xxx…
\nConnected to telnet.example.com.\nEscape character is '^]'.\n\nlol (tty0)
\n\nlogin:\n"
\stoptyping

EOF


\stoptext
