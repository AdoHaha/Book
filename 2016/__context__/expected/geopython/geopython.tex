\usemodule[pycon-2016]
\starttext

\Title{Krótkie wprowadzenie do GeoPythona}
\Author{Małgorzata Papież}
\MakeTitlePage

Od kilku lat, w Polsce coraz większą popularność zdobywają aplikacje
oparte na wykorzystaniu {\em danych przestrzennych}. To dzięki nim
przestaliśmy być zależni od papierowych map i bez wychodzenia z domu
możemy zobaczyć każdy zakątek Ziemi. Co jeśli dostępne aplikacje takie
jak Google Earth, Google Maps nam nie wystarczają i chcemy czegoś
więcej? Zobaczmy więc, jak wykorzystując Pythona możemy zaprojektować
aplikacje pozwalające nam na tworzenie własnych map czy przeglądanie
najnowszych zdjęć satelitarnych.

\subsection[dane-przestrzenne]{Dane przestrzenne}

Rozwój technologii komputerowych w ostatnich latach sprawił, że
przetwarzanie bardzo dużej ilości danych w czasie rzeczywistym stało się
możliwe. Przyspieszenie procesów analizy dużych zbiorów danych
przyczyniło się wzrostu zainteresowania danymi przestrzennymi. Dane
przestrzenne przechowują informacje o obiekcie świata rzeczywistego
biorąc pod uwagę jego położenie przestrzenne. Oprócz informacji
przestrzennej podanej w postaci szerokości i długości geograficznej mogą
zawierać również inne istotne informacje takie jak data utworzenia,
związki przestrzenne obiektów ze sobą tzw. topologię czy właściwości
danego obiektu czyli jego atrybuty. Dane przestrzenne dzieli się na
wektorowe i rastrowe. Pierwsze z nich reprezentowane są przez obiekty
geometryczne takie jak punkty, linie, powierzchnie, których kształt i
położenie definiowany jest przez współrzędne. Taki sposób reprezentacji
używany jest z racji swojej dużej dokładności, możliwości wyodrębniania
poszczególnych obiektów oraz małej pojemności plików. Do obiektów
wektorowych w rzeczywistości należą granice działek, ulice, budynki,
sieci: elektryczne, gazowe, wodne, telefoniczne itd. W modelu rastrowym
dane są przechowywane w postaci pojedynczych pikseli, w regularnej
siatce pikseli. Dzięki temu znacznie wydajniej radzi sobie z analizą i
modelowaniem zjawisk zachodzących w przestrzeni np. stanu
zanieczyszczenia, opadów atmosferycznych. Niestety wymaga dużo pamięci
RAM oraz powoduje utratę części informacji. Źródła danych
przestrzennych:

\startitemize[packed]
\item
  mapy i plany,
\item
  digitalizacja i wektoryzacja papierowych map,
\item
  odbiorniki GPS,
\item
  zdjęcia satelitarne i lotnicze,
\item
  pomiary geodezyjne, stacje pomiarowe i wywiady terenowe.
\stopitemize

\subsection[python-i-dane-przestrzenne]{Python i dane przestrzenne}

Znaczna część aplikacji korzystająca z danych przestrzennych napisana
została w języku Python. Taki wybór podyktowany został możliwościami
jakie daje ten język programowania. Po pierwsze pozwala korzystać za
darmo praktycznie ze wszystkich bibliotek związanych z danymi
przestrzennymi, w szczególności z {\em GDAL/OGR}. Dodatkowo ułatwia
pracę na danych rastrowych dzięki bibliotece {\em numpy}. Graficzna
biblioteka {\em PyQt} pozwala budować interfejsy graficzne, które mogą
być dołączane jako nakładki do istniejących już aplikacji przestrzennych
i przenoszone między systemami operacyjnymi bez potrzeby modyfikacji ich
kodu.

\subsection[gdalogr]{GDAL/OGR}

GDAL ({\em Geospatial Data Abstraction Library}) jest biblioteką
rozwijaną przez fundację {\em OSGEO} jako wolne oprogramowanie, służącą
do operacji na danych rastrowych. Zawiera w sobie również bibliotekę OGR
({\em Simple Features Library}) do danych wektorowych. Obecnie obsługuje
około 140 formatów danych rastrowych oraz 80 wektorowych. Obie
biblioteki napisane zostały w języku C++, ale posiadają bindingi dla
innych języków, w tym dla Pythona. Istotną rzeczą jest, że GDAL jest tak
naprawdę zbiorem odrębnych programów tzw. {\em utility programs}, które
możemy wywołać z linii komend. Wszystkie dostępne operacje znajdują się
na stronie biblioteki GDAL {[}1{]}.

\subsection[pierwsza-aplikacja-z-użyciem-gdalogr]{Pierwsza aplikacja z
użyciem GDAL/OGR}

Aby rozpocząć pracę z biblioteką GDAL pierwszym krokiem jaki należy
wykonać jest sprawdzenie czy posiadamy zainstalowaną bibliotekę:

\starttyping
import sys
try:
  from osgeo import ogr, gdal
except ImportError:
  sys.exit('GDAL/OGR is not installed.')
\stoptyping

GDAL nie jest dołączany do standardowej biblioteki modułów Pythona.
Warto jednak przed jego instalacją sprawdzić czy nie posiadamy już
zainstalowanej wersji, gdyż GDAL z racji swojej dużej użyteczności i
popularności jest często instalowany razem z innymi programami (Google
Earth, QGis, ArcGIS). Tych, którzy nie posiadają GDAL, odsyłam na stronę
{[}2{]}, na której krok po kroku wytłumaczone zostało jak zainstalować
bibliotekę na różnych systemach.

Zacznijmy od najprostszego przykładu prezentującego w jaki sposób
następuje odczyt danych rastrowych:

\starttyping
from osgeo import ogr, gdal
dataset = gdal.Open('test.tif', gdal.GA_ReadOnly)
if dataset is None:
    print 'File not open!'
\stoptyping

Z modułu GDAL należy wywołać metodę \type{Open} ze ścieżką dostępu jako
parametrem (najlepiej w postaci bezwzględnej). Drugi parametr określa
sposób otwarcia pliku. Domyślnie parametr ten ustawiony jest na odczyt
rastra, dlatego \type{gdal.GA_ReadOnly} może być pominięte. Mając
wczytanego rastra możemy dokonać sprawdzenia takich wartości jak ilość
kanałów \type{RasterCount}, ilość wierszy \type{RasterXSize}, ilość
kolumn \type{RasterYSize}, z których składa się raster.

\starttyping
dataset = gdal.Open('test.tif')

bands = dataset.RasterCount
cols = dataset.RasterXSize
rows = dataset.RasterYSize

print 'Number of bands: ', bands
print 'X: ', cols
print 'Y: ', rows
\stoptyping

Więcej niż jeden kanał w rastrze oznacza, że dane z tego samego
położenia zostały zarejestrowane w różnych zakresach promieniowania.
Możemy również sprawdzić podstawowe statystyki dotyczące każdego kanału.

\starttyping
for band in range(bands):
    print 'Band no.: ', band
    srcband = dataset.GetRasterBand(band)
    if srcband:
        stats = srcband.GetStatistics(True, True)
        print 'Min: %.3f'%stats[0]
        print 'Max: %.3f'%stats[1]
\stoptyping

Teraz spróbujmy otworzyć plik wektorowy. Przed otwarciem pliku ustawiamy
sterownik \type{Driver}, który jest obiektem odpowiadającym za poprawne
wczytanie odpowiedniego typu danych. Ważne jest również by przy
pierwszym otwarciu pliku ustawić prawa dla sterownika, w zależności od
tego czy chcemy odczytywać czy zapisywać dane. Domyślnym prawem jest
prawo do odczytu oznaczane zerem. Jedynka oznacza możliwość modyfikacji
pliku i jego ponownego zapisu. Nie wszystkie wspierane przez OGR formaty
posiadają opcję zapisu. Metoda Open zwraca obiekt zwany źródłem danych:

\starttyping
from osgeo import ogr
driver = ogr.GetDriverByName('ESRI Shapefile')
datasource = driver.Open('test.shp', 0)
\stoptyping

Źródło danych składa się z warstw, które pobieramy za pomocą funkcji
\type{GetLayer}. Najbardziej podstawowy format wektorowy Shapefile
posiada tylko jedną warstwę, dlatego użycie indeksu oznaczającego numer
warstwy jest opcjonalne. Przy pozostałych formatach ustawienie indeksu
jest obowiązkowe. W celu sprawdzenia liczby warstw możemy wywołać
następującą funkcję:

\starttyping
numLayer = datasource.GetLayerCount()
\stoptyping

Po pobraniu warstwy jesteśmy w stanie odczytać podstawowe informacje o
obiektach w niej zawartych. Możemy sprawdzić z ilu obiektów składa się
warstwa \type{GetFeatureCount}. Najważniejsze jednak jest to, że możemy
pobrać każdy obiekt po to, by móc odczytać jego geometrię, nazwę, oraz
wartości jakie w sobie przechowuje:

\starttyping
datasource = driver.Open('test.shp', 0)
for feat in range(numLayer):
    layer = datasource.GetLayerByIndex(feat)
    print 'Layer name: ', layer.GetName()
    numfeat = layer.GetFeatureCount()
    print 'Number of features:  ', numfeat
\stoptyping

Jak widzimy, odczyt danych wektorowych jest bardziej skomplikowany i
dobranie się do struktury danych wymaga przejścia przez kilka poziomów,
co przy bardzo dużej ilości danych powoduje opóźnienia.

\subsection[ale-skąd-te-dane]{Ale skąd te dane?}

W podanych przykładach korzystaliśmy z przykładowych danych rastrowych i
wektorowych. Jest to próbny zestaw danych, który jest ściągany podczas
instalacji aplikacji do danych przestrzennych korzystających z GDAL.
Pisanie własnej aplikacji wymaga od nas jednak rzeczywistych danych
dostosowanych do naszych potrzeb. Oto kilka źródeł z których możemy
pobierać bezpłatnie potrzebne nam dane:

\startitemize[packed]
\item
  Centralny Ośrodek Dokumentacji Geodezyjnej i Kartograficznej {[}3{]},
\item
  Centralna Baza Danych Geologicznych {[}4{]},
\item
  Geoportal 2 {[}5{]},
\item
  OpenStreetMap {[}6{]},
\item
  USGS {[}7{]},
\item
  ESA/Sentinel {[}8{]}.
\stopitemize

\subsection[wizualizacja-danych]{Wizualizacja danych}

Odczyt danych i ich analiza to jednak nie wszystko. Aby nasza aplikacja
miała możliwość podglądu danych potrzebujemy biblioteki graficznej. O
ile z wyświetleniem rastrów nie mamy problemów (możemy do ich
wyświetlenia użyć dowolnej graficznej biblioteki Pythona, np. PyQt) o
tyle wyświetlenie wektorów jest bardziej problematyczne. Podobnie jak w
przypadku rastrów możemy skorzystać z gotowych komponentów graficznych
wbudowanych w bibliotekę PyQt np. {\em QPainter} do rysowania obiektów.
Niestety rozwiązanie to ma jedna wadę. Nie sprawdza się dla wektorów
mających powyżej kilku tysięcy wierzchołków. Dla porównania,
wyświetlenie konturu jednego województwa zajmuje ułamki sekund,
wyświetlenie całej mapy Polski z konturami wszsytkich województw zajmuje
już kilkanaście sekund. Z pomocą przychodzą biblioteki dedykowane do
wizualizacji danych wektorowych. Jedną z nich jest {\em matplotlib} wraz
z rozszerzeniem {\em Basemap}. Jest on odpowiednikiem znanego z Matlaba
narzędzia zwanego Mapping Toolbox. Jego główną zaletą jest to, że sam
dokonuje odwzorowania kartograficznego czyli transformacji współrzędnych
geograficznych na współrzędne rysunku. Dodatkowo zawiera sporo funkcji
ułatwiających rysowanie danych przestrzennych m.in. rysowanie
równoleżników, południków, rysowanie i wygładzanie granic i wybrzeży.
Minusem jest jednak słabo opisana dokumentacja.

\subsection[pozostałe-biblioteki-przestrzenne]{Pozostałe biblioteki
przestrzenne}

Wspomniane {\em GDAL/OGR} nie jest jedyną biblioteką wspomagającą
przetwarzanie danych przestrzennych. Obecnie dostępnych darmowo mamy
kilkanaście bibliotek, szczególnie do manipulacji danymi wektorowymi. Z
nich najbardziej znane to Fiona i Shapely. Obie są typowymi bibliotekami
Pythona, które pozwalają na odczyt danych wektorowych, tworzenie nowych
geometrii, sprawdzanie poprawności geometrii oraz wszelkiego rodzaju
operacje geometryczne. Prosty odczyt danych wektorowych za pomocą
biblioteki Fiona:

\starttyping
import fiona
data = fiona.open('test.shp')
print 'Ilość obiektów w warstwie: ', len(data)

rec = next(data)
print rec.keys()
print rec['geometry']['type']
print rec['properties']
\stoptyping

\subsection[podsumowanie]{Podsumowanie}

Wybór odpowiedniej biblioteki jest w znacznym stopniu zależny od stopnia
zaawansowania operacji, które będą wykonywane w projektowanej aplikacji.
Pod tym względem niewątpliwie liderem jest GDAL/OGR, który dostarcza
najwięcej gotowych funkcjonalności. Niemniej GDAL/OGR może być
problematyczny ze względu na fakt, że jego struktura oparta jest na C++.
Fiona i Shapely opierają się na standardach Pythona m.in. korzystając z
plików, słowników czy iteratorów, co przyspiesza pracę i zmniejsza
prawdopodobieństwo popełnienia błędu.

\subsection[bibliografia]{Bibliografia}

\startitemize[n,packed][stopper=.,width=2.0em]
\item
  Strona GDAL z listą wszystkich operacji:
  http://www.gdal.org/gdal\type{_}utilities.html
\item
  Instalacja GDAL/OGR:
  http://www.gis.usu.edu/\lettertilde{}chrisg/python/2009/install.html
\item
  Centralny Ośrodek Dokumentacji Geodezyjnej i Kartograficznej:\crlf
  http://www.codgik.gov.pl/index.php/darmowe-dane.html
\item
  Centralna Baza Danych Geologicznych: http://baza.pgi.gov.pl/
\item
  Geoportal: http://www.geoportal.gov.pl/web/guest/DOCHK
\item
  OpenStreetMap: http://download.geofabrik.de/
\item
  USGS: http://earthexplorer.usgs.gov/
\item
  ESA/Sentinel: https://scihub.copernicus.eu/dhus/\#/home
\item
  Systemy GIS:
  http://wazniak.mimuw.edu.pl/images/9/9a/Systemy\type{_}mobilne\type{_}wyklad\type{_}8.pdf
\item
  Dane przestrzenne: http://www.igik.edu.pl/pl/a/Dane-przestrzenne-def
\item
  Wykorzystanie języka Python w GIS:
  http://gis-support.pl/wykorzystanie-jezyka-programowania-python-w-quantum-gis/
\stopitemize


\stoptext
