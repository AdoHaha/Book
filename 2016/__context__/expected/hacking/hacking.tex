\usemodule[pycon-2016]
\starttext

\Title{Hacking MicroPython}
\Author{Radomir Dopieralski}
\MakeTitlePage

When using MicroPython, you have even bigger chance to need your own
library written in C than in other Python implementations. You might
want to use one of the existing libraries for your platform, get access
to some peripherals or other features of the hardware that is not
exposed to Python by default, or even just do something faster and with
smaller memory overhead. To do that, you will need to extend the
existing firmware with your own C code, and this article is going to
show you how to do it, using the ESP8266 port as the example.

\subsection[a-tour-around-the-micropython-repository]{A Tour Around the
MicroPython Repository}

The repository is located on GitHub, at
https://github.com/micropython/micropython -- you can easily fork it and
clone it to your computer. Inside you will see a number of directories:

The \type{docs} directory contains documentation that you can compile
into HTML using Sphinx. The \type{logo} directory contains MicroPython's
logo in various format, for use in documentation and any other materials
on MicroPython. The \type{tools} directory contains a number of utility
programs useful for interaction with various MicroPython boards. The
\type{tests} directory contains automated unit and integration tests.

The most interesting is the \type{py} directory, where the code for the
MicroPython interpreter and runtime live. This is the main source of
information about the API itself, as it's not documented otherwise.
There is also a \type{lib} directory that contains various C libraries.
The \type{drivers} directory holds libraries for interfacing with
additional hardware that you might connect to your boards, and the
\type{extmod} directory contains MicroPython modules implemented in C
that are shared between several ports.

All the other directories contain individual ports of MicroPython for
particular platforms. Each of them contains its own \type{Makefile} and
\type{README} files, which tell you how to build that port and flash it
onto your board. For the ESP8266 port, we are going to go into the
\type{esp8266} directory, of course.

\subsection[setting-up-the-development-environment]{Setting Up the
Development Environment}

As has been said, the \type{README} file contains detailed instructions
on how to download the SDK and compiler, set it up, download and install
all the necessary dependencies, and finally compile the firmware and
flash it to the development board. I will not be repeating the
instructions here, because that is counter-productive and they can
change at any moment anyways. Just follow the \type{README}, and make
sure you can compile your own version of the firmware.

\subsection[adding-your-own-source-file]{Adding Your Own Source File}

In order to add your own MicroPython module written in C, you need to
create a new C file and add references to it to several files, so that
the file is picked up during the compilation.

First of all, you need to add it to the \type{Makefile}, to the list of
source files in the \type{SRC_C} variable. Make sure to follow the same
format as the other files in there. The order doesn't matter much.

\starttyping
SRC_C = \
    main.c \
    system_stm32.c \
    stm32_it.c \
        ...
        mymodule.c
        ...
        adc.c \
    $(wildcard boards/$(BOARD)/*.c)
\stoptyping

The second file you will need to add to is \type{esp8266.ld}, which is
the map of memory used by the compiler. You have to add it to the list
of files to be put in the \type{.irom0.text} section, so that your code
goes into the instruction read-only memory (iROM). If you fail to do
that, the compiler will try to put it in the instruction random-access
memory (iRAM), which is a very scarce resource, and which can get
overflown if you try to put too much there.

Now just create an empty \type{mymodule.c} file, and run the compilation
to see that it is now included in the firmware.

\subsection[creating-a-python-module]{Creating a Python Module}

We have our file, but it doesn't actually do anything. It's empty, and
there is no new python module that we could import. Time to change that.

From the C side, modules in MicroPython are simply structs with a
certain structure. Open the \type{mymodule.c} file and put this code
inside:

\starttyping
#include "py/nlr.h"
#include "py/obj.h"
#include "py/runtime.h"
#include "py/binary.h"
#include "portmodules.h"


STATIC const mp_map_elem_t mymodule_globals_table[] = {
    { MP_OBJ_NEW_QSTR(MP_QSTR___name__), MP_OBJ_NEW_QSTR(MP_QSTR_mymodule) },
};

STATIC MP_DEFINE_CONST_DICT(mp_module_mymodule_globals, mymodule_globals_table);

const mp_obj_module_t mp_module_mymodule = {
    .base = { &mp_type_module },
    .name = MP_QSTR_mymodule,
    .globals = (mp_obj_dict_t*)&mp_module_mymodule_globals,
};
\stoptyping

What does this code do? It just defines a python module, using
\type{mp_obj_module_t} type, and then initializes some of its fields,
such as the base type, the name, and the dictionary of globals for that
module. In that dictionary, it defines one variable, \type{__name__},
with the name of our module in it. That's it.

Now, for this module to actually be available for import, we need to add
it to \type{mpconfigport.h} file to \type{MICROPY_PORT_BUILTIN_MODULES}:

\starttyping
extern const struct _mp_obj_module_t mp_module_mymodule;

#define MICROPY_PORT_BUILTIN_MODULES \
    { MP_OBJ_NEW_QSTR(MP_QSTR_umachine), (mp_obj_t)&machine_module }, \
    ...
    { MP_OBJ_NEW_QSTR(MP_QSTR_mymodule), (mp_obj_t)&mp_module_mymodule }, \
\stoptyping

Now you can try compiling the firmware and flashing it to your board.
Then you can run \type{import mymodule} and see it imported.

\subsection[adding-a-function]{Adding a Function}

Now let's add a simple function to that module. Edit \type{mymodule.c}
again and add this code right after the includes:

\starttyping
#include <stdio.h>


STATIC mp_obj_t mymodule_hello(void) {
    printf("Hello world!\n");
    return mp_const_none;
}
STATIC MP_DEFINE_CONST_FUN_OBJ_0(mymodule_hello_obj, mymodule_hello);
\stoptyping

This creates a function object \type{mymodule_hello_obj} which takes no
arguments, and when called, executes the C function
\type{mymodule_hello}. Also note, that our function has to return
something -- so we return \type{None}. Now we need to actually add that
function object to our module:

\starttyping
STATIC const mp_map_elem_t mymodule_globals_table[] = {
    { MP_OBJ_NEW_QSTR(MP_QSTR___name__), MP_OBJ_NEW_QSTR(MP_QSTR_mymodule) },
    { MP_OBJ_NEW_QSTR(MP_QSTR_hello), (mp_obj_t)&mymodule_hello_obj },
};
\stoptyping

Now when you compile and flash the firmware, you will be able to import
the module and call the function inside it.

\subsection[function-arguments]{Function Arguments}

The \type{MP_DEFINE_CONST_FUN_OBJ_0} macro that we used to define our
function is a shortcut for defining a function with no arguments. We can
also define a function that takes a single argument with
\type{MP_DEFINE_CONST_FUN_OBJ_1} -- the C function then needs to take an
argument of type \type{mp_obj_t}:

\starttyping
STATIC mp_obj_t mymodule_hello(mp_obj_t what) {
    printf("Hello %s!\n", mp_obj_str_get_str(what));
    return mp_const_none;
}
STATIC MP_DEFINE_CONST_FUN_OBJ_1(mymodule_hello_obj, mymodule_hello);
\stoptyping

Note that the \type{mp_obj_str_get_str} function will automatically
raise the right exception on the python side if the argument we gave it
is not a python string. This is very convenient.

It's also possible to define functions with variable number of
arguments, or even with keyword arguments -- you can easily find
examples of that in the modules already included in MicroPython. I will
not be covering this in detail.

\subsection[classes]{Classes}

Let's try to add a class to our module. A class is similar to a module
-- it's also a C struct with certain fields:

\starttyping
STATIC const mp_rom_map_elem_t mymodule_hello_locals_dict_table[] = {
}
STATIC MP_DEFINE_CONST_DICT(mymodule_hello_locals_dict,
                            mymodule_hello_locals_dict_table);

const mp_obj_type_t mymodule_hello_type = {
    { &mp_type_type },
    .name = MP_QSTR_Hello,
    .print = mymodule_hello_print,
    .make_new = mymodule_hello_make_new,
    .locals_dict = (mp_obj_dict_t*)&mymodule_hello_locals_dict,
};
\stoptyping

It needs two functions: one for creating the class and allocating all
the memory it needs, and one for printing the objects of that class
(similar to python's \type{__repr__}). Let's add them near the top of
our file:

\starttyping
typedef struct _mymodule_hello_obj_t {
    mp_obj_base_t base;
    uint8_t hello_number;
} mymodule_hello_obj_t;


mp_obj_t mymodule_hello_make_new(const mp_obj_type_t *type, size_t n_args,
                                 size_t n_kw, const mp_obj_t *args) {
    mp_arg_check_num(n_args, n_kw, 1, 1, true);
    pyb_spi_obj_t *self = m_new_obj(mymodule_hello_obj_t);
    self->base.type = &mymodule_hello_type;
    self->hello_number = mp_obj_get_int(args[0])
    return MP_OBJ_FROM_PTR(self);
}


STATIC void pyb_spi_print(const mp_print_t *print, mp_obj_t self_in, mp_print_kind_t kind) {
    pyb_spi_obj_t *self = MP_OBJ_TO_PTR(self_in);
    mp_printf(print, "Hello(%u)", self->hello_number);
}
\stoptyping

We define a struct to hold all our class data, with one additional field
\type{hello_number}. Then we define functions to create and to print it.

Of course you also need to add the class to the module's globals.
Compile it and try creating and printing objects of our new class.

\subsection[adding-methods]{Adding Methods}

Methods in MicroPython are just functions in the class's locals dict.
You add them the same way as you do to modules, just remember that the
first argument is a pointer to the data struct:

\starttyping
STATIC mp_obj_t mymodule_hello_increment(mp_obj_t self_in) {
    pyb_spi_obj_t *self = MP_OBJ_TO_PTR(self_in);
    self->hello_number += 1;
    return mp_const_none;
}
MP_DEFINE_CONST_FUN_OBJ_1(mymodule_hello_increment_obj,
                          mymodule_hello_increment);
\stoptyping

Also, don't forget to add them to the locals dict:

\starttyping
STATIC const mp_rom_map_elem_t mymodule_hello_locals_dict_table[] = {
    { MP_ROM_QSTR(MP_QSTR_read), MP_ROM_PTR(&mymodule_hello_increment_obj) },
}
\stoptyping

And that's all.

\subsection[going-further]{Going Further}

I hope this will get you started. However, it's not possible to cover
all the possible things you could want to do. You have to look at the
existing code, both in the \type{py} directory, to see what is
available, and in all the port directories, to see how it is used.

There is also a very lively community at the MicroPython forum, where
you can ask for help: http://forum.micropython.org

Happy hacking!


\stoptext
