\enableregime[utf8]

\newif \ifFinalPrint 
\FinalPrintfalse
\FinalPrinttrue

\input oddpage.sty

%% Chcę tylko pdf,
\setupoutput[pdftex]
\pdfcompresslevel=0 % Lepiej sie rsyncuje!
\pdfminorversion=4

% Kwaw mówił, że rozmiar B5 jest taki właśnie
\definepapersize
        [KuranB5]
        [height=250mm, width=170mm]


\newbox   \PingBugBox 
\setbox   \PingBugBox = \box\voidb@x

\definelayer[pagepictures][width=\paperwidth,height=\paperheight]



\ifFinalPrint
   \setuppapersize[KuranB5][KuranB5]
   \def \BUG#{\setbox \PingBugBox = \vbox}
   \setupbackgrounds[page][page][background=pagepictures]\relax
\else
  \setuppapersize[KuranB5][A4]
  \definecolor[tlo1][k=0.2]
  \definelayer[PingBugLayer]
     [width=\paperwidth,
      height=\paperheight]

  \setupbackgrounds[paper][paper][background={color,PingBugLayer},
                                backgroundcolor=tlo1]
  \setupbackgrounds[page][page][background={color,pagepictures},
                                backgroundcolor=white]

  \useexternalfigure[biedrona][../sty/biedrona.pdf]
  \def \BUGBiedrona#1{%  #1 -- rotation, #2 -- width
      \setbox \scratchbox \hbox{\rotate[rotation=%
             \numexpr45*\PingBugSubCnt\relax]{%
              \externalfigure[biedrona][width=#1]}}%
      \setbox \scratchbox \hbox{\raise\dp\scratchbox\box\scratchbox}%
      \dp\scratchbox=0pt\ht\scratchbox=0pt
      \box\scratchbox}
      

  \newcount \PingBugCnt
  \newcount \PingBugSubCnt
  \unprotect
  \def \BUG#{\ifvmode
       \errmessage{\string \BUG in vertical mode, no good}
     \fi
     \global \advance \PingBugCnt 1
     \ifnum 0\getvariable{PingBug}{\number \PingBugCnt}
          = 0\getvariable{PingBug}{\number \numexpr
                                      \PingBugCnt-1\relax}\relax
        \global \advance \PingBugSubCnt 1
     \else
        \global \PingBugSubCnt=0
     \fi
     \hbox to0pt{\hss \BUGBiedrona{8pt}\hss}%
     \expanded {\relax
        \noexpand\writeutilitycommand{%
        \noexpand \setgvariables[PingBug][\number \PingBugCnt
           =\noexpand \number \noexpand \realpageno]}}%
        \global\setbox \PingBugBox \vbox \bgroup
           \ifvoid \PingBugBox \else 
              \unvbox \PingBugBox
              \vskip1ex
           \fi
        \leftskip=0pt \rightskip=0pt
        \startalignment[right]
        \hsize=18cm
        \bgroup
        \aftergroup \stopalignment
        \aftergroup \egroup
        \aftergroup \BUG!A
        \noindent
        \llap{\BUGBiedrona{8pt}}%
        \tfx \setupinterlinespace
        \let \bgroup}

  \def \BUG!A {%
     \ifnum 0\getvariable{PingBug}{\number \PingBugCnt}
          = 0\getvariable{PingBug}{\number \numexpr
                                      \PingBugCnt+1\relax}\relax
          \message{BUG: \number \PingBugCnt\space same page.
             \number \PingBugCnt:\number \numexpr
                                      \PingBugCnt+1\relax;
             --0\getvariable{PingBug}{\number \PingBugCnt}--
             --0\getvariable{PingBug}{\number \numexpr
                                      \PingBugCnt+1\relax}--
                                      }%
     \else
          \message{BUG: \number \PingBugCnt\space shipping}%
        \ifvoid \PingBugBox\else
           \setlayer[PingBugLayer][x=1cm,y=26cm,
                page=\getvariable{PingBug}{\number\PingBugCnt}]{%
              \framed[background=color,backgroundcolor=white,
              framecolor=darkgreen]
              {~\kern10pt\vtop{\kern3pt
               \setbox\scratchbox=\vsplit \PingBugBox to 3cm
               \unvbox\scratchbox
              \kern10pt}~}}%              
        \fi
     \fi}
  \protect


\fi


\setupcolors[state=start]

\setuptolerance[tolerant]

\setupitemize[indentnext=no]

\setuplayout[
        marking=off,
        topspace=1.5cm,
        header=20pt,
        bottomspace=1cm,
        backspace=2cm,
        cutspace=2cm,
        margin=0pt,
        width=fit,
        height=fit,
        location=doublesided
]


\setuptyping[indentnext=no,bodyfont=8pt]

% Bez tego nie widać Antykwy Toruńskiej, nie wiem dlaczego
\loadmapfile [pdftex.map]

\definefloat[rys][rysunki]
\definefloat[tabela][tabele]

\setuphead[section][number=no]


% Standardowy dziubek do \startitemize
\definesymbol[1][\hbox{~$\diamond$}]
\def \startlist{\startitemize[l]}
\def \stoplist{\stopitemize}

% Kwadrat: To byl jakies makro, ale ono sie
% podstawialo w Bayesian A/B testing with Python - Bogdan Kulynych (PyCon PL 2014)
% \def|{\errmessage{Invalid tenteges}}

% Nasze cyferki hex, używa się \inhex number, bez {}
\def \hexdigit #1{%
        \ifcase #1\relax 0\or 1\or 2\or 3\or 4\or 5\or
        6\or 7\or 8\or 9\or A\or B\or C\or D\or E\or
        F\else \gulpundefined\fi}
\def \inhex  {\begingroup \afterassignment \inhexA \count0=}
\def \inhexA {\count1=0
   \loop \ifnum \count0>15 
     \advance \count0 by-16
      \advance \count1 by 1
   \repeat
   \edef \x{\endgroup \hexdigit{\count1}\hexdigit{\count0}}%
   \x}



\newcount \NoOfArticles
\def \ArtPage#1{\csname ArtPage::#1\endcsname}
\def \ArtPageSet#1#2{\global \expandafter \expandafter \expandafter \def \ArtPage{#1}{#2}}
\let \Article=\relax

\begingroup
    \let \par\empty
    \NoOfArticles=0
    \toks0={}
    \def \a{
        \edef \fileline{\fileline\space0\space0\relax}%
        \expandafter \b \fileline\relax
        }%
    \def \A{
       \edef \fileline{\fileline\space}
       \expandafter \B \fileline \relax
    }%
    \def \B #1 {\b #1 0 0 }
    \def \b #1 {
       \if$#1$\expandafter \d \else
          \def\r{#1}%
          \afterassignment \c
          \count0=\fi}%
    \def \c#1\relax{%
        \edef \R{\noexpand \ArtPageSet{\r}{\number\count0 }}\R
        \edef \R{\noexpand \ArtPageSet{\r::firstpage}{\number\numexpr\count1+1\relax}}\R
        \advance \count1 by \count0
        \edef \R{\noexpand \ArtPageSet{\r::lastpage}{\number\count1}}\R
        \ifFirstOdd \ifodd \count1 \advance \count1 1\fi \fi
        \edef \R{\Article{\r}{\noexpand \ArtPage{\r}}}%
        \toks0=\expandafter{\the\toks0\R}%
    }
    \def \d#1\relax{}
    \doprocessfile \scratchread{artlist.txt}\A
    \xdef \PingArticles{\the\toks0}
    \count1=0
    \doprocessfile \scratchread{artpages.inc}\a

\endgroup


\def \CheckPingPages{%\loggingall
%\showthe \numexpr \ArtPage{\jobname}\relax
%\showthe \numexpr \lastpage\relax
\ifnum \ArtPage{\jobname}=\lastpage
\else \loggingall \begingroup
   \ArtPageSet{\jobname}{\number \lastpage}%
   \immediate\openout \scratchwrite=artpages.inc\relax
   \def \Article##1##2{\immediate\write\scratchwrite{##1 ##2}}%
   \PingArticles
   \immediate \closeout \scratchwrite
   \endgroup
\fi%\tracingnone
}
%\appendtoks \CheckPingPages \to \everystarttext
%\show \starttext
%\CheckPingPages
\appendtoks \enableregime[utf-8]\to \everystarttext



% I z tego powodu właśnie wyłączamy automat stronicujący
%\setuppagenumbering[state=stop]
%\expandafter \show \csname ArtPage::\jobname::firstpage\endcsname
\setuppagenumber[number=0\ArtPage{\jobname::firstpage}]


\def \noang #1){\leavevmode{\english \it #1\/})}
\def \ang {\leavevmode \tf ang.~\noang}

\newtoks \HeaderAuthor
\newtoks \HeaderTitle

\newtoks \PageAuthor
\newtoks \PageTitle

\newbox \AbstractBox \setbox\AbstractBox=\vbox{}


\usetypescriptfile [type-gyr]
\usetypescript [type-gyre][qx]
\usetypescript [antykwa-torunska][qx]

\definetypeface [pagella] [rm] [serif] [pagella] [default] [encoding=qx]
\definetypeface [pagella] [tt] [mono]  [cursor]  [default] [encoding=qx]
\definetypeface [pagella] [mm] [math]  [pagella] [default] [encoding=qx]


\definetypeface [bonum] [rm] [serif] [bonum] [default] [encoding=qx]
\definetypeface [bonum] [tt] [mono]  [cursor]  [default] [encoding=qx,rscale=1.05]
\definetypeface [bonum] [mm] [math]  [bonum] [default] [encoding=qx,rscale=1.1]



\setupbodyfont [bonum,9pt]

\mainlanguage[polish]\language[polish]
\hyphenation{
        także
        source-forge
}

     \fontdimen 3 \the\font = 0.25\fontdimen 6 \the\font
     \fontdimen 4 \the\font = 0.12\fontdimen 6 \the\font

%\showframe
%\showgrid

        


\def \pingheader{\vbox{%
   \hbox to \hsize {%
     \ifnum \realpageno=1 \hss\else
     \ifodd \pageno
        \hfil\quad\HeaderTitle\quad\hfil\llap{\inhex\pageno}%
     \else
        \rlap{\inhex\pageno}\hfil\quad \HeaderAuthor\quad\hfil
     \fi\fi}}}
   
\def \AddToToks#1#2{#1=\expandafter{\the #1#2}}

\def \Author#1{\begingroup
        \edef \h{\the \HeaderAuthor}%
        \ifx \h \empty
           \endgroup 
        \else
           \endgroup \AddToToks \HeaderAuthor {, }%
        \fi
        \AddToToks \HeaderAuthor {#1}%
        \AddToToks \PageAuthor{\AuthorLine{#1}}%
}

\def \AuthAffil#1{\AddToToks \PageAuthor{\AuthorAffilLine{#1}}}

\def \Title#1{\HeaderTitle{#1}\PageTitle{#1}}

% \ignorespacesandpars ignores all spaces and \par
% It assumes ~ is active and doesn't ignore it
\begingroup 
  % define the following with ~ fixed as active
  \catcode`\~=13
  \def ~ {\endgroup
     \def \ignorespacesandpars{\begingroup
       \let ~\relax
       % it's tempting to use \let\par\empty but it's dangerous,
       % for instance what if we encounter \csname \number
       % 0...\par..
       \let \par=\space 
       \begingroup
       \aftergroup \futurelet
       \aftergroup ~\aftergroup \ignorespacesandparsA
       \begingroup
          \def ~{\begingroup\aftergroup \noexpand \expandafter
          \aftergroup \endgroup}%          
          \edef~{\endgroup~~~~~~~~}~\endgroup}%
     \def \ignorespacesandparsA{%
       \ifx ~^^20% one space
         \aftergroup \ignorespacesandpars
         \afterassignment \endgroup
         % as a side effect a space is eaten 
         \count0=0%
       \else
         \expandafter \endgroup
       \fi}}~


\def \startabstract{\setbox \AbstractBox = \vbox \bgroup
   \startnarrower[2*left,2*right]
        \def \stopabstract{\stopnarrower \egroup}
        \noindent \tfx {\sc Streszczenie:~}\ignorespacesandpars}

\def \AuthorLine#1{#1\crlf}
\def \AuthorAffilLine#1{~~~#1\crlf}

\font \TitleFont = qx-anttb at 18pt

\newif \iftitpage \titpagefalse

\def \MakeTitlePage{
        %% \ifodd \pageno \else         
        %%        \ifFinalPrint\else
        %%        ===tu będzie pusta strona, bo parzysta==\par
        %%        ===tego tekstu na finalnym wydruku nie będzie==\par
        %%        \fi
        %%         \noheaderandfooterlines\null\vfill\break
        %% \fi
        \setbox0 = \vbox {\titpagetrue
            \startalignment[right,broad,nothyphenated]
                 \TitleFont\setupinterlinespace
                  \noindent 
                  \let \TitleBreak=\crlf
                 \the \PageTitle\par
            \stopalignment
            \blank[big]
            \startalignment[right]
                \noindent {\tfa
                \the \PageAuthor}
            \stopalignment
            \blank[big]
            \vfill 
            \unvbox \AbstractBox
            \blank[big]
            \hbox{}
        }
        \ifdim \ht0<.5\vsize
          \setbox0=\vbox to 0.5\vsize{\unvbox0\vfil}
        \fi
        \noheaderandfooterlines
        \box0
        \CheckPingPages
}

\setupheadertexts[{%
   \let \TitleBreak=\empty
   \ifodd \pageno
      \hfill~~~\the \HeaderTitle~~~\hfill
      \llap{\inhex\pageno}%
   \else
      \rlap{\inhex\pageno}%
      \hfill~~~\the \HeaderAuthor~~~\hfill
   \fi}]
        

\setupindenting[medium,yes]
\setuplabeltext[pl][figure=Rys.~]
\setupfootnotes[margindistance=5pt]


% różności
\def \emph#{\leavevmode \bgroup \it \bgroup
        \aftergroup \/%
        \aftergroup \egroup
        \let \bgroup}


\def \startbibl {\startitemize[n]}
\def \stopbibl{\stopitemize}
\def \bibitem{\item}


\iffalse % OLD URLS
% URL
\def \URLciachciach{:/\URLciach}
\def\URLciach{\egroup
        \discretionary{\hbox{/$\Rightarrow$}}{\hbox{$\Rightarrow/$}}{/}%
        \hbox\bgroup}
\def \URLunciachciach #1://#2\URLunciachciach {%
        \ifx$#2$%
            #1%
        \else
            \URLunciachciach #1\URLciachciach#2\URLunciachciach
        \fi}
\def \URLunciach #1/#2\URLunciach{%
        \ifx$#2$%
           #1%
        \else
           \URLunciach#1\URLciach #2\URLunciach
        \fi}
\def \URL#{\leavevmode \begingroup
        \catcode`\_=12 \catcode`\#=12
        \catcode`\%=12 \catcode`\&=12
        \catcode`~=13
        \def~{\char`\~}%
        \tt \URLA}
\def \URLA#1{\hbox{%
          \URLunciachciach
          \URLunciach
          #1%
          /\URLunciach
          ://\URLunciachciach
        }\endgroup\afterURL}
\def \afterURL{}

\else

\def \URL#{\leavevmode
  \bgroup % it will be ended by closing \}
    % define \a as expandafter \URLprepare \string with
    % sufficient number of \expandafter to eat the following
    % \else, \or or \fi
    \def \a{%
      \expandafter
        \expandafter
      \expandafter
        \URLprepare
      \expandafter
        \string
    }%
    \def \do##1{\catcode`##1=12\relax}%
    \do\&\do\%\do\\\toks0={}%
    \count0=1% braces check
    \afterassignment \URLprepare
    \let \bgroup=}

\def \URLprepare{\futurelet \URL \URLprepareA}

\def \URLprepareA{%
  \ifcase 
      \ifcat \noexpand \URL \bgroup  0\else
      \ifcat \noexpand \URL \egroup  1\else
      \ifcat \noexpand \URL \space   2\else
      \ifcat \noexpand \URL +3\else  4\fi\fi\fi\fi
      \space % this space gets eaten by number
    % Case 0  begingroup, 
    \CaseZero
    \advance \count0 1 \a 
  \or
    % Case 1: endgroup
    \advance \count0-1
    \ifnum \count0=0
      \URLtypeset
    \else
       \expandafter\a
    \fi
  \or
    % Case 2: space
     \CaseThree
    \toks0=\expandafter{\the\toks0 \URLspace}%
    \afterassignment \URLprepare
    \count1=0% 
  \or
    % case 3: other character, put it to toks0 interpretting
    \expandafter \URLprepareB
  \else
    % case 4: non-other character, make it other
    \a
  \fi}

\def \URLprepareB#1{%
   \ifcase 0%
      \ifnum`#1=`&1\fi
      \ifnum`#1=`/2\fi
      \ifnum`#1=`?1\fi
      \toks0=\expandafter{\the\toks0 #1}%
   \or
      \toks0=\expandafter{\the\toks0 \URLbreakable{#1}}%
   \else
      \toks0=\expandafter{\the\toks0 \URLciach}%
   \fi
   \URLprepare}

\def \URLciach#1{%
  \ifx #1\URLciach \expandafter \firstoftwoarguments
  \else \expandafter \secondoftwoarguments \fi
  \URLciachciach
  {\URLbreakable{/}#1}}

\def \URLciachciach{\setbox0=\hbox{/}/\kern-0.3\wd0/}

\def \URLtypeset{\tt 
  \def\URLbreakable##1{%
    \discretionary{\hbox{##1$\Rightarrow$}}{\hbox{$\Rightarrow/$}}{##1}}%
  \the\toks0 }
\def \afterURL{}

\fi

% Sometimes we need to have indented paragraph in itemize or
% parshape. Therefore I created \pseudopar. It indents but
% doesn't end the current paragraph. After \pseudopar you can
% have arbitrary number of spaces and \pars

\def \afterwhite#1{% #1 MUST be a single token.
  \begingroup
     \aftergroup #1\relax\relax
     \edef \par{\let \par= \space}\par
     \afterwhiteA
}

\def \afterwhiteA{\futurelet \afterwhite \afterwhiteB}

\def \afterwhiteB{%
  \ifx \afterwhite \par
     \def \afterwhite{\let \afterwhite = }%
     \afterassignment \afterwhiteA
  \else
     \def \afterwhite{\endgroup}%
  \fi
  \afterwhite}

\def \pseudopar{\hfil\break\null\kern\parindent\afterwhite\relax}


% Bibliografie
\definenumber [bibnum]
\def \bibitemindent{25pt}
\def \bibit{\dosingleargument \dobibit}
\def \dobibit[#1]{\par\incrementnumber[bibnum]
\begingroup
\noindent\begingroup
\llap{\hbox to \bibitemindent{[\getnumber[bibnum]]\hss}}%
\begingroup 
   \edef \a{\endgroup
   \if$#1$\relax
   \else
      \noexpand\reference
      [bib:#1]{\getnumber[bibnum]}%
   \fi}\a
\def \title##{\bgroup \it \let \bgroup=}%
\def \aut##{\bgroup \tf \let \bgroup=}%
\def \par{\endgroup\parshape 1 \bibitemindent 
  \dimexpr \hsize-\bibitemindent\par\endgroup}\ignorespaces}
\newif \ifbibcitenext
\def \dobibcite#1{\ifbibcitenext,\kern0.1em\relax\else\bibcitenexttrue\fi
  \in{}[bib:#1]}
\def \bibcite[#1]{[%
    \bibcitenextfalse
    \processcommalist[#1]\dobibcite]}

\def\cite[#1]{\leavevmode \nobreak \kern 0.3em\nobreak\bibcite[#1]}

% Vreserve
% Usage:
% \vreserve[20pt][-3000] % reserve 20pt with penalty -3000
% default vreserve is \baselineskip, default penalty is -300
% default unit is \baselineskip
\def \vreserve{\dodoubleargument \dovreserve}
\def \dovreserve[#1][#2]{\ifhmode\par\fi
     \begingroup 
        \afterassignment \gobbleuntilrelax
        \count0=#2 -300 \relax
        \afterassignment \gobbleuntilrelax
        \dimen0=#1\baselineskip\relax
        \edef \a {\endgroup
        \vskip 0pt plus \the\dimen0
        \penalty \the\count0
        \vskip 0pt plus -\the\dimen0
        }\a}
