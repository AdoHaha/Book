\usemodule[pycon-yyyy]
\starttext

\Title{Python w projektach sprzętowych}
\Author{Piotr Maliński}
\MakeTitlePage

\subsection[skąd-ta-popularność-projektów-hardware-plus-software]{Skąd
ta popularność projektów \quotation{hardware plus software}?}

Nieustający rozwój technologii cyfrowych popularyzuje dostęp do
Internetu, jak i coraz lepszą dostępność różnego rodzaju urządzeń
elektronicznych podłączonych do niego. Przeciętny Kowalski nie tylko ma
smarphona, ale się z nim praktycznie nie rozstaje. Ma dostęp do
Internetu poprzez sieć 3G, jak i poprzez coraz liczniejsze hotspoty
WiFi. Jako konsument zaznajomiony z nowoczesną technologią i siecią
staje się potencjalnym nabywcą kolejnych urządzeń i cyfrowych
udogodnień. Kosz na śmieci może być inteligentny i wysłać Ci na
smartphona listę produktów, które zużyłeś. Razem z inteligentną lodówką
przygotują Ci listę zakupów, a inteligentny sklep spożywczy podliczy
cenę produktów bez wyjmowania ich z koszyka. Zwykły kosz, czy lodówka to
już za mało. Nastały czasy, w których słusznie lub nie wszystko musi być
w sieci, musi być \quotation{inteligentne}, \quotation{nowoczesne}.

Trendy, czy moda cyfrowej rewolucji, tworzy rynek dla nowych produktów
sprzętowych, jak i daje zajęcie dla programistów nie mających wcześniej
do czynienia z takimi projektami. Inteligentna lodówka musi przecież
wysyłać gdzieś w chmurę dane tak, by aplikacja webowa, czy mobilna,
mogła zadziałać swoją magię. W tym cyfrowym wyścigu udział biorą duże
firmy, jak i młode innowacyjne startupy, więc zapotrzebowanie na
software, czy prototypowanie produktów sprzętowych, jest spore i chyba
na tyle ciekawe, by się tym zainteresować.

\subsection[miejsce-dla-programistów-pythona-w-świecie-rzeczy]{Miejsce
dla programistów Pythona w świecie rzeczy}

Produkcyjnie świat rzeczy (ang. IoT - Internet of Things), czy internetu
rzeczy, ma niewiele wspólnego z Pythonem, czy innymi językami
skryptowymi. Niemniej zanim nowy inteligentny produkt trafi do
produkcji, musi być zaprojektowany i w miarę potrzeb połączony z
usługami i aplikacjami w sieci. Potrzebne są prototypy i testy. Zamiast
zatrudniać do tego od razu programistę C/ASM ze znajomością
mikrokontrolerów ATmega, czy STM, można zaprototypować coś szybciej z
wykorzystaniem Pythona, JavaScriptu, Lua, czy także C/C++ na platformach
dla „twórców”. Arduino, czy Raspberry Pi, nie wymagają od nas znajomości
asemblera, czy niuansów mikrokontrolerów. Oferują łatwe w użyciu, dobrze
udokumentowane API i biblioteki, które można wykorzystać do
prototypowania elektroniki.

Programista Pythona ma do wyboru kilka platform sprzętowych. MicroPython
dostępny jest na kilku potężnych mikrokontrolerach STM, czy też
popularnym i tanim ESP8266 z WiFi. Podobnie platforma Zerynth, który
pozwala programować w Pythonie, którego kompiluje przed wrzuceniem na
mikrokontroler. Wysokopoziomowe platformy Tinkerforge czy Phidgets także
obsługują Pythona, podobnie jak bardziej wyspecjalizowana własnościowa
platforma firmy Synapse. Jeżeli potrzebujemy większej mocy
obliczeniowej, to możemy skorzystać z Raspberry Pi i bardzo wielu innych
\quotation{komputerów na płytce}, czy nawet klasycznych komputerów z
procesorami x86 Intela i AMD.

\subsection[kiedy-i-do-czego-stosować-pythona-w-projekcie-sprzętowym]{Kiedy
i do czego stosować Pythona w projekcie sprzętowym?}

Załóżmy na razie, że nasz projekt ma być produkowany w dużych ilościach
- że to nie będzie jedna, dwie sztuki używane lokalnie przez nas.

Platformy Pythonowe, czy ogólnie wszystkie platformy do prototypowania
elektroniki, są dobre, jak sama nazwa wskazuje, do prototypowania
elektroniki. Jest to coś, o czym niektóre projekty na Kickstarterze
potrafią zapomnieć. Prototypując układ mamy możliwość szybkiego
sprawdzenia niektórych komponentów oraz całości jako funkcjonalny
produkt. Prototyp umożliwia też prace nad połączeniem go z zewnętrznym
oprogramowaniem. Gdy to mamy za sobą i prototyp działa na stole, trzeba
przejść do dalszych testów - funkcjonalności, jak i pierwszych testów w
warunkach, w jakich będzie pracował końcowy produkt (np. ujemne
temperatury). Dla przykładu podam przypadek pomp infuzyjnych - medyczne
urządzenia dozujące dawki leków dla pacjentów. W zależności od
producenta spotyka się w nich dwie typy klawiatur - numeryczną, gdzie
trzeba wprowadzić poprawną ilość, oraz strzałkową, gdzie trzeba
zwiększać/zmniejszać dawkę leku, aż osiągnie się poprawną wartość. Testy
z udziałem pielęgniarek wykazały, że klawiatury strzałkowe dają mniej
niezauważonych błędów przy wprowadzaniu dawki - pielęgniarka ciągle
patrzy się na wyświetlacz próbując ustawić prawidłową wartość zamiast
patrzeć się na numeryczną klawiaturę.

Gdy mamy dopieszczony prototyp, trzeba wezwać specjalistów od
mikrokontrolerów i elektroniki, którzy takie prototypy przeleją na
produkcyjne układy. W tej fazie redukuje się koszty optymalizując dobór
komponentów (dość często nie potrzeba aż tak rozbudowanych kontrolerów,
jak te używane przez np. MicroPythona), jak i całość przelewa się na
płytki PCB. To, co wyjdzie z kilku pierwszych iteracji, to też będą w
pewnym zakresie prototypy - trzeba zmontować i przetestować, by upewnić
się, czy elektronika działa poprawnie (czasami elementy mogą być źle
rozmieszczone i powodować zakłócenia). Jeżeli od razu zamówimy kilkaset,
czy więcej PCB, to zostaniemy z samymi kosztami, bo hardware nie da się
poprawić tak łatwo, jak oprogramowanie. Tworzenie sprzętu to wiele
iteracji - od płytki stykowej po kilka wersji PCB.

Specjalista od mikrokontrolerów przyda nam się także do oprogramowania
docelowego układu. Przy wysokopoziomowych platformach do prototypowania
niektóre kwestie nie są aż takie ważne - zarządzanie energią, usypianie
i wybudzanie mikrokontrolera, efektywność sterowania niektórych
komponentów. Staje się to szczególnie ważne, gdy nasz produkt ma działać
na zasilaniu bateryjnym, jak i przy doborze komponentów (np. inny,
tańszy model wyświetlacza bez biblioteki do Arduino, czy Raspberry Pi).
Oszczędność kilku/kilkunastu lub więcej złotych robi ogromną różnicę,
gdy chcemy zlecić produkcję wielu sztuk naszego urządzenia. Jeszcze
większą różnicę robi dobór części, które są dostępne w wystarczających
ilościach.

W przypadku własnych projektów na niską skalę nie trzeba iść aż tak
daleko, ale zawsze warto przejść z surowego prototypu na płytce stykowej
na PCB. Istnieją serwisy, w których możemy zamówić kilka płytek naszego
autorstwa. Mając dwie trwałe płytki - jedna z mikrokontrolerem (Arduino,
MicroPython), a druga z resztą układu znacząco podnosi się trwałość i
niezawodność.

\subsection[pythonowe-platformy-do-prototypowania-elektroniki]{Pythonowe
platformy do prototypowania elektroniki}

Python dostępny jest na kilku platformach, projektach związanych z
elektroniką. Tinkerforge, czy Phidgets są bardziej ukierunkowane na
prototypowanie większych i rozproszonych projektów. Zerynth stara się
oferować także narzędzia produkcyjne, natomiast mikrokontrolery z
MicroPythonem, płytki Raspberry Pi i podobne projekty skupiają się na
\quotation{mniejszej} elektronice.

MicroPython to implementacja interpretera Pythona działająca na
mikrokontrolerze. Ma zaimplementowaną część biblioteki standardowej i
nie jest kompatybilna z większością bibliotek do klasycznego Pythona.
Zaleta mikrokontrolera z MicroPythonem jest taka, że wrzucamy do pamięci
flash plik z kodem Pythona i gotowe. Nie trzeba kompilować, a sam język
jest bardzo dobry w projektach edukacyjnych z udziałem dzieci i
młodzieży. MicroPython jest dostępny na oficjalnej płytce deweloperskiej
projektu. Można go też wgrać na ESP8266 i kilka innych mikrokontrolerów,
na które został przeportowany. Firma PyCom oferuje swoje płytki
deweloperskie z MicroPythonem (WiPy, LoPy itd.). Podobnie robi Adafruit
ze swoim forkiem pod układy SAM obecne w produktach tej firmy. Firma ST
od niedawna oferuje też produkcyjny układ SPWF04 (podobny do ESP8266) z
MicroPythonem \quotation{prosto z pudełka}. Płytka edukacyjna MicroBit
także go wykorzystuje. Wybór jest więc dość spory. Do wad można zaliczyć
ograniczoną ilość bibliotek do komponentów, co może utrudnić
prototypowanie.

Zerynth to komercyjna i częściowo otwarta platforma, której celem jest
obsługa projektu sprzętowego od prototypu po produkcję. Pozwala
programować w Pythonie i C, a za pomocą Zerynth VM pozwala generować
gotowy firmware na produkcyjne MCU. Do tego wsparcie dla usług w
chmurze, aplikacja mobilna i aktualizacje OTA. Na chwilę obecną trudno
mi określić, na ile projekt ten jest efektywny w tym, co twierdzi, że
robi. Usługa rozwija się, niemniej popularność jest raczej mała i
przynajmniej w przypadku Pythona ograniczeni jesteśmy dostępnością
bibliotek do komponentów. Zaletą na pewno jest IDE i dodatkowe
narzędzia, wadą - vendor lockin.

Jako trzecią platformę możemy potraktować komputery, w szczególności
Raspberry Pi, które łączą komputer z zestawem GPIO i dość dużą ilością
bibliotek dla dodatkowych komponentów. Mając do dyspozycji USB, WiFi,
Bluetooth, Ethernet możemy interfejsować i komunikować się z różnoraką
elektroniką i usługami sieciowymi. USB można też wykorzystać do
komunikacji szeregowej z mikrokontrolerami. Jeżeli musimy wykorzystać
sprzęt \quotation{wyższego poziomu}, np. kamerę przemysłową, to zapewne
będziemy potrzebować MS Windows lub Linuksa i API producenta dla Pythona
(lub API .NET i IronPython, ew. interfejs COM dla starszego sprzętu).
Zostając w temacie kamer przemysłowych - sporo producentów ostatnio
dostarcza wsparcie dla Pythona (np. PointGrey/FLIR). W przypadku
zasilania bateryjnego należy uwzględnić, że nawet oszczędne płytki z
procesorami ARM będą pobierały znacznie więcej prądu niż mikrokontroler,
jak i nie będzie można ich aż tak efektywnie usypiać.

Oddzielną grupę stanowią projekty takie jak Tinkerforge, czy Phidgets.
Projekty te oferują wiele komponentów i płytek kontrolujących. Całość
jest zintegrowana wysokopoziomowym API dostępnym dla licznych języków. W
tego typu projektach skryptując np. silnik krokowy nie musimy
implementować wysyłania serii impulsów, tylko korzystamy z gotowej
metody wykonującej określony ruch silnika. Projekty tego typu nadają się
do prototypowania większych rozproszonych projektów sprzętowych - np.
automatyzacji domu. Niekomercyjne projekty \quotation{we własnym
zakresie} też będą pasować. Ceny komponentów są wyraźnie wyższe od
masowo dostępnych (chińskich) odpowiedników, ale w cenę wliczona jest
integracja i gotowe API. Elementy montażowe to dodatkowe udogodnienie
przy prototypowaniu. Należy jednak pamiętać że to projekty zamknięte na
siebie - nie oferują zazwyczaj niskopoziomowej integracji z komponentami
spoza projektu (np. komunikacji I2C, SPI).

Czasami Pythonowe platformy mogą okazać się niewystarczające. Pośród
mikrokontrolerów najpopularniejszą platformą do prototypowania jest
Arduino wykorzystujące C. Ilość bibliotek do różnych, nawet mało
popularnych komponentów i duża społeczność sprawiają, że ta platforma
może okazać się jedyną pozwalającą stworzyć prototyp szybko i
efektywnie. Dodatkową zaletą jest szeroki wybór mikrokontrolerów - wiele
firm zapewnia wsparcie w IDE Arduino dla swoich płytek deweloperskich.


\stoptext
