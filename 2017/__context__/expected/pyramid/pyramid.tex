\usemodule[pycon-yyyy]
\starttext


\section[pyramidalny-mikroframework---marcin-bardź]{Pyramid(alny)
mikroframework - Marcin Bardź}

\section[wstęp]{Wstęp}

Pyramid jest frameworkiem webowym rozwijanym przez inicjatywę
społecznościową Pylons Project, która za cel postawiła sobie tworzenie
pythonowych technologii webowych spełniających najwyższe standardy,
takie jak 100\letterpercent{} pokrycia w dokumentacji i w testach, czy
wsparcie dla Pythona 2.x i 3.x. Autorzy przyznają, że podczas tworzenia
Pyramida mocno inspirowali się Zope, Pylons i Django, a ja bym do tej
listy dodał jeszcze Ruby on Rails, którym inspirowały się wszystkie
powyższe frameworki.

Teraz może wytłumaczę się z tego niedorzecznego tytułu - otóż jest to
najbardziej zwarta charakterystyka frameworka Pyramid, jaką udało mi się
wymyślić. Piramidalny to - za słownikiem PWN - \quotation{olbrzymi,
kolosalny, zwykle o pomyłce, głupstwie itp.} i takie właśnie było moje
wrażenie, gdy zobaczyłem (nomen omen) Pyramid po raz pierwszy. Jest to
niesamowicie rozbudowane narzędzie, nie będące jednocześnie
full-stackowym frameworkiem.

Pyramid nie jest może najpopularniejszy, ale zawsze pojawia się w
zestawieniach web frameworków i zyskał sobie uznanie starych wiarusów
spod znaku ZOPE i Plone.

Niniejszy artykuł nie jest wprowadzeniem do Pyramida - uważam, że nie ma
potrzeby powielać oficjalnego (skądinąd całkiem dobrego) tutoriala (link
w źródłach). Zamiast tego wybrałem i opisałem kilka interesujących, moim
zdaniem, cech frameworka, które wyróżniają go spośród tłumu konkurentów.

Ponadto, choć jestem wyznawcą zasady, że jedna linijka kodu znaczy
więcej niż 0xFF słów, zabrakło niestety miejsca na przykłady. Dlatego
moim celem będzie zaintrygowanie czytelnika na tyle, by sam sięgnął do
materiałów.

\section[przytłaczający]{Przytłaczający}

Pierwsze spojrzenie na Pyramida może przerazić nieprzygotowanego
pythonowca: większa część \quotation{hello world} przypomina magiczne
zaklęcia, a dokumentacja jest tak obszerna, że przeczytanie samego spisu
treści zajmuje parę minut. Nie mówiąc już o tutorialu, który składa
się~z, bagatela, 10 rozdziałów i 52 podrozdziałów! Do tego dochodzi
instalacja, która pobiera chyba połowę PyPI, a samo spojrzenie na
submoduły wywołuje zawrót głowy.

Nie jest to zachęcające pierwsze spojrzenie i z tego powodu dla wielu
może być spojrzeniem ostatnim. Przyjrzyjmy się zatem, czy rzeczywiście
jest aż tak źle.

\subsection[rozmiar]{Rozmiar}

Podstawowy Pyramid zawiera aż 32 moduły, jednakże nie ma się czego bać,
gdyż całość podzielona jest na logiczne obszary, z których jedne są
większe, a inne całkiem małe (na przykład moduł \type{pyramid.decorator}
zawiera tylko jedną funkcję \type{reify()}). Taki podział bardzo ułatwia
poszukiwania i pogłębianie wiedzy, wszystko związane z danym tematem
zebrane jest w jednym miejscu.

\subsection[dokumentacja]{Dokumentacja}

Jak już wspomniałem wcześniej, obszerność dokumentacji może onieśmielać
- twórcy Pyramida dbają o to, by dokumentacja była zawsze kompletna, co
przy rozmiarze frameworka przekłada się na niemały kawałek literatury.

Na szczęście w materiałach łatwo znaleźć to, czego się~szuka - każdy z
czterdziestu (sic!) rozdziałów skupia się na jednym tylko zagadnieniu,
dzięki czemu nie trzeba skakać po wielu miejscach, żeby zorientować się
w konkretnym temacie.

\subsection[dodatki-add-ons]{Dodatki ({\em add-ons})}

Poza podstawowym frameworkiem mamy jeszcze całą rodzinkę dodatków,
zarówno tworzonych i utrzymywanych w ramach Pylons Project (około 100
sztuk), jak i niezależnych. Daje to sumę około 350 pakietów w PyPI.

Dodatki są bardzo zróżnicowane od prostych typu \type{paginate}, przez
rozszerzenia funkcjonalności jak \type{pyramid_jinja2} czy
\type{pyramid_ldap}, a na wielkich aplikacjach (CMS, platforma dla
mikroserwisów itp.) kończąc.

\subsection[ciągły-i-nieubłagany-rozwój]{Ciągły i nieubłagany rozwój}

Od momentu powstania Pyramid rozwija się w stałym, szybkim tempie -
kolejna wersja pojawia się około dwóch razy w roku, przynosząc nowe
funkcjonalności i usprawnienia. Autorzy nie boją się~dokonywać poważnych
zmian, jeśli są to zmiany na lepsze, co czasem może skutkować tym, że
nasz projekt po aktualizacji nagle staje się przestarzały, a w dłuższej
perspektywie przestaje działać z nowo wydaną wersją Pyramida.

Ponadto twórcy Pyramida nie poddają się zjawisku
{\em not-invented-here}, czyli uporczywej implementacji wszystkiego od
zera tylko po to, aby pozbyć się zależności od \quotation{obcych}
bibliotek. Na przykład jedną z ciekawszych zmian między Pyramidem 1.7 a
1.8 była rezygnacja z rozwijanego przez długi czas mechanizmu
{\em scaffoldów} (szablonów automatyzujących np. tworzenie nowego
projektu) na rzecz popularnej biblioteki zewnętrznej
\type{cookiecutter}. Pociągało to za sobą poważne zmiany zarówno w
kodzie, jak i w dokumentacji oraz testach (pamiętamy -
100\letterpercent{} pokrycia). Jednakże praca została wykonana, a bilans
zysków i strat był na pewno dodatni.

\subsection[zależności]{Zależności}

Pomimo licznych zewnętrznych bibliotek wymaganych do funkcjonowania (dla
wersji 1.9 naliczyłem ich 11 plus drugie tyle, żeby odpalić
\quotation{Hello World!}) Pyramid działa stabilnie i nie widać zgrzytów
między modułami. Wynika to po części z tego, że większość zależności
należy do Pylons Project, zaś niektóre moduły zewnętrzne były pierwotnie
częścią Pyramida i zostały \quotation{oderwane} od frameworka po to, by
umożliwić ich niezależne wykorzystanie (twórcy stawiają na otwartość).

Najbardziej integralną zależnością Pyramida jest \type{WebOb} -
biblioteka opakowująca żądania HTTP w środowisku WSGI, używana w wielu
innych projektach (jak choćby TurboGears, czy Google App Engine) i
będąca częścią Pylons Project.

\section[totalna-swoboda]{Totalna swoboda}

Pyramid bardzo mało narzuca programiście, dając mu swobodę wyboru, z
którą wielu nie wie co zrobić. Z jednej strony mamy nieskrępowaną
wolność kodowania, która pozwala rozwinąć skrzydła programiście, a także
umożliwia pracę z ulubionymi bibliotekami. Z drugiej jednak strony,
programista może poczuć się zawieszony w próżni, nie wiedząc w jaki
sposób zaimplementować daną funkcjonalność.

\subsection[struktura-projektu]{Struktura projektu}

Pyramid nie zakłada jedynej słusznej struktury projektu - dopuszczalne
są zarówno aplikacje jednoplikowe, jak i rozbudowane, wielomodułowe. W
praktyce jednak, w dokumentacji i w szablonach (\type{cookiecutters})
zastosowany jest układ plików, który sprawdza się w większości
zastosowań.

\subsection[baza-danych]{Baza danych}

Pyramid nie jest w żaden sposób zależny od jakiejkolwiek bazy danych i
zostawia użytkownikowi decyzję o użyciu bazy i o jej rodzaju. Nie
oznacza to, że programista pozostawiony jest samemu sobie - istnieją
oficjalne szablony projektów współpracujące z SQLAlchemy i z ZODB.
Ponadto w PyPI można znaleźć wsparcie dla Cassandry, Redisa, MongoDB i
wielu innych.

\subsection[szablony-templates]{Szablony ({\em templates})}

Nie zdziwi cię czytelniku fakt, że i w tym miejscu panuje pełna swoboda.
Dostępne są oficjalne integracje z Jinja2, Chameleonem i Mako, a także
wiele nieoficjalnych - do wyboru, do koloru. Dzięki mechanizmowi
{\em rendererów} używanie szablonów jest proste i przyjemne.

\section[formularze]{Formularze}

Tworzenie i walidacja formularzy nie jest częścią Pyramida, ale z pomocą
przychodzi nam Pylons Project i biblioteka \type{deform}. Ponadto wśród
oficjalnych dodatków możemy znaleźć jeszcze \type{WTForms}.

\section[killer-features]{{\em Killer features}}

Po tych wszystkich przytłaczających informacjach czas przejść do
przyjemniejszych tematów. Poniżej przedstawiam najciekawsze, moim
zdaniem, elementy Pyramida, które niekoniecznie są unikalne i
niepowtarzalne, ale obecność ich wszystkich w jednym frameworku czyni go
bardzo potężnym.

\subsection[widoki]{Widoki}

Widok to kawałek kodu powiązany z konkretną ścieżką aplikacji. W
Pyramidzie może to być zwykła funkcja albo klasa bądź obiekt posiadający
metodę \type{__call__()}. Ostatecznym celem widoku jest utworzenie
obiektu typu \type{Response} i zwrócenie go jako wynik wywołania
funkcji.

\subsection[konfiguracja]{Konfiguracja}

Pyramid, poza imperatywnym mechanizmem konfiguracji (funkcje typu
\type{add_view()}) udostępnia także bardzo wygodny deklaratywny
mechanizm, oparty na dekoratorach. Dzięki temu zamiast wspomnianego
\type{add_view()} wystarczy dodać przed naszą funkcją widoku dekorator
\type{@view_config()}, który zostanie odnaleziony podczas startu
aplikacji.

Dzięki bibliotece \type{Venusian} (należącej oczywiście do Pylons
Project) aplikacja może zostać przeskanowana pod kątem konfiguracji
zaszytej w rozsianych po całym kodzie dekoratorach. Dzięki takiemu
podejściu konfiguracja znajduje się razem z powiązanym z nią kodem i w
razie zmian nie ma konieczności dokonywania zmian w dwóch miejscach.
Minusem jest to, że skanowanie konfiguracji wykonywane jest od nowa przy
każdym uruchomieniu aplikacji, co może opóźnić nieco jej start.

\subsection[traversal]{Traversal}

Pyramid posiada dwa (a jakże) mechanizmy służące do mapowania adresu URL
na kod aplikacji. Pierwszy z nich, zwany {\em URL Dispatch}, to
klasyczny {\em routing} i jest podobny do mechanizmów obecnych w innych
frameworkach, a co za tym idzie nudny - tworzymy listę mapującą wzorce
ścieżki (wyrażenia regularne) na widoki.

Drugi mechanizm nazywa się {\em Traversal} i wykradziony został z ZOPE.
Opiera się on na tworzeniu drzewiastej struktury obiektów
reprezentujących zasoby. Przypomina to rozgałęzioną strukturę katalogów
na dysku albo zagnieżdżone pythonowe słowniki.

Poszukiwanie zasobu rozpoczyna się od rozbicia ścieżki na części
oddzielone znakiem \type{/}. Następnie tworzony jest zasób początkowy
(\type{root}) i podawany jest do niego pierwszy segment ścieżki (jak w
słowniku: \type{root[path[0]]}). Jeśli operacja się powiedzie (czyli
zasób zachowuje się jak słownik i zwraca wartość dla podanego klucza),
to otrzymaną wartość traktujemy tak samo jak \type{root} i idziemy
rekurencyjnie w głąb.

W momencie gdy skończą nam się części ścieżki albo gdy nie uda się
pobrać zasobu podrzędnego, algorytm przejścia dobiega końca, a jego
wynikiem jest ostatni (najbardziej zagnieżdżony) z uzyskanych zasobów.

W następnym kroku poszukiwane są widoki skojarzone ze znalezionym
zasobem. Oczywiście nic się nie marnuje i jeśli pozostał jeszcze jakiś
nieprzetrawersowany fragment ścieżki, to zostanie on przekazany do
funkcji widoku. Co więcej, istnieje jeszcze jedna opcja - możemy
stworzyć kilka widoków skojarzonych z tym samym zasobem, różniących się
tylko nazwą. W takiej sytuacji pierwszy nieskonsumowany fragment ścieżki
jest traktowany jako nazwa widoku (często jest to nazwa w stylu:
\type{show}, \type{edit}, \type{delete}).

Warto tutaj dodać, że choć~{\em traversal} zachowuje się jakby wchodził
w głąb zagnieżdżonych słowników, to same zasoby słownikami być nie muszą
(i zwykle nie są). Jak już zdążyli nas przyzwyczaić twórcy Pyramida,
także w tej kwestii mamy całkowitą swobodę - wystarczy, że stworzymy
klasę z metodą \type{__getitem__}, która na przykład pobierze dane z
bazy i już mamy nasz zasób.

Brzmi skomplikowanie i takie jest w istocie, ale korzyści z zastosowania
{\em traversala} mogą być bardzo znaczące. Przede wszystkim struktura
zasobów ma charakter drzewiasty i znacznie lepiej niż płaski
{\em routing} odzwierciedla strukturę adresu URL. Ponadto,
{\em traversal} nie jest podatny na trudne w wykryciu błędy wewnątrz
wyrażeń regularnych ani też nie jest wrażliwy na kolejność wpisów. Po
trzecie, bardzo łatwo dodać nową funkcjonalność do takiej struktury,
tworząc nową gałąź w odpowiednim miejscu istniejącego drzewa, co nigdy
nie wpływa to na pozostałe zasoby.

Początkującym (a także zaawansowanym) amatorom {\em traversala} może
pomóc malutka biblioteczka o nazwie \type{TraversalKit}, która dostarcza
klasę bazową dla zasobu oraz ułatwia tworzenie drzewa zasobów.

\subsection[brak-pierwiastka-magicznego]{Brak pierwiastka magicznego}

Pyramid stawia przejrzystość na pierwszym miejscu, dlatego próżno w nim
szukać magii składniowej w stylu \type{py.test} czy \type{SQLAlchemy}. Z
jednej strony może to zniechęcać początkującego użytkownika (nic się
samo nie wyczaruje), ale tę cechę docenią bardziej doświadczeni
programiści, gdyż framework nie blokuje twórcy swoimi z definicji
ograniczonym DSLami. Odsuwa to także groźbę \quotation{walki z
frameworkiem}, która często pojawia się w większych projektach.

Jedynym odstępstwem od zasady \quotation{niemagiczności} jest wspomniana
wcześniej deklaratywna konfiguracja.

\section[ciekawostki]{Ciekawostki}

Na koniec zebrałem garść ciekawostek dotyczących Pyramida,
przedstawionych w maksymalnie skondensowanej formie.

{\bf Początki} - pierwotnie Pylons Project pracował nad web frameworkiem
o nazwie Pylons, jednakże po wydaniu wersji 1.0 podjęto odważną decyzję
i zamiast tworzyć go dalej wzięto na warsztat bibliotekę
\type{repoze.bfg}, skupiając wysiłki na jej rozwijaniu. Na koniec
zmieniono nazwę na bardziej chwytliwą i tak oto powstał Pyramid. Jest to
chyba najdobitniejszy przykład pragmatycznego podejścia programistów
Pylons Project.

{\bf Pliki statyczne} - w przeciwieństwie do większości frameworków, w
Pyramidzie nie musimy się obawiać serwowania plików statycznych
bezpośrednio przez aplikację. Mamy do dyspozycji liczne narzędzia
pomagające nam w tym zadaniu, jest też wbudowany {\em cache buster}.

{\bf Waitress} - to stworzony przez Pylons Project prosty (ale w pełni
funkcjonalny) serwer WSGI, który może być z powodzeniem używany z
aplikacjami Pyramida. Jego podstawową cechą jest przenośność i łatwość
zastosowania zarówno w środowisku developerskim, jak i produkcyjnym.

{\bf Interfejsy} - budowa wewnętrzna Pyramida w znacznym stopniu opiera
się na interfejsach z biblioteki \type{zope.interface}. Może to
początkowo powodować pewną awersję, bo kod troszkę \quotation{śmierdzi}
Javą, ale twórcy wiedzieli co robią i nie przegięli struny - hierarchia
klas jest bardzo prosta i niezbyt głęboka, dlatego trudno się w niej
pogubić.

{\bf Sesje} - jak na Pyramida przystało, do dyspozycji mamy kilka
silników sesji. Dostępne od ręki są to tylko nieszyfrowane sesje
ciasteczkowe (podpisywane lub nie), ale w oficjalnych rozszerzeniach
znajdziemy szyfrowane sesje oparte o PyNaCl albo serwerowe używające
Redisa. Jeśli powyższe rozwiązania nam nie pasują, zawsze możemy
stworzyć własną klasę sesji implementując interfejs \type{ISession}.

{\bf Uwierzytelnianie} - Pyramid obsługuje liczne metody
uwierzytelniania (od {\em basic auth} przez rozwiązania z ciasteczkami
aż po cięższy oręż typu LDAP). Oczywiście nie ma przeszkód, żeby
stworzyć~własny autentykator, wystarczy zaimplementować interfejs
\type{IAuthenticationPolicy}.

{\bf Autoryzacja} - w pakiecie podstawowym mamy do dyspozycji tylko
jeden mechanizm autoryzacji oparty o listy ACL, który jest
wszechstronny, acz nie najłatwiejszy w użyciu. Na szczęście istnieje
wiele dodatków, które powiększają ilość opcji w tym zakresie, a poza tym
dość łatwo można stworzyć samemu autorską autoryzację (na pewno zgadłeś,
że wystarczy zaimplementować interfejs \type{IAuthorizationPolicy}).

{\bf Debug Toolbar} - to przepotężne narzędzie bardzo pomagające w
diagnozowaniu błędów w naszej aplikacji. Wystarczy je otworzyć w
zakładce przeglądarki, by otrzymać wszelkie informacje na temat żądań
HTTP, nagłówków, wydajności, logów, a nawet zapytań do bazy danych.

{\bf Tweens} - Pyramid posiada możliwość \quotation{wciśnięcia} kodu
pomiędzy zapytanie a naszą aplikację. Brzmi to trochę jak
{\em middleware} WSGI, z tą~różnicą, że {\em tween} uruchamiany jest w
kontekście aplikacji, gdzie ma dostęp do jej stanu wewnętrznego. W taki
właśnie sposób zaimplementowany jest m.in. Debug Toolbar.

{\bf Generowanie URLi} - Pyramid potrafi tworzyć URLe na podstawie tras
({\em routes}), dzięki czemu możemy uniknąć~topornego i bardzo podatnego
na błędy ręcznego \quotation{klejenia} adresów.

{\bf Szybkość} - pomimo złożonej funkcjonalności Pyramid okazuje się być
całkiem szybki, rozstawiając po kątach znaczną część liczącej się
konkurencji. Nie wykazuje się też nadmierną zasobożernością, dzięki
czemu utrzymanie aplikacji jest względnie tanie.

\section[podsumowanie]{Podsumowanie}

Pyramid, dzięki połączeniu wielu elementów, tworzy bardzo spójne i
wszechstronne środowisko do tworzenia aplikacji webowych. Po wykonaniu
pierwszych trudnych kroków staje się wygodny w użyciu i nad wyraz
intuicyjny. Bardzo dobrze nadaje się zarówno do małych projektów, jak i
do wielkich aplikacji, a jego szybkość i niezawodność powodują, że jest
tani w utrzymaniu.

Dlatego też mam nadzieję, że moim tekstem zachęciłem cię do przyjrzenia
się (może ponownego) temu niedocenionemu frameworkowi, jakim jest
Pyramid.

\section[źródła]{Źródła}

\startitemize[packed]
\item
  https://trypyramid.com - Pyramid strona domowa
\item
  https://pylonsproject.org - Pylons Project
\item
  https://docs.pylonsproject.org/projects/pyramid/en/latest/ -
  dokumentacja Pyramida
\item
  https://pylonsproject.org/projects.html - projekty rozwijane przez
  Pylons Project
\item
  https://trypyramid.com/resources-extending-pyramid.html - oficjalne
  dodatki ({\em add-ons})
\item
  https://webob.org - strona domowa \type{WebOb}, wrappera WSGI
  używanego przez Pyramid
\item
  https://docs.pylonsproject.org/projects/pyramid/en/latest/quick\type{_}tour.html
  - przewodnik po Pyramidzie
\stopitemize


\stoptext
