\usemodule[pycon-yyyy]
\starttext

\Title{with python: security()}
\Author{Piotr Dyba}
\MakeTitlePage

\section[about-author]{About author}

Piotr Dyba is a Team Leader and software engineer at F-Secure's Rapid
Detection Service where he and his Python team are developing a
distributed network of honeypots, network sensors and tooling for RDS.
In his spare time he is a leading mentor and teacher at PyLadies Poznan.
Pythonically in love with asyncio and personally with hiking. The last
interesting fact about Piotr is that he used to be a photojournalist and
visited Afghanistan as war photographer.

\subsection[introduction]{Introduction}

Introduction to cybersecurity for developers. Cybersecurity is a
critical part of all distributed applications. By understanding and
implementing proper security measures, guard your own resources against
malicious attackers as well as provide a secure environment for all
relevant parties.

\subsection[disclaimers]{Disclaimers:}

The views and opinions expressed in this article focus on the
developer's perspective an are solely those of the author. I have
received no payments for mentioning any of the open-source or commercial
products mentioned in this article.

\subsection[what-is-cybersecurity-and-why-is-it-so-important-nowadays]{What
is cybersecurity and why is it so important nowadays?}

Cybersecurity encompasses means and protocols to defend resources and
devices (computer systems, IoT devices, smartphones) from disruption,
damage or theft. It is not only about firewalls and antivirus software,
but rather it starts with the hardware, including policies for handling
computers and servers at a company, access rights to server rooms and
even which devices are allowed to be brought into the company building.
For instance, it is common to ban external pen-drives. To be able to
assume that our hardware is safe, we need to harden our software on both
workstations and servers and impose proper policies for the users, data
handling, access.

Looking at the number and severity of leaks over time {[}23{]}, it is
easy to observe a huge increase in their occurrences over the years and
severity rising from 1 000 to above 1 000 000 000 user's data.

\placefigure{NotPetya malware screen}{\externalfigure[000_notpetya.png]}

Remember the screen (rys. 1) from June 2017? NotPetya laid waste on many
companies across the globe in just one day, some of which are still
recovering. The most known cases were companies such as TNT and Raben.
Raben is a good example of well prepared recovery that took less than 24
hours. In that time, Raben moved their whole server infrastructure from
Windows to Linux, which was quite an achievement.

\placefigure{Attackers per day in the first 6 months of
2017}{\externalfigure[001_attacks_per_day.png]}

The graph (rys. 2) shows the number of attackers per day in first 6
months of 2017, which varied from 2 to almost 16 millions per day. An
attack means an action against one of F-secure's research honeypots
deployed on the public internet. A honeypot is a server that pretends to
be an easy target for an attacker, where all his or her actions can be
monitored and registered, but which cannot crucially affect the critical
parts of the broader infrastructure. It's just a trap that pretends to
expose services like ssh, mysql, smtp etc., in our case written in pure
Python.

\placetable[none]{}
\starttable[|l|l|l|l|]
\HL
\NC O.No.
\NC username
\NC password
\NC count
\NC\AR
\HL
\NC 1
\NC root
\NC root
\NC 1165236
\NC\AR
\NC 2
\NC admin
\NC admin
\NC 66522
\NC\AR
\NC 3
\NC user
\NC user
\NC 23994
\NC\AR
\NC 4
\NC test
\NC test
\NC 13302
\NC\AR
\NC 5
\NC pi
\NC raspberry
\NC 35938
\NC\AR
\NC 6
\NC support
\NC support
\NC 35331
\NC\AR
\NC 7
\NC ubnt
\NC ubnt
\NC 33170
\NC\AR
\NC 8
\NC oracle
\NC oracle
\NC 9294
\NC\AR
\NC 9
\NC guest
\NC guest
\NC 23524
\NC\AR
\NC 10
\NC git
\NC git
\NC 10829
\NC\AR
\HL
\stoptable

It was deducted that a lot of the above attacks were conducted by bots
trying to log in, using a few default username-password combinations. It
could be assumed that such attacks are sometimes successful, otherwise
the attackers would stop scanning in such a manner.

\subsection[common-attack-vectors-on-frontend-application]{Common attack
vectors on frontend application}

How can we hack our app? We can start from analysing it on our own, but
probably someone else has already thought about it.

Of course they had, and it was not someone but thousands of people.
There is a huge project called Open Web Application Security Project -
OWASP in short {[}1{]}. OWASP is not only gathering all common threats
but also offers examples of attacks, ways to measure their severity and
much more. It can be used both by technical and less technical
personnel, such as a Project Manager, as most of the vulnerabilities
also have a business level explanation.

OWASP offers a vast repository of cybersecurity knowledge, encompassing
not only on threats, but also on tools, books, events and other
interesting sites.

Every few years, OWASP publishes a list of most common attacks. The last
one is from 2013 and a fresh one is coming up this year, but if you
compare between the recent lists, the changes are minimal over time,
which leads to a sad conclusion that many people still have not yet
learned from others' mistakes.

OWASP TOP 10 from 2013:

\startitemize[n,packed][stopper=.,width=2.0em]
\item
  A1 Injection
\item
  A2 Broken Authentication and Session Management
\item
  A3 Cross-Site Scripting (XSS)
\item
  A4 Insecure Direct Object References
\item
  A5 Security Misconfiguration
\item
  A6 Sensitive Data Exposure
\item
  A7 Missing Function Level Access Control
\item
  A8 Cross-Site Request Forgery (CSRF)
\item
  A9 Using Components with Known Vulnerabilities
\item
  A10 Unvalidated Redirects and Forwards
\stopitemize

Try all of the OWASP TOP10 and more using one of the projects: BeeBox
project {[}24{]}, OWASP Broken Web Applications Project (OWASP BWAP)
{[}25{]}.

\subsection[common-attack-vectors-on-a-python-application]{Common attack
vectors on a Python application}

Attackers can try injecting code into our Python application, but this
can only happen if they have found a new loophole in Python itself, or
if we are using eval/exec or pickle, especially for user input.

Example of Python payload (not in one line for readability)

\starttyping
import subprocess
subprocess.Popen([
    '/bin/bash',
    '-c',
    'rm -rf --no-preserve-root /'
])
\stoptyping

SQL injection\ldots{} Again, we are quite safe here, unless we are using
our own SQL engine instead of mature ORMs such as SQLAlchemy or Django
ORM.

Example of SQL payload:

\starttyping
admin; DROP DATABASE users;
\stoptyping

The main threat to a Python application is located between chair and
keyboard\ldots{} the developer. Fortunately, we can also mitigate this
to a point, using security static code analysis which is described later
on. If you are interested in how to exploit pickle there is a link that
can explain this really well on an example of an already fixed bug in
the Twisted framework {[}26{]}.

But why people would even use eval or exec if it is so dangerous?

\starttyping
python2 -m timeit -s 'code = "a,b =  2,3; c = a * b"' 'exec(code)'
10000 loops, best of 3: 18.7 usec per loop
\stoptyping

\starttyping
python2 -m timeit -s 'code = compile("a,b =  2,3; c = a * b",
    "<string>", "exec")' 'exec(code)'
1000000 loops, best of 3: 0.664 usec per loop
\stoptyping

\starttyping
python3.5 -m timeit -s 'code = "a,b =  2,3; c = a * b"' 'exec(code)'
10000 loops, best of 3: 22.3 usec per loop
\stoptyping

\starttyping
python3.5 -m timeit -s 'code = compile("a,b =  2,3; c = a * b",
    "<string>", "exec")' 'exec(code)'
1000000 loops, best of 3: 0.544 usec per loop
\stoptyping

It can make a simple script much faster on Python 2.7 \lettertilde{}30
(18.7usec vs.~0.664usec) times and, what is interesting, over 40 times
faster on Python 3.5 (22.3usec vs.~0.544usec).

Eval may simplify code. A known example of string calculator shows that
really well.

String calculator:

\starttyping
def count(equation):
    a, sign, b = equation.split(' ')
    a, b = int(a), int(b)
    if sign == '+':
        return a + b
    elif sign == '-':
        return a - b
    elif sign == '*':
        return a * b
    elif sign == '/':
        return a / b
    else:
        return 'Unsupported sign'

print(count('2 + 3'))
\stoptyping

String calculator using only eval:

\starttyping
print(eval('2 + 3'))
\stoptyping

From 12 lines of code for the most basic example of a equation, it can
be reduced to just one line using eval, which will work even with more
complex equations.

\subsection[threat-modeling]{Threat modeling}

Threat modeling is a process for analyzing the security of an
application or a system. It is a structured approach that enables you to
identify, quantify, and address the security risks associated with the
target of the modeling.

\subsection[identification-of-assets]{1. Identification of assets}

Let's conduct an example threat modeling from Batman's perspective (rys.
3) and identify our assets, which are: the bat cave, our butler Alfred
and information in form of emails and texts.

\subsection[identification-and-quantification-of-threats]{2.
Identification and quantification of threats}

Next, let's distinguish threats, so: the police, our arch enemy joker
and the press. Now, let's quantify those threats. Alfred is
irreplaceable and has access to all our other assets, so he is our
highest risk and thus has the highest priority for defense. Our bat cave
is also precious, but it can be rebuilt. Lastly, information included in
emails and text messages could allow our actions and position to be
tracked, but we can handle the journalists and the police ourselves.

\subsection[addressing]{3. Addressing}

For our main asset, Alfred, we can obscure his location and his identity
which is not that easy in the modern world. The bat cave is a much
simpler task, as we can install security systems, traps, misleading
bases of operations etc. There is a ton of possibilities here. For
emails and texts, encryption and caution when writing something delicate
should be enough.

\placefigure{Batman Threat modeling}{\externalfigure[002_batman.png]}

\subsection[identification-of-assets---decompose-the-application]{1.
Identification of assets - decompose the application}

So, as in the batman example shows, the first step is identifying
assets, their purpose and use cases. The next step is to specify entry
points to each asset and then how it interacts with external parts of
the service or 3rd party services. The last step is defining the Access
Control Lists (ACL), for example which actions are allowed for an
anonymous user, a registered user and an admin to do. If development of
an application is already in progress, there is a high chance that part
of the work is already done by reusing data flow diagrams or application
UML. An even better source of how an application should work are
behavioural or integration tests.

\subsection[identification-and-quantification-of-threats-1]{2.
Identification and quantification of threats}

There are few frameworks that we can work with, such as STRIDE (which
stands for Spoofing identity, Tampering with data, Repudiation,
Information disclosure, Denial of service, Elevation of privilege) and
ASF (Application Security Frame), both of which should give us reliable
information regarding:

\startitemize[packed]
\item
  auditing and logging
  \startitemize[packed]
  \item
    auditing is used to answer the question \quotation{Who did what?}
    and possibly why.
  \item
    logging is more focused on what's happening.
  \stopitemize
\item
  authentication and authorization
  \startitemize[packed]
  \item
    authentication is the a process of ascertaining that somebody really
    is who he or she claims to be
  \item
    authorization is a process to determine who is allowed to do what
  \stopitemize
\item
  configuration management - how and where do we store configs
\item
  data validation and protection in storage and transit
  \startitemize[packed]
  \item
    where do we store the data
  \item
    where do we validate it before recording
    \startitemize[packed]
    \item
      backend side
    \item
      both backend and frontend side
    \item
      only frontend (highly not advised)
    \stopitemize
  \item
    protection in transit
    \startitemize[n,packed][stopper=.]
    \item
      basic approach: have all communication over TLS, so encrypted
    \item
      advanced approach: have the data additionally encrypted before
      sending it, which is becoming the corporate level security
      standard
    \item
      expert approach: send it over a dedicated VPN tunnel, government
      level security
    \stopitemize
  \stopitemize
\item
  exception management - how do we track exception occurrence and what
  are the procedures to handle them
\stopitemize

Now we just need to measure the severity of the threats we have. We can
try approaching that by ourselves, determining which assets are the most
important for us or, for example, use CVSS. CVSS, or Common
Vulnerability Scoring System, is based on factors such as:

\startitemize[packed]
\item
  Attack vector
\item
  Attack complexity (hard to measure)
\item
  Privileges
\item
  User interaction
\item
  Scope
\item
  Confidentiality
\item
  Integrity
\item
  Availability
\stopitemize

All of the above together can give a reference rating, but if one of the
parameters is hard to define, like in the case of attack complexity, try
both and either calculate an average or leave it as a range from two
edge CVSS outcomes. It is important to remember that this is just a
reference point, not an oracle for what should be done. As mentioned
before, having use case UMLs or, even better, abuse case UMLs may come
in handy here.

\subsection[addressing-1]{3. Addressing}

We can address issues in 4 ways, by:

\startitemize[n,packed][stopper=.]
\item
  Completely removing the threat
\item
  Reducing/mitigating the threat
\item
  Acknowledging it and doing nothing
\item
  Pretending there is no issue
\stopitemize

Obviously, the best options is to remove the threat, but sometimes it's
either impossible or the costs of removing it are too high. In such
cases we can try to mitigate it. Taking the risk by leaving it as is or
marking it as \quotation{address later}, where it does not affect our
business nor customers, maybe fine in some cases, for instance when an
attacker is able to traverse over a directory with long and random file
names of non important pictures of other users' cats. So, there is a
chance that someone will type some random gibberish and they will see
the picture of a user's cat but they still won't know whose cat it is
and we don't really care if that happens. But if the attacker can
traverse not only over cat pictures, but also over config files etc.,
and we still do nothing about it, then we are asking to be hacked.

\subsection[threat-model-of-a-simple-blog-app]{Threat Model of a simple
blog app}

\subsection[specification]{Specification}

A threat model of a simple app using AngularJS for frontend, Sanic for
backend and SQL database with a few simple endpoints such as:

\startitemize[packed]
\item
  / - static home for serving the JavaScript and HTML parts
\item
  /api/login - dedicated authentication endpoint
\item
  /api/blog - blog list view
\item
  /api/blog/ - detail post view
\item
  /api/user - user list view
\item
  /api/user/ - detail user view
\stopitemize

\subsection[use-cases-to-understand-how-the-application-is-used]{1.1
Use cases to understand how the application is used}

So let's get back to our Blog app. We have 3 basic use cases for our
app:

\startitemize[packed]
\item
  Everyone can enter the site and view the blog and blog posts
\item
  Registered users can add new posts
\item
  Admins can manage users and delete posts
\stopitemize

\subsection[interactions-with-external-entities]{1.2 Interactions
with external entities}

Let's now project these use cases on interactions with the database.

\placetable[none]{}
\starttable[|l|l|l|l|l|l|l|]
\HL
\NC DB Action
\NC Login
\NC Logout
\NC Read
\NC Write
\NC Delete
\NC Manage
\AR
\NC
\NC
\NC
\NC blog
\NC blog
\NC blog
\NC Users
\NC\AR
\HL
\NC DB Read
\NC Yes
\NC No
\NC Yes
\NC Yes
\NC No
\NC Yes
\NC\AR
\NC DB Write
\NC Yes
\NC Yes
\NC No
\NC Yes
\NC No
\NC Yes
\NC\AR
\NC DB Delete
\NC No
\NC No
\NC No
\NC No
\NC Yes
\NC Yes
\NC\AR
\HL
\stoptable

The important fact is that only the admin can manage users and delete
blog posts, which means it is quite easy to defend the application
against data lose.

\subsection[identifying-entry-points]{1.3 Identifying entry
points}

\placetable[none]{}
\starttable[|l|l|l|l|l|l|l|]
\HL
\NC Method
\NC /
\NC /api/
\NC /api/
\NC /api/
\NC /api/
\NC /api/
\NC\AR
\NC
\NC
\NC login
\NC blog
\NC blog/
\NC user
\NC user/
\NC\AR
\HL
\NC GET
\NC Yes
\NC No
\NC Yes
\NC Yes
\NC Yes
\NC Yes
\NC\AR
\NC POST
\NC No
\NC Yes
\NC Yes
\NC Yes
\NC Yes
\NC Yes
\NC\AR
\NC DEL
\NC No
\NC No
\NC No
\NC Yes
\NC No
\NC Yes
\NC\AR
\HL
\stoptable

\subsection[identifying-trust-levels---access-control-lists]{1.4
Identifying trust levels - Access Control Lists}

\placetable[none]{}
\starttable[|l|l|l|l|l|l|l|]
\HL
\NC user type
\NC Login
\NC Logout
\NC Read
\NC Write
\NC Delete
\NC Manage
\NC\AR
\NC
\NC
\NC
\NC blog
\NC blog
\NC blog
\NC Users
\NC\AR
\HL
\NC Anonymous
\NC Yes
\NC No
\NC Yes
\NC No
\NC No
\NC No
\NC\AR
\NC User
\NC No
\NC Yes
\NC Yes
\NC Yes
\NC No
\NC No
\NC\AR
\NC Admin
\NC No
\NC Yes
\NC Yes
\NC Yes
\NC Yes
\NC Yes
\NC\AR
\HL
\stoptable

Let's think about our database actions on API layer and focus on Get,
Post, Delete methods shown in those two tables.

\placetable[none]{}
\starttable[|l|l|l|l|l|l|l|]
\HL
\NC User type
\NC /
\NC /api/
\NC /api/
\NC /api/
\NC /api/
\NC /api/
\NC\AR
\NC
\NC
\NC login
\NC blog
\NC blog/
\NC user
\NC user/
\NC\AR
\HL
\NC Anonymous
\NC G
\NC P
\NC G
\NC G
\NC -
\NC G
\NC\AR
\NC User
\NC G
\NC -
\NC GP
\NC GP
\NC -
\NC G
\NC\AR
\NC Admin
\NC G
\NC -
\NC GP
\NC GPD
\NC GP
\NC GPD
\NC\AR
\HL
\stoptable

(G - Get, P - Post, D - Delete)

We know we can disable del and post methods for home, and for login we
can only accept the post method. Note that the other endpoint's
specification will depend on the project's approach to creating proper
endpoints, so it may differ from the example above.

\subsection[identifying-assets]{1.5 Identifying assets}

what is the most precious thing we have in our app?

In case of a blog, it is the information, that is, our users and blog
posts, but depending on our business model, it can also be
confidentiality. These can be affected by:

\startitemize[packed]
\item
  Someone getting admin access and deletes everything.
\item
  Someone getting user access and creates spam or spreads malware.
\stopitemize

Our second most valued asset may be our code base, which can be targeted
in different ways than the information itself by getting open sourced or
sold, following which even read only access will allow an attacker to
more easily find vulnerabilities and exploit them.

At this point we have completed all points of the Decomposition phase of
threat modeling.

\subsection[identification-and-quantification-of-threats-frontend]{2.1
Identification and quantification of threats: Frontend}

So, for our Front end we can expect six of OWASP top 10:

\startitemize[n,packed][stopper=.]
\item
  A2 Broken Authentication and Session Management
\item
  A3 Cross-Site Scripting (XSS)
\item
  A5 Security Misconfiguration
\item
  A7 Missing Function Level Access Control
\item
  A8 Cross-Site Request Forgery (CSRF)
\item
  A10 Unvalidated Redirects and Forwards
\stopitemize

We are using a well-known framework, which is really good, unless we or
our developers do something stupid, because AngularJS mitigates or even
handles all of those issues.

\subsection[identification-and-quantification-of-threats-python-backend]{2.2
Identification and quantification of threats: Python Backend}

OWASP TOP 10:

\startitemize[n,packed][stopper=.]
\item
  A1 Injection
\item
  A2 Broken Authentication and Session Management
\item
  A4 Insecure Direct Object References
\item
  A5 Security Misconfiguration
\item
  A6 Sensitive Data Exposure
\item
  A7 Missing Function Level Access Control
\item
  A8 Cross-Site Request Forgery (CSRF)
\stopitemize

From Threat Modeling of a Python app, we can see that we can have even
more vulnerabilities than on the frontend side, but most of them are
mitigated out of the box, as long as we follow three important rules
during development:

\startitemize[n,packed][stopper=.]
\item
  Use common sense;
\item
  Do not use uncommon external libraries without proper check up, for
  example if they are not sending the data to NSA/KGB/etc., but most
  importantly they are not sending the data to a unknown attacker and
  being sold later on;
\item
  Do not use outdated libraries as it may open code to vulnerabilities.
\stopitemize

\subsection[addressing-issues]{3.0 Addressing issues}

Someone gets admin access and deletes all data - we can mitigate or even
remove this threat by:

\startitemize[packed]
\item
  Adding multifactor authentication at least for Admins
\item
  Moving the admin part of an application to a separate microservice
  with access only from specific IPs/IP ranges
\stopitemize

Someone gets user access and creates spam - we can mitigate that by:

\startitemize[packed]
\item
  Limiting posts per day
\item
  Edits per hour
\item
  Shorter sessions
\item
  Captcha
\stopitemize

Part of mitigation is also having proper testing at unit level - having
not only happy paths but including edge cases helps.

\subsection[ok-so-we-got-hacked-but-how-bad-can-it-be]{OK so we
got hacked, but how bad can it be?}

The are four circles of being owned:

\startitemize[n,packed][stopper=.]
\item
  Being hacked by a bot or a script kiddie
\item
  Being hacked using one of the well-known and documented threats
\item
  Being hacked using a known vulnerability
\item
  Being hacked using an unknown vulnerability - 0 day
\stopitemize

The 1st and 2nd circle are really bad as it means we have almost no
security and we should really change priorities. It will most probably
be picked up by the internet media and we will be lampooned for how easy
and fun it was hacking our app. Really shame on you! The 3rd level means
that protocols regarding emerging threats should be better and faster.
Last but not least, getting owned by a new, shiny vulnerability is
obviously bad for business but at least not that shameful. At the same
time, it shows how serious the attackers needs to be and how good were
the defences, as it drew the attacker to the last resort he/she had. For
the fourth circle, the attacker will try to keep their actions unnoticed
as long as possible, thus there should not be a fallback in the press
about it.

\quotation{The smart thing is to learn from others' mistakes.}

Not seeing the destructive outcome of an attack does not necessarily
mean that you were not hacked. There is a chance that you have been
owned long time ago and you still do not know about it! Apart from
invisible actions such as gathering data or some increased trolling, an
attacker may also install some nasty ransomware and demand a high
ransom.

Gartner report from 2016 says that it takes on average 200 days since a
hack for a company to find about it (citation needed). Imagine what an
attacker can do during that time!

\section[tooling]{Tooling}

Source Code Analysis (Reading through the abstract syntax tree and
looking for possible security bugs) tools:

\startitemize[packed]
\item
  Bandit - a tool designed to find common security issues in Python
  code, it is written in Python and can be easily extended with custom
  security policies. It works in a similar manner to pylint or pep8
  tool. You can get it directly from PyPi.
\item
  SonarQube - more advanced than bandit, comes with a plugin for Jenkins
  and support for JS, HTML and 20 other languages. The tool has an
  integrated web UI and many more useful features. You can spawn your
  own instance or buy it as a service. Jenkins read -
  \quotation{Continuous Inspection}.
\stopitemize

Automatic Scanning tools:

\startitemize[packed]
\item
  ZAP (Zed Attack Proxy) - Free, Open Source, Jenkins ready
\item
  Burp - Free/Paid, Jenkins ready in \lettertilde{}2017
\item
  Metasploit - Free, OpenSource, Jenkins ready
\item
  SQLMap - Free, OpenSource
\item
  scapy - Free, OpenSource, Python
\stopitemize

Zap is an open source alternative to burp, developed under the OWASP
project and it already has a dedicated Jenkins plugin. Burp can scan for
vulnerabilities, intercept browser traffic and automate custom attacks,
it does not have a Jenkins plugin yet, but it was announced that this
year something should be ready for continuous development. Both ZAP and
Burp can also be used manually, so you can define your own attack
patterns and payloads. Metasploit is a framework that focuses on system
level scans but can be extended with custom ones. SQLMap scans
application only for SQL injections. If want you to test a custom
protocol, you will probably need to use scapy, which is a Python library
for preparing dedicated TCP, ICMP packages and UDP datagrams.

Commercial solutions and managed services: The advantage of the service
approach is quite nice, as the report from a cybersecurity consultant
does not include false-positives from the scans. There are three major
players in this filed:

\startitemize[packed]
\item
  Qualys
\item
  Nessus
\item
  F-Secure Radar
\stopitemize

\subsection[useful-terminology]{Useful terminology}

\subsection[penetration-testing]{Penetration Testing}

A penetration test is an authorized attack on a system, application
and/or infrastructure. The reasoning behind undergoing pentesting
(Penetration Testing) is quite obviously to find security weaknesses or
to fulfil a compliance needed by a 3rd party. There are usually two main
objectives for a pentester to achieve: get privileged access and/or
obtain restricted information. Even when having an on site pentesting
team, it is still recommended to use a 3rd party company as well.
Penetration tests should be considered before major releases or
periodically after a longer development cycle. In the best case scenario
automated security tests using for example Zap are also in place.

A penetration tester is a person who performs a penetration test. They
can also be called by: Pentester, Hacker, White Hat or Security
consultant, just remember not to call them Black Hat or Cracker which
basically means a criminal and will make them really sad and you do not
want that to happen.

There are three major approaches while pentesting: white, grey and black
box.

White box means full transparency and full access for a pentester to our
production systems, especially:

\startitemize[packed]
\item
  A pentester has full access to the application and systems they are
  testing
\item
  The test covers the widest range of attack vectors
\item
  A pentester has access to our source code and documentation
\stopitemize

Grey box narrows the access, but still requires source code and
sometimes a developer instance before starting the main pentest, the
pentesting becomes targeted and works within boundaries:

\startitemize[packed]
\item
  A pentester does not have access to the system
\item
  A pentester has different types of accounts such as user and/or
  moderator, but not admin-level application access
\item
  A pentester has access to our source code and documentation
\stopitemize

Black box, as the name suggests, limits access, so the pentester's
perspective is similar to a real attacker:

\startitemize[packed]
\item
  A pentester does not have knowledge about the system or only knows its
  basics
\item
  A pentester has no access to source code, but obtaining the source
  code may be one of the steps when executing an attack or be one if its
  goals
\item
  A pentester has no initial access to the app above end-user level
\stopitemize

\subsection[ciso]{CISO}

Chief Information Security Officer is a person responsible for:

\startitemize[packed]
\item
  Computer security/incident response team
\item
  Disaster recovery and business continuity management
\item
  Identity and access management
\item
  Information privacy
\item
  Information regulatory compliance PCI, Data Protection Act, GIODO in
  Poland
\item
  Information risk management
\item
  Security architecture and development, so process and tools
\item
  IT Security
\item
  Security awareness in the company
\item
  Managing pentests and red teamings
\stopitemize

\subsection[red-teaming]{Red Teaming}

A red teaming drill is usually a multi-level attack on a company that
only CEO, CISO or CTO are aware of. A Red Teaming Exercise may consist
of:

\startitemize[packed]
\item
  Physical security testing like breaking into the office, server rooms,
  conference rooms, planting bugs or lock picking
\item
  Network and IT security testing done by, for example, planting
  Raspberry Pi on the net, routing data through a sniffer or leaving
  behind some rubberducky
\item
  Phishing and social engineering
\item
  Software pentesting
\stopitemize

If the red timing finishes undetected, it means the company that
undergoes the test has huge security problems as the attacker during the
drill, after achieving all goals, starts being \quotation{noisy} till a
point someone should notice\ldots{}

\subsection[references]{References}

\startitemize[n,packed][stopper=.,width=2.0em]
\item
  http://owasp.org
\item
  https://vulnhub.com
\item
  https://github.com/sbilly/awesome-security
\item
  https://github.com/nixawk/pentest-wiki
\item
  https://safeandsavvy.f-secure.com/
\item
  https://reddit.com/r/netsec/
\item
  https://blogs.cisco.com/
\item
  https://youtube.com/channel/UClcE-kVhqyiHCcjYwcpfj9w
\item
  http://gynvael.coldwind.pl/
\item
  https://nakedsecurity.sophos.com/
\item
  https://risky.biz/netcasts/risky-business/
\item
  https://badcyber.com/
\item
  https://packetstormsecurity.com
\item
  https://labs.mwrinfosecurity.com/
\item
  https://ctftime.org/ctf-wtf/
\item
  http://overthewire.org/wargames/
\item
  https://picoctf.com/
\item
  https://microcorruption.com/
\item
  https://www.offensive-security.com/when-things-get-tough/
\item
  Batman threat modeling.
  https://mobilisationlab.org/wp-content/uploads/\crlf
  2015/08/batman-threat-model-1200.png
\item
  sshttp - hiding SSH servers behind HTTP.
  https://github.com/stealth/sshttp
\item
  Justin Mayer. Replacing passwords with multiple factors: email, otp,
  and hardware keys. EuroPython 2017
\item
  Data leaks in time.
  http://www.informationisbeautiful.net/visualizations/worlds-biggest-data-breaches-hacks/
\item
  BeeBox project. https://sourceforge.net/projects/bwapp/files/bee-box/
\item
  OWASP Broken Web Applications Project (OWASP BWAP).\crlf
  https://www.owasp.org/index.php/OWASP\type{_}Broken\type{_}Web\type{_}Applications\type{_}Project
\item
  Exploiting pickle. https://blog.nelhage.com/2011/03/exploiting-pickle/
\stopitemize


\stoptext
