\usemodule[pycon-yyyy]
\starttext

\Title{Inteligentny dom - technologie komunikacji między urządzeniami}
\Author{Krzysztof Czarnota}
\MakeTitlePage

System zarządzania inteligentnym budynkiem wymaga sprawowania kontroli
nad urządzeniami zainstalowanymi w obiekcie. Sprawna i odporna na błędy
komunikacja jest niezbędna w celu realizacji tego zadania. Niestety,
różnorodna charakterystyka wykorzystywanych urządzeń utrudnia
zastosowanie jednego standardu komunikacji.

Poniżej przedstawiony został przegląd wybranych interfejsów i protokołów
komunikacyjnych spotykanych w urządzeniach automatyki budynkowej wraz z
krótkim opisem podstawowych cech i zastosowań danego rozwiązania.

\subsection[uart]{UART}

UART (Universal Asynchronous Receiver and Transmitter) to interfejs
przeznaczony do nawiązania komunikacji z urządzeniem poprzez port
szeregowy. Komunikacja odbywa się przez dwie jednokierunkowe linie TX
oraz RX. Dane wysyłane są asynchronicznie w postaci ramek. Ramka danych
definiuje długość danych, obecność bitu parzystości oraz ilość bitów
stopu. Format ten zapewnia podstawową kontrolę błędów. Nawiązanie
komunikacji wymaga używania tego samego formatu ramki i prędkości
transmisji przez oba urządzenia. Interfejs typowo obsługuje prędkości
transmisji od 110 do 115200 bodów. Maksymalna długość przewodu łączącego
urządzenia wynosi nawet 100 m i spada wraz ze wzrostem prędkości
transmisji {[}1, 2{]}.

Jest to jeden z najbardziej podstawowych i najczęściej stosowanych
interfejsów ze względu na jego łatwość użycia oraz uniwersalność. W
przypadku urządzeń automatyki budynkowej interfejs ten będzie często
spotykany w bardziej zaawansowanych sterownikach urządzeń (sterowniki
ogrzewania, sterowniki instalacji solarnej, centrale alarmowe).

\subsection[spi]{SPI}

SPI (Serial Peripheral Interface) to bardzo popularny interfejs
komunikacji synchronicznej pomiędzy systemami mikroprocesorowymi a
układami peryferyjnymi w konfiguracji master-slave. Komunikacja odbywa
się z wykorzystaniem trzech linii: MOSI (Master Output Slave Input),
MISO (Master Input Slave Output) oraz SCLK (Serial CLocK). Podłączenie
jednocześnie wielu urządzeń jest możliwe dzięki dodatkowej linii SS
(Slave Select). Magistrala ta umożliwia połączenie urządzeń na
niewielkie odległości. Typowo są to urządzenia mieszczące się na jednej
płytce drukowanej. Sam interfejs SPI definiuje tylko sygnały niezbędne
do jego realizacji, nic nie mówiąc o formacie wymienianych danych, który
zależy od urządzenia peryferyjnego. Prędkość transmisji zależna jest od
częstotliwości sygnału SCLK, która może wynosić nawet kilka MHz {[}1,
3{]}.

Komunikacja za pomocą interfejsu SPI najczęściej spotykana jest w
urządzeniach peryferyjnych stosowanych w systemach wbudowanych, takich
jak: karty SD, pamięci EEPROM, przetworniki ADC/DAC, czy też
wyświetlacze ciekłokrystaliczne.

\subsection[i2c]{I2C}

I2C (Inter-Integrated Circuit) to dwukierunkowa magistrala służąca do
synchronicznego przesyłania danych w urządzeniach elektronicznych.
Komunikacja odbywa się z wykorzystaniem dwóch dwukierunkowych linii SDA
-- (Serial Data Line) oraz SCL (Serial Clock Line). Ze względu na swoją
budowę interfejs ten nadaje się do komunikacji na małe odległości (do
kilkunastu cm). Każde urządzenie I2C ma swój unikalny 7-bitowy adres.
Pierwsze 4 bity określają identyfikator urządzenia. Jest on nadawany
przez producenta układu i umożliwia zorientowanie się co do typu
urządzenia (pamięć, przetwornik I2C, zegar). Pozostałe 3 bity to
fizyczny adres urządzenia. Najmłodszy bit adresu służy do wyboru typu
następnej operacji (odczyt lub zapis). Prędkość komunikacji może odbywać
się w dwóch trybach: standardowym (100 kHz) i szybkim (400 kHz) {[}1,
3{]}.

Interfejs I2C znajduje zastosowanie w układach peryferyjnych, gdy
prostota i niski koszt są ważniejsze od wysokich prędkości transmisji.
Magistralę I2C znajdziemy w układach takich jak: termometry, czujniki
ciśnienia, mierniki przyśpieszenia czy też zegary czasu rzeczywistego.

\subsection[wire]{1-Wire}

1-Wire to interfejs stworzony do łączenia urządzeń na duże odległości
(nawet do kilkuset metrów), przy czym transmisja odbywa się stosunkowo
wolno. Interfejs do komunikacji wykorzystuje tylko jedną dwukierunkową
linię danych. Urządzenia slave podłączone do tego interfejsu mogą
posiadać własne zasilanie lub mogą być zasilane bezpośrednio z linii
danych, wykorzystując zasilanie pasożytnicze. Interfejs zakłada
istnienie na magistrali tylko jednego urządzenia master i dowolnej
liczby urządzeń slave. Urządzenia slave identyfikowane są przy pomocy
unikalnego 8-bajtowego identyfikatora, nadawanego urządzeniu w czasie
produkcji. Urządzenia slave same nie wykazują żadnej aktywności, a
wszelkie transfery na magistrali inicjowane są przez urządzenie master.
Typowa prędkość komunikacji to 16 kbps {[}1, 3{]}.

Interfejs 1-Wire jest typowo wykorzystywany do komunikacji pomiędzy
niewielkimi i niedrogimi urządzeniami, takimi jak: termometry cyfrowe,
czujniki meteorologiczne czy zamki elektroniczne (iButton).

\subsection[modbus]{Modbus}

Modbus jest prostym i niezawodnym protokołem zapewniającym komunikację
między wieloma urządzeniami w architekturze master-slave. Jest to
obecnie standard otwarty, który znalazł szerokie zastosowanie systemach
automatyki, zarówno przemysłowej jak i domowej. Komunikacja odbywa się
przeważnie przez interfejs szeregowy (RS232, RS485, UART), ale istnieje
również wariant protokołu nazwany Modbus TCP, który zapewnia komunikację
przez sieć TCP/IP. Komunikacja może odbywać się w dwóch trybach:
znakowym (ASCII) oraz binarnym (RTU). Urządzenie zdalne widziane jest
jako 16-bitowe rejestry. Protokół definiuje funkcje odpowiedzialne za
odczyt i zapis danych na urządzeniu. Zapewnia także diagnostykę,
potwierdzenie wykonania rozkazów oraz sygnalizację błędów {[}4{]}.

Protokół Modbus jest standardem komunikacyjnym wspieranym przez cały
szereg producentów sterowników i innych urządzeń wykorzystywanych nie
tylko w automatyce. Komunikację za pomocą tego protokołu możemy nawiązać
z urządzeniami takimi jak: sterowniki pieca, pompy ciepła, centrale
wentylacyjne czy sterowniki solarne.

\subsection[knx]{KNX}

KNX (Standard KONNEX) to standard komunikacji dla automatyki budynkowej,
który umożliwia wspólną komunikację pomiędzy wszystkimi odbiornikami
energii elektrycznej w budynku. KNX jest systemem rozproszonym. Każdy
element podłączony do instalacji wyposażony jest w procesor i elementy
niezbędne do jego samodzielnej pracy {[}5{]}.

System KNX opcjonalnie zapewnia zdalny dostęp do wszystkich instalacji w
budynku. Bardzo często stosowany jest w hotelach do sterowania
oświetleniem, ogrzewaniem, wentylacją i innymi urządzeniami znajdującymi
się w budynku.

\subsection[lontalk]{LonTalk}

LonTalk (Local Operating Network) to protokół wykorzystywany do
komunikacji węzłów w rozproszonym systemie automatyki LonWorks. System
LonWorks składa się niezależnych urządzeń zwanych węzłami, które
posiadają zdolność komunikowania się ze sobą po wspólnym medium.
Protokół LonTalk jest standardem otwartym i można zaimplementować go w
dowolnym urządzeniu. Pojedyncze węzły posiadają własną inteligencję i
mogą przetwarzać różne programy niemal niezależnie od siebie, jednak
wszystkie udostępniają informacje urządzeniom z innych obszarów. Wymiana
danych jest sterowana zdarzeniowo {[}6{]}.

Ze względu na swoją złożoność LonWorks nadaje się do realizacji wielu
zadań z zakresu automatyki budynkowej, takich jak na przykład kontrola
dostępu, układy HVAC, sygnalizacja przeciwpożarowa, sterowanie
oświetleniem i pracą wind.

\subsection[m-bus]{M-Bus}

M-Bus (Meter-Bus) jest europejskim standardem opracowanym do przesyłania
wskazań z przyrządów pomiarowych. Standard zapewnia obsługę kilkuset
urządzeń podłączonych do magistrali o długości wynoszącej nawet kilka
kilometrów. Transmisja danych jest odporna na błędy, ale stosunkowo
wolna. Prędkość przesyłania danych waha się w granicach od 300 do 9600
bodów. Protokół zakłada niezbyt częste odczytywanie mierzonych wartości
{[}7{]}.

Zgodnie ze swoim przeznaczeniem protokół M-Bus stosowany jest w różnych
miernikach znajdujących się w instalacjach domowych, takich jak:
liczniki energii elektrycznej, gazomierze i ciepłomierze.

\subsection[z-wave]{Z-Wave}

Z-Wave to bezprzewodowy protokół umożliwiający stworzenie zdalnie
sterowanej sieci urządzeń w topologii mesh. Twórcą oraz właścicielem
technologii jest duńska firma Zensys. Wykorzystanie protokołu wymaga
przynależności do stowarzyszenia Z-Wave Alliance. Technologia
wykorzystuje różne częstotliwości w zależności od regionu świata i
pracuje w paśmie od 865 do 956 Mhz. Każde urządzenie (z wyjątkiem
urządzeń zasilanych bateryjnie) podłączone do sieci jest również
przekaźnikiem sygnału dla innych urządzeń, co znacznie zwiększa zasięg
sieci. Przed nawiązaniem komunikacji nowe urządzenie musi zostać
dołączone do sieci przez proces parowania z głównym kontrolerem. Każda
sieć Z-Wave posiada swój 4-bajtowy identyfikator, który zapisywany jest
w urządzeniach podczas parowania. Pojedyncze urządzenia identyfikowane
są poprzez 1-bajtowy identyfikator, który musi być unikalny w danej
sieci. Protokół został zaprojektowany tak, aby zapewniać niezawodną
transmisję małych pakietów danych z niskimi opóźnieniami. Przepustowość
wynosi 40 kb/s w układach serii 200 lub 200 kb/s w układach serii 400.

Protokół Z-Wave jest stosowany w szerokiej gamie urządzeń przeznaczonych
dla inteligentnych budynków takich jak: włączniki ścienne, żarówki,
termostaty, rolety, siłowniki okien czy sterowniki centralnego
ogrzewania.

\subsection[zigbee]{ZigBee}

ZigBee to protokół bezprzewodowej transmisji danych w sieciach typu mesh
działających w paśmie 868 MHz, 915 MHz lub 2,4 GHz. Charakteryzuje się
niewielkim poborem energii oraz niską prędkością transmisji. Wyróżniamy
trzy typy urządzeń ZigBee: * koordynator (ZC) - jest to węzeł początkowy
sieci, do którego przyłączają się inne urządzenia, w sieci może
występować tylko jedno takie urządzenie, * router (ZR) - przekazuje dane
z innych urządzeń lub routerów, * urządzenie końcowe (ZED) - przesyła
dane do routera, do którego jest przyłączone {[}9{]}.

Standard ZigBee wykorzystywany jest w urządzeniach, które wysyłają mało
danych oraz wymagają długiego czasu pracy na zasilaniu bateryjnym.
Typowe zastosowanie znajduje on w różnego rodzaju czujnikach (dymu,
zalania, otwarcia okien) i inteligentnych przełącznikach (świtała,
rolet).

\subsection[wifi]{Wifi}

Wifi to standard przeznaczony do budowy bezprzewodowych sieci
komputerowych. Urządzenia pracują w paśmie częstotliwości 2,4 GHz lub 5
GHz. Powszechność urządzeń wykorzystujących standard wifi spowodowała,
że technologia ta została wykorzystana także w urządzeniach automatyki
budynkowej.

Ze względu na spore zapotrzebowanie na energię wsparcie dla komunikacji
wifi oferowane jest przeważnie w urządzeniach na stałe podłączonych do
zasilania, takich jak inteligentne żarówki czy sterowniki centralnego
ogrzewania. Technologia ta jest także często wykorzystywana w projektach
hobbystycznych za sprawą taniego i łatwo dostępnego modułu wifi ESP8266.

\subsection[podsumowanie]{Podsumowanie}

Przedstawione powyżej technologie stanowią tylko część rozwiązań
stosowanych w urządzeniach automatyki domowej. Rynek inteligentnych
domów oraz internetu rzeczy rozwija się bardzo dynamicznie, co powoduje
powstawanie coraz to nowych standardów. Producenci systemów oraz
urządzeń preferują własne rozwiązania, starając się~zagarnąć jak
największą część rynku dla siebie. Powyższe czynniki oraz różnorodność
wymagań nie sprzyjają pojawieniu się spójnego standardu komunikacji
urządzeń automatyki domowej. Niemniej opisane rozwiązania posiadają już
bardzo silną pozycję na rynku, a ich znajomość na pewno ułatwi
odnalezienie się w gąszczu wykorzystywanych interfejsów i protokołów
komunikacyjnych.

\subsection[bibliografia]{Bibliografia}

\startitemize[n,packed][stopper=.]
\item
  Tomasz Francuz - Język C dla mikrokontrolerów AVR. Od podstaw do
  zaawansowanych aplikacji:
  http://helion.pl/ksiazki/jezyk-c-dla-mikrokontrolerow-avr-od-podstaw-do-zaawansowanych-aplikacji-tomasz-francuz,jcmikr.htm
\item
  Interfejs UART:
  https://en.wikipedia.org/wiki/Universal\type{_}asynchronous\type{_}receiver/transmitter
\item
  Magistrale szeregowe: http://www.epanorama.net/links/serialbus.html
\item
  Protokół Modbus: http://www.modbus.org
\item
  System KNX: https://www.knx.org/
\item
  System LonWorks: https://en.wikipedia.org/wiki/LonWorks
\item
  Magistrala M-Bus: http://www.m-bus.com
\item
  Protokół Z-wave: http://www.z-wave.com
\item
  Protokół ZigBee: http://www.zigbee.org
\stopitemize


\stoptext
