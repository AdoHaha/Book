\usemodule[pycon-yyyy]
\starttext


\section[django-in-the-real-world---israel-fermín-montilla]{Django in
the Real World - Israel Fermín Montilla}

\subsection[introduction]{Introduction}

There are hundreds of django-based projects running out there, in most
of the cases, the default setup is enough. Some others need to scale so
they can serve thousands of requests per minute. Even though there's no
recipe for optimization, there are some strategies you can implement to
take your project to that level, some of them are very simple, some
others are very complex.

Optimizing a system, isn't always related to the technology or the tools
you're using. Of course, there are efficient and inefficient ways of
using certain language or framework, but it's usually more about having
some Software Engineering concepts clear and knowing how to apply them
to your current technology of choice after measuring your performance
metrics.

\subsection[basic-concepts]{Basic Concepts}

\startitemize
\item
  {\bf Performance:} is the amount of work accomplished by a computer
  system. There are several metrics to measure how a system performs,
  depending on the case, the most common metrics are:
  \startitemize[packed]
  \item
    {\bf Response time:} total amount of time it takes to a system to
    respond to a request for service;
  \item
    {\bf Throughput:} is the maximum rate at which {\em something} can
    be processed. For web systems it's usually measured in {\em requests
    per minute};
  \item
    {\bf High availability:} it's a characteristic of systems which aims
    to ensure a certain level of operational performance, usually
    {\em uptime}, for a higher period of time;
  \item
    Low utilization of available resources.
  \stopitemize
\item
  {\bf Scalability:} the capability of a system to process or handle a
  {\bf growing} workload or its potential to be {\bf enlarged} to
  accommodate that growth is known as Scalability. We say a system is
  {\bf scalable} if its performance improves after adding more hardware
  proportionally to the added capacity;
\item
  {\bf Bottleneck:} a {\em bottleneck} occurs when the capacity of an
  application is severely limited by a single component. The
  {\em bottleneck} has the lowest {\em throughput} of all the parts of
  the transaction;
\item
  {\bf Pareto principle:} the {\em pareto principle} states that, for
  many events, roughly 80\letterpercent{} of the effects come from
  20\letterpercent{} of the causes, so, by fixing that
  20\letterpercent{}, we could achieve an 80\letterpercent{}
  improvement.
\stopitemize

\subsection[basic-django-deployment]{Basic django deployment}

The usual django stack runs django with either {\em uwsgi} or
{\em gunicorn} behind a web server that could be either Apache or nginx
and uses {\em postgres} for database persistence, the architecture would
be something like this:

\placefigure{}{\externalfigure[/home/israel/Projects/pycon/Book/2017/presentations/django_in_the_real_world/diagram.jpg]}

This would be how you'd deploy your personal project to your VPS for the
first time.

\subsection[profile-first]{Profile first}

Before optimizing anything, you'll have to measure what's going on and
where the true bottlenecks are, you can use tools like
\type{django-debug-toolbar}to have an overall idea about the performance
and even the raw \type{SQL} queries being ran.

To measure code performance in terms of CPU and memory usage, you can
use a combination of \type{cProfile} and \type{snakeviz} or \type{vprof}
which offers visualization tools out of the box. If you have some budget
to invest, a tool like \type{newrelic} is extremely useful.

\subsection[common-bottlenecks]{Common bottlenecks}

However, you can't keep that architecture forever, as traffic grows,
some parts of the system will suffer and you'll have to use some
strategies to relief that pain.

\section[database-level]{Database level}

Usually the first part of your system showing some symptoms will be your
database, this will usually be your first bottleneck and the first thing
you'll have to optimize.

\subsection[increased-response-times]{Increased response times}

This is the main symptom, and it could be the product of either slow
reads or slow writes, which leads to slow reads due to resource locking.

\startitemize
\item
  {\bf Slow writes:} usually product of over-indexing a table, all
  indexes are updated on write time on \type{INSERT}, \type{UPDATE} and
  \type{DELETE}. Too many indexes on a table will produce slow writes,
  no indexes at all might end up on slow reads, you need to know your
  data model and the questions it needs to answer and index based on
  workload. Premature optimization is bad.
\item
  {\bf Slow reads:} could be the product of a sub-optimal data model for
  the type of queries it needs to answer, solutions will depend on the
  traffic on the tables and the relations between them.
  \startitemize[packed]
  \item
    {\bf Add Indices:} a good hint about which fields to index is to
    check the ones that appear the most on the \type{WHERE} clauses of
    the executed SQL queries. It's usually a good idea to index foreign
    keys.
  \item
    {\bf Denormalize:} usually it's a bad idea to add \type{ManyToMany}
    relations between model we know will grow indefinitely and quick,
    consider denormlizing them into \type{JSONField} or
    \type{ArrayField} if using Postgres or just duplicate the data, this
    will drastically improve the performance on those queries.
  \item
    {\bf Database caching:} recommended for tables with a low
    modification rate, here you can make use of a tool like
    \type{django-cachalot} which will handle everything by you with few
    lines of code and supports several caching backends with some
    limitations with documented solutions depending on the chosen cache
    backend.
  \item
    {\bf Replication:} this is usually the first step you take, separate
    the write operations from the read operations, this way both
    operations are handled by different instances. You might have some
    replication delay depending on traffic and write contention. Using
    django's database routers is highly recommended here to abstract the
    upper layers from handling the databases for write and read.
  \item
    {\bf Know your ORM:} one of the most powerful libraries withing
    django framework is the \type{ORM}, \type{django.db.models.Model}
    class is really powerful and efficient but so underused in most
    cases that is surprising, first of all they have lazy evaluation,
    which means it only hits the database when it needs to and gives you
    only the data you need to retrieve, so, it is highly convenient to
    have a look at the aggregation framework and the \type{Q()} and
    \type{F()} expressions if you need to build complex querysets as
    well as the aggregation framework if you need to perform some
    grouping and calculations over the data, the I/O overhead of loading
    a whole set of rows into memory is considerably higher than getting
    your DBMS to give you the result of a computation which would be
    only an \type{integer} or a \type{float} number.
  \stopitemize
\stopitemize

\section[caching]{Caching}

Caching and database caching are two different things, when we talk
about database caching we keep a cache of the results of the most common
queries or the ones where the data doesn't change a lot, whereas when we
talk about caching, what we want to achieve is to speed up the system's
response time by keeping a copy of something in a place where we can
retrieve it quickly.

\startitemize[packed]
\item
  {\bf Static files:} stylesheets, javascript files, images, icons,
  everything your site needs to look nice and have a great UX can be
  cached using memcached, for example, or using a Content Delivery
  Network (CDN) if you know it will be accessed from different and
  distant places. This way you can offload some work from your web
  server and avoid hitting the disk and having to load all those assets
  on every request.
\item
  {\bf Templates:} one of django's coolest features is its templates
  system, which allows you to cache either a whole view or certain
  blocks of a template that are highly demanded, it only takes you to
  add one templatetag and you're good to go, you can cache static blocks
  that won't change or blocks that are associated to a dynamic piece of
  information like a session or user id.
\item
  {\bf Session storage:} django offers several session storage backends,
  File based and database backed, but sessions are something you're
  going to be accessing a lot, so, it is recommended to have a fast
  storage backend, redis can be really handy for this and it's easy to
  setup, you only need to make sure that your whole dataset fits into
  memory which won't be a problem at all if you use it only for
  sessions.
\stopitemize

\subsection[conclusion]{Conclusion}

Django is a full stack framework, which means it provides an entire
development platform to build your whole system, a lot of out of the box
behavior and middlewares. You need to develop and plan smart to make it
scale easily by measuring, prioritizing and taking the right action.
Optimization and performance tuning are hard to do, but if you back your
hypothesis with data from your monitoring system, you can achieve
favorable results.


\stoptext
