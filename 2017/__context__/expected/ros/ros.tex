\usemodule[pycon-yyyy]
\starttext

\Title{Rapidly Prototyping Robotics with ROS and IPython (workshop)}
\Author{Lidia Lipińska, Marcin Kolesiński, Igor Zubrycki}
\MakeTitlePage

TLDR:

We will guide you through basics of robotics and ROS. You will control
robots using Python through Jupyter Notebooks.

Learn Python in a different way and move real things.

Or use this knowledge in a crazy startup project to prototype cool
robotic application.

\subsection[workshop-agenda-3h]{Workshop agenda (3h):}

\startitemize[packed]
\item
  Introduction to robotics and ROS ecosystem
\item
  Setting up and running your first ROS Node
\item
  ROS concepts explained: topic \type{&} service
\item
  Using IPython Interactive Widgets to control a robot
\item
  Visualizing robot senses using ROS and Bokeh library
\item
  Tour of ROS tools with examples: Rviz, Rosbag, TF (transforms)
\item
  Wrapup: how to continue learning ROS, Q\type{&}A session
\stopitemize

\subsection[is-this-for-me]{Is this for me?}

This workshop is intended for people familiar with basics of Python
(loops, conditionals). You don't have to know anything about robotics we
will introduce you from fundamentals and give an opportunity to control
robots from IPython environment. Knowledge of GNU Linux basics (we will
be using Ubuntu) is welcome, however not necessary.

Whether you are warming up with Python and want to try something more,
or you are a pro and plan to do some robotic project but never did
anything with ROS - we invite you to our workshop!

\subsection[introduction-and-invitation]{Introduction and invitation}

Robotics is growing and interesting field for both work and fun.

Robots (moving, intelligent devices) have hundreds of interacting
components that enable them to do what they are supposed to do:

\startitemize[packed]
\item
  play and teach children {[}1{]}
\item
  explore underwater {[}2{]} or space {[}3{]}
\item
  be a companion {[}4{]} or a teleoperation {[}5{]} tool
\item
  and so much more.
\stopitemize

Did you see a robot that can jump and paraglide {[}6{]}? Or the creepy
one {[}7{]} that can walk on its own and open doors? Or the one that
hitchhiked through Canada {[}8{]}?

One can say that this is possible because the hardware is getting better
or that we have access to cheap components. While this is the case,
software rules the world and this software is based on ROS.

ROS (Robot Operating System) is a set of different tools that simplify
robot prototyping and more and more production.

Like in most activities, in designing robots there is a huge potential
to bootstrap. That is, while one robot is different than the other they
all have similar needs: there is need to connect hardware (motor
drivers, cameras, other sensors), multiple computers and programs, need
to control it from one place, visualize how it is going and finally do
some robot stuff (plan to go somewhere or do something, talk and be
social or just be cute {[}9{]}.

ROS is a tool for such bootstrapping and in this workshop, you will
learn how you can start interacting with such a system or maybe build
your own.

We will be doing this workshop using IPython and especially, running
most of the code from Jupyter Notebooks. This is because it is a perfect
prototyping tool for ROS! ROS programs, called Nodes, can be written in
many languages Python, C++, Java \ldots{} What IPython offers is ease of
modifying a running program and understanding what is going on. This
greatly speeds up prototyping -- you can see what is wrong and repair
it, while still running the program!

We invite everybody interested in robotics and ROS to our workshop but
actually, there is more. Making robots move using IPython is a great way
to learn some interesting programming and Python concepts such as
Asynchronous Events, Objects. We will try to explain these concepts as
we go and you will have a nice motivation to become a Python master --
if you are not one yet ;)

In short, this workshop will be Python beginner friendly, as much as
possible.

\subsection[plan-for-the-workshop]{Plan for the workshop}

\section[installing-software-and-checking-if-everything-is-working]{Installing
software and checking if everything is working}

We will be using a VirtualBox machine with Ubuntu 14.04 and ROS Indigo
installed. ROS is mainly Linux based as it uses built-in Linux tools and
you can not run the main ROS program -- roscore from Windows. Also, it
is kind of hard, to use multiple ROS versions in the same systems, so
while you can have your own computer with Linux and some other version
of ROS, we will not have time to debug it if something does not work.

We will connect the virtual machine to a roscore server so that you will
be able to read a secret message that our Ono (social robot) prepared
for you :)

\section[introduction-to-ros-and-to-the-state-of-the-art-for-robotics]{Introduction
to ROS and to the state-of-the art for robotics}

We will go through some interesting parts of ROS and we will give you
some examples of how ROS is used in our work and in some cool work
around the world.

Robotics has its own language that you may or may not be familiar with,
we will say what a sensor or a servo is and what is a difference between
a normal Linux box an embedded computer and a microcontroller / Arduino.
This all is so that you will have easier time figuring out the practical
part of the workshop -- ask questions ;)

\section[making-robots-move]{Making robots move}

You will be introduced to an ESP bot which is a small robot with a
servo- controlled head and a light sensitive sensor. To make it move you
will create an ROS {\em publisher} that will control the servo via an
IPython Interactive widget. That is, where you will move a slider robot
will move. All the needed hardware will be provided by us, excluding PC
computer which you have to bring on your own.

We will also build a function that will {\em subscribe} to message
stream (a topic) with data from robot's light sensor.

We will real time visualize this data in IPython using a Bokeh plot.

Your challenge will be to make a program that does something with this
data and sends it back to the robot.

But wait, there's more :) Try to read Ono's sensors or change Ono's
emotions using a different Interactive element -- a list.

We will ask Ono about his emotions using a different way to communicate
-- ROS service. You will ask a Robot sending him necessary data and he
will respond to you.

\section[ros-tools-tour]{ROS tools tour}

ROS has a lots of tools that could be helpful for robot prototyping.
There is no chance to show them all but we will show you some common use
cases and tools that could fit.

There is also a great visualization tool in ROS, called Rviz.
Visualization is something super necessary when building and debugging
robots. While tools such as Bokeh plots are great for couple of sensors
readouts, in Rviz you can see camera images or 3d point clouds and you
can also see how your robot is moving on a robot model.

You will build your own visualization of a real robot that is moving. We
will also show how can you calculate relative positions of two robots
using TF tool -- that simplifies another super important robot issue --
calculations using transform frames (rotations, translations, and
operations on quaternions).

Finally, you will see how you can start multiple {\em nodes} using
roslaunch and how you can record what is happening using Rosbag.

\subsection[some-references-and-helpful-links]{Some references and
helpful links}

For a nice introduction to most of the tools used here, official ROS
tutorials {[}10{]} are a very good start.

There is a Patrick Goebel book {[}11{]} which shows how you can use and
combine these tools (and couple more) to create a mobile robot with arms
that can make map of its own environment to plan his movement and not
bump into things

For an introduction to robotics, I recommend watching some on-line
videos or participating in free MOOCs. Actually one of the first MOOCs
-- AI for robotics {[}12{]} on Udacity is a way that Igor started using
Python seriously {[}13{]}.

\subsection[references]{References}

\startitemize[n,packed][stopper=.,width=2.0em]
\item
  https://www.youtube.com/watch?v=sF0tRCqvyT0
\item
  https://www.youtube.com/watch?v=EgtZAUDqxHE
\item
  https://vimeo.com/146183080
\item
  https://www.youtube.com/watch?v=dx0zxr3D\type{_}zU
\item
  https://www.youtube.com/watch?v=p92415PxgCw
\item
  https://www.youtube.com/watch?v=\type{_}luhn7TLfWU
\item
  https://www.youtube.com/watch?v=rVlhMGQgDkY
\item
  http://mir1.hitchbot.me/
\item
  https://www.youtube.com/watch?v=3g-yrjh58ms
\item
  http://wiki.ros.org/ROS/Tutorials
\item
  http://wiki.ros.org/Books/ROSbyExampleVol2
\item
  https://www.udacity.com/course/artificial-intelligence-for-robotics--cs373
\item
  https://www.udacity.com/course/robotics-nanodegree--nd209
\stopitemize


\stoptext
