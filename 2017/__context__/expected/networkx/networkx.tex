\usemodule[pycon-yyyy]
\starttext

\Title{Network Analysis using Python (workshop)}
\Author{Mridul Seth}
\MakeTitlePage

Politics, Mathematics, Law, Biology, Computer Science, Finance - all of
these subjects have one thing in common: they can be modeled using
networks. NetworkX is a Python language software package for the
creation, manipulation and study of the structure, dynamics, and
functions of complex networks. This workshop will introduce the basics
of network theory and working with graphs using Python and the NetworkX
package. This will be a hands on tutorial and will require writing a lot
of code snippets. The participants should be comfortable with basic
Python (loops, dictionaries, lists) and some (minimal) experience with
working inside a Jupyter Notebook.

Broadly the tutorial is divided into four parts:

\startitemize[n][stopper=.]
\item
  Introduction

  \startitemize[packed]
  \item
    Basics of graph theory and various examples of networks in real
    life.
  \item
    Introduction to the NetworkX API and various data structures
  \stopitemize
\item
  Tools usage
  \startitemize[packed]
  \item
    Work with small synthetic networks (generated using random graph
    generators) to understand the structure of the network.
  \item
    Analyse the network and study various properties of the network like
    centrality, connectivity, shortest paths, cliques.
  \stopitemize
\item
  Analyze data

  \startitemize[packed]
  \item
    We'll use the various techniques we have learnt so far and model a
    network out of real world data like co-authorship network(
    http://www-personal.umich.edu/\lettertilde{}mejn/netdata/cond-mat-2005.zip)
    or a similar network and study the structure and properties of this
    network.
  \stopitemize
\item
  Real world data

  \startitemize[packed]
  \item
    We'll work on some interesting problems like temporal networks and
    visualisation of networks.
  \item
    We'll model the US Airport Network with respect to time and will try
    to make sense out of it.
  \stopitemize
\stopitemize

By the end of the tutorial everyone should be comfortable with hacking
on the NetworkX API, modelling data as networks and basic analysis on
networks using Python.

This text gives a brief overview of the content of the workshop.

\section[packages-and-other-requirements]{Packages and other
requirements}

You should have python3 and pip installed, and using
\useURL[url1][https://virtualenv.pypa.io/en/stable/][][virtualenv]\from[url1]
is recommended and the following Python packages should be installed.

\startitemize[packed]
\item
  Jupyter
\item
  matplotlib
\item
  networkx
\item
  pandas
\item
  numpy
\stopitemize

You can also use
\useURL[url2][https://conda.io/docs/][][conda]\from[url2] to install
these packages.

The workshop will also require datasets to work on which will be
uploaded on GitHub repository
https://github.com/MridulS/pyconpl-2017-networkx.

The repo will contain the Jupyter notebooks that will be required during
the workshop.

\subsection[introduction]{Introduction}

\subsubsection[what-are-networks-graphs-and-how-can-we-use-networkx-and-python-to-represent-them]{What
are Networks (Graphs) and how can we use NetworkX and Python to
represent them?}

A graph G is represented by a set of nodes and a set of edges (rys. 1).
An edge between two nodes in a graph signifies a relationship between
those two nodes. Edges can be directed or undirected.

\placefigure{Network}{\externalfigure[network.png]}

NetworkX uses dictionaries underneath to store node and edge data. It's
dict-o-dict-o-dict-o-dict to be precise.

\starttyping
G['node1']
G['node1']['node2']
G['node1']['node2']['some_id']['some_attrb'] # multigraphs
\stoptyping

\subsubsection[basics-of-networkx]{Basics of NetworkX}

There are multiple graph classes implemented in NetworkX:

\startitemize[packed]
\item
  Graph - a graph object which only allows single edges (undirected)
  between nodes.
\item
  DiGraph - a graph object which only allow single edges (directed)
  between nodes, i.e.~an edge from \quote{A} to \quote{B} doesn't mean
  that there exists an edge from \quote{B} to \quote{A}.
\item
  MultiGraph - a graph object which allows multiple edges(undirected)
  between two edges.
\item
  MultiDiGraph - a graph object which allows multiple edges(directed)
  between two edges.
\stopitemize

Hybrid graphs? Not yet available in NetworkX.

\starttyping
# Create an empty graph object with no nodes and edges.
G = nx.Graph() # DiGraph, MultiGraph, MultiDiGraph
\stoptyping

\starttyping
# Add nodes to our graph object
# In NetworkX, nodes can be any hashable object e.g. a text string,
# an image, an XML object, another Graph, a customized node object, etc.

G.add_node('1')
G.add_node(1)
G.add_node('second')
\stoptyping

\type{G.nodes()} will give a list of nodes.

\starttyping
# Now let's talk about edges.
# Edge between two nodes means that they share some property/relationship
# G.add_node(H)
G.add_edge(0, 'second')
G.add_edge(2, 3)
G.add_edge('second', 'node4')
\stoptyping

\type{G.edges()} will give a list of edges.

\subsubsection[measures-and-properties-of-a-network]{Measures and
properties of a network}

Finding the important nodes in a network is a task frequently
encountered by network scientists. There are various approaches to this
like using the nodes with highest degree, i.e connections. or nodes
which have a high number of paths going through them as removing that
node can impair mobility in the network. Deciding the measures to use
depends on the applications and the requirements of the network
scientist.

Usually one of the most important tasks in network analysis is to find
the shortest path and there are various algorithm to dot that. We will
use the algorithms implemented in the NetworkX package to study the
network.

Structures in a network is also a major network analysis tool like
finding triangles, cliques, open triangles. A rudimentary recommendation
engine can be made by looking for open triangles in a network.

From wikipedia, A connected component (or just component) of an
undirected graph is a subgraph in which any two vertices are connected
to each other by paths, and which is connected to no additional vertices
in the supergraph.

NetworkX has algorithms implemented to help us with finding components
of a network.

\starttyping
print [len(c) for c in sorted(nx.connected_components(authors_graph),
    key=len, reverse=True)]
\stoptyping

This piece of code gives us the components of a network
\type{authors_graph} where \type{authors_graph} is the Arxiv GR-QC
(General Relativity and Quantum Cosmology) collaboration network.

source: http://snap.stanford.edu/data/index.html\#canets

The output:

\starttyping
[4158, 14, 12, 10, 9, 9, 8, 8, 8, 8, 8, 8, 7, 7, 7, 7, 7, 7, 7,
7, 6, 6, 6, 6, 6, 6, 6, 6, 6, 6, 6, 6, 5, 5, 5, 5, 5, 5, 5, 5,
5, 5, 5, 5, 5, 5, 5, 5, 5, 4, 4, 4, 4, 4, 4, 4, 4, 4, 4, 4, 4,
4, 4, 4, 4, 4, 4, 4, 4, 4, 4, 4, 4, 4, 4, 4, 4, 4, 4, 3, 3, 3,
3, 3, 3, 3, 3, 3, 3, 3, 3, 3, 3, 3, 3, 3, 3, 3, 3, 3, 3, 3, 3,
3, 3, 3, 3, 3, 3, 3, 3, 3, 3, 3, 3, 3, 3, 3, 3, 3, 3, 3, 3, 3,
3, 3, 3, 3, 3, 3, 3, 3, 3, 3, 3, 3, 3, 3, 3, 3, 3, 3, 3, 3, 3,
3, 3, 3, 3, 3, 3, 3, 3, 3, 3, 3, 3, 3, 3, 3, 3, 3, 3, 3, 3, 3,
3, 3, 3, 3, 3, 3, 3, 3, 3, 3, 3, 2, 2, 2, 2, 2, 2, 2, 2, 2, 2,
2, 2, 2, 2, 2, 2, 2, 2, 2, 2, 2, 2, 2, 2, 2, 2, 2, 2, 2, 2, 2,
2, 2, 2, 2, 2, 2, 2, 2, 2, 2, 2, 2, 2, 2, 2, 2, 2, 2, 2, 2, 2,
2, 2, 2, 2, 2, 2, 2, 2, 2, 2, 2, 2, 2, 2, 2, 2, 2, 2, 2, 2, 2,
2, 2, 2, 2, 2, 2, 2, 2, 2, 2, 2, 2, 2, 2, 2, 2, 2, 2, 2, 2, 2,
2, 2, 2, 2, 2, 2, 2, 2, 2, 2, 2, 2, 2, 2, 2, 2, 2, 2, 2, 2, 2,
2, 2, 2, 2, 2, 2, 2, 2, 2, 2, 2, 2, 2, 2, 2, 2, 2, 2, 2, 2, 2,
2, 2, 2, 2, 2, 2, 2, 2, 2, 2, 2, 2, 2, 2, 2, 2, 2, 2, 2, 2, 2,
2, 2, 2, 2, 2, 2, 2, 2, 2, 2, 2, 2, 2, 2, 2, 2, 2, 2, 2, 2, 1]
\stoptyping

As we can see from the output there are a lot of authors who work in
pairs (they haven't collaborated with anyone other than their partner)
and triangles. But the majority, 4158, work in a big collaboration
network, even though we have a lone wolf there.

We can find many other interesting properties of a dataset by doing
exploratory analysis of the network.

\subsubsection[real-world-data]{Real world data}

There are multiple sources to find real word network datasets like
\useURL[url3][http://snap.stanford.edu/data/index.html][][SNAP]\from[url3]
and \useURL[url4][http://konect.uni-koblenz.de][][KONECT]\from[url4]. We
will use a real world network and analyse the network using the various
tools and measures we learn in this tutorial.

\subsection[references]{References}

\startitemize[n,packed][stopper=.]
\item
  Official NetworkX tutorial.
  https://github.com/networkx/notebooks/blob/\crlf
  master/tutorial.ipynb
\item
  NetworkX documentation. https://networkx.readthedocs.io/en/stable/
\item
  EuroSciPy 2016 tutorial. https://github.com/MridulS/euroscipy-networkx
\item
  PyCon/SciPy 2017 tutorial.
  https://github.com/ericmjl/Network-Analysis-Made-Simple
\item
  Video Tutorial.
  https://www.youtube.com/watch?v=E4VKzFmByhE\type{&}t=3139s
\item
  DataCamp course.
  https://www.datacamp.com/courses/network-analysis-in-python-part-1
\item
  Datasets (Social Network Analysis project at Stanford).
  http://snap.stanford.edu/\crlf
  data/index.html
\item
  KONECT (KONECT project at UNamur). http://konect.uni-koblenz.de
\stopitemize


\stoptext
