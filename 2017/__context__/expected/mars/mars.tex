\usemodule[pycon-yyyy]
\starttext


\section[python-w-łaziku-marsjańskim-michał-barciś]{Python w łaziku
marsjańskim? --- Michał Barciś}

Już od ponad dwóch lat pracujemy wspólnie z zespołem Continuum z
Uniwersytetu Wrocławskiego nad prototypem łazika marsjańskiego Aleph1. W
tym czasie rozważaliśmy wiele rozwiązań i różnych technologii, a
ostatecznie duża część robota została zaprogramowana w Pythonie. Podczas
prezentacji opowiem o samym łaziku oraz drodze, którą przebyliśmy w
czasie tworzenia go, ale w tym artykule chciałbym trochę poszerzyć ten
temat i przedstawić przegląd najciekawszych zastosowań tego języka
programowania. Wspomnę tutaj o wielu bibliotekach oraz technologiach,
których nie będę dokładnie opisywał --- zainteresowany czytelnik na
pewno bez problemu znajdzie więcej informacji na ich temat w Internecie.

\section[mocne-strony]{Mocne strony}

Jeszcze kilka lat temu, gdy poruszany był temat Pythona częstą reakcją
było \quotation{To ten język, w którym pisze się skrypty, nadaje się to
do czegokolwiek innego?}, \quotation{Poza szkołami nie ma miejsca dla
Pythona, to dobry język tylko dla początkujących}, albo nawet, ze
zdziwieniem, \quotation{To w Pythonie da się napisać aplikację
okienkową?}.

Podczas pracy w STX Next zauważyłem jednak, że coraz częściej nawet nasi
klienci zdają sobie sprawę z tego, że Python to nie tylko aplikacje
Internetowe, czy proste skrypty służące do zarządzania serwerami.
Dostarczane rozwiązania muszą być wielośrodowiskowe, a aplikacja
Internetowa nie może już tylko działać w przeglądarce --- ma rozpoznawać
obrazy z kamery, analizować zbierane dane, wizualizować je w czasie
rzeczywistym, a czasami nawet ostatecznie włączyć światło albo otworzyć
zamek w drzwiach. Na szczęście praktycznie w każdej z tych dziedzin
możemy korzystać z lubianej i znanej technologii --- z Pythona.

\subsection[serwisy-internetowe]{Serwisy Internetowe}

O stronach Internetowych w Pythonie nie muszę się chyba dużo rozpisywać.
Django, Pylons, Flask i wiele innych sprawiają, że gdy już uporamy się z
problemem, który framework wybrać, samo tworzenie oprogramowania to
pestka. Jednak według mnie, głównym atutem Pythona na stronach
Internetowych jest łatwość integrowania go z innymi usługami, które
często udostępniają API w tych samych technologiach. Dzięki temu nasza
strona internetowa staje się już tylko \quotation{frontendem} do dużo
bardziej złożonych zastosowań.

\subsection[aplikacje-desktopowe]{Aplikacje desktopowe}

Coraz szybciej odchodzą w niepamięć, ponieważ prawie wszystko można już
zrobić w przeglądarce Internetowej. Wciąż jednak czasami preferujemy
\quotation{natywne} aplikacje, a Python sprawdza się w nich świetnie ---
szczególnie wtedy, kiedy aplikacja często wchodzi w interakcję z
użytkownikiem. Dzięki pakietom takim jak TkInter, WxPython czy PyQt
tworzenie aplikacji jest szybkie i przyjemne, a interfejs użytkownika
dobrze integruje się z systemem operacyjnym.

Jednak chyba największą zaletą Pythona w aplikacjach desktopowych jest
używanie go jako narzędzia do rozszerzeń i skryptowania rozwiązań.
Blender, VIM i wiele innych właśnie w ten sposób z powodzeniem
wykorzystuje Pythona.

\subsection[devops-i-skrypty]{Devops i skrypty}

Python jako język skryptowy z wbudowanym REPL i wieloma jego
zamiennikami takimi jak iPython, bpython, ptpython naturalnie nadaje się
do zarządzania systemem operacyjnym i plikami. Dodatkowo pakiety takie
jak Plumbum zapewniają, że znając Pythona, już nigdy więcej nie będziemy
musieli pisać skryptów w Bashu.

\subsection[gry-komputerowe]{Gry komputerowe}

Co prawda raczej ciężko o duże tytuły napisane w 100\letterpercent{} w
Pythonie, jednak właśnie ten język bardzo często wykorzystywany jest do
tworzenia scenariuszy i skryptów sterujących przebiegiem gry, które nie
wymagają wysokiej wydajności i szybkości działania. Dzięki temu
scenarzyści mogą szybko i skutecznie tworzyć wciągające światy i
postaci.

\subsection[nauczanie-maszynowe]{Nauczanie maszynowe}

Dla mnie dużym zdziwieniem było jakim cudem właśnie w tej dziedzinie
Python zdobył taką dużą popularność. Jednak już po pierwszym projekcie
zrozumiałem, że to dzięki możliwości łatwego i dobrze ustandaryzowanego
przetwarzania danych z niewielkim narzutem pamięciowym. Dodatkowo
biblioteki takie jak numpy umożliwiają szybkie wykonywanie operacji
arytmetycznych na tych danych. Wiele algorytmów nauczania maszynowego
można sprowadzić do operacji na macierzach. Tam, gdzie mimo to wydajność
jest problemem stosuje się silniki pisane na przykład w C/C++ i
udostępniające API do Pythona, dzięki czemu możemy szybko i wydajnie
eksperymentować na modelach.

Dodatkowo, społeczność wytworzyła wiele darmowych narzędzi
umożliwiających łatwą pracę z dużą ilością danych. Od bibliotek
służących do wizualizacji, takich jak matplotlib, poprzez łatwą
prezentację danych w Jupyter Notebook (kiedyś iPython notebook), aż po
frameworki do pracy z danymi takie jak Pandas, czy scikit-learn.

\subsection[mikrokontrolery-i-automatyka]{Mikrokontrolery i automatyka}

O dziwo, nawet przy bardzo niskopoziomowym programowaniu coraz częściej
słyszy się o Pythonie --- a to głównie za sprawą projektu MicroPython,
który umożliwia uruchamianie kodu Pythona na systemach wbudowanych. Czy
ma to sens, czy od takich systemów nie oczekuje się raczej szybkości,
niezawodności, wykorzystania każdego cyklu procesora? Wg mnie ma ---
nawet najmniejsze procesory są coraz bardziej wydajne i często
potrzebują wykonać tylko najprostsze operacje, np. odebrać pakiet z
sieci i włączyć światło. Chcielibyśmy móc tego typu urządzenia
programować szybko i być pewni, że logika przez nie wykonywana zgadza
się z tym, co oczekujemy i właśnie to jest nam w stanie zapewnić krótki
i przejrzysty kod w Pythonie.

\subsection[robotyka]{Robotyka}

To chyba jedno z najciekawszych zastosowań Pythona i z roku na rok coraz
bardziej popularnych, co widać na przykład po szybkim rozwoju projektu
ROS (Robot Operating System), który jest swoistym frameworkiem do
tworzenia różnych wysoce-zautomatyzowanych urządzeń i systemów. Roboty
to nie tylko niskopoziomowy kod często wymagający pracy w czasie
rzeczywistym oraz maksymalnego wykorzystania zasobów. W każdym robocie
jest jakaś część logiki odpowiedzialna za podejmowanie wysokopoziomowych
decyzji, takich jak na przykład wybór miejsca do którego robot ma się
przemieścić, czy który przycisk wcisnąć. Python doskonale się do tego
nadaje --- zazwyczaj nie są to operacje wymagające pracy w czasie
rzeczywistym, a świetna integracja z systemami do nauczania maszynowego
umożliwia pełne wykorzystanie zasobów. Dodatkowo dużo łatwiej w Pythonie
stworzyć prototyp na podstawie którego wyodrębnione zostaną komponenty,
które należy przepisać np. do C w celu poprawy wydajności.

\section[a-kiedy-python-może-nie-wystarczyć]{A kiedy Python może nie
wystarczyć?}

Niestety Python nie zawsze będzie idealnym rozwiązaniem. Najczęściej
przywoływanym problemem są aplikacje czasu rzeczywistego, szczególnie te
wymagające bardzo szybkich odpowiedzi. I faktycznie raczej nie użyłbym
tego języka podczas programowania regulatora silników dużych prędkości,
jednak w momencie gdy mówimy o decyzjach podejmowanych w setnych
sekundy, czemu nie? Garbage collector zawsze można wyłączyć w kluczowych
momentach, a dzisiejsze procesory często mają bardzo duży zapas mocy
obliczeniowej. W ostateczności nasz program można traktować jako
prototyp, który pozwoli nam łatwo znaleźć miejsca wymagające
optymalizacji, albo po prostu za szablon aplikacji stworzonej w innej
technologii.

Python może również nie sprawdzić się w programach, które wymagają dużej
ilości obliczeń --- symulacje, silniki gier komputerowych, dekodowanie
danych. Na szczęście w takim wypadku zawsze możemy wyodrębnić krytyczne
części kodu i zrealizować je w innej technologii.

Podsumowując, wszechstronność Pythona jest wyjątkowo mocną cechą tego
języka i warto z niej korzystać --- w końcu możemy tworzyć zupełnie nowe
rzeczy i rozwijać nasze zainteresowania w wielu dziedzinach w
technologii, którą już dobrze znamy i lubimy, dzięki czemu próg wejścia
jest znacznie mniejszy.

\section[źródła]{Źródła}

\startitemize[packed]
\item
  https://www.python.org/about/success/usa/ --- Python success stories
\item
  https://wiki.python.org/moin/Applications --- lista aplikacji
  napisanych w Pythonie
\item
  https://en.wikipedia.org/wiki/List\type{_}of\type{_}Python\type{_}software
  --- inna (dłuższa) lista aplikacji napisanych w Pythonie
\item
  https://worthwhile.com/blog/2016/07/19/django-python-advantages/ ---
  Python i web development
\item
  https://wiki.python.org/moin/WebFrameworks --- Frameworki webowe w
  Pythonie
\item
  http://plumbum.readthedocs.io/ --- oficjalna strona Plumbum
\item
  http://scikit-learn.org/ --- SciKit learn
\item
  https://micropython.org/ --- MicroPython
\item
  http://www.ros.org/ --- ROS
\item
  https://stackoverflow.com/a/15011981/540717 --- Python i aplikacje
  czasu rzeczywistego
\stopitemize


\stoptext
