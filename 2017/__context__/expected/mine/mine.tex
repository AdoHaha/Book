\usemodule[pycon-yyyy]
\starttext

\Title{Minecraft sterowany Pythonem}
\Author{Milena Piasecka}
\MakeTitlePage

Minecraft zapewnia niczym nieskrępowaną kreatywność i swobodę tworzenia.
Python jest prosty do czytania i zapisywania, daje duże możliwości tak
początkującemu, jak i zaawansowanemu programiście. Dzięki zintegrowaniu
Minecrafta z Pythonem możemy uzyskać jeszcze większą kontrolę nad
Minecraftem i tym samym wyzwolić większe pokłady kreatywności.

Dlaczego warto?

\startitemize[packed]
\item
  Poznanie ukrytej magii Minecrafta - urozmaicenie gry i odkrycie jej
  nieznanych możliwości
\item
  Zapoznanie się z Pythonem - dla początkujących programistów
\item
  Nauczanie programowania, w szczególności wśród dzieci (od ok. 10 lat)
  i młodzieży
\item
  Ułatwienie poruszania się i budowania w świecie Minecraft.
\stopitemize

\subsection[startujemy]{Startujemy}

Co jest potrzebne: -
\useURL[url1][\%5Bhttps://minecraft.net/en-us/download/\%5D][][Minecraft]\from[url1]
- Minecraft w wersji podstawowej -
\useURL[url2][https://www.python.org/downloads/][][Python]\from[url2] -
minimum Python 3 -
\useURL[url3][https://www.java.com/en/download/][][Java]\from[url3] -
najlepiej ostatnia wersja - API Minecraft Python - pobierz Minecraft
Tools dla
\useURL[url4][https://sourceforge.net/projects/python-with-minecraft-windows/][][Windows]\from[url4],
\useURL[url5][https://sourceforge.net/projects/python-with-minecraft-mac/files/?source=navbar][][MAC
OS]\from[url5],
\useURL[url6][https://github.com/py3minepi/py3minepi][][Rapsberry Pi lub
Ubuntu]\from[url6] -
\useURL[url7][https://getbukkit.org/spigot][][Server Minecraft
Spigot]\from[url7] - wersja Spigot zgodna z wersją Minecrafta

Jeśli na komputerze mamy już Minecrafta, Pythona i Javę, należy
ściągnąć, rozpakować folder \quotation{Minecraft Tools} i zainstalować
API Minecrafta poprzez uruchomienie \quotation{Install\type{_}API}
spośród wyodrębnionych plików. Wtedy należy pobrać taką wersję serwera
Spigot, która jest zgodna z posiadaną wersją Minecrafta. Pobrany plik
\quotation{Spigot.jar} należy podmienić w folderze \quotation{server} w
\quotation{Minecraft Tools}. Wtedy można już uruchomić serwer poprzez
\quotation{Start\type{_}server} z folderu \quotation{Minecraft Tools}.
Okna serwera nie należy zamykać.

Wtedy wystarczy już tylko: 1) Uruchomić Minecrafta, 2) Przejść w tryb
Multiplayer, 3) Dodać serwer: nazwa dowolna, w adresie wpisać
\quotation{localhost} 4) Wejść w tryb Multiplayer poprzez utworzony
serwer.

\section[testujemy]{Testujemy}

Żeby przetestować połączenie serwera i API, a także tworzyć programy
współpracujące z Minecraftem, należy mieć uruchomiony serwer oraz
Minecrafta w trybie Multiplayer (z ustawionym serwerem Spigot).

W edytorze Idle wpisujemy:

\starttyping
>>> from mcpi.minecraft import Minecraft
>>> mc = Minecraft.create ()
\stoptyping

Te 2 pierwsze łączą program z Minecraftem. Jeśli po uruchomieniu
programu nie pojawił się komunikat o błędzie, możemy kontynuować
wpisując np.

\starttyping
>>> mc.player.setTilePos(0,120,0)
\stoptyping

Dzięki temu poleceniu nasz bohater w Minecrafcie powinien unieść się
wysoko nad ziemię. Jeśli tak się stało, wszystko działa sprawnie.

\subsection[misje]{Misje}

W Minecrafcie jest wiele przydatnych akcji, jakie można uruchomić
sterując Pythonem. Warto je znać, aby móc się szybciej i efektywniej:
wybudować, wyżywić, schronić przed wrogimi bytami czy teleportować.

\section[teleportacja]{Teleportacja}

Położenie w Minecrafcie można łatwo ustalić za pomocą klawisza F3.

Jeśli znamy docelowe współrzędne, można szybko teleportować się w
wybrane miejsce:

\starttyping
from mcpi.minecraft import Minecraft
mc = Minecraft.create ()
x=10
y=110
y=12
mc.player.setPos (x, y, z)
\stoptyping

W ten sposób można np. szybko uciec przed zombie w bezpieczne miejsce -
jak wnętrze domu, jeśli wcześniej znamy jego współrzędne.

Jeśli chcemy np. \quotation{obejść} znane sobie miejsca, tj. przenieść
się gdzieś, rozejrzeć się i przenieść znów w inne miejsce, przydatna
będzie funkcja {\em sleep}. Żeby ją wykorzystać, musimy dodać do
programu moduł {\em time} poprzez:

\starttyping
>>> import time
\stoptyping

Wtedy możemy wykorzystać funkcję {\em sleep}

\starttyping
>>> time.sleep (5)
\stoptyping

żeby przenieść się w wybrane miejsce na 5 sekund, a następnie wybrać w
kolejne miejsce:

\starttyping
from mcpi.minecraft import Minecraft
mc = Minecraft.create ()
mc.player.setPos (10, 110, 12)
time.sleep (5)
mc.player.setPos (20, 110, 50)
\stoptyping

\section[stawianie-bloków]{Stawianie bloków}

W Minecrafcie każdy typ bloku ma swój ID. Pełną listę bloków dostępnych
w Minecrafcie znajdziesz tu:
\useURL[url8][http://minecraft-pl.gamepedia.com/Warto\%C5\%9Bci_danych][][Lista
bloków]\from[url8]. Znając ID bloku oraz współrzędne, gdzie chcemy dany
blok postawić, możemy w szybki sposób go umieścić w danym miejscu.

\starttyping
from mcpi.minecraft import Minecraft
mc = Minecraft.create ()
mc.setBlock (10, 110, 12, 103)
\stoptyping

W powyższym przypadku postawiliśmy arbuza (kod 103).

\section[szybkie-budowanie]{Szybkie budowanie}

To, co jest nam od początku potrzebne w świecie Minecraft, to
odpowiednie schronienie. Dlatego zazwyczaj pierwszy dzień w grze
przeznaczamy na budowę domu. Jest to szczególnie istotne w trybie
przetrwanie, gdzie dom stanowi schronienie przed wrogimi bytami, jak np.
zombie.

Do tworzenia jednego bloku wykorzystaliśmy funkcję {\em setBlock ()}, a
do budowania większych bloków użyjemy funkcji {\em setBlocks ()}, która
pozwoli nam stworzyć prostopadłościan.

\starttyping
from mcpi.minecraft import Minecraft
mc = Minecraft.create ()
poz = mc.player.getPos ()
x = poz.x
y = poz.y
z = poz.z
szer = 10
wys = 5
dlug = 6
typBloku = 4
powietrze = 0
mc.setBlocks (x,y,z,x+szer, y+wys, z+dlug, typBloku)
\stoptyping

W ten sposób utworzony prostopadłościan jest pełny w środku. Teraz
trzeba stworzyć mniejszy prostopadłościan zbudowany z bloków powietrza
poprzez zmniejszenie argumentów wcześniej zbudowanego prostopadłościanu
o 1.

Istotne jest to, że w powyższym przykładzie użyliśmy funkcji {\em getPos
()}, która zwraca współrzędne jako wartości rzeczywiste, w
przeciwieństwie do funkcji {\em getTilePos ()}, która zwraca współrzędne
jako liczby całkowite.

\subsection[źródła]{Źródła:}

\startitemize[packed]
\item
  \useURL[url9][http://minecraft-pl.gamepedia.com/Warto\%C5\%9Bci_danych][][Minecraft
  - gamepedia]\from[url9]
\item
  \quotation{{\em Nauka programowania z Minecraftem}} - autor: Craig
  Richardson, Warszawa, 2016.
\stopitemize


\stoptext
