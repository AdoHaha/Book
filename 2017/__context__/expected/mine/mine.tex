\usemodule[pycon-yyyy]
\starttext

\Title{Minecraft sterowany Pythonem (warsztat)}
\Author{Milena Piasecka}
\MakeTitlePage

Minecraft {[}1{]} zapewnia niczym nieskrępowaną kreatywność i swobodę
tworzenia. Python jest prosty do czytania i zapisywania, daje duże
możliwości tak początkującemu, jak i zaawansowanemu programiście. Dzięki
zintegrowaniu Minecrafta z Pythonem możemy uzyskać jeszcze większą
kontrolę nad Minecraftem i tym samym wyzwolić większe pokłady
kreatywności {[}2{]}.

Zalety integracji Minecrafta z Pythonem są następujące:

\startitemize[packed]
\item
  poznanie ukrytej magii Minecrafta - urozmaicenie gry i odkrycie jej
  nieznanych możliwości;
\item
  zapoznanie się z Pythonem - dla początkujących programistów;
\item
  nauczanie programowania, w szczególności wśród dzieci (od ok. 10 lat)
  i młodzieży;
\item
  ułatwienie poruszania się i budowania w świecie Minecraft.
\stopitemize

\subsection[startujemy]{Startujemy}

Do rozpoczęcia prac są potrzebne następujące programy:

\startitemize[packed]
\item
  Minecraft {[}3{]} w wersji podstawowej;
\item
  Python {[}4{]} - minimum Python 3;
\item
  Java {[}5{]} - najlepiej ostatnia wersja;
\item
  API Minecraft Python - pobierz Minecraft Tools dla Windows {[}6{]},
  MAC OS {[}7{]}, Raspberry Pi {[}11{]} lub Ubuntu {[}8{]};
\item
  Server Minecraft Spigot {[}9{]} - wersja Spigot zgodna z wersją
  Minecrafta.
\stopitemize

Jeśli na komputerze mamy już Minecrafta, Pythona i Javę, należy
ściągnąć, rozpakować folder \quotation{Minecraft Tools} i zainstalować
API Minecrafta poprzez uruchomienie \quotation{Install\type{_}API}
spośród wyodrębnionych plików. Wtedy należy pobrać taką wersję serwera
Spigot, która jest zgodna z posiadaną wersją Minecrafta. Pobrany plik
\quotation{Spigot.jar} należy podmienić w folderze \quotation{server} w
\quotation{Minecraft Tools}. Wtedy można już uruchomić serwer poprzez
\quotation{Start\type{_}server} z folderu \quotation{Minecraft Tools}.
Okna serwera nie należy zamykać.

Wtedy wystarczy już tylko:

\startitemize[n,packed][stopper=.]
\item
  uruchomić Minecrafta;
\item
  przejść w tryb Multiplayer;
\item
  dodać serwer: nazwa dowolna, w adresie wpisać \quotation{localhost};
\item
  wejść w tryb Multiplayer poprzez utworzony serwer.
\stopitemize

\section[testujemy]{Testujemy}

Żeby przetestować połączenie serwera i API, a także tworzyć programy
współpracujące z Minecraftem, należy mieć uruchomiony serwer oraz
Minecrafta w trybie Multiplayer (z ustawionym serwerem Spigot).

W edytorze Idle wpisujemy:

\starttyping
>>> from mcpi.minecraft import Minecraft
>>> mc = Minecraft.create ()
\stoptyping

Te 2 pierwsze łączą program z Minecraftem. Jeśli po uruchomieniu
programu nie pojawił się komunikat o błędzie, możemy kontynuować
wpisując np.

\starttyping
>>> mc.player.setTilePos(0,120,0)
\stoptyping

Dzięki temu poleceniu nasz bohater w Minecrafcie powinien unieść się
wysoko nad ziemię. Jeśli tak się stało, wszystko działa sprawnie.

\subsection[misje]{Misje}

W Minecrafcie jest wiele przydatnych akcji, jakie można uruchomić
sterując Pythonem. Warto je znać, aby móc się szybciej i efektywniej:
wybudować, wyżywić, schronić przed wrogimi bytami czy teleportować.

\section[teleportacja]{Teleportacja}

Położenie w Minecrafcie można łatwo ustalić za pomocą klawisza F3.

Jeśli znamy docelowe współrzędne, można szybko teleportować się w
wybrane miejsce:

\starttyping
from mcpi.minecraft import Minecraft
mc = Minecraft.create ()
x=10
y=110
y=12
mc.player.setPos (x, y, z)
\stoptyping

W ten sposób można np. szybko uciec przed zombie w bezpieczne miejsce -
jak wnętrze domu, jeśli wcześniej znamy jego współrzędne.

Jeśli chcemy np. \quotation{obejść} znane sobie miejsca, tj. przenieść
się gdzieś, rozejrzeć się i przenieść znów w inne miejsce, przydatna
będzie funkcja {\em sleep}. Żeby ją wykorzystać, musimy dodać do
programu moduł {\em time} poprzez:

\starttyping
>>> import time
\stoptyping

Wtedy możemy wykorzystać funkcję {\em sleep}

\starttyping
>>> time.sleep (5)
\stoptyping

żeby przenieść się w wybrane miejsce na 5 sekund, a następnie wybrać w
kolejne miejsce:

\starttyping
from mcpi.minecraft import Minecraft
mc = Minecraft.create ()
mc.player.setPos (10, 110, 12)
time.sleep (5)
mc.player.setPos (20, 110, 50)
\stoptyping

\section[stawianie-bloków]{Stawianie bloków}

W Minecrafcie każdy typ bloku ma swój ID, a pełna lista bloków
dostępnych w Minecrafcie jest na {[}10{]}. Znając ID bloku oraz
współrzędne, gdzie chcemy dany blok postawić, możemy w szybki sposób go
umieścić w danym miejscu.

\starttyping
from mcpi.minecraft import Minecraft
mc = Minecraft.create ()
mc.setBlock (10, 110, 12, 103)
\stoptyping

W powyższym przypadku postawiliśmy arbuza (kod 103).

\section[szybkie-budowanie]{Szybkie budowanie}

To, co jest nam od początku potrzebne w świecie Minecraft, to
odpowiednie schronienie. Dlatego zazwyczaj pierwszy dzień w grze
przeznaczamy na budowę domu. Jest to szczególnie istotne w trybie
przetrwanie, gdzie dom stanowi schronienie przed wrogimi bytami, jak np.
zombie.

Do tworzenia jednego bloku wykorzystaliśmy funkcję {\em setBlock ()}, a
do budowania większych bloków użyjemy funkcji {\em setBlocks ()}, która
pozwoli nam stworzyć prostopadłościan.

\starttyping
from mcpi.minecraft import Minecraft
mc = Minecraft.create ()
poz = mc.player.getPos ()
x = poz.x
y = poz.y
z = poz.z
szer = 10
wys = 5
dlug = 6
typBloku = 4
powietrze = 0
mc.setBlocks (x,y,z,x+szer, y+wys, z+dlug, typBloku)
\stoptyping

W ten sposób utworzony prostopadłościan jest pełny w środku. Teraz
trzeba stworzyć mniejszy prostopadłościan zbudowany z bloków powietrza
poprzez zmniejszenie argumentów wcześniej zbudowanego prostopadłościanu
o 1.

Istotne jest to, że w powyższym przykładzie użyliśmy funkcji {\em getPos
()}, która zwraca współrzędne jako wartości rzeczywiste, w
przeciwieństwie do funkcji {\em getTilePos ()}, która zwraca współrzędne
jako liczby całkowite.

\subsection[bibliografia]{Bibliografia}

\startitemize[n,packed][stopper=.,width=2.0em]
\item
  Minecraft - gamepedia. https://minecraft-pl.gamepedia.com/
\item
  Craig Richardson. Nauka programowania z Minecraftem. PWN, Warszawa,
  2016.
\item
  https://minecraft.net/en-us/download/
\item
  https://www.python.org/downloads/
\item
  https://www.java.com/en/download/
\item
  https://sourceforge.net/projects/python-with-minecraft-windows/
\item
  https://sourceforge.net/projects/python-with-minecraft-mac/\crlf
  files/?source=navbar
\item
  https://github.com/py3minepi/py3minepi
\item
  https://getbukkit.org/spigot
\item
  http://minecraft-pl.gamepedia.com/Wartości\type{_}danych
\item
  https://dev.bukkit.org/projects/raspberryjuice
\stopitemize


\stoptext
