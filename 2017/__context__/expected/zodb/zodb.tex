\usemodule[pycon-yyyy]
\starttext

\Title{The ZODB Ecosystem}
\Author{Christopher Lozinski}
\MakeTitlePage

\subsection[abstract]{Abstract}

ZODB is an object-oriented database written in Python and optimized in
C. It is widely used in the US (United States of America), but not as
well known in Poland. This talk introduces the database, presents
important concepts, and reviews the tools available.

\subsection[about-the-author]{About the Author}

Christopher Lozinski has been using the ZODB since 1999. He is an MIT
graduate and Silicon Valley expat who has been using dynamically bound
languages for 35 years. First it was Lisp, then Objective-C, and now
Python. On the ZODB, he built PythonLinks.info, PolandTrade.info,
PrivaCV.com, Zopache.com, and the recently decommissioned
SpecialtyJobMarkets.com. He is a dual Polish-US citizen, and polyglot.

\subsection[introduction]{Introduction}

You can download the talk slides {[}1{]} and watch the presentation
{[}2{]}. ZODB {[}3{]} is an object-oriented database written in Python
and optimized in C.

\subsection[so-easy-to-use]{So Easy to Use}

ZODB is really easy to use from Python. Just subclass off of class
persistent. Persistent, and your objects become persistent. Here is a
code sample from the Tutorial {[}4{]}.

\starttyping
import persistent

class Account(persistent.Persistent):

    def __init__(self):
        self.balance = 0.0

    def deposit(self, amount):
        self.balance += amount

    def cash(self, amount):
        assert amount < self.balance
        self.balance -= amount
\stoptyping

Did you notice how trivial it was to create a persistent object. Magic!

You can also subclass off of class Persistent Containers (Btrees), and
class Persistent Set.

\subsection[acid-properties]{ACID Properties}

ZODB is an ACID-compliant database {[}5{]}:

\startitemize[packed]
\item
  Atomicity requires that each transaction be \quotation{all or
  nothing};
\item
  Consistency ensures that any transaction will bring the database from
  one valid state to another;
\item
  Isolation ensures that the concurrent execution of transactions
  results in a system state that would be obtained if transactions were
  executed sequentially, i.e., one after the other;
\item
  Durability ensures that once a transaction has been committed, it will
  remain so, even in the event of power loss, crashes, or errors.
\stopitemize

\subsection[zodb-data-model]{ZODB Data Model}

Simple ZODB applications store data as a tree of objects, essentially a
tree of dictionaries. More sophisticated applications store Data as a
graph of objects. Really sophisticated applications can store data as
any Python data structure. If the data is reachable from the root object
it is persistent. If not, it is garbage collected.

\subsection[a-specific-data-model]{A Specific Data Model}

PythonLinks.info/the-world is a model of the Python software industry
represented as a hierarchy. \quotation{The-world} is a category
containing regions. Regions contain countries. Countries contain cities.
Cities contain companies. Companies can contain jobs, products, and
other links such as YouTube videos, product reviews, and blogs. The tree
representing the Python industry can be displayed as a table.

\subsection[traversal]{Traversal}

To access a particular object, one has to start at the root, and
traverse the tree. Since the nodes are essentially dictionaries, with
text keys, it is very natural to map a URL to a particular object by
traversing the tree.

\subsection[canonical-urls]{Canonical URLs}

ZODB applications access objects by traversing the tree. It would be
nice if every object had its own unique name, so that even when the tree
structure changed, it would still be easy to find the object using its
canonical URL.

PythonLinks provides canonical URLs. Every node has a unique name which
can be accessed using a dictionary in the root node. Attempts to create
a new node with the same name, result in an integer being appended to
the new node name: Node, Node-1, Node-2, etc. When objects are renamed,
or deleted, the root dictionary is updated.

\subsection[zodb-history-and-undo]{ZODB History and Undo}

ZODB is a versioned database. Each object preserves its historical
versions until they are garbage collected. That makes it possible to
edit an object, save the edits, and then revert to the previous version
of the object. One can even do diffs of previous objects, in order to
decide which one is best.

\subsection[zodb-storage-options]{ZODB Storage Options}

There is a rich variety of ways to store your ZODB objects:

\startitemize[n,packed][stopper=.]
\item
  {\bf FileStorage} is the basic way of storing ZODB (Python) objects on
  the file system. There is also Blob storage for storing images, files
  and other large objects on the file system.
\item
  {\bf ZEO} (Zope Enterprise Objects) is the basic client-server version
  of the ZODB. ZEO clients share data from the ZEO server. Each
  participating Python application server includes a ZEO client. They
  all talk to the ZEO server. The ZEO server will then store the data in
  FileStorage and BlobStorage
\item
  {\bf RelStorage} stores ZODB data in a relational database. It
  performs better than ZEO. It runs on top of MySQL, Oracle and
  PostgreSQL.
\item
  {\bf NewtDB} is a version of RelStorage optimized for PostgreSQL. It
  stores data as both Python pickles, as well as JSONB PostgreSQL data
  structures. This allows one to benefit from the native PostgreSQL
  catalogs.
\stopitemize

\subsection[caching]{Caching}

Caching is an important feature of the ZODB. Every ZEO client includes
an object cache. The object cache is usually in RAM. There is an
optional file based cache. If objects are updated on one ZEO client,
they are invalidated on the other clients, and reloaded as needed.
Generally this works fine, but in cases where the object is actively
used on all ZEO clients, and frequently updated, it does lead to more
network traffic.

\subsection[scalability]{Scalability}

For mostly-read applications ZODB scales brilliantly. Multiple disk
heads can read simultaneously from different parts of the file used in
FileStorage. Historically its performance under heavy writing was
limited because writes all happened at the end of a single file. Only
one disk head at a time could write. That problem has now been fixed.
Writes can now occur simultaneously on multiple files on a single
computer. The database can grow even larger by hosting the blobs on
remote servers, such as Amazon S3.

At Zope Corporation, several hundred newspaper content-management
systems and web sites were hosted using a multi-database configuration
with most data in a main database and a catalog database. The databases
had several hundred gigabytes of ordinary database records plus multiple
terabytes of blob data.

Of course if you need multiple computers to store your object data (not
blobs), then you may want to upgrade to NEO (Not NEO4J). NEO is a GPL'd
cousin to the ZODB.

\subsection[multiple-views-on-every-object]{Multiple Views on Every
Object}

The modern way to use a ZODB data model on the web is to support views
on objects. There can be multiple views on any object. There can be the
same view on different objects. That same view can be one class, or else
using the Zope Component Architecture, there can be different views with
the same name on different classes of objects, depending on the
interfaces which they have defined.

\subsection[zodb-related-platforms]{ZODB-related Platforms}

So which platform should you use with ZODB? I am not an expert in all of
these platforms, but I will give you my best advice on when to use each
of them.

\section[plone]{Plone}

If you need a Content Management System, and Plone and its plugins do
what you need. Brilliant. Go for it. Plone is mature, stable, heavily
used. They have an annual international conference, and many regional
conferences. I recommend it highly. There are lots of consultants you
can hire.

But Plone is not an application development platform. If Plone does what
you need, you are in great luck. But I advise against making changes to
Plone. Small changes you may be able to make, but anything more major,
and you are more likely than not to run into trouble. There is just so
much code in there. Zope 2 code, Zope 3 code, and Five (a merger of Zope
2 and Zope 3). Then there is the Content Management Framework (CMF), and
all of Plone. Way way too complex to even consider doing custom app
development on it.

\section[pyramid]{Pyramid}

The next obvious choice is Pyramid {[}6{]}. If you are building a
corporate application, Pyramid is a great option. You can use the ZODB,
or a relational database. You can use routing or traversal. You can
configure it three different ways. You can choose from lots of different
GUI and other libraries.

\section[grok]{Grok}

Me, I chose to use Grok {[}7{]} with Zope 3. I am not building a
corporate application. I am building startup applications. I don't care
about computer performance, if my applications are overloaded, that is a
great problem to have. I can then optimize things. Above all else, I
care about speed and cost of development. Related to that is software
flexibility.

\section[zopache]{Zopache}

Sadly Grok lacks a TTW (Through the Web) development environment, so I
built one. Zopache.com is a JavaScript IDE which I built on top of the
ZODB using Grok. ZODB developers might call it a TTW development
platform.

Basic Zopache objects include HTML, CSS, JavaScript, Python Scripts,
Image, File and Jade. HTML objects can be edited with either the WYSIWYG
HTML editor, or the technical Ace Editor. CSS JavaScript and Python
Scripts can also be edited with the Ace Editor.

Zopache also supports containers. Trees can be built out of containers.
Folders support cut, copy, rename, paste and delete operations. HTML
folders can be edited with the Ace {[}8{]} and CKEditor {[}9{]}.
JavaScript Folders allow one to build a well organized tree of
JavaScript code, but serve it as a single minified file or as a human
readable and searchable file, which clickable links to the original
JavaScript object. History and Undo are supported.

Using Zopache, in your browser, you create a tree of folders, and
populate and edit those objects. Every object has its own URL, and can
be displayed or edited. There is no need to know Unix. But you do have
the option to grow the application with real Python code.

\section[flask]{Flask}

Flask {[}10{]} is a very lightweight framework which can be used with
the ZODB. Flask does routing, so one has to manually map from URLs to
objects.

\section[other-frameworks]{Other Frameworks}

There are many Python frameworks, and many other less often used ways of
working with the ZODB. I won't mention them all here. They lead to
market confusion. But it is only natural in the Python world for there
to be many different solutions. So one really needs a survey of the ZODB
ecosystem to get you started in the right direction.

\subsection[pyramid-vs-grok]{Pyramid vs Grok}

So let us compare them. Of course Grok can be used in different ways, so
I will focus on how I use it.

Pyramid is very focused on computer performance, how fast will the
software run? I am focused on developer performance, how fast can I
develop my applications. They want close to the metal code. I want high
levels of abstractions.

In particular the Pyramid folks are mosty strongly against TTW
development. I invite you to take a look at the quote on the SubstanceD
demo page {[}11{]}. Clearly they are very conflicted about minimal TTW
development, let alone the idea of building a full JavaScript IDE on top
of the ZODB.

The next issue is the target market. Pyramid is firmly focused on the
consulting market. Every client is different and needs a different
solution. They want the ability to do it this way or that way. Use a
relational database or the ZODB. Use dispatch or traversal. Configure it
three different ways. There are multiple ways to do things.

I am more focused on the product market. There is no legacy code here.
Just the web, which I have to work with. I want to build it as fast and
flexibly as is possible. I like the Python idea that there should be one
way to do things. I want a purist OO approach. Like Ruby on Rails, I
want an opinionated high level framework. Grok gives me that.

Pyramid is focused on multiple deployments. So TTW makes no sense. I am
focused on building a single website, so storing stuff in the ZODB is
much more modern than storing it on a 40 year old filesystem.

But let us do a closer technical comparison. Actually I am not even sure
what to compare. For a fair comparison, Pyramid does not stand on its
own. It needs a bunch of libraries to match what Grok does.

The first thing I want to do is automatic CRUD. Define the schema and
get CRUD for free. zope.formlib does this for me. z3c.form does this for
me for trees. Pyramid does not include CRUD. One has to add it manually.
I am still not sure which CRUD package to recommend. Reportedly the one
used in SubstanceD does not do CRUD. So which one does? The next one I
found has just been re-built for use with relational databases. Also not
what I need.

I also use components. Pyramid only supports view components. But the
Zope Component Architecture is now available with PyAMS. That was not
true when I started.

And of course there is no way that somethng like the high level
zope.securitypolicy has been ported to Pyramid. It is just incompatible
with the performance-optimized Pyramid security model.

\subsection[other-databases]{Other Databases}

There are many other NoSQL databases on the market. My job is not to
evaluate them all, but just to get you to ask the right questions:

\startitemize[n,packed][stopper=.]
\item
  Are they ACID compliant?
\item
  Do they provide a graph database?
\item
  Do they support Persistent Python?
\item
  Can you add arbitrary instance variables at run-time?
\item
  Do they support history and undo?
\stopitemize

\subsection[about-pythonlinks.info]{About PythonLinks.info}

PythonLinks.info is the world's largest Python Directory. One branch
includes a geographic directory of companies using Python. For each
region, for each country for each city, Python Links lists the companies
in that city, and provides important links for that company. Their name
is hyperlinked to their web site. Their products and jos are listed. It
is also possible to add YouTube videos, blogs, product reviews, and
reviews of what it is like to work at the company.

This directory is all in JavaScript, and JSON, so it is quite easy to
add teh table of companies in your city to your website. Let me know if
you are interested in doing so.

\subsection[references]{References}

\startitemize[n,packed][stopper=.,width=2.0em]
\item
  https://pythonlinks.info/presentations/zodbtalk.pdf
\item
  https://PythonLinks.info/zodb
\item
  ZODB. http://www.zodb.org
\item
  ZODB Tutorial. http://www.zodb.org/en/latest/tutorial.html
\item
  ACID. https://en.wikipedia.org/wiki/ACID
\item
  Pyramid. https://trypyramid.com
\item
  Grok. http://grok.zope.org
\item
  Ace editor. https://ace.c9.io
\item
  CKEditor. https://ckeditor.com
\item
  Flask. http://flask.pocoo.org
\item
  SubstanceD. http://demo.substanced.net/
\stopitemize


\stoptext
