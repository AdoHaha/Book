\usemodule[pycon-yyyy]
\starttext

\Title{Micropython w domowej automatyce}
\Author{Szymon Pyżalski, Nikoloz Glonti}
\MakeTitlePage

\section[korzystanie-z-gpio-mikrokontrolerów]{Korzystanie z GPIO mikrokontrolerów}

Dla osoby mającej doświadczenie w Pythonie rozpoczęcie pracy ze
środowiskiem MicroPython zwykle nie nastręcza problemów od strony
programistycznej. Z drugiej strony wymaga ona wiedzy elektronicznej w
celu właściwego bezpiecznego i skutecznego korzystania z GPIO. W tym
artykule omówimy podstawowe techniki związane z GPIO starając się
jednocześnie odświeżyć podstawowe prawa fizyki dotyczące elektroniki.

\subsection[gpio-jako-wyjście]{GPIO jako wyjście}

\section[bezpośrednie-zasilanie-z-gpio]{Bezpośrednie zasilanie z
GPIO}

W przypadku elementów działających z małą mocą i wymagających
niewielkich napięć, takich jak pojedyncze diody LED możliwe jest
zasilanie bezpośrednio z pinu z ewentualnym zastosowaniem rezystora.
Przykładowa sytuacja - próbujemy zasilić niskoprądową diodę LED przy
użyciu chipa ESP8266.

Podłączenie przedstawia ryc 1

\placefigure{ryc. 1 Dioda z rezystorem
ograniczającym}{\externalfigure[ryc1.svg]}

Z dokumentacji technicznej odczytujemy następujące parametry:

\startitemize[packed]
\item
  Spadek napięcia diody 2,5 V
\item
  Prąd diody: 2 mA
\item
  Napięcie GPIO: 3,3 V
\item
  Maksymalny prąd dla GPIO chipa: 12 mA
\stopitemize

Ponieważ napięcie i maksymalny prąd jakiego jest w stanie dostarczyć
GPIO są większe od wymaganych przez diodę, możemy zasilić ją
bezpośrednio. Aby dobrać wielkość rezystora przypomnijmy sobie II prawo
Kirchhoffa i prawo Ohma. Pierwsze z nich mówi nam, że spadek napięcia na
rezystorze musi wynieść tyle ile wynosi różnica między napięciem na GPIO
a spadkiem napięcia na diodzie. Prawo Ohma określa zaś, że wartość
opornika jest stosunkiem spadku napięcia i natężenia prądu. Otrzymujemy
wzór:

\starttyping
R = \frac{V_S - V_{LED}}{I}
\stoptyping

Podstawiając wartości: 3,3V, 2,5V i 2mA otrzymujemy żądaną oporność
400Ω. Możemy zatem użyć bliskiego tej wartości opornika 470Ω dostępnego
w większości zestawów.

Ostatnią wartością, którą warto policzyć jest jeszcze moc rezystora,
którą uzyskujemy ze znanego wzoru:

\starttyping
P = U \cdot I
\stoptyping

Moc naszego rezystora w tym obwodzie wyniesie więc 1,6 mW. Jest to dużo
mniej niż maksymalna moc standardowych rezystorów wynosząca 250 mW,
zatem moc rezystora nie będzie stanowiła problemu.

\subsection[uwaga]{Uwaga:}

Prawo Ohma znajduje zastosowanie wyłącznie dla elementów takich jak
rezystory. Elementy półprzewodnikowe cechują się nieliniową zależnością
między napięciem a natężeniem prądu. Błędem byłoby stwierdzenie, że
dioda z powyższego przykładu ma 1,2 kΩ i oczekiwanie, że dla innych
napięć dioda zachowa ten stosunek U / I. Dioda pracować może tylko w
wąskim zakresie napięcia. Poniżej niego w ogóle nie włączy się,
natomiast powyżej jej oporność gwałtownie spadnie, co szybko doprowadzi
do przepalenia.

\section[tranzystor-jako-przełącznik]{Tranzystor jako przełącznik}

Wachlarz zastosowań tranzystorów w elektronice analogowej i cyfrowej
jest bardzo szeroki. My jednak skupimy się na jednym - wykorzystaniu
tranzystora do kontroli obwodu o wysokim natężeniu prądu przy pomocy
znacznie niższego. W tym celu zastosujemy konfigurację
\quotation{wspólnego emitera}:

\placefigure{ryc. 2 Tranzystor NPN jako przełącznik
diody}{\externalfigure[ryc2.svg]}

W układzie tym w momencie, gdy pin znajduje się w stanie niskim,
tranzystor znajduje się w stanie zatkania - brak przewodnictwa między
kolektorem a emiterem. Gdy pin przejdzie w stan wysoki - tranzystor
osiąga stan nasycenia - prąd między kolektorem i emiterem przewodzony
jest z minimalnym oporem. Spróbujmy przy jego pomocy zasilić
\quotation{żarówkę} LED na 12V o mocy 1,5 W.

Parametry:

\startitemize[packed]
\item
  Wzmocnienie tranzystora: β = 200
\item
  Maksymalny prąd kolektor-emiter: 1A
\item
  Napięcie stanu nasyconego baza-emiter: 0.7 V
\stopitemize

Mając daną moc i napięcie źródła światła możemy wyliczyć płynący przez
nie prąd jako 125 mA. Jest to również prąd kolektor-emiter. Wymagany
prąd baza-emiter wyniesie zatem 125 mA / 200 = 0,63 mA. Stąd też przy
zastosowaniu wzoru powyżej otrzymujemy oporność 4,73 kΩ. Użyjemy zatem
na bazie opornik o wartości 4,7 kΩ, który powinien nam zagwarantować
wymagany prąd baza-emiter jeśli pin GPIO przejdzie w stan wysoki 3,3V.

\section[tranzystor-a-przekaźnik]{Tranzystor a przekaźnik}

Mimo że tranzystor i przekaźnik mogą spełniać podobną rolę w obwodzie,
istnieje między nimi duża różnica. Przekaźnik przełącza styki
mechanicznie przy pomocy elektromagnesu. Tranzystor nie posiada
elementów mechanicznych. Jego działanie opiera się na technologii
półprzewodnikowej. Aby wybrać, która z tych opcji jest dla nas właściwa,
należy rozważyć wady i zalety obu rozwiązań.

\section[zalety-przekaźnika]{Zalety przekaźnika}

\startitemize[packed]
\item
  Może przewodzić zwykle prąd o znacznie wyższym natężeniu niż
  tranzystor.
\item
  Może przewodzić zarówno prąd zmienny, jak i stały.
\item
  Sposób zasilania cewki jest niezależny od tego, jaki prąd przez niego
  płynie, zatem obwód jest bardziej uniwersalny.
\stopitemize

\section[zalety-tranzystora]{Zalety tranzystora}

\startitemize[packed]
\item
  Ma o kilka rzędów wielkości mniejsze niż przekaźnik opóźnienie
  pomiędzy kontrolującym go sygnałem, a odpowiedzią.
\item
  Ze względu na brak elementów mechanicznych, nie ulega zmęczeniu przy
  przełączaniu.
\item
  Wymaga znacznie mniejszego prądu na bazie, niż przekaźnik do
  uruchomienia cewki.
\item
  Jego praca jest bezgłośna.
\item
  Nie posiadając elementów indukcyjnych nie wywołuje problemu
  samoindukcji w momencie wyłączenia (przekaźnik może wymagać
  zastosowania diody ochronnej, która pochłonie przepięcie).
\stopitemize

Tak więc w sytuacji, gdy mamy do czynienia z prądem zmiennym, lub o
dużym natężeniu, wybierzemy przekaźnik. Podobnie, gdy nie znamy
parametrów odbiornika energii, który będzie kontrolowany. W przypadku
sygnału wypełniającego (PWM) musimy zastosować tranzystor ze względu na
małe opóźnienie i zdolność do wytrzymania licznych przełączeń. W wielu
innych sytuacjach wybór pozostaje kwestią priorytetów.

\subsection[gpio-jako-wejście]{GPIO jako wejście}

\section[rezystory-podciągające]{Rezystory podciągające}

Piny niepodłączone mogą negatywnie wpływać na pracę mikrokontrolera.
Pełnią one role swego rodzaju anteny -- zbierają zakłócenia cyfrowe(na
przykład silne szpilki sygnałowe). Taki stan pinu określamy w języku
angielskim jako {\em floating}. Rezystory podciągające służą do
uniknięcia zakłóceń cyfrowych, które mogą prowadzić do nieprawidłowej
pracy układu. {\externalfigure[ryc3.svg]} Na rycinie widzimy przykład
pinu podciągniętego do stanu wysokiego przez rezystor. Jeśli zewrzemy
przełącznik, wysoka wartość rezystora sprawia, że linia 3,3 V nie jest
już w stanie ustalić stanu pinu na wysoki, gdyż jego potencjał ustalany
jest przez bazę. Stąd pin osiągnie stan niski. Wartości rezystorów
zależą od częstotliwości zmian stanów na linii. Na przykład dla linii
I2C należy stosować 4,7kOm, dla przycisków 10kOm

\section[bouncing-sygnału]{Bouncing sygnału}

Zbudowaliśmy już układ według powyższego schematu i postanowiliśmy go
przetestować. Napiszmy zatem prosty kod:

\starttyping
import machine

cur_value = 0
toggle_count = 0
pin = machine.Pin(5, machine.Pin.IN)

while True:
    value = pin.value()
    if value != cur_value:
        print(toggle_count)
        cur_value = value
        toggle count += 1
\stoptyping

Ten prosty program ma nam zliczać kolejne zmiany stanu pina 16.
Teoretycznie licznik powinien wzrosnąć o 1 przy każdym wciśnięciu i
zwolnieniu przycisku. Czasem jednak obserwujemy, że wciśnięcie powoduje
kilka zmian stanu przycisku. Zjawisko to zwane po angielsku
{\em bouncing} jest czymś normalnym na mechanicznych stykach. Możemy
wobec niego zastosować dwa podejścia.

\subsection[podejście-hardwareowe]{Podejście hardware'owe}

W tym podejściu stosujemy odpowiednią modyfikację obwodu. Najprostszym
rozwiązaniem jest zastosowanie tzw. układu RC składającego się z
rezystora i kondensatora.

\placefigure{ryc. 4 Przycisk z układem RC}{\externalfigure[ryc4.svg]}

Obwód ten różni się od poprzedniego obecnością dodatkowego rezystora i
kondensatora. Po zamknięciu przełącznika przejście pinu w stan wysoki
zostanie opóźnione do czasu naładowania się kondensatora. Wartości
rezystora i kondensatora należy dobrać doświadczalnie tak, aby
wyeliminować {\em bouncing} jednak by wprowadzone opóźnienie nie było
zbyt wysokie.

\subsection[podejście-softwareowe]{Podejście software'owe}

Zjawisko {\em bouncingu} możemy obsłużyć również software'owo. W tym
celu wprowadzamy stosowne opóźnienie w kodzie:

\starttyping
    import time
    # ...
        value = pin.value()
        if value != cur_value:
            time.sleep_ms(300)
    # ...
\stoptyping

\subsection[materiały-internetowe]{Materiały internetowe}

\startitemize[packed]
\item
  http://www.electronics-tutorials.ws/ - wyczerpujące samouczki różnych
  zagadnień elektronicznych.
\item
  https://learn.sparkfun.com/tutorials/ - samouczki nastawione na pracę
  z mikrokontrolerami
\item
  http://micropython-on-wemos-d1-mini.readthedocs.io/en/latest/ -
  materiały ze szkolenia micropythonowego Radomira Dopieralskiego
\item
  https://docs.micropython.org/en/latest/esp8266/esp8266/tutorial/intro.html
\item
  oficjalna dokumentacja micropythona na ESP 8266
\stopitemize


\stoptext
