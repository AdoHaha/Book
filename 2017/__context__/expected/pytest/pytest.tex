\usemodule[pycon-yyyy]
\starttext

\Title{Parametrize all the tests! (workshop)}
\Author{Miro Hrončok}
\MakeTitlePage

In this workshop you will learn about {\bf pytest} {[}1{]} and why it is
better than unittest or nose. You will see and try out examples on easy
and later on more complex scenarios. You will learn how to write
parametric tests and you will parametrize some test for a common problem
such as fizzbuzz. Later we'll dive in into fixtures and how they are
different than setup and teardown. Fixtures can be parametric as well,
so we'll code some examples for that as well. At the end, you'll learn
about cross products of parameters and fixtures.

This text can serve as a brief introduction to the topic, but some of
the parts are intentionally missing and will be presented on the
workshop.

\subsection[what-is-pytest]{What is pytest}

Pytest is a framework for writing tests for your Python code. When you
look at testing in Python, you'll probably learn about {\bf unittest}
{[}2{]}, because it is a part of Python's standard library. However,
unittest is not very flexible and its syntax and usage is very much
inspired with JUnit -- a Java unit testing framework. I guess writing
code in Python using a Java inspired API is something no Pythonista
would desire. :punch:

The lack of flexibility when running tests written in unittest inspired
another project to be created, a once popular tool called {\bf nose}
{[}3{]}. While nose can make running tests easier and more pleasant, the
main part -- how do we write our tests -- remained the same. Not to
mention nose has been in maintenance mode for the past several years.
:vhs:

{\bf Pytest} on the other hand reinvents the way test are being written.
Instead of Java-like nonsense, you write very readable, Pythonic code.
:snake:

Compare this:

\starttyping
import unittest

class TestStringMethods(unittest.TestCase):

    def test_upper(self):
        self.assertEqual('foo'.upper(), 'FOO')

    def test_isupper(self):
        self.assertTrue('FOO'.isupper())
        self.assertFalse('Foo'.isupper())

if __name__ == '__main__':
    unittest.main()
\stoptyping

With this:

\starttyping
def test_upper():
    assert 'foo'.upper() == 'FOO'

def test_isupper():
    assert 'FOO'.isupper()
    assert not 'Foo'.isupper()
\stoptyping

Pytest is using a simple assert statement to verify {\em things} in
tests. Unlike using assert {[}4{]} in regular Python code, it is
monkeypatched to provide more information about what happened when the
assertion fails:

\starttyping
def test_upper():
    assert 'foo'.upper() == 'FOo'
\stoptyping

\starttyping
================================== FAILURES ===================================
_________________________________ test_upper __________________________________

    def test_upper():
>       assert 'foo'.upper() == 'FOo'
E       AssertionError: assert 'FOO' == 'FOo'
E         - FOO
E         + FOo

test_example.py:2: AssertionError
========================== 1 failed in 0.02 seconds ===========================
\stoptyping

\subsection[lets-write-some-tests-for-fizzbuzz]{Let's write some tests
for fizzbuzz}

Fizzbuzz {[}5{]} is a group word game for children to teach them about
division. Players take turns to count incrementally, replacing any
number divisible by three with the word \quotation{fizz}, and any number
divisible by five with the word \quotation{buzz}. If the number is
dividable by both, one shall say \quotation{fizzbuzz}.

Let's first write a function that takes an integer and returns a string
with \quotation{fizz}, \quotation{buzz}, \quotation{fizzbuzz} or the
number depending on the rules of the game.

\starttyping
# fizzbuzz.py
def fizzbuzz(number):
   ...
\stoptyping

(The body of the above function is intentionally left out as an exercise
for the reader.)

Let's create a \type{test_fizzbuzz.py} file in the \type{tests}
directory. When running \type{python -m pytest}, the tests in this
directory will be automatically collected if the name of the files
starts with \type{test_}.

\starttyping
# tests/test_fizzbuzz.py
from fizzbuzz import fizzbuzz


def test_regular():
    assert fizzbuz(1) == '1'


def test_fizz():
    assert fizzbuz(3) == 'fizz'
\stoptyping

Now run the tests with \type{python -m pytest}. (Note that if you have
an virtual environment set up in the very same directory, pytest will
likely try to collect the tests from within that. You can either use the
\type{--ignore} switch to ignore your virtual environment directory or
specify the path to your tests directory as a first positional
argument.)

\starttyping
$ python -m pytest --ignore=env
$ python -m pytest tests/
\stoptyping

Writing tests that check that \type{fizzbuzz(5)} is \type{'buzz'} and
\type{fizzbuzz(15)} is \type{'fizzbuzz'} is left as and exercise for the
reader.

At the end of this part, you should have a fizzbuzz implementation and
tests for 1, 3, 5 and 15.

\subsection[parametrize-our-tests-and-why-is-it-better]{Parametrize our
tests and why is it better}

In reality, we would need to test for other numbers as well, such as 2,
6, 333 etc. Without parametric tests, we have two options:

Copy pasting tests around, renaming them and changing the tested value.

\starttyping
def test_fizz3():
    assert fizzbuz(3) == 'fizz'


def test_fizz6():
    assert fizzbuz(6) == 'fizz'


def test_fizz9():
    assert fizzbuz(9) == 'fizz'
\stoptyping

Uh\ldots{} copy pasting code around like that is wrong. :thumbsdown:
:poop:

\thinrule

Or running one test over a range of numbers:

\starttyping
def test_fizz():
    for number in (3, 6, 9, 333, 3003):
        assert fizzbuz(number) == 'fizz'
\stoptyping

This does not look so bad on the first glance, but let us intentionally
break the implementation for large numbers and see how the tests result
looks like.

\starttyping
def fizzbuzz(number):
    if number > 100:
        return 'Oh no, a bug!'
    ...
\stoptyping

From the tests results, do you see what's wrong? :confused:

\thinrule

Instead, we'll parametrize the test. That means, we'll have one test
written -- as a template -- and it will be run for multiple values.

\starttyping
import pytest


@pytest.mark.parametrize('n', (3, 6, 9, 333, 3003))
def test_fizz(n):
    assert fizzbuz(n) == 'fizz'
\stoptyping

See what happens when you run the tests and try running them with the
verbose (\type{-v}) switch.

Let's write parametric tests for all our cases. Note that in our
example, we've used a tuple to represent possible values of our
parameter. But it can be any (terminal) iterable, such as list, set,
range or a custom generator. Try to write some parametric test for
fizzbuzz using a range or a (list) comprehension {[}6{]}.

\subsection[lets-do-something-more-complicated]{Let's do something more
complicated}

Imagine we'd have a function that accepts boundaries and returns a
generator of fizzbuzz results, something like this:

\starttyping
def fizzbuzz_range(*args, **kwargs):
    for n in range(*args, **kwargs):
        yield fizzbuzz(n)
\stoptyping

Now let us write a parametric tests that takes 3 parameters: start, stop
and result:

\starttyping
@pytest.mark.parametrize(['start', 'stop', 'result'],
                         [(0, 3, ['0', '1', '2']),
                          (3, 7, ['fizz', '4', 'buzz', '6']),
                          ...])
def test_fizz(start, stop, result):
    assert list(fizzbuz_range(start, stop)) == result
\stoptyping

Notice how \type{@pytest.mark.parametrize} can parametrize multiple
values.

\subsection[fixtures-and-how-they-work]{Fixtures and how they work}

Sometimes, you want to tests several facts about one thing. Especially
if that thing is hard to create, you might want to prepare it beforehand
and use it repetitively in multiple tests. In that case, you might want
to create a {\em fixture} -- something that's needed for multiple tests
to operate.

Fixture can be created a s a simple function with a decorator:

\starttyping
@pytest.fixture
def game():
    '''A fizzbuzz sequence as in the children's game'''
    return list(fizzbuzz_range(1, 101))
\stoptyping

Using that fixture from tests boils down to a function parameter with
the same name:

\starttyping
def test_game_has_100_items(game):
    assert len(game) == 100


def test_game_startswith_1(game):
    assert game[0] == '1'


def test_game_endswith_buzz(game):
    assert game[-1] == 'buzz'

...
\stoptyping

But fixtures are far more powerful than that. They allow you to create a
context for your tests using a \type{yield} statement. Here is a
hypothetical example where each test using this fixture would connect to
a database and close the connection afterwards:

\starttyping
@pytest.fixture
def db():
    database = ExampleDatabase(user='me', password='test')
    connection = database.connect()
    yield connection
    connection.close()
\stoptyping

Bare in mind that if a test fails, the last line of this fixture would
never get executed. To make sure the connection get's closed even on
failed test, you would use a classic \type{try-finally} block:

\starttyping
@pytest.fixture
def db():
    database = ExampleDatabase(user='me', password='test')
    connection = database.connect()
    try:
        yield connection
    finally:
        connection.close()
\stoptyping

Or, if the database API has a context manager, we could use that as
well:

\starttyping
@pytest.fixture
def db():
    database = ExampleDatabase(user='me', password='test')
    with database.connect() as connection:
        yield connection
\stoptyping

Sometimes, you might want to share a database connection across all the
tests, for that, you might explicitly set fixture's scope. Without
specifying a scope, one fixture is recreated every time a test uses it.
Instead, we'll set the {\em module} scope and this time one instance of
a fixture is used for all the tests that require it:

\starttyping
@pytest.fixture(scope='module')
...
\stoptyping

One fixture can use another fixture, so imagine a scenario, where we
want to reuse a connection for all the tests, but we want to set up
database content before every test and clean it afterwards. We will
create 2 fixtures -- one scoped for the module, one with the default
scope:

\starttyping
@pytest.fixture(scope='module')
def connection():
    database = ExampleDatabase(user='me', password='test')
    with database.connect() as connection:
        yield connection

@pytest.fixture
def db(connection):
    connection.query(...)  # fill in the data
    try:
        yield connection
    finally:
        connection.query(...)  # delete everything
\stoptyping

Note that using another fixture from a fixture can be done by using it's
name as a name of a parameter. You can use a module scoped fixture from
a default scoped fixture, but not the other way around (for obvious
reasons).

\subsection[already-available-fixtures-in-pytest]{Already available
fixtures in pytest}

Not only you can create your own fixtures, but you can benefit form some
fixtures available in pytest by default. You can see the list of them
with \type{python -m pytest -q --fixtures}.

They include stuff like a temporary directory or more magical things
like \type{capsys} that allows you to introspect what would normally be
printed on standard output and stderr.

Pytest normally hides every output if the test passes and displays it if
the test fails. This is useful for debugging failed tests. You can leave
you prints in and it doesn't harm. But sometimes you actually want to
check would would have been printed. The \type{capsys} fixture helps
here:

\starttyping
def fizzbuzz_game():
    for n in fizzbuzz_range(1, 101):
        print(n)
\stoptyping

\starttyping
def test_fizzbuzz_game_prints_100_lines(capsys):
    fizzbuzz_game()
    out, _ = capsys.readouterr()
    assert len(out.strip().split('\n')) == 100
\stoptyping

\thinrule

You can also use the builtin fixtures to create your own. Here we create
a git repository and let the test execute within:

\starttyping
@pytest.fixture
def gitrepo(tmpdir):
    with tmpdir.as_cwd():
        subprocess.run(['git', 'init'])
        yield tmpdir
\stoptyping

\subsection[parametrizing-fixtures]{Parametrizing fixtures}

As well as tests, fixtures can also be parametric {[}7{]}. This is very
helpful if you need to run your tests with multiple backends or if you
use one parameter repetitively across multiple tests. Let's get back to
our hypothetical database example and make it parametric:

\starttyping
@pytest.fixture(scope='module', params=['postgres', 'sqlite'])
def connection(request):
    if request.param == 'postgres':
        database = PostgresDB(...)
    else:
        database = SQLiteDB(...)
    with database.connect() as connection:
        yield connection
\stoptyping

Now every test that uses the {\em connection} fixture (even
transitively) will run twice, once with \type{PostgresDB()} and once
with \type{SQLiteDB()}.

Note that parametric fixtures have a little different syntax and need to
accept the special \type{request} function parameter that holds the
entire context about the test being run.

\thinrule

In our fizzbuzz example, we might want to test multiple facts about
fizzbuzz calls for 3, 6, 9, 333 etc. Instead of repeating the parameters
every time or creating a global variable with list of numbers, we can
crate a parametric fixture:

\starttyping
@pytest.fixture(scope='module', params=[3, 6, 9, 333])
def fizznum(request):
    return request.param


def test_fizznum_is_not_devidable_by_5(fizznum):
    '''A "metatest" that tests the fizznum fixture itself''''
    assert fizznum % 5 != 0


def test_fizz(fizznum):
    assert fizzbuzz(fizznum) == 'fizz'
\stoptyping

Notice how we created the fixture module scoped? That's because reusing
an integer cannot harm (it's immutable anyway).

\subsection[cross-products-of-parameters-andor-parametric-fixtures]{Cross
products of parameters and/or parametric fixtures}

All the parameters can be combined together to create a cross product of
test templates. Either you can combine two or more
\type{@pytest.mark.parametrize} decorators, more fixtures, or both
together.

Note that the bellow example is not very well written and I don't
encourage to write code like that, but it serves a purpose. :trollface:

\starttyping
@pytest.fixture(scope='module', params=[3, 6, 9, 333])
def fizznum(request):
    return request.param


@pytest.fixture(scope='module', params=[5, 10, 500])
def buzznum(request):
    return request.param


def test_fizzbuzz(fizznum, buzznum):
    assert fizzbuzz(fizznum * buzznum) == 'fizzbuzz'
\stoptyping

You can even create a fixture that creates a cross product of fixtures:

\starttyping
@pytest.fixture(scope='module')
def fizzbuzznum(fizznum, buzznum):
    return fizznum * buzznum


def test_fizzbuzz(fizzbuzznum):
    assert fizzbuzz(fizzbuzznum) == 'fizzbuzz'
\stoptyping

Don't overdo that. If you want to test for all possible values without
{\em actually testing all possible values}, you might want to look at
{\bf hypothesis} {[}8{]}.

\subsection[where-to-go-next]{Where to go next}

I highly recommend reading trough the pytest documentation {[}1{]} as it
is very understandable and has plenty of examples. It has multiple
sections covering parametric tests and fixtures. Dive in!

\subsection[references]{References}

\startitemize[n,packed][stopper=.]
\item
  pytest documentation. https://docs.pytest.org/
\item
  unittest in Python documentation.
  https://docs.python.org/3/library/unittest.html
\item
  nose documentation. http://nose.readthedocs.io/
\item
  assert on Python Wiki.
  https://wiki.python.org/moin/UsingAssertionsEffectively
\item
  Fizz buzz on Wikipedia. https://en.wikipedia.org/wiki/Fizz\type{_}buzz
\item
  List comprehensions.
  https://docs.python.org/3/tutorial/datastructures.html\#list-comprehensions
\item
  Parametrizing fixtures.
  https://docs.pytest.org/en/latest/fixture.html\#parametrizing-fixtures
\item
  Hypothesis documentation. https://hypothesis.readthedocs.io/
\stopitemize


\stoptext
