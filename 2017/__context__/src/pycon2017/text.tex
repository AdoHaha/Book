\usemodule[pycon-yyyy]
\starttext
\Title{{\tt \char`\_\char`\_init\char`\_\char`\_}}
\HeaderTitle{}
\gobbleoneargument \MakeTitlePage
\CheckPingPages


% Strona pierwsza, czyli okładka znaczy

\null
\noheaderandfooterlines
\vskip 0.2\vsize
\centerline{%
  \tfd PyCon PL 2017
  %\externalfigure[ping-ascii.100][width=0.7\hsize]
}


\vskip 0pt plus 1filll

\centerline{Polska Grupa Użytkowników Linuxa}
\break
\noheaderandfooterlines

\def \ListaPlac#1:{\blank[2*big]
     \begingroup
     \let \PAR=\par
     \def \par{\PAR\endgroup}
     \catcode`\^^M=13
     \noindent \begingroup \sc #1:%
     %\lccode `\~=`\^^M%
     \lowercase{\endgroup\def~{\hfil\break\hbox
     to2em{}\ignorespaces
}}
}%


\ListaPlac Komitet Organizacyjny:
 Filip Kłębczyk (przewodniczący)
 Piotr Wójcik
 Anna Wszeborowska
 Grzegorz Neblik
 Daria Ratyńska
 Ewa Jodłowska
 Kinga Kięczkowska
 Maciej Szulik
 Radomir Dopieralski
 Jakub Wasielak
 Dawid Osuchowski
 Michał Rokita
 Piotr Wasilewski

\ListaPlac Komitet Programowy:
 Piotr Wójcik (przewodniczący)
 Maciej Szulik
 Radomir Dopieralski
 Piotr Kasprzyk
 Jakub Wiśniowski
 Marcin Jabrzyk
 Grzegorz Szczepańczyk
 Jarosław Szymla
 Mateusz Flieger
 Paweł Stoworowicz
 Katarzyna Jachim
 Jacek Rzeszutek
 Wojciech Lichota
 Piotr Roksela
 Christian Barra (PyData)

% \ListaPlac Redakcja:
%  Sebastian Pawluś
%  Jakub Janoszek
%  Piotr Kasprzyk
%  % Ja nie robię, ja tylko makra --K.L.
%  % Krzysztof Leszczyński
%  Filip Kłębczyk
%  Maciej Wiśniowski
%  Piotr Wójcik

\vfill \break

\ListaPlac Główny partner:~\par%\blank[2*big]
\centerline{\externalfigure[src/PSF.png][width=0.45\hsize]}

\ListaPlac Złoci Sponsorzy:~\par%\blank[2*big]
\centerline{}
\centerline{\externalfigure[src/dreamlab.png][width=0.45\hsize]\hskip10mm\externalfigure[src/samsung.png][width=0.45\hsize]}
\vskip10mm
\centerline{\externalfigure[src/jpmorgan.png][width=0.45\hsize]}

\ListaPlac Srebrni Sponsorzy:\par
\vskip12mm
\centerline{\externalfigure[src/codilime.png][width=0.40\hsize]\hskip12mm\externalfigure[src/stx.png][width=0.40\hsize]}
\vskip12mm
\centerline{\externalfigure[src/compendium.png][width=0.40\hsize]}

\vfil\break
\ListaPlac Brązowi Sponsorzy:\par
\vskip2mm
\centerline{\externalfigure[src/xcaliber.png][width=0.40\hsize]\hskip12mm\externalfigure[src/RockwellAutomation.png][width=0.40\hsize]}
\vskip12mm
\centerline{\externalfigure[src/CodersLab.png][width=0.40\hsize]\hskip20mm\externalfigure[src/gp.png][width=0.15\hsize]}

\ListaPlac Grafitowi Sponsorzy:\par
\vskip12mm
\centerline{\externalfigure[src/Helion.png][width=0.40\hsize]\hskip12mm\externalfigure[src/itpwn.png][width=0.40\hsize]}
\centerline{\externalfigure[src/logo_JetBrains_1.png][width=0.40\hsize]\hskip20mm\externalfigure[src/ORM_logo_cmyk.png][width=0.40\hsize]}

\ListaPlac Partner techniczny:\par
\vskip12mm
\centerline{\externalfigure[src/litex.png][width=0.40\hsize]}
\break

\iffalse
  % wersja prezesa

Witajcie. Znowu.


\else

\begingroup
\setupwhitespace[big]
\setupindenting[none,never]

Witamy na dziesiątej edycji konferencji PyCon PL!

Nasze wydarzenie już po raz trzeci gości w Hotelu Ossa położonym w środkowej
Polsce. Dla nas - organizatorów, kolejne pojawienie się w miejscu, które jest
nam znane z poprzednich edycji, to spore ułatwienie w przygotowaniu polskiego
PyCona. Liczymy więc, że uda nam się uniknąć paru organizacyjnych wpadek
z poprzednich lat, ale znając życie, pewnie wygenerujemy jakieś nowe. Czy tak
w rzeczywistości będzie, to już przekonacie się sami. Przepraszamy też
za wszelkie niedogodności związane z opracowywaniem nowego systemu
rejestracji, który jest jeszcze w powijakach, ale już w przyszłym roku
z pewnością pokaże swoją siłę. Po roku przerwy wraca za to zupełnie nowa
konferencyjna aplikacja mobilna i jesteśmy bardzo ciekawi, jak Wam się spodoba.

Przy okazji PyCona PL 2017 odbywają się też kolejne warsztaty PyLadies,
tym razem jednak nieco inne, bo parodniowe i dla osób zupełnie początkujących.
Wierzymy, że różnorodność społeczności Pythona, to jedno z jej kół zamachowych
i należy ją aktywnie wspierać. Wydaje nam się, że na naszej konferencji
udaje się to akurat dosyć dobrze, czego chyba wyrazistym dowodem jest też to,
że przy okazji tej edycji będziemy gościć po raz pierwszy uczestników z Litwy,
Łotwy i Turcji, ale prawdopodobnie także z dalekiej Korei Południowej.

W ostatnim czasie język Python staje się coraz bardziej popularny w edukacji,
zarówno na świecie, jak i w naszym kraju. Mnoży się liczba inicjatyw
edukacyjnych, przybywa ciekawych projektów, takich jak BBC micro:bit i szkoleń,
czasem w trybie nawet bardzo intensywnym. Wszystkiemu temu sprzyja też ciągły
wzrost zapotrzebowania na programistów Pythona. Najlepszym jednak dowodem na
dziejące się na naszym krajowym podwórku zmiany jest zapowiedziana możliwość
wybrania Pythona na maturze w roku szkolnym 2018/2019 z jednoczesnym
wycofywaniem legendarnego już w zastosowaniach dydaktycznych Pascala.

W Polsce przyszły rok będzie za to szczególnie bogaty w konferencje związane
z~naszym językiem. Co chyba oczywiste, następnej jesieni czeka nas jubileuszowa
10 edycja PyCona PL i chcemy, by była równie niezapomniana jak pierwsza, która
odbyła się w~2008 roku. Gorąco więc zachęcamy do jej wsparcia organizacyjnego,
ale również finansowego, szczególnie przez firmy, które czerpią pełnymi
garściami z dobrodziejstw Pythona. To jednak nie wszystko - na wiosnę 2017
czeka nas coś zupełnie nowego - konferencja PyData Poland 2017. Będzie to
duże, parodniowe wydarzenie poświęcone intensywnie rozwijającej się tematyce
uczenia maszynowego i data science, będzie to też pierwsza edycja konferencji
z rodziny PyData w Europie Środkowo-Wschodniej. Tego po prostu nie można przegapić!

Na koniec tego wstępniaka, już tradycyjnie pozostaje sobie życzyć dobrze
spędzonego czasu, udanej integracji i sporo wyniesionej wiedzy na PyCon PL 2017.

\endgroup

\fi

\vfil\break
\begingroup

\setupwhitespace[big]

\null\vfill
\noindent{\sc Kolofon:}

\blank[big]\parindent=0pt \rightskip=0pt
plus 1fill

Wszystkie prace związane z~przygotowaniem publikacji do~druku
zostały wykonane wyłącznie w~systemie Linux (m.in. Ubuntu 16.04.1 LTS).

Skład został wykonany w systemie \TeX\ z~wykorzystaniem,
opracowanego na~potrzeby konferencji, środowiska redakcyjnego
opartego na~formacie \CONTEXT.

Teksty referatów złożono krojem Bonum z kolekcji \TeX-Gyre.
Tytuły referatów złożono krojem Antykwa Toruńska, opracowanym
przez Janusza Nowackiego na~podstawie rysunków Zygfryda
Gardzielewskiego.


\endgroup
\break


\iffalse
\def \defEvent{\dodoubleargumentwithset \dodefEvent}
\def \dodefEvent[#1][#2]{\getparameters[Event::#1::][#2]}

\def \vskipreserve #1#2{\vskip #1\nobreak
                        \vskip 0pt plus \dimexpr #2\relax\penalty-1000
                        \vskip 0pt plus -\dimexpr #2\relax}

\def \startDay#1{\vskipreserve{3\baselineskip}{0.3\vsize}
     \centerline{\tfa #1}
     \blank[big]
}

\def \AgendaEvent[#1]{\par
     \def \AgendaEventfrom{\errmessage{Undefined AgendaEvent from}}
     \def \AgendaEventto{\errmessage{Undefined AgendaEvent to}}
     \def \AgendaEventid{\errmessage{Undefined AgendaEvent id}}
     \getparameters[AgendaEvent][#1]%
     \begingroup\edef \a{\endgroup
       \noexpand\doAgendaEvent
       \AgendaEventid \space \AgendaEventfrom \space \AgendaEventto\relax}\a
     }

\def \numifty#1{\ifnum 99=0#1\relax\infty \else #1\fi}

\setvalue{start::ref}{}
\setvalue{stop::ref}{}
\setvalue{start::event}{\qquad --- \vrule width 0pt height 5ex depth 0ex\relax}
\setvalue{stop::event}{ ---\vrule width 0pt height 0ex depth 3ex\relax}

\def \doAgendaEvent #1 #2:#3 #4:#5\relax{
     \hbox to \hsize \bgroup
     \hbox to 6em{\hfil
       $\numifty{#2}^{\numifty{#3}}$~--~%
       $\numifty{#4}^{\numifty{#5}}$\quad}%
     \vtop \bgroup
       \advance \hsize -9em
       \noindent \rightskip=0pt plus 1fil
       \vrule width 0pt height 3ex
       \if $\getvalue{Event::#1::author}$\else
          {\sc \getvalue{Event::#1::author}}\hfil\break
       \fi
       \getvalue{start::\getvalue{Event::#1::type}}
      {\it \getvalue{Event::#1::title}}%
       \getvalue{stop::\getvalue{Event::#1::type}}
      \ifnum 0=0\ArtPage{#1::firstpage}\relax
      \else
      \parfillskip=0pt \dotfill\null
      \rlap{\tt \inhex\ArtPage{#1::firstpage}}
      \setgvalue{Art::#1::found}{1}
      \fi\par
     \egroup\hfil\egroup}
\def \stopDay{}

\catcode `\_=13
\def_{{\tt \char `\_}}
% \input agenda

\iffalse
\startDay{pozostałe artykuły}
      \setgvalue{Art::ping2008::found}{1}

\def \Article#1#2{\ifnum1=0\getvalue{Art::#1::found}\relax
     \else #1\par\doAgendaEvent #1 ~:~ ~:~\relax\fi}

\PingArticles

\fi
\fi
%\xxxxxxxxxxxxxxxxx


\centerline{\bf Spis treści}
\blank[2*big]
\toks0={}

\newread \testarticle
\def \Article#1#2{\message{**ARTICLE: #1 ***}%
     \toks0=\expandafter{\the\toks0\relax\Article{#1}}
     \toks1={}% Authors
     \toks2={}% Title
     \openin \testarticle=#1/#1.tex\relax
     \ifeof \testarticle
        \closein \testarticle
     \else
        \closein \testarticle
        \expandafter \ScanArt \normalinput #1/#1.tex\relax\ScanArt
     \fi}

\def \ScanArt {\afterassignment \ScanArtA \let \next}
\def \ScanArtA{%
     \ifcase 0%
            \ifx\next \ScanArt 1\fi
            \ifx\next \Title   2\fi
            \ifx\next \Author  2\fi
            \ifx\next \MakeTitlePage 2\fi
            \relax
            % case 0, unknown token
            \expandafter \ScanArt
            \or
            % case 1, found \ScanArt, i.e. endofinput
            \expandafter \FinishScan
            \or
            % case 2
            \expandafter \next
            \else
              \errmessage{internal if/fi error}
            \fi}

\def \Title#1{\toks2=\expandafter{\the\toks2\relax #1\relax}%
     \ScanArt}
\def  \Author#1{\toks1=\expandafter{\the\toks1, #1}\ScanArt}
\def \MakeTitlePage{\endinput\ScanArt}
\def \FinishScan{\edef\next{{\the\toks1}{\the\toks2}}%
     \toks0=\expandafter{\the\toks0\next}}
\PingArticles
\def \Article#1#2#3{\bTR
     \bTD \tt 0x00\inhex \ArtPage{#1::firstpage}\relax\eTD
     \bTD #3\relax~--~\it \gobbleoneargument#2\relax\eTD
     \eTR}
\bTABLE
        \the\toks0
\eTABLE

\stoptext
